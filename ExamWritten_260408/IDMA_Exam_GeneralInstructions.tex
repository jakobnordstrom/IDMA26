
\begin{abstract}
%     \noindent
%     \textbf{Due:} \duedate.
%   
  \noindent
  \textbf{Submission:}
  Please write your solutions
%     on paper. Leave
  with
  ample margins on all
  sides, and make sure your handwriting is legible.
  \emph{Start your solution of every new problem on a new
    page. % sheet of paper.
%       Please make sure to mark every sheet
    Please mark every page
    with your name,
%       KU ID,
    exam number or something else that uniquely identifies your exam,}
  so that it is easy to see for every
%     sheet
  page
  which exam submission it is part of.
%   
%     Please submit your solutions  via \emph{Digital eksamen}.
%     It will be helpful if you state clearly 
%     your full name and your KU ID at the top of the first page.
%     From the point of view of grading, we 
%     are of course grateful if you typeset your solutions in
%     \LaTeX{} or some other math-aware  typesetting system. This is
%     \emph{not} a requirement, however, and how you write your
%     solutions will not affect  the grading as long as you
%     make sure that everything that you write is clearly legible
%     (and leave reasonable margins on all sides).
%   

  \noindent
  \textbf{Exposition:}
  Please try to be precise
%     and to the point
  in your solutions and
  refrain from vague statements.  
  Never, ever just state an answer, but always make sure to
  \emph{explain why} the answer is what it is.
  Provide clear references to any
%     course material used.
  facts in the course literature used.
  \emph{Write your solutions in such a way that 
    a    %%%    an imaginary 
    fellow
    student of yours could read, understand, and verify your
    solutions.}
%   
%     In addition to what is stated below, the general rules in the
%     official course information always apply.
%   
    
  \noindent
  \textbf{Collaboration:}
  All problems should be solved individually. No communication or
  collaboration is allowed, and solutions will be checked for
  plagiarism.

  \noindent
  \textbf{Reference material:}
  Textbooks and handwritten notes (including lecture notes) are allowed.
%     
%     Other typewritten material, such as problem sets or previous exams
%     with written solutions,  is not allowed.
%     
  Other typewritten material, including (but not limited to) problem
  sets or previous exams with solutions,  is not allowed.
%     
%     All course material is allowed, including  textbooks,
%     lecture notes, exercise sheets, and individual notes.
%   
%     You are \emph{not} supposed, and will not need,
%     to copy substantial parts of text verbatim,
%     and if your solutions make heavy use of reference material, then make
%     sure to provide clear references to what you are using.
%     
  Please note that you deviate from definitions and
%     algorithms as  described
  algorithm descriptions
  in the course material at your own risk! If, however,
  slight variations of algorithms were presented in the textbooks
  and/or in class,
%     then
  \mbox{such minor details do not matter.} 

  \noindent
  \textbf{Grading:}
  A score of 
  \thresholdforpass
  is guaranteed to be enough to pass the exam.

  \noindent
  \textbf{About the problems:}
  Note that the problem are of quite different types, and are
  \emph{not sorted in increasing order of difficulty.}
  \emph{Please read through the whole exam first,} 
  before you start working on any
  single problem, so that you can plan which order of dealing with the
  problems makes most sense to you.
  Note that this is a fairly large exam, and
%     so you can definitely get 
  you can get 
  a top grade without solving all problems.
  % You are not necessarily expected to be able to solve all of the problems.
  Also, partial answers to problems can sometimes give substantial
  amounts of points.
%   
%     Please do not hesitate to 
%   %     ask 
%   %     send a private message on \emph{Absalon}
%     alert the exam administrators
%     if any problem statement is   unclear. 
%   
  \textbf{\emph{Good luck!}}
\end{abstract}



%%% Local Variables:
%%% mode: latex
%%% TeX-master: "DMFS_Exam_230412"
%%% End:

% VERSION WITHOUT SOLUTIONS
\documentclass{jn-pset}
% Uncomment to get VERSION WITH SOLUTIONS and also do
%   \setboolean{versionwithsolutions}{true}
% a few lines down
%   \documentclass[solutions]{jn-pset}

\usepackage{ifthen}
\newboolean{versionwithsolutions}
% Uncomment to insert text that should only be there in version WITH solutions
\setboolean{versionwithsolutions}{true}
% Uncomment to insert text that should only be there in version WITHOUT solutions
\setboolean{versionwithsolutions}{false}



%%%%
%%%% QUICK INSTRUCTIONS FOR FORMATTING OF PROBLEMS
%%%%
%    
% Code a problem by
%    \begin{problem}
%    \end{problem}
%
% Code a subproblem inside a problem by (note the percent signs!)
%    \begin{subproblem}%
%        \label{problem:labelhere}%
%        Text here
%    \end{subproblem}
%
% Get a small vertical space by issuing the command
%    \smallskip
%
% For instance, to give a hint to a problem (after having completed
% the problem statement) code in the following way:
%
%    \smallskip
%     \noindent
%     \emph{Hint:}
%     Consider the following super-useful hint for this particular problem...
%

% PROBLEM-SET-SPECIFIC MACROS (UPDATE FOR EACH PROBLEM SET)
% *** NOT USED FOR EXAM ***
%
% \newcommand{\psetno}{5}
%%% The formatting macro \duedate is not used for the exam
%   \newcommand{\duedate}{Wednesday April 10 at 13:00 CET}
% Dummy formatting for draft versions
%   \newcommand{\thresholdA}{$\langle A \rangle$~points\xspace}
%   \newcommand{\thresholdB}{$\langle B \rangle$~points\xspace}
%   \newcommand{\thresholdC}{$\langle C \rangle$~points\xspace}
%   \newcommand{\thresholdD}{$\langle D \rangle$~points\xspace}
%   \newcommand{\thresholdE}{$\langle E \rangle$~points\xspace}
% Add actual grading thresholds here with \renewcommand 
%   \renewcommand{\thresholdA}{$200$~points\xspace}
%   \renewcommand{\thresholdB}{$170$~points\xspace}
%   \renewcommand{\thresholdC}{$140$~points\xspace}
%   \renewcommand{\thresholdD}{$110$~points\xspace}
%   \renewcommand{\thresholdE}{$80$~points\xspace}

% Threshold guaranteed to pass the exam
\newcommand{\thresholdforpass}{$220$~points\xspace}


% Compact lists
\usepackage{paralist}

% COURSE-SPECIFIC MACROS

% COURSE-SPECIFIC MACROS FOR IDMA 2025

\newcommand{\coursenameabbrev}{IDMA}
\newcommand{\coursenameshort}{Introduktion til diskret matematik og algoritmer}
\newcommand{\coursenamelong}{NDAB23002U Introduktion til diskret matematik og algoritmer}
%    
\newcommand{\submissionemail}{jn@di.ku.dk\xspace}
%   \newcommand{\courseinstructor}{Jakob Nordstr\"om}
\newcommand{\courseinstructor}
        {Jakob Nordstr\"om and Srikanth Srinivasan}
\newcommand{\courseperiod}{2024/2025}

% PACKAGES, MACROS, ET CETERA

\usepackage[T1]{fontenc}
\usepackage[utf8]{inputenc}
%%% Apparently a newer version of babel doesn't play well with nada-ten
%    \usepackage[english]{babel}

\usepackage{hyperref}

\usepackage{amsmath}
\usepackage{amssymb}
\usepackage{amsfonts}
\usepackage{mathtools}

% Provide calligraphic \mathscr font
\usepackage{mathrsfs}

% Enable use of MetaPost generated PostScript files
\usepackage{ifpdf}
\usepackage{graphicx}  
\ifpdf         
\DeclareGraphicsRule{*}{mps}{*}{}
\fi            

% For getting subfigures 1(a), 1(b) etc
% The package "subfigure" is obsolete, so switch to subcaption
%    \usepackage[sf,SF]{subfigure}
\usepackage{subcaption} 

% Sam Buss's package for formatting proofs
\usepackage{bussproofs}

% Extensions to verbatim commands
\usepackage{verbatim}

% Smiley
\usepackage{wasysym}

% To choose how to enumerate lists
\usepackage{enumerate}

%%%%
%%%% THIS FILE IS INTENDED TO BE READ-ONLY --- PLEASE DO NOT EDIT.
%%%% PLEASE CONTACT JAKOB NORDSTRÖM AT jn@di.ku.dk REGARDING ANY ISSUES.
%%%%

%
% DETECTION OF DOCUMENT TYPE
%----------------------------
%
% Version date: September 24, 2022
%
% First versions by Jakob Nordström <jn@di.ku.dk>
% Cleaned up version by Marc Vinyals <vinyals@kth.se>
% Minor later additions by Jakob Nordström and Susanna F. de Rezende
%
% This file needs to be included or input(ted) so that the conditional
% macro definitions in other LaTeX files will not generate compilation
% errors. Most document classes are detected using \@ifclassloaded.
%
% Use the file 'testdoctypedetection.tex' to double-check that these
% Boolean detectors work.
%
%
%%% The report class is not detected correctly in view of later
%%% updates, but this should be easy to fix when needed. 
%%% [Jakob Nordström, May 16, 2016]
% detectedReport is set to true if none of detectedArticle, detectedThesis,
% detectedSTOC, detectedFOCS, detectedSIAM, detectedIEEE, detectedICS,
% or detectedPoster is true.
%


\usepackage{ifthen}

\provideboolean{detectedSTOC}
\provideboolean{detectedFOCS}
\provideboolean{detectedElsevier}
\provideboolean{detectedNOW}
\provideboolean{detectedLMCS}
\provideboolean{detectedIEEE}
\provideboolean{detectedPoster}
\provideboolean{detectedSIAM}
\provideboolean{detectedLNCS}
\provideboolean{detectedACM}
\provideboolean{detectedACMconf}
\provideboolean{detectedSigplanconf}
\provideboolean{detectedToC}
\provideboolean{detectedLIPIcs}
\provideboolean{detectedAAAI}
\provideboolean{detectedIJCAI}
\provideboolean{detectedCompCplx}
\provideboolean{detectedEasyChair}
\provideboolean{detectedJAIR}
\provideboolean{detectedArticle}
\provideboolean{detectedReport}
\provideboolean{detectedThesis}

\makeatletter

\@ifclassloaded{sig-alternate}
{\setboolean{detectedSTOC}{true}}
{\setboolean{detectedSTOC}{false}}

\@ifclassloaded{elsarticle}
{\setboolean{detectedElsevier}{true}}
{\setboolean{detectedElsevier}{false}}

\@ifclassloaded{now}
{\setboolean{detectedNOW}{true}}
{\setboolean{detectedNOW}{false}}

\@ifclassloaded{lmcs}
{\setboolean{detectedLMCS}{true}}
{\setboolean{detectedLMCS}{false}}

\@ifclassloaded{IEEEtran} {
  \setboolean{detectedIEEE}{true}
  \ifCLASSOPTIONconference {
    \setboolean{detectedFOCS}{true}
  }
  \else {
    \setboolean{detectedFOCS}{false}
  }
  \fi
}
{
  \setboolean{detectedFOCS}{false}
  \setboolean{detectedIEEE}{false}
}

%%% Obsolete SIAM class file
%   \@ifclassloaded{siamltex1213} 
%   {\setboolean{detectedSIAM}{true}}
%   {\setboolean{detectedSIAM}{false}}
\@ifclassloaded{siamart171218}
{\setboolean{detectedSIAM}{true}}
{\setboolean{detectedSIAM}{false}}

\@ifclassloaded{llncs}
{\setboolean{detectedLNCS}{true}}
{\setboolean{detectedLNCS}{false}}

\@ifclassloaded{acmsmall}
{\setboolean{detectedACM}{true}}
{\setboolean{detectedACM}{false}}

\@ifclassloaded{acmart}
{\setboolean{detectedACMconf}{true}
 \setboolean{detectedACM}{true}}
{\setboolean{detectedACMconf}{false}}

\@ifclassloaded{sigplanconf}
{\setboolean{detectedSigplanconf}{true}}
{\setboolean{detectedSigplanconf}{false}}

\@ifclassloaded{toc}
{\setboolean{detectedToC}{true}}
{\setboolean{detectedToC}{false}}

\@ifclassloaded{lipics}
{\setboolean{detectedLIPIcs}{true}}
{\@ifclassloaded{lipics-v2019}
  {\setboolean{detectedLIPIcs}{true}}
  {\@ifclassloaded{oasics-v2019}
    {\setboolean{detectedLIPIcs}{true}}
    {\setboolean{detectedLIPIcs}{false}}
  }
}

%   
%   \@ifclassloaded{lipics}
%   {\setboolean{detectedLIPIcs}{true}}
%   {\@ifclassloaded{lipics-v2016}
%     {\setboolean{detectedLIPIcs}{true}}
%     {\@ifclassloaded{oasics-v2016}
%       {\setboolean{detectedLIPIcs}{true}}
%       {\setboolean{detectedLIPIcs}{false}}
%     }
%   }
%   

\@ifclassloaded{cc}
{\setboolean{detectedCompCplx}{true}}
{\setboolean{detectedCompCplx}{false}}

\@ifclassloaded{easychair}
{\setboolean{detectedEasyChair}{true}}
{\setboolean{detectedEasyChair}{false}}

%%%
%%% For JAIR, detect that the "jair" package is being used
%%%
\@ifpackageloaded{jair}
{\setboolean{detectedJAIR}{true}}        
{\setboolean{detectedJAIR}{false}}        

%%%
%%% AAAI and IJCAI have special style files that needs to be detected.
%%% It seems they update the name with the year of the conference also.
%%%
\@ifpackageloaded{aaai}
{\setboolean{detectedAAAI}{true}}        
{\@ifpackageloaded{aaai18}
  {\setboolean{detectedAAAI}{true}}
  {\@ifpackageloaded{aaai20}       
    {\setboolean{detectedAAAI}{true}}
    {\setboolean{detectedAAAI}{false}}}}

\@ifpackageloaded{ijcai18}
{\setboolean{detectedIJCAI}{true}}        
{\@ifpackageloaded{ijcai19}
  {\setboolean{detectedIJCAI}{true}}        
  {\setboolean{detectedIJCAI}{false}}}

\@ifclassloaded{sciposter}
{\setboolean{detectedPoster}{true}}
{\setboolean{detectedPoster}{false}}

\@ifclassloaded{article}
{\setboolean{detectedArticle}{true}}
{\setboolean{detectedArticle}{false}}

\makeatother

\ifthenelse{\not \isundefined{\examen} 
  \and \not \isundefined{\disputationsdatum} 
  \and \not \isundefined{\disputationslokal}}   
  {\setboolean{detectedThesis}{true}}
  {\setboolean{detectedThesis}{false}}

%%%
%%% Not entirely sure whether detectedReport is set correctly in view
%%% of later updates [Jakob Nordström, May 16, 2016]
%%%
           
\ifthenelse{\boolean{detectedArticle} \or \boolean{detectedThesis}
  \or \boolean{detectedSTOC}    \or \boolean{detectedFOCS}
  \or \boolean{detectedSIAM}    \or \boolean{detectedIEEE}
  \or \boolean{detectedACMconf} \or \boolean{detectedACM}
  \or \boolean{detectedPoster}}
{\setboolean{detectedReport}{false}}
{\setboolean{detectedReport}{true}}



%%%%
%%%% THIS FILE IS INTENDED TO BE READ-ONLY --- PLEASE DO NOT EDIT.
%%%% PLEASE CONTACT JAKOB NORDSTRÖM AT jakobn@kth.se REGARDING ANY ISSUES.
%%%%

% GENERAL MACROS TO USE IN LaTeX-FILES
%======================================
%
% AUTHOR
%   Jakob Nordström <jakobn@kth.se>
%   Some improvements added by Marc Vinyals <vinyals@kth.se>
%
% VERSION
%   Last updated June 7, 2019
%    
% KNOWN ISSUES:
%   References with page numbers such as \refsecP, \refthP, etc will not
%   work with the Elsevier and SIAM document classes. No work-arounds
%   have been added, so compilation will fail if these macros are used
%   for Elsevier or SIAM articles.

%
% MACRO NAMING CONVENTION FOR MATHEMATICAL MACROS WITH DELIMITERS 
%-----------------------------------------------------------------
%
% For mathematical macros with delimiters there are usually
% three different flavours corresponding to different sizes of 
% the delimiters as follows:
%
%    \newcommand{\mycommand}[1]{<command> ( {#1} )}
%    \newcommand{\Mycommand}[1]{<command> \bigl( {#1} \bigr)}
%    \newcommand{\MYCOMMAND}[1]{<command> \left( {#1} \right)}
%

%
% REQUIRED PACKAGES
%-------------------
%

\usepackage{ifthen}
\usepackage{xspace}
% varioref does not seem to mix well with the Elsevier document class
\ifthenelse
{\boolean{detectedElsevier} \or \boolean{detectedSIAM} 
  \or \boolean{detectedLIPIcs}}
{}
{\usepackage{varioref}}


%
% M I S C E L L A N E O U S 
%---------------------------
%

\DeclareMathAlphabet{\mathsfsl}{OT1}{cmss}{m}{sl}


%
% G E N E R A L   F O R M A T T I N G   R U L E S 
%-------------------------------------------------
%
% To achieve some kind of consistency in the notation
%

% Format of functions to integers or real numbers
\newcommand{\formatfunctiontonumbers}[1]{\mathrm{#1}}

% Format of functions to sets
\newcommand{\formatfunctiontoset}[1]{\mathit{#1}}

% Dots in x_1 \lor ... \lor x_n and the like
% (make a generic macro that can be changed according to publisher
% requirements)  
\newcommand{\formuladots}{\cdots}


%
% B I G - O H   N O T A T I O N 
%-------------------------------
%

\newcommand{\BIGOH}[1]{\mathrm{O} \left( #1 \right)}
\newcommand{\Bigoh}[1]{\mathrm{O} \bigl( #1 \bigr)}
\newcommand{\bigoh}[1]{\mathrm{O} ( #1 )}
\newcommand{\LITTLEOH}[1]{\mathrm{o} \left( #1 \right)}
\newcommand{\Littleoh}[1]{\mathrm{o} \bigl( #1 \bigr)}
\newcommand{\littleoh}[1]{\mathrm{o} ( #1 )}
\newcommand{\BIGTHETA}[1]{\Theta \left( #1 \right)}
\newcommand{\Bigtheta}[1]{\Theta \bigl( #1 \bigr)}
\newcommand{\bigtheta}[1]{\Theta ( #1 )}
\newcommand{\BIGOMEGA}[1]{\Omega \left( #1 \right)}
\newcommand{\Bigomega}[1]{\Omega \bigl( #1 \bigr)}
\newcommand{\bigomega}[1]{\Omega ( #1 )}
\newcommand{\LITTLEOMEGA}[1]{\omega \left( #1 \right)}
\newcommand{\Littleomega}[1]{\omega \bigl( #1 \bigr)}
\newcommand{\littleomega}[1]{\omega ( #1 )}
\newcommand{\POLYBOUND}[1]{\mathrm{poly} \left( #1 \right)}
\newcommand{\Polybound}[1]{\mathrm{poly} \bigl( #1 \bigr)}
\newcommand{\polybound}[1]{\mathrm{poly} ( #1 )}
\newcommand{\POLYLOGBOUND}[1]{\mathrm{polylog} \left( #1 \right)}
\newcommand{\Polylogbound}[1]{\mathrm{polylog} \bigl( #1 \bigr)}
\newcommand{\polylogbound}[1]{\mathrm{polylog} ( #1 )}
            
\DeclareMathOperator{\polylog}{polylog}


%
% G E N E R A L  M A T H E M A T I C A L   N O T A T I O N
%----------------------------------------------------------
%

% N, Z, Q, R as symbols for classes of numbers
\ifthenelse{\boolean{detectedToC}}{}
{
  \newcommand{\Q}         {\mathbb{Q}}
  \newcommand{\R}         {\mathbb{R}}
  \newcommand{\Rplus}     {\mathbb{R}^{+}}
  \newcommand{\N}         {\mathbb{N}}
  \newcommand{\Nplus}     {\mathbb{N}^{+}}
  \newcommand{\Nzero}     {\mathbb{N}_{0}}
  \newcommand{\Z}         {\mathbb{Z}}
}

% Sigma sum sign with indices to the right, not below and above
\newcommand{\sumnodisplay}{{\textstyle \sum}}

% Absolute value and norm
\providecommand{\abs}[1]{\lvert#1\rvert}
\providecommand{\Abs}[1]{\bigl\lvert#1\bigr\rvert}
\providecommand{\ABS}[1]{\left\lvert#1\right\rvert}
\providecommand{\norm}[1]{\lVert#1\rVert}
\providecommand{\Norm}[1]{\bigl\lVert#1\bigr\rVert}
\providecommand{\NORM}[1]{\left\lVert#1\right\rVert}

% Exists unique
\newcommand{\existsunique}{\exists!}

% Rounding
\newcommand{\ceiling}[1]{\lceil #1 \rceil}
\newcommand{\Ceiling}[1]{\bigl \lceil #1 \bigr \rceil}
\newcommand{\CEILING}[1]{\left \lceil #1 \right \rceil}

\newcommand{\floor}[1]{\lfloor #1 \rfloor}
\newcommand{\Floor}[1]{\bigl \lfloor #1 \bigr \rfloor}
\newcommand{\FLOOR}[1]{\left \lfloor #1 \right \rfloor}

\newcommand{\intpart}[1]{\lceil #1 \rfloor}
\newcommand{\Intpart}[1]{\bigl \lceil #1 \bigr \rfloor}
\newcommand{\INTPART}[1]{\left \lceil #1 \right \rfloor}

% Max and min
% Don't use \maxof and \minof to avoid conflict with calc package
\newcommand{\MAXOFEXPR}[2][]{\max_{#1} \left\{ #2 \right\}}
\newcommand{\MINOFEXPR}[2][]{\min_{#1} \left\{ #2 \right\}}
\newcommand{\Maxofexpr}[2][]{\max_{#1} \bigl\{ #2 \bigr\}}
\newcommand{\Minofexpr}[2][]{\min_{#1} \bigl\{ #2 \bigr\}}
\newcommand{\maxofexpr}[2][]{\max_{#1} \{ #2 \}}
\newcommand{\minofexpr}[2][]{\min_{#1} \{ #2 \}}

\newcommand{\maxofset}[3][:]{\max \{ #2 #1 #3 \}}
\newcommand{\minofset}[3][:]{\min \{ #2 #1 #3 \}}
 
\newcommand{\MAXOFSET}[3][:]%
     {\ifthenelse{\equal{#1}{;}}%
     {\MAXOFEXPR{ #2 \,;\, #3 }}
     {\ifthenelse{\equal{#1}{:}}%
     {\MAXOFEXPR{ #2 \,:\, #3 }}
     {\max \twincommandJN{\left\{}{#2}{\left#1}{\right}{\,#3}{\right\}}}}}
\newcommand{\MINOFSET}[3][:]%
     {\ifthenelse{\equal{#1}{;}}%
     {\MINOFEXPR{ #2 \,;\, #3 }}
     {\ifthenelse{\equal{#1}{:}}%
     {\MINOFEXPR{ #2 \,:\, #3 }}
     {\min \twincommandJN{\left\{}{#2}{\left#1}{\right}{\,#3}{\right\}}}}}

\newcommand{\Maxofset}[3][:]%
     {\ifthenelse{\equal{#1}{;}}%
     {\Maxofexpr{ #2 \,;\, #3 }}
     {\ifthenelse{\equal{#1}{:}}%
     {\Maxofexpr{ #2 \,:\, #3 }}
     {\max \twincommandJN{\bigl\{}{#2}{\bigl#1}{\bigr}{\,#3}{\bigr\}}}}}
\newcommand{\Minofset}[3][:]%
     {\ifthenelse{\equal{#1}{;}}%
     {\Minofexpr{ #2 \,;\, #3 }}
     {\ifthenelse{\equal{#1}{:}}%
     {\Minofexpr{ #2 \,:\, #3 }}
     {\min \twincommandJN{\bigl\{}{#2}{\bigl#1}{\bigr}{\,#3}{\bigr\}}}}}


%
% A L G E B R A
%---------------
%

% Some linear algebra
\newcommand{\transpose}[1]{\ensuremath{#1^{\top}}}
\newcommand{\innerproduct}[2]{\langle #1, #2 \rangle}
\newcommand{\Innerproduct}[2]{\bigl\langle #1, #2 \bigr\rangle}
\newcommand{\INNERPRODUCT}[2]{\left\langle #1, #2 \right\rangle}

% Generic field
\newcommand{\fieldstd}{\mathbb{F}}
\newcommand{\fieldf}{\mathbb{F}}
\newcommand{\F}{\mathbb{F}}

% Finite fields
\newcommand{\GF}[1]{\mathrm{GF} ( #1 )}
\newcommand{\gf}[1]{\mathrm{GF} ( #1 )}
\newcommand{\GFmul}[1]{\mathrm{GF} ( #1 )^{*}}
\newcommand{\gfmul}[1]{\mathrm{GF} ( #1 )^{*}}


%
% P R O B A B I L I T Y   T H E O R Y 
%-------------------------------------

% AMS-TeX defines an operator name \Pr
\DeclareMathOperator{\Expop}{E}
\DeclareMathOperator{\Varianceop}{Var}

% Probability
\newcommand{\PROB}[2][]{\Pr_{#1} \left[ #2 \right]}
\newcommand{\Prob}[2][]{\Pr_{#1} \bigl[ #2 \bigr]}
\ifthenelse{\boolean{detectedLMCS}}
{\renewcommand{\prob}[2][]{\Pr_{#1} [ #2 ]}}
{\newcommand{\prob}[2][]{\Pr_{#1} [ #2 ]}}

% Expectation
\newcommand{\EXPECTATION}[2][]{\Expop_{#1} \left[ #2 \right]}
\newcommand{\Expectation}[2][]{\Expop_{#1} \bigl[ #2 \bigr]}
\newcommand{\expectation}[2][]{\Expop_{#1} [ #2 ]}
\newcommand{\VARIANCE}[1]{\Varianceop \left( #1 \right)}
\newcommand{\Variance}[1]{\Varianceop \bigl( #1 \bigr)}
\newcommand{\variance}[1]{\Varianceop ( #1 )}

% 
% INTERLUDE: MATCHING MIDDLE SEPARATORS (FROM THE UK TeX FAQ)
% 
% 
% One of the few glaring omissions from TeX's mathematical typesetting
% capabilities is a means of setting separators in the middle of
% mathematical expressions. In all sorts of mathematical enterprises one
% may find oneself needing a \middle command, to be used in expressions
% like \left\{ x \in \mathbb{N} \middle| x \mbox{ even} \right\} to
% specify the set of even natural numbers. The e-TeX system defines just
% such a command, but users of Knuth's original need some support.
% Donald Arseneau's braket package provides commands for set
% specifications (as above) and for Dirac brackets (and bras and kets).
% The package uses the e-TeX built-in command if it finds itself running
% under e-TeX.
% 
% See ftp://cam.ctan.org/tex-archive/macros/latex/contrib/misc/braket.sty .
% 
% Or one can do as below.
%

\newcommand{\twincommandJN}[6]%
    {#1#2#3\vphantom{#2#5}\mspace{-2.05mu}#4.#5#6}

% Perhaps this is superfluous---in text mode there is no need for measuring
% with \vphantom, I think, since \bigl[ and \bigr] are what they are 
% independent of what is inside (are they not?).
%
% The length -2.25mu probably should be set instead by doing sth like
%    
%    \newlength{\lengthJN}
%    \settowidth{\lengthJN}{$\left.\right.$}
%    \setlength{\lengthJN}{0.5\lengthJN}
%
% and then using \mspace{-\lengthJN}, but the difference appears to be
% very small so I have not implemented this.


% CONDITIONAL EXPECTATION
\newcommand{\condexp}[2]{\Expop{#1  \mid  #2}}
\newcommand{\CondExp}[2]%
    {\Expop\twincommandJN{\bigl[}{#1}{\bigl|}{\bigr}{\,#2}{\bigr]}}
\newcommand{\CONDEXP}[2]%
     {\Expop\twincommandJN{\left[}{#1}{\left|}{\right}{\,#2}{\right]}}

% CONDITIONAL PROBABILITY
\newcommand{\condprob}[3][]{\prob[#1]{#2  \mid  #3}}
\newcommand{\Condprob}[3][]%
    {\Pr_{#1}\twincommandJN{\bigl[}{#2}{\bigl|}{\bigr}{\,#3}{\bigr]}}
\newcommand{\CONDPROB}[3][]%
    {\Pr_{#1}\twincommandJN{\left[}{#2}{\left|}{\right}{\,#3}{\right]}}

%
% Example code:
%    
%    \begin{displaymath}
%     \CONDEXP{\sum_{i=1}^kX_i}{Z}\quad\mbox{and}\quad%
%     \CONDPROB{B\land C}{\bigwedge_{i\in S}A_i}
%    \end{displaymath}
%    
%    $\CondExp{\sum_{i=1}^k X_i}{Z}$  
%    and
%    $\CondProb{B\land C}{\bigwedge_{i\in S}A_i}$
%    


%
% F U N C T I O N S
%-------------------
%

% DESCRIPTION OF FUNCTION
\newcommand{\funcdescr}[3]{\ensuremath{ #1 : #2 \to #3}}

% DOMAIN
\newcommand{\domainof}[1]{\ensuremath{\mathrm{dom} ( #1 )}}
\newcommand{\Domainof}[1]{\ensuremath{\mathrm{dom}\bigl( #1 \bigr)}}

% INVERSE IMAGE
\newcommand{\invimageof}[2]{{\ensuremath{{#1}^{-1} \left( #2 \right)}}}

%
% G R A P H S
%-------------
%

\newcommand{\edges}[1]{E( #1 )}
\newcommand{\Edges}[1]{E\bigl( #1 \bigr)}
\newcommand{\vertices}[1]{V( #1 )}
\newcommand{\Vertices}[1]{V\bigl( #1 \bigr)}

\newcommand{\vdegree}[2][]{\mathrm{deg}_{#1}(#2)}
\newcommand{\Vdegree}[2][]{\mathrm{deg}_{#1}\bigl(#2\bigr)}
\newcommand{\vneighbour}[2][]{N_{#1}({#2})}
\newcommand{\Vneighbour}[2][]{N_{#1}\bigl({#2}\bigr)}

% Boundary
\newcommand{\boundary}[1]{\ensuremath{\partial #1}}

\newcommand{\pathstd}{\ensuremath{P}}
\newcommand{\pathalt}{\ensuremath{Q}}
\newcommand{\pathfromto}[3]{#1 : #2 \rightsquigarrow #3}


%
% S E T S   A N D   T U P L E S
%-------------------------------
%

\newcommand{\set}[1]{\{ #1 \}}
\newcommand{\Set}[1]{\bigl\{ #1 \bigr\}}
\newcommand{\SET}[1]{\left\{ #1 \right\}}

\newcommand{\setdescr}[3][\mid]{\set{ #2 #1 #3 }}
\newcommand{\Setdescr}[3][|]%
     {\ifthenelse{\equal{#1}{;}}%
     {\Set{ #2 \,;\, #3 }}
     {\ifthenelse{\equal{#1}{:}}%
     {\Set{ #2 \,:\, #3 }}
     {\twincommandJN{\bigl\{}{#2\,}{\bigl#1}{\bigr}{\,#3}{\bigr\}}}}}
\newcommand{\SETDESCR}[3][|]%
     {\twincommandJN{\left\{}{#2\,}{\left#1}{\right}{\,#3}{\right\}}}

\newcommand{\setbrackets}[1]{[ #1 ]}
\newcommand{\Setbrackets}[1]{\bigl[ #1 \bigr]}
\newcommand{\SETBRACKETS}[1]{\left[ #1 \right]}

\newcommand{\setdescrbrackets}[3][\mid]{{\setbrackets{ #2 #1 #3 }}}
\newcommand{\Setdescrbrackets}[3][|]%
     {\twincommandJN{\bigl[}{#2}{\bigl#1}{\bigr}{\,#3}{\bigr]}}
\newcommand{\SETDESCRBRACKETS}[3][|]%
     {\twincommandJN{\left[}{#2}{\left#1}{\right}{\,#3}{\right]}}

\newcommand{\SETSIZE}[1]{\left\lvert#1\right\rvert}
\newcommand{\Setsize}[1]{\bigl\lvert#1\bigr\rvert}
\newcommand{\setsize}[1]{\lvert#1\rvert}

% Set complement
\newcommand{\setcompl}[1]{\overline{#1}}

% Intersection and union
\newcommand{\intersection}{\cap}
\newcommand{\Intersection}{\bigcap}
\newcommand{\Intersectionnodisplay}{\textstyle \bigcap}

\newcommand{\union}{\cup}
\newcommand{\Union}{\bigcup}
\newcommand{\Unionnodisplay}{\textstyle \bigcup}

% Intersection and union with some space
\newcommand{\unionSP}{\, \union \, }
\newcommand{\intersectionSP}{\, \intersection \, }

% Disjoint union 
%
% Can see no difference between below definition and \mathbin{\dot{\cup}}
\newcommand{\disjointunion}{\overset{.}{\cup}}
\newcommand{\disjointunionSP}{\disjointunion}
\newcommand{\Disjointunion}{\overset{.}{\bigcup}}
\newcommand{\disjunion}{\disjointunion}
\newcommand{\Disjunion}{\Disjointunion}

% First n positive integers
\newcommand{\nset}[1]{[{#1}]}
\newcommand{\Nset}[1]{\bigl[ {#1} \bigr]}


%
% L O G I C
%-----------
%

%
% Logic connectives
%
% Logic or is \lor. Logic and is \land. Logic not is \lnot.
% They can be used in math mode only.
\newcommand{\Lor}{\bigvee}
\newcommand{\Land}{\bigwedge}

% nodisplay = indices to the right, not below and above
\newcommand{\Lornodisplay}{{\textstyle \bigvee}}
\newcommand{\Landnodisplay}{{\textstyle \bigwedge}}

\newcommand{\limpl}{\rightarrow}
\newcommand{\lequiv}{\leftrightarrow}

% Prefixed NOT (pfnot):  \lnot x
% Overlined NOT (olnot): \overline{x}
% \stdnot{lit} is the standard NOT notation for variables and literals

\newcommand{\pfnot}[1]{\lnot #1}
\newcommand{\olnot}[1]{\overline{#1}}
\newcommand{\stdnot}[1]{\olnot{#1}}
    
% syntactic equivalence
\newcommand{\synteq}{\doteq}

% constants "true" and "false"
\newcommand{\FALSE}{\mathit{FALSE}}
\newcommand{\TRUE}{\mathit{TRUE}}

\newcommand{\false}{\bot}
\newcommand{\true}{\top}

\newcommand{\falsenum}{0}
\newcommand{\truenum}{1}

%
% Notation and terms for CNF/DNF formulas
%
% Standard notation for parameters in k-CNF formulas
\newcommand{\nvar}{n}
\newcommand{\nvars}{\nvar}
\newcommand{\nclause}{m}
\newcommand{\nclauses}{\nclause}
\newcommand{\clwidth}{k}
\newcommand{\density}{\Delta}

% Formatting of k-CNF/k-DNF in running text
\newcommand{\xdnf}[1]{\mbox{\ensuremath{#1}-DNF}\xspace}
\newcommand{\kdnf}{\xdnf{\clwidth}}
\newcommand{\xdnfform}[1]{\mbox{\ensuremath{#1}-DNF} formula\xspace}
\newcommand{\kdnfform}{\xdnfform{\clwidth}}

\newcommand{\xcnf}[1]{\mbox{\ensuremath{#1}-CNF}\xspace}
\newcommand{\kcnf}{\xcnf{\clwidth}}
\newcommand{\xcnfform}[1]{\mbox{\ensuremath{#1}-CNF} formula\xspace}
\newcommand{\kcnfform}{\xcnfform{\clwidth}}
\newcommand{\xclause}[1]{\mbox{\ensuremath{#1}-clause}\xspace}
\newcommand{\kclause}{\xclause{\clwidth}}

\newcommand{\Excnf}[1]{\mbox{\ensuremath{\mathrm{E}#1}-CNF}\xspace}
\newcommand{\Ekcnf}{\Excnf{\clwidth}}
\newcommand{\Excnfform}[1]{\mbox{\ensuremath{\mathrm{E}#1}-CNF} formula\xspace}
\newcommand{\Ekcnfform}{\Excnfform{\clwidth}}

\newcommand{\xterm}[1]{\mbox{\ensuremath{#1}-term}\xspace}
\newcommand{\kterm}{\xterm{\clwidth}}
\newcommand{\Exterm}[1]{\mbox{\ensuremath{\mathrm{E}#1}-term}\xspace}
\newcommand{\Ekterm}{\Exterm{\clwidth}}

%
% Distributions for random k-CNF formulas
%
% fixed # random clauses chosen with replacement
% \randkcnfnclwrepl{clause width}{#variables}{#clauses}
%
\newcommand{\randkcnfnclwrepl}[3][\clwidth]%
        {\ensuremath{\mathcal{F}^{#2, #3}_{#1}}}
\newcommand{\randkcnfnclwreplstd}% 
        {\randkcnfnclwrepl{\clwidth}{\nvar}{\nclause}}


%
% C O M P L E X I T Y   T H E O R Y 
%-----------------------------------
%

% NOTATION FOR LANGUAGES/PROBLEMS
\newcommand{\problemlanguageformat}[1]{\textsc{#1}\xspace}
\newcommand{\problemlanguageformatnospace}[1]{\textsc{#1}}
\newcommand{\langstd}{\ensuremath{L}}
\newcommand{\langcompl}[1]{\ensuremath{\overline{#1}}}

\newcommand{\PROP}{\problemlanguageformat{PROP}}
\newcommand{\TAUT}{\problemlanguageformat{TAUT}}
\newcommand{\TAUTOLOGY}{\problemlanguageformat{Tautology}}
\newcommand{\SAT}{\problemlanguageformat{Sat}}
\newcommand{\CNFSAT}{\problemlanguageformat{CnfSat}}
\newcommand{\SATISFIABILITY}{\problemlanguageformat{Satisfiability}}
\newcommand{\THREESAT}{\text{$3$-}\problemlanguageformat{Sat}}
\newcommand{\TWOSAT}{\text{$2$-}\problemlanguageformat{Sat}}
\newcommand{\ONEINTHREESAT}{\text{$1$-}\problemlanguageformatnospace{In}%
        \text{-$3$-}\problemlanguageformat{Sat}}
\newcommand{\MAXSAT}{\problemlanguageformat{MaxSat}}
\newcommand{\MAXTWOSAT}{\problemlanguageformatnospace{Max}%
        \text{-$2$-}\problemlanguageformat{Sat}}
\newcommand{\CIRCUITSAT}{\problemlanguageformat{CircuitSat}}
\newcommand{\NAESAT}{\problemlanguageformat{NotAllEqualSat}}
\newcommand{\DOMINATINGSET}{\problemlanguageformat{DominatingSet}}
\newcommand{\VERTEXCOVER}{\problemlanguageformat{VertexCover}}
\newcommand{\MAXCUT}{\problemlanguageformat{MaxCut}}
\newcommand{\DOMATICNUMBER}{\problemlanguageformat{DomaticNumber}}
\newcommand{\MONOCHROMTRI}{\problemlanguageformat{MonochromaticTriangle}}
\newcommand{\CLIQUECOVER}{\problemlanguageformat{CliqueCover}}
\newcommand{\INDSET}{\problemlanguageformat{IndependentSet}}
\newcommand{\GCLIQUE}{\problemlanguageformat{Clique}}
\newcommand{\GCOLOURING}{\problemlanguageformat{Colouring}}
\newcommand{\SUBGRAPHISO}{\problemlanguageformat{Subgraph\-Isomorphism}}
\newcommand{\GKERNEL}{\problemlanguageformat{Kernel}}
\newcommand{\MINMAXMATCHING}{\problemlanguageformat{MinMaxMatching}}
\newcommand{\CUBICSUBGRAPH}{\problemlanguageformat{CubicSubgraph}}

% NOTATION FOR COMPLEXITY CLASSES
\newcommand{\complclassformat}[1]%
        {\textrm{\upshape{\textsf{#1}}}\xspace}
\newcommand{\cocomplclass}[1]%
        {\textrm{\upshape{\textsf{co#1}}}\xspace}

\newcommand{\TIMEclass}[1]{\ensuremath{\complclassformat{TIME}\bigl(#1\bigr)}}
\newcommand{\SPACEclass}[1]{\ensuremath{\complclassformat{SPACE}\bigl(#1\bigr)}}
\newcommand{\DTIMEclass}[1]{\ensuremath{\complclassformat{DTIME}\bigl(#1\bigr)}}
\newcommand{\DTIMEadviceclass}[2]%
    {\ensuremath{\complclassformat{DTIME}\bigl(#1\bigr)/{#2}}}
\newcommand{\Pclass}{\complclassformat{P}}
\newcommand{\NP}{\complclassformat{NP}}
\newcommand{\NPclass}{\NP}
\newcommand{\coNP}{\cocomplclass{NP}}
\newcommand{\coNPclass}{\coNP}
\newcommand{\CoNP}{\coNP}
\newcommand{\Ppoly}{\complclassformat{P/poly}}
\newcommand{\Pslashpoly}{\Ppoly}
\newcommand{\Pslashpolyclass}{\Pslashpoly}
\newcommand{\PSPACE}{\complclassformat{PSPACE}}
\newcommand{\PSPACEclass}{\PSPACE}
\newcommand{\NPSPACE}{\complclassformat{NPSPACE}}
\newcommand{\NPSPACEclass}{\NPSPACE}
\newcommand{\pspace}{\PSPACE}
\newcommand{\EXPTIME}{\complclassformat{EXPTIME}}
\newcommand{\exptime}{\complclassformat{EXPTIME}}
\newcommand{\EXPSPACE}{\complclassformat{EXPSPACE}}
\newcommand{\expspace}{\complclassformat{EXPSPACE}}
\newcommand{\EXP}{\complclassformat{EXP}}
\newcommand{\NEXP}{\complclassformat{NEXP}}
\newcommand{\coNEXP}{\cocomplclass{NEXP}}
\newcommand{\Lclass}{\complclassformat{L}}
\newcommand{\LOGSPACE}{\complclassformat{L}}
\newcommand{\NL}{\complclassformat{NL}}
\newcommand{\NLclass}{\NL}
\newcommand{\coNL}{\cocomplclass{NL}}
\newcommand{\NCclass}{\complclassformat{NC}}
\newcommand{\DP}{\complclassformat{DP}}
\newcommand{\DPclass}{\DP}
\newcommand{\SIGMACLASS}[1]{\ensuremath{\Sigma^p_{#1}}}
\newcommand{\Sigmaclass}[1]{\SIGMACLASS{#1}}
\newcommand{\PICLASS}[1]{\ensuremath{\Pi^p_{#1}}}
\newcommand{\Piclass}[1]{\PICLASS{#1}}
\newcommand{\PH}{\complclassformat{PH}}
\newcommand{\EXPEXP}{\complclassformat{EXPEXP}}
\newcommand{\NEXPEXP}{\complclassformat{NEXPEXP}}
\newcommand{\coNEXPEXP}{\cocomplclass{NEXPEXP}}        
\newcommand{\BPP}{\complclassformat{BPP}}
\newcommand{\ZPP}{\complclassformat{ZPP}}
\newcommand{\RP}{\complclassformat{RP}}
\newcommand{\coRP}{\cocomplclass{RP}}
\newcommand{\MIParg}[1]{\ensuremath{\complclassformat{MIP}[#1]}}
\newcommand{\MIP}{\complclassformat{MIP}}
\newcommand{\IP}{\complclassformat{IP}}
\newcommand{\PCPnot}{\complclassformat{PCP}}
% PCP{completeness}{soundness}{randomness}{queries}{alphabet size}
\newcommand{\PCPalph}[5]%
    {\ensuremath{\complclassformat{PCP}_{{#1},{#2}}[{#3}, {#4}, {#5}]}}
\newcommand{\PCP}[4]%
    {\ensuremath{\complclassformat{PCP}_{{#1},{#2}}[{#3}, {#4}]}}

\newcommand{\SAC}{\complclassformat{SAC}}
\newcommand{\NC}{\complclassformat{NC}}
\newcommand{\ACzero}{\ensuremath{\complclassformat{AC}^{0}}}
\newcommand{\SC}{\complclassformat{SC}}
\newcommand{\TISP}{\complclassformat{TISP}}
\newcommand{\LOGCFL}{\complclassformat{LOGCFL}}


%
% T E X T   F O R M A T T I N G
%-------------------------------
%

% introducing a new term or mentioning a familiar one for the first time
%\newcommand{\introduceterm}[1]{{\textsl{#1}}}
\newcommand{\introduceterm}[1]{{\emph{#1}}}

% Spacing before punctuation in displayed equations according to LLNCS
% (But SIAM disagrees, so it is convenient to have a macro for this.)
\newcommand{\eqperiod}{\enspace .}
\newcommand{\eqcomma}{\enspace ,}

% For description lists with items in italics (to avoid confusion
% with theorem headers, for instance)
\newcommand{\italicitem}[1][]{\item[\textit{#1}]}


%
% A B B R E V I A T I O N S   O F   F R E Q U E N T     E X P R E S S I O N S
%-----------------------------------------------------------------------------
%

% GENERAL EXPRESSIONS
\newcommand{\wrt}{with respect to\xspace}
\newcommand{\wrtabbrev}{w.r.t.\ }
\ifthenelse{\boolean{detectedToC}}{}
  {\newcommand{\eg}{for instance\xspace} % should be surrounded by commas 
    \newcommand{\Eg}{For instance\xspace}}
\ifthenelse{\boolean{detectedToC}}{}{
\newcommand{\ie}{i.e.,\ }
\newcommand{\Ie}{I.e.,\ }
}
\newcommand{\ieComma}{i.e.,\ }  %%%% Special hack to use when adapting to ToC
\newcommand{\ieNoComma}{i.e.\ }
\ifthenelse{\boolean{detectedLIPIcs} \or \boolean{detectedIJCAI}}
{\renewcommand{\st}{\errmessage{Please do not use st}}}
{\ifthenelse{\isundefined{\st}}
  {\newcommand{\st}{such that\xspace}}
  {}}      
%   \newcommand{\st}{such that\xspace}}

\newcommand{\etal}{et al.\@\xspace}
\newcommand{\etalS}{et al\@. }

\newcommand{\vs}{vs.\ }

% TYPICAL MATHEMATICAL EXPRESSIONS
\newcommand{\ifaoif}{if and only if\xspace}
\newcommand{\wolog}{without loss of generality\xspace}
\newcommand{\Wolog}{Without loss of generality\xspace}
\newcommand{\iid}{independently and identically distributed\xspace}
\ifthenelse{\boolean{detectedIEEE}}{}{\newcommand{\QED}{Q.E.D.}}
\newcommand{\qedlong}{which was to be proved\xspace}

\newcommand{\whp}{with high probability\xspace}
\newcommand{\Whp}{With high probability\xspace}
\newcommand{\aas}{asymptotically almost surely\xspace}
\newcommand{\Aas}{Asymptotically almost surely\xspace}


%
% R E F E R E N C E S 
%---------------------
%
% Macros with capital initial letter intended for use at start of sentence.
%

%
% REFERENCES TO SECTIONAL UNITS AND PAGE INTERVALS
%
% The LLNCS document class and instructions says that in the running text, 
% but not at the beginning of a sentence, Sect., Chap. and Fig. should be 
% abbreviated.
%
% LLNCS wants, e.g., "Section 4.3" with capital S, SIAM doesn't.
%

% Sections
\newcommand{\refsec}[1]{Section~\ref{#1}}
\newcommand{\refsecP}[1]{Section~\vref{#1}}
\newcommand{\Refsec}[1]{Section~\ref{#1}}
\newcommand{\RefsecP}[1]{Section~\vref{#1}}
\newcommand{\reftwosecs}[2]{Sections~\ref{#1} and~\ref{#2}}
\newcommand{\refthreesecs}[3]{Sections~\ref{#1}, \ref{#2}, and~\ref{#3}}

% Chapters
\newcommand{\refch}[1]{Chapter~\ref{#1}}
\newcommand{\Refch}[1]{Chapter~\ref{#1}}
\newcommand{\reftwochs}[2]{Chapters~\ref{#1} and~\ref{#2}}
\newcommand{\refthreechs}[3]{Chapters~\ref{#1}, \ref{#2} and~\ref{#3}}

% Appendices
\newcommand{\refapp}[1]{Appendix~\ref{#1}}
\newcommand{\Refapp}[1]{Appendix~\ref{#1}}
\newcommand{\reftwoapps}[2]{Appendices~\ref{#1} and~\ref{#2}}

% Figures
\newcommand{\reffig}[1]{Figure~\ref{#1}}
\newcommand{\reffigP}[1]{Figure~\vref{#1}}
\newcommand{\Reffig}[1]{Figure~\ref{#1}}
\newcommand{\ReffigP}[1]{Figure~\vref{#1}}
\newcommand{\reftwofigs}[2]{Figures~\ref{#1} and~\ref{#2}}
\newcommand{\refthreefigs}[3]{Figures~\ref{#1}, \ref{#2}, and~\ref{#3}}


%
% REFERENCES TO THEOREM-LIKE ENVIRONMENTS
%
% Requires \usepackage{varioref}
%
% Adapted to LLNCS document class and instructions. Definition, Theorem etc
% should be capitalized when followed by a number.
%

% Theorems
\newcommand{\refth}[1]{Theorem~\ref{#1}}
\newcommand{\reftwoths}[2]{Theorems~\ref{#1} and~\ref{#2}}
\newcommand{\refthreeths}[4][and]{Theorems~\ref{#2}, \ref{#3}, {#1}~\ref{#4}}
\newcommand{\refthm}[1]{Theorem~\ref{#1}}
\newcommand{\reftwothms}[2]{Theorems~\ref{#1} and~\ref{#2}}
\newcommand{\refthreethms}[4][and]{Theorems~\ref{#2}, \ref{#3}, {#1}~\ref{#4}}

% Lemmas
\newcommand{\reflem}[1]{Lemma~\ref{#1}}
\newcommand{\reftwolems}[2]{Lemmas~\ref{#1} and~\ref{#2}}
\newcommand{\refthreelems}[4][and]{Lemmas~\ref{#2}, \ref{#3}, {#1}~\ref{#4}}

% Propositions
\newcommand{\refpr}[1]{Proposition~\ref{#1}}
\newcommand{\reftwoprs}[2]{Propositions~\ref{#1} and~\ref{#2}}

% Corollaries
\newcommand{\refcor}[1]{Corollary~\ref{#1}}
\newcommand{\reftwocors}[2]{Corollaries~\ref{#1} and~\ref{#2}}

% Definitions
\newcommand{\refdef}[1]{Definition~\ref{#1}}
\newcommand{\reftwodefs}[2]{Definitions~\ref{#1} and~\ref{#2}}
\newcommand{\refthreedefs}[3]{Definitions~\ref{#1}, \ref{#2}, and~\ref{#3}}

% Remarks
\newcommand{\refrem}[1]{Remark~\ref{#1}}
\newcommand{\reftworems}[2]{Remarks~\ref{#1} and~\ref{#2}}

% Observations
\newcommand{\refobs}[1]{Observation~\ref{#1}}
\newcommand{\reftwoobs}[2]{Observations~\ref{#1} and~\ref{#2}}

% Facts
\newcommand{\reffact}[1]{Fact~\ref{#1}}

% Conjectures
\newcommand{\refconj}[1]{Conjecture~\ref{#1}}
\newcommand{\reftwoconjs}[2]{Conjectures~\ref{#1} and~\ref{#2}}

% Examples
\newcommand{\refex}[1]{Example~\ref{#1}}
\newcommand{\reftwoexs}[2]{Examples~\ref{#1} and~\ref{#2}}

% Properties
\newcommand{\refproperty}[1]{Property~\ref{#1}}
\newcommand{\reftwoproperties}[2]{Properties~\ref{#1} and~\ref{#2}}
 
% Claims
\newcommand{\refclaim}[1]{Claim~\ref{#1}}
\newcommand{\reftwoclaims}[2]{Claims~\ref{#1} and~\ref{#2}}

% References at start of sentence
\newcommand{\Refth}[1]{Theorem~\ref{#1}}
\newcommand{\Reflem}[1]{Lemma~\ref{#1}}
\newcommand{\Refpr}[1]{Proposition~\ref{#1}}
\newcommand{\Refcor}[1]{Corollary~\ref{#1}}
\newcommand{\Refdef}[1]{Definition~\ref{#1}}
\newcommand{\Refrem}[1]{Remark~\ref{#1}}
\newcommand{\Refobs}[1]{Observation~\ref{#1}}
\newcommand{\Refconj}[1]{Conjecture~\ref{#1}}
\newcommand{\Refex}[1]{Example~\ref{#1}}
\newcommand{\Refclaim}[1]{Claim~\ref{#1}}

% References with page numbers
\newcommand{\refthP}[1]{Theorem~\vref{#1}}
\newcommand{\reflemP}[1]{Lemma~\vref{#1}}
\newcommand{\refprP}[1]{Proposition~\vref{#1}}
\newcommand{\refcorP}[1]{Corollary~\vref{#1}}
\newcommand{\refdefP}[1]{Definition~\vref{#1}}
\newcommand{\refremP}[1]{Remark~\vref{#1}}
\newcommand{\refobsP}[1]{Observation~\vref{#1}}
\newcommand{\refconjP}[1]{Conjecture~\vref{#1}}
\newcommand{\refexP}[1]{Example~\vref{#1}}
\newcommand{\refpropertyP}[1]{Property~\vref{#1}}

% Some more references
\newcommand{\refrule}[1]{rule~\ref{#1}}
\newcommand{\reftworules}[2]{rules~\ref{#1} and~\ref{#2}}

\newcommand{\refpart}[1]{part~\ref{#1}}
\newcommand{\Refpart}[1]{Part~\ref{#1}}
\newcommand{\reftwoparts}[2]{parts~\ref{#1} and~\ref{#2}}
\newcommand{\Reftwoparts}[2]{Parts~\ref{#1} and~\ref{#2}}

\newcommand{\refitem}[1]{item~\ref{#1}}
\newcommand{\Refitem}[1]{Item~\ref{#1}}
\newcommand{\reftwoitems}[2]{items~\ref{#1} and~\ref{#2}}
\newcommand{\Reftwoitems}[2]{Items~\ref{#1} and~\ref{#2}}

\newcommand{\refcase}[1]{case~\ref{#1}}
\newcommand{\Refcase}[1]{Case~\ref{#1}}
\newcommand{\reftwocases}[2]{cases~\ref{#1} and~\ref{#2}}
\newcommand{\Reftwocases}[2]{Cases~\ref{#1} and~\ref{#2}}

% References to equations (alias for \eqref just for simplicity)
%
% The definition of \refeq overwrites a command in the mathtools package
% if this package has been loaded

\ifthenelse
{\isundefined{\refeq}}
{\newcommand{\refeq}[1]{\eqref{#1}}}
{\renewcommand{\refeq}[1]{\eqref{#1}}}
\newcommand{\refeqP}[1]{\eqref{#1} on page~\pageref{#1}}




%%%
%%% SOME LOCAL MACROS FOR THE IDMA COURSE
%%%

% GCD
%   \DeclareMathOperator{\gcd}{gcd}

% Two-norm
\newcommand{\twonorm}[1]{\lVert#1\rVert_2}
\newcommand{\Twonorm}[1]{\bigl\lVert#1\bigr\rVert_2}
\newcommand{\TWONORM}[1]{\left\lVert#1\right\rVert_2}

% Formal language grammars
%   \newcommand{\produces}{\rightarrow}
%   \newcommand{\terminalf}[1]{\mathtt{#1}}
%   \newcommand{\tokenf}[1]{\text{\textbf{#1}}}
%   \newcommand{\emptystring}{\varepsilon}
%   \newcommand{\numtoken}{\tokenf{num}}

% For checkmark and xmark
\usepackage{pifont}
\newcommand{\xmark}{\ding{55}}

% For formatting induction proofs
\newcommand{\indproofstep}[1]%
{
  \smallskip
  \noindent
  \textbf{\textit{#1:}}
}

\newcommand{\indbase}[1]{\indproofstep{Base case (#1)}}
\newcommand{\indstep}{\indproofstep{Induction step}}
\newcommand{\indclaim}{\indproofstep{Claim}}





\ifthenelse{\boolean{versionwithsolutions}}
{\newcommand{\examheading}{\mbox{Exam April 2, 2025} \\
    Problems and Solutions}}
{\newcommand{\examheading}{\mbox{Exam April 2, 2025}}}


% For getting watermark "DRAFT" across all pages (for instance, 
% when posting preliminary version of problem set)
%   \usepackage{draftwatermark}
%   % \SetWatermarkFontSize{20 cm}
%   \SetWatermarkScale{5}


%%%
%%% TITLE
%%%

\author{\courseinstructor}
\course{\coursenamelong{}}
\semester{\courseperiod}
\title{\coursenameshort: 
  \examheading}



\begin{document}

\maketitle

\ifthenelse{\boolean{versionwithsolutions}}
{
\noindent
\emph{%
  Please note that the main purpose of this document is to explain
  what the correct solutions are and how to arrive at them.  Thus,
  while the text below is certainly intended to provide good examples
  of how to solve problems and reason about solutions, these examples
  do not necessarily specify exactly how the handed-in exams were
  expected to look like.
%
  This is especially so since for many problems there are more than
  one correct way of solving them.  
%   
%     In general, the solutions provided below tend to be much more
%     detailed than what would be expected for the exam submissions.
%   
  As communicated during the course, the course notes
  published on Absalon, which contain many problems that we worked out
  in class, are probably the best indicator of what level of detail is
  expected from the exam solutions.
}


}
{
\begin{abstract}
%     \noindent
%     \textbf{Due:} \duedate.
%   
  \noindent
  \textbf{Submission:}
  Please write your solutions
%     on paper. Leave
  with
  ample margins on all
  sides, and make sure your handwriting is legible.
  \emph{Start your solution of every new problem on a new
    page. % sheet of paper.
%       Please make sure to mark every sheet
    Please mark every page
    with your name,
%       KU ID,
    exam number or something else that uniquely identifies your exam,}
  so that it is easy to see for every
%     sheet
  page
  which exam submission it is part of.
%   
%     Please submit your solutions  via \emph{Digital eksamen}.
%     It will be helpful if you state clearly 
%     your full name and your KU ID at the top of the first page.
%     From the point of view of grading, we 
%     are of course grateful if you typeset your solutions in
%     \LaTeX{} or some other math-aware  typesetting system. This is
%     \emph{not} a requirement, however, and how you write your
%     solutions will not affect  the grading as long as you
%     make sure that everything that you write is clearly legible
%     (and leave reasonable margins on all sides).
%   

  \noindent
  \textbf{Exposition:}
  Please try to be precise
%     and to the point
  in your solutions and
  refrain from vague statements.  
  Never, ever just state an answer, but always make sure to
  \emph{explain why} the answer is what it is.
  Provide clear references to any
%     course material used.
  facts in the course literature used.
  \emph{Write your solutions in such a way that 
    a    %%%    an imaginary 
    fellow
    student of yours could read, understand, and verify your
    solutions.}
%   
%     In addition to what is stated below, the general rules in the
%     official course information always apply.
%   
    
  \noindent
  \textbf{Collaboration:}
  All problems should be solved individually. No communication or
  collaboration is allowed, and solutions will be checked for
  plagiarism.

  \noindent
  \textbf{Reference material:}
  Textbooks and handwritten notes (including lecture notes) are allowed.
%     
%     Other typewritten material, such as problem sets or previous exams
%     with written solutions,  is not allowed.
%     
  Other typewritten material, including (but not limited to) problem
  sets or previous exams with solutions,  is not allowed.
%     
%     All course material is allowed, including  textbooks,
%     lecture notes, exercise sheets, and individual notes.
%   
%     You are \emph{not} supposed, and will not need,
%     to copy substantial parts of text verbatim,
%     and if your solutions make heavy use of reference material, then make
%     sure to provide clear references to what you are using.
%     
  Please note that you deviate from definitions and
%     algorithms as  described
  algorithm descriptions
  in the course material at your own risk! If, however,
  slight variations of algorithms were presented in the textbooks
  and/or in class,
%     then
  \mbox{such minor details do not matter.} 

  \noindent
  \textbf{Grading:}
  A score of 
  \thresholdforpass
  is guaranteed to be enough to pass the exam.

  \noindent
  \textbf{About the problems:}
  Note that the problem are of quite different types, and are
  \emph{not sorted in increasing order of difficulty.}
  \emph{Please read through the whole exam first,} 
  before you start working on any
  single problem, so that you can plan which order of dealing with the
  problems makes most sense to you.
  Note that this is a fairly large exam, and
%     so you can definitely get 
  you can get 
  a top grade without solving all problems.
  % You are not necessarily expected to be able to solve all of the problems.
  Also, partial answers to problems can sometimes give substantial
  amounts of points.
%   
%     Please do not hesitate to 
%   %     ask 
%   %     send a private message on \emph{Absalon}
%     alert the exam administrators
%     if any problem statement is   unclear. 
%   
  \textbf{\emph{Good luck!}}
\end{abstract}



%%% Local Variables:
%%% mode: latex
%%% TeX-master: "DMFS_Exam_230412"
%%% End:
}

%%%
%%% NUMBER REPRESENTATION PROBLEM
%%%

\begin{problem}
  (50 p)
% 10 p for problem (a)
% 20 p each for problems (b) and (c)
%
  This problem is about different representations of integers.

\begin{subproblem}%
  \ifthenelse{\boolean{versionwithsolutions}}
  {(10 p)}
  {\ignorespaces}
  Write the binary number $(110)_2$ in decimal notation.
\end{subproblem}

\begin{solution}
  We have
  $(110)_2
  =
  1 \cdot 2^2 + 
  1 \cdot 2^1 + 
  0 \cdot 2^0
  =
  6
  $.
\end{solution}

\begin{subproblem}%
  \ifthenelse{\boolean{versionwithsolutions}}
  {(20 p)}
  {\ignorespaces}
  Write the decimal number $110$ in binary notation.
\end{subproblem}

\begin{solution}
  Dividing by~$2$ and outputting the remainders will give us the
  bits in the binary expansion in reverse order. We obtain 
  \begin{align*}
    110 &= 55 \cdot 2 + 0 \\
    55  &= 27 \cdot 2 + 1 \\
    27  &= 13 \cdot 2 + 1 \\
    13  &=  6 \cdot 2 + 1 \\
    6   &=  3 \cdot 2 + 0 \\
    3   &=  1 \cdot 2 + 1 \\
    1   &=  0 \cdot 2 + 1 
  \end{align*}
  from which we see that
  $
  (110)_{10}
  = 
  (1101110)_2
  $.
  Just to verify that we have not made any mistake, we can check our answer by 
  computing
  $
  (1101110)_2
  =
  1 \cdot 2^6 + 
  1 \cdot 2^5 + 
  0 \cdot 2^4 + 
  1 \cdot 2^3 + 
  1 \cdot 2^2 + 
  1 \cdot 2^1 + 
  0 \cdot 2^0 
  =
  64 + 
  32 + 
  8 + 
  4 + 
  2
  =
  110
  $.
\end{solution}

  
\begin{subproblem}%
  \ifthenelse{\boolean{versionwithsolutions}}
  {(20 p)}
  {\ignorespaces}
  Write the octal number $(2025)_8$ in decimal notation.
\end{subproblem}

\begin{solution}
  We have
 $(2025)_8
 =
 2 \cdot 8^3 + 
 0 \cdot 8^2 + 
 2 \cdot 8^1 + 
 5 \cdot 8^0
 =
 2 \cdot 512 + 
 2 \cdot 8 +
 5
 = 
 1045
 $.
\end{solution}

\end{problem}

%%%
%%% GCD PROBLEM
%%%

\begin{problem}
  (50 p)
% 20 p for easy problem
% 30 p for harder problem
%
%%% VERY EXPLICIT STATEMENT
%     Run the Euclidean algorithm on the numbers
%     $m$ and~$n$ below to find
%     $d = \gcd (m, n)$, 
%   %     showing all  steps in the execution,
%     showing all recursive calls,
%     and then express~$d$ as a linear combination of $m$ and~$n$.
%%%     LESS EXPLICIT STATEMENT
  Use the algorithm we have learned for determining
  $d = \gcd (m, n)$ for the numbers below,
  showing details of all function calls made, 
  and then express~$d$ as a linear combination of $m$ and~$n$.
  
\begin{subproblem}%
  \ifthenelse{\boolean{versionwithsolutions}}
  {(20 p)}
  {\ignorespaces}
  $m = 38$ and $n=14$.
\end{subproblem}

\begin{solution}
  We use the Euclidean algorithm, which is based on the observation
  that
  $
  \gcd (m, n) 
  =
  \gcd (n, m \bmod n) 
  $.
  
  For
  $m = 38$
  and
  $n = 14$
  we obtain
  \begin{align*}   
    38 &= 2 \cdot 14 + 10 \\
    14 &= 1 \cdot 10 +  4 \\
    10 &= 2 \cdot  4 +  2 \\
    4  &= 2 \cdot  2 +  0
  \end{align*}
  from which we see that 
  $
  \gcd (38, 14) = 2
  $.
  By considering the above equalities in reverse order we
  can write the greatest common divisor
  \begin{align*}
    2 &=
        10 -  2 \cdot 4
    \\
      &=
        10 -  2 \cdot ( 14 - 1 \cdot 10 )
    \\
      &=
        3 \cdot 10 -  2 \cdot  14 
    \\
      &=
        3 \cdot
        ( 38 - 2 \cdot 14  )
        -  2 \cdot  14 
    \\
      &=
        3 \cdot 38
        -  8 \cdot  14
  \end{align*}
  as a linear combination of
  $38$ and~$14$ as desired.
\end{solution}


\begin{subproblem}%
  \ifthenelse{\boolean{versionwithsolutions}}
  {(30 p)}
  {\ignorespaces}
  $m = 117$ and $n=69$.
\end{subproblem}

\begin{solution}
  For
  $m = 117$
  and
  $n = 69$
  we obtain
  \begin{align*}   
    117 &= 1 \cdot 69 + 48 \\
    69  &= 1 \cdot 48 + 21 \\
    48  &= 2 \cdot 21 +  6 \\
    21  &= 3 \cdot 6 +  3 \\   
    6  &= 2 \cdot  3 +  0
  \end{align*}
  from which we see that 
  $
  \gcd (117, 69) = 3
  $.
  Processing these equalities in reverse order we obtain
  \begin{align*}
     3 &= 
         21 - 3 \cdot  6          
    \\
       &=
         21 - 3 \cdot ( 48 - 2 \cdot 21 )
    \\
       &=
         7 \cdot 21 - 3 \cdot  48    
    \\
       &=
         7 \cdot ( 69 - 1 \cdot 48 )
         - 3 \cdot  48
    \\
       &=
         7 \cdot  69 - 10 \cdot  48
    \\   
       &=
         7 \cdot  69 - 10 \cdot (117 - 1 \cdot 69 )         
    \\   
       &=
         17 \cdot  69 - 10 \cdot 117 
  \end{align*}
  as a linear combination of
  $117$ and~$69$ as desired.
\end{solution}


\end{problem}

%%%
%%%   MATHEMATICAL INDUCTION PROBLEM
%%%
\begin{problem}
  (60 p)
  Provide formal proofs of the following claims
  using proof techniques that we have learned during the course.


  \begin{subproblem}%
    \ifthenelse{\boolean{versionwithsolutions}}
    {(30 p)}
    {\ignorespaces}
    Define
    $a_1 = 1$
    and
    $a_n = 2 a_{n-1} + 1$
    for
    $n \geq 2$.
    Prove that for all positive integers
    $n \in \Nplus$
    it holds that 
    $a_n = 2^n - 1$.
  \end{subproblem}

\begin{solution}
We prove this equality by induction over~$n$.

\begin{description}
\item[\emph{Base case ($n=1$):}]
  We have
  $a_1 = 1 = 2^1 - 1$
  by definition.

\item[\emph{Induction step:}]
  Assume that the equality holds for $n-1$ and consider $n$.
  We have
  \begin{align*}
    a_n &= 2 a_{n-1} + 1
    \\                           
    &= 2 \cdot (2 ^{n-1} - 1) + 1
    &&[\text{by the induction hypothesis}]
    \\
    &= 2 ^{n} - 2 + 1    
    \\
    &= 2 ^{n} - 1
  \end{align*}
  which is the desired equality.
\end{description}
The claim now follows by the induction principle.  
\end{solution}


\begin{subproblem}%
  \ifthenelse{\boolean{versionwithsolutions}}
  {(30 p)}
  {\ignorespaces}
  Prove that for all non-negative integers
  $n \in \N$
  it holds that
  $
  3 \mid 4^n + 5 
  $.
\end{subproblem}

\begin{solution}
%     This is very similar to a problem we did in class, 
%     and we will solve this one in the same way by arguing by
  We prove this by 
  induction over~$n$.
  \begin{description}
  \item[\emph{Base case ($n=0$):}]
    For $n=0$ we have that
    $
    4^0 + 5 
    =
    6
    $
    is divisible by~$3$.
    
  \item[\emph{Induction step:}]
    Suppose that
    $
    3 \mid 4^n + 5
    $,
    which is the same  as saying that
    $
    4^n + 5 = 3 \cdot M
    $
    for some integer~$M$.
    Fixing this $M$, and
    working on the expression \mbox{for $n+1$}, we get
    \begin{align*}
      4^{n+1} + 5
      &= 
        3 \cdot 4^{n}
        + 
        4^{n} + 5
      &&[\text{since $4^{n+1} = (3 + 1) \cdot 4^n$}]
      \\
      &= 
        3 \cdot 4^{n}
        + 
        3 \cdot M
      &&[\text{by the induction hypothesis}]
      \\
      &=
        3 \cdot ( 4^{n} + M )
    \end{align*}
    which shows that this expression is divisible by~$3$.
  \end{description}
  The claim follows by the induction principle.
\end{solution}


\end{problem}

%%%
%%% RELATIONS
%%%

\begin{problem}%
  \label{problem:relations}%
  (90 p)  
  Let
  $A = \set{1, 2, 3, 4}$
  and consider the following binary relations on $A$:
  \begin{align*}
    R &= \set{ (2,1), (3,1), (3,2), (4,1), (4,2), (4,3) } \\
    S &= \set{ (1,1), (2,2), (3,3), (4,4) } \\
    T &= \set{ (1,1), (1, 4), (2,2), (2,3), (3,2), (3,3), (4,1), (4,4) }
  \end{align*}

%%% 5 points per correctly identified property
  \begin{subproblem}
    \ifthenelse{\boolean{versionwithsolutions}}
    {(60 p)}
    {\ignorespaces}
    For each of the relations above, determine whether it is
    \begin{compactenum}
    \item
      reflexive,      
    \item
      symmetric,
    \item
      antisymmetric,
    \item
      transitive.
    \end{compactenum}
    Please make sure to explain, briefly but clearly, what these properties
    mean and why they are satisfied for a relation when
    they are. For any relation that
    fails to satisfy a property, make sure to provide a specific
    counterexample.

  \end{subproblem}

  \begin{solution}
    Recall that a relation $R$ is:
    \begin{itemize}
    \item
      \emph{reflexive} if for all $x$ it holds that $(x,x) \in R$;
      
    \item
      \emph{symmetric} if whenever
      $(x,y) \in R$
      it also holds that
      $(y,x) \in R$;

    \item
      \emph{anti-symmetric} if 
      $(x,y) \in R$
      and
      $(y,x) \in R$ implies
      $x=y$;
      
    \item
      \emph{transitive} if whenever
      $(x,y) \in R$
      and
      $(y,z) \in R$
      it also holds that
      $(x,z) \in R$.      
    \end{itemize}

    The relation~$R$ specified in the problem statement
    is anti-symmetric
    (simply since there is no pair $(x, y)$
    such that
    $(x, y) \in R$
    and
    $(y, x) \in R$)
    and transitive
    (which can be verified by case analysis, or by observing that
    $R$ is the greater-than relation).
    It is not reflexive since, e.g., 
    $(1, 1) \notin R$,
    and it is not symmetric since, e.g.,
    $(2, 1) \in R$
    but
    $(1, 2) \notin R$.

    The relation~$S$ is the identity relation,
    which vacuously satisfies all the properties listed.

    The relation~$T$ can be verified to be 
    reflexive, symmetric, and transitive.
    It is not anti-symmetric since
    $(2,3) \in T$
    and
    $(3,2) \in T$
    but
    $2 \neq 3$.    
\end{solution}

%%% 10 points per correctly identified relation
  \begin{subproblem}
    \ifthenelse{\boolean{versionwithsolutions}}
    {(30 p)}
    {\ignorespaces}
    Which of the relations above, if any, are equivalence relations or
    partial orders? Please make sure to justify your answers.
  \end{subproblem}

\begin{solution}
  An \emph{equivalence relation} is a relation that is
  reflexive, symmetric, and transitive.
  The relations $S$ and~$T$ satisfy these conditions
  as argued above.

  A \emph{partial order}
  is a reflexive, anti-symmetric, and transitive relation.
  The identity relation~$S$ is formally speaking also a partial order,
  since it satisfies all the  required properties
  (but it is of course a very boring partial order). 

  It might be worth pointing out that
  the relation~$R$ is \emph{not} a partial order, since it is not reflexive.
  This is just a special case of the general fact that
  non-strict order relations define partial orders but strict order relations
  do not.
\end{solution}
\end{problem}


%%%
%%% PROPOSITIONAL LOGIC
%%%

\begin{problem}%
  \label{problem:logic}%
  (60 p)  
  For each of the propositional logic formulas below,
  determine whether it is a tautology or not.
  If the formula is not a tautology, show how to
  add a single connective to make it into a tautology.
  Please make sure to justify your answers
  (e.g., by presenting truth tables, or by using rules for rewriting logic formulas
  that we have learned in class).

%%% TAUTOLOGY  
  \begin{subproblem}%
    \label{problem:formula-1}%
    \ifthenelse{\boolean{versionwithsolutions}}
    {(30 p)}
    {\ignorespaces}
    $
    \lnot \bigl(
    ( p \limpl q) \lor r
    \bigr)
    \limpl
    \bigl(
    (\lnot q \land \lnot r) \land p
    \bigr)
    $
  \end{subproblem}

\begin{solution}
  This formula is a tautology, i.e., it is always true.
  To see this, note first that we know that
  \begin{equation}
    \label{eq:impl}
    p \limpl q
    \ \equiv \
    \lnot p \lor q
  \end{equation}
  since the only way the implication
  $p \limpl q$
  can be false is that $p$ is true and $q$ is false,
  and this is one of the basic equivalences we have learned in the course.
  Furthermore, from De Morgan's laws it is easy to derive that
  \begin{equation}
    \label{eq:de-morgan}
    \lnot ( a \lor b \lor c)
    \ \equiv \
    \lnot a \land \lnot b \land \lnot c
  \end{equation}
  (i.e., the only way a disjunction can be false is that all its
  disjuncts are false).
  
  By combining
  \eqref{eq:impl}
  and~\eqref{eq:de-morgan},
  and using that conjunction is commutative, we see that the premise
  $
  \lnot \bigl(
  ( p \limpl q) \lor r
  \bigr)
  $
  is in fact equivalent to the conclusion
  $
  (\lnot q \land \lnot r) \land p
  $,
  and so the implication
  in Problem~\ref{problem:formula-1}
  will always evaluate to true.
\end{solution}

  \begin{subproblem}
    \ifthenelse{\boolean{versionwithsolutions}}
    {(30 p)}
    {\ignorespaces}
    $
    \bigl(
    (p \land q) \limpl r
    \bigr)
    \lequiv
    \bigl(
    (q \lor r) \lor \lnot p
%%% This should be    
%       (\lnot q \lor r) \lor \lnot p
    \bigr)
    $    
  \end{subproblem}

\begin{solution}
  This is not a tautology.
  If we set $p$ and~$q$ to true
  but $r$ to false, we get that
  $
  (p \land q) \limpl r
  $
  evaluates to  false but
  $
  (q \lor r) \lor \lnot p
%%% This should be    
%       (\lnot q \lor r) \lor \lnot p
  $
  evaluates to true, and so the whole formula is false.

  If we change $q$ on the right-hand side to $\lnot q$,
  we obtain the formula
  \begin{equation}
    \label{eq:patched}
    \bigl(
    (p \land q) \limpl r
    \bigr)
    \lequiv
    \bigl(
    (\lnot q \lor r) \lor \lnot p
    \bigr)
    \ ,
  \end{equation}    
  which is a tautology. To see this, we can rewrite the left-hand side
  of~\eqref{eq:patched}
  as
  \begin{subequations}
  \begin{align}
    (p \land q) \limpl r
    \ &\equiv \
        \lnot (p \land q) \lor r
    \\
    \ &\equiv \
        (\lnot p \lor \lnot q) \lor r
    \\
    \ &\equiv \
        (\lnot q \lor r) \lor \lnot p
  \end{align}
  \end{subequations}
  by using
  the property~\eqref{eq:impl} of implication,
  De Morgan's laws,
  and commutativity of disjunction.
  This show that the left-hand and right-hand sides of
  of the formula~\eqref{eq:patched} are equivalent,
  and so the formula will always evaluate to true.
\end{solution}

\end{problem}


%%%
%%% TRICKY INDUCTION PROBLEM: GCD OF FIBONACCI NUMBERS
%%%

\begin{problem}%
  (80 p)
  Recall that the Fibonacci numbers are defined as
  \begin{align*}
    F_1 &= 1 \\
    F_2 &= 1 \\
    F_n &= F_{n - 1} +  F_{n - 2} &&\text{for $n \geq 3$.}
  \end{align*}
  %
  Prove that consecutive Fibonacci numbers
  $F_{n+1}$
  and
  $F_{n}$
  are relatively prime, and show that for  $n \geq 2$
  the  Euclidean algorithm when run on $F_{n+1}$ and $F_n$
  makes exactly $n-1$~function calls to determine that this is so
  (i.e., it reaches remainder~$0$ after exactly $n-1$
  function calls).

\begin{solution}
  We prove that two consecutive Fibonacci numbers are relatively
  prime, i.e., that they have greatest common divisor~$1$, by using
  the Euclidean algorithm. While doing so, we count the number of
  function calls, i.e., the number of times the relation
  $
  \gcd (m, n) 
  =
  \gcd (n, m \bmod n) 
  $
  is applied before reaching the trivial base case where $n$
  divides~$m$
  and the remainder is~$0$.

  \begin{description}
  \item[\emph{Base case ($n=2$):}]
    For
    $F_3 = 2$
    and
    $F_2 = 1$,
    we clearly have
    $
    \gcd ( 2, 1 ) = 1
    $.
    If we run the Euclidean algorithm, we get that
    $
    F_3 = 2 \cdot F_2 + 0
    $,
    and so we reach remainder~$0$ after a single step.
    
  \item[\emph{Induction step:}]
    Suppose that for $n-1$ it holds that
    $
    \gcd ( F_{n - 1} , F_{n - 2}) = 1
    $
    and that the Euclidean algorithm reaches remainder~$0$ after
    $(n-1) - 1 = n-2 $ function calls.

    As a first step when computing
    $
    \gcd ( F_{n} ,  F_{n - 1})
    $,
    the Euclidean algorithm divides
    $F_{n}$ by  $F_{n - 1}$
    to compute the remainder.
    This remainder is~$F_{n-2}$, since by the definition of Fibonacci
    numbers we have that
    $F_n = 1 \cdot F_{n - 1} +  F_{n - 2}$,
    and so the equalities
    \begin{equation}
      \gcd ( F_{n} ,  F_{n - 1})
      =
      \gcd ( F_{n-1} , F_n \bmod F_{n - 1})
      =
      \gcd ( F_{n-1} ,  F_{n - 2})
    \end{equation}
    hold.
    By the induction hypothesis,  the Euclidean algorithm
    computes
    $\gcd ( F_{n-1} ,  F_{n - 2}) = 1  $
    with $n-2$ additional recursive calls. 
    Hence, we conclude that the Euclidean algorithm will
    determine that
    $F_n$ and~$F_{n-1}$
    are relatively prime after $n-1$ function calls.
  \end{description}
  The claim in the problem statement now follows by the induction principle.
\end{solution}


\end{problem}



%%%
%%% COMBINATORICS
%%%

\begin{problem}%
  \label{problem:combinatorics}%
  (90 p)
  For a few years now the Copenhagen metropolitan area 
  (including Lund) has had an unusually large number of researchers in 
  computational complexity theory, and a team of such
  researchers have decided to submit a joint grant application  to
  create the
  \emph{Copenhagen Computational Complexity Centre}
  focusing on research in this scientific field.
  Since gender balance is a serious issue in computer science, 
  a noteworthy aspect of the team of co-applicants is that
  the male professors
  Amir, Jakob, and Srikanth at the University of Copenhagen
  are balanced by the female professors
  Nutan and Paloma at the IT University of Copenhagen and
  Susanna at Lund University.

  For the subproblems below, please make sure to answer not just with
  numbers but with more combinatorial-looking expressions, and to expand these
  expressions out to show that you understand the meaning of any
  notation used. Also make sure to explain how you reason to reach
  your answers.

  \begin{subproblem}
    \ifthenelse{\boolean{versionwithsolutions}}
    {(50 p)}
    {\ignorespaces}
    Together with the application documents, the co-applicants are
    planning to enclose a group photo, and much thought has gone into
    how to choose the seating arrangement.
%       for this photo.
    All the researchers will be placed in a single row, but they have
    agreed that a great way to highlight the gender
    balance would be to make sure that male and female researchers
    alternate, so that every second person in the row
    is male or female, respectively.
    In how many different ways can the $6$~researchers be arranged on
    the photo to satisfy this constraint?
  \end{subproblem}

\begin{solution}
  The seating arrangement is uniquely specified by determining whether
  the leftmost person on the photo is male or female, and then by
  specifying the internal order of the female and male researchers,
  respectively.
  This gives us:
  \begin{itemize}
  \item
    $2$ choices for a male or female researcher at the leftmost position;
  \item
    $3! = 6$ ways of arranging the $3$~female researchers
    from left to right;
  \item
    likewise
    $3! = 6$ ways of arranging the $3$~male researchers
    from left to right;
  \end{itemize}
  for a total of
  $
  2 \cdot 3! \cdot 3!
  =
  2 \cdot 6 \cdot 6
  =
  72
  $
  different arrangements.
\end{solution}
  
  \begin{subproblem}
    \ifthenelse{\boolean{versionwithsolutions}}
    {(40 p)}
    {\ignorespaces}
    Any serious research centre application these days should also
    identify a steering committee for the centre.
    After long deliberations, the co-applicants have decided that this
    committee should:
    \begin{compactitem}
    \item
      consist of $4$ persons all in all;

    \item
      include co-applicants representing all $3$ partner
      institutions, i.e, 
      the University of Copenhagen,
      the IT University of Copenhagen,
      and
      Lund University;

    \item
      have perfect gender balance, i.e., two male and two female members.
      
    \end{compactitem}
    %
    In how many different ways can the steering committee be composed?
    
  \end{subproblem}

\begin{solution}
  Since Susanna is the only co-applicant from Lund University, she has
  to be on the steering committee.

  This means that we need exactly one more female committee member,
  who will be from ITU,
  and this gives us $2$~choices for either Nutan or Paloma.

  Finally, we need two male members, who will both have to come from
  the University of Copenhagen.
  We can  think of choosing either two persons among Amir, Jakob, and
  Srikanth in
  $\binom{3}{2} = 3$ ways,
  or choosing one person to leave out in
  $\binom{3}{1} = 3$ ways.

  Summing up (or, rather, multiplying together), we see that the
  steering committee can be composed in
  $
  1 \cdot 2 \cdot \binom{3}{2}
  =
  1 \cdot 2  \cdot 3 = 6
  $
  ways.  
\end{solution}
  
  
\end{problem}


%%%
%%% GRAPH PROBLEM: SHORTEST PATH
%%%

\begin{problem}%
  \label{problem:shortest-path}%
  (90 p)
  When Jakob has international visitors, he needs to give them travel
  directions from Kastrup Airport to the
  Department of Computer Science (DIKU) at Universitetsparken.  Jakob
  is aware of the following relevant public transport options in
  Copenhagen with travel times as stated
%     (in either direction):
  (in either direction, and with time for switching transport mode included):
  \begin{compactitem}
  \item
    Between Kastrup Airport and Kongens Nytorv by metro: 13 minutes.
    
  \item
    Between Kastrup Airport and København H by train: 20 minutes.
    
  \item
    Between København H and Nørreport by train: 3 minutes.
    
  \item
    Beween København H and Kongens Nytorv by metro: 4 minutes.
    
  \item
    Between Kongens Nytorv and Vibenshus Runddel by metro: 10 minutes.

  \item
    Between Vibenshus Runddel and Universitetsparken by foot: 9 minutes.
    
  \item
    Between Nørreport and Universitetsparken by bus: 8 minutes.

  \end{compactitem}
%
  What is not so clear to Jakob is how he should use this information
  to find
%     a good route
  as fast a route as possible
  between the airport and DIKU to suggest to his
  visitors.
  
  \begin{subproblem}
    \ifthenelse{\boolean{versionwithsolutions}}
    {(20 p)}
    {\ignorespaces}
    Help Jakob by modelling this problem as a graph.
    Explain what the vertices and edges represent and what other
    information you need to add to the graph.
    Make sure to show concretely what graph you obtain
    for Jakob's problem above.
  \end{subproblem}


\begin{solution}
  We create a graph with $6$~vertices:
  \begin{compactitem}
  \item vertex 
    $A$ for Kastrup Airport, 

  
\begin{figure}[t] %%  [tp]
  \centering
  \includegraphics{Figures/public-transport-graph-25.1}
%     \includegraphics{Figures/top-sort-graph-24.1}
%
  \caption{Graph modelling public transport options from
    Kastrup Airport to DIKU in
    Problem~\ref{problem:shortest-path}.
  }
  \label{fig:public-transport-graph}
\end{figure}


%%% Local Variables:
%%% mode: latex
%%% TeX-master: "../IDMA_Exam_250402"
%%% End:

    
  \item vertex 
    $K$ for Kongens Nytorv,

  \item vertex 
    $H$ for København H,

  \item vertex 
    $N$ for Nørreport,

  \item vertex 
    $V$ for Vibenshus Runddel,

  \item vertex 
    $U$ for Universitetsparken and DIKU.

  \end{compactitem}
%  
  Between these vertices we add edges as specified by the travel options 
%     in the problem statement
  with weights equal to the travel
  times, and these edges are undirected since 
%     it is clearly stated that
  the time given is for  travel in either direction:
%     This yields the  following edges:
  \begin{compactitem}
  \item edge 
    $(A, K)$ with weight~$13$,
    
  \item edge 
    $(A, H)$ with weight~$20$,
    
  \item edge 
    $(H, N)$ with weight~$3$,
    
  \item edge 
    $(H, K)$ with weight~$4$,

  \item edge 
    $(K, V)$ with weight~$10$,

  \item edge 
    $(V, U)$ with weight~$9$,

  \item edge 
    $(N, U)$ with weight~$8$.
  \end{compactitem}
%
  This yields the graph in
  \reffig{fig:public-transport-graph}.
  
\end{solution}


\begin{subproblem}
  \ifthenelse{\boolean{versionwithsolutions}}
    {(70 p)}
    {\ignorespaces}
    Propose a suitable graph algorithm to solve Jakob's problem.
    Explain what this algorithm is and why it is the right choice for
    this problem.  Make a dry-run of the algorithm and explain the
    relevant steps in the execution (similarly to what has been done
    in class and in the lecture notes).
    
    If your algorithm uses any auxiliary data structures, then explain in detail
    for the first two vertices processed how these data structures
    change. For the rest of the algorithm execution, just
    report what the relevant outputs of the data structures are without
    going into any details.

    What travel directions for Jakob's visitors does your algorithm produce?    
  \end{subproblem}
  

\begin{solution}
  What Jakob is looking for is the fastest route from Kastrup to DIKU,
  i.e., the shortest path in a graph where the edge weights are the
  travel times between different locations.
  We have learned in the course to compute such shortest paths by
  using Dijkstra's algorithm.

  
\begin{figure}[t] %%  [tp]
  \begin{subfigure}[b]{0.32\textwidth}
    \centering
    \includegraphics{Figures/public-transport-heaps-25.1}
    \caption{Before dequeuing $A$.}
    \label{fig:heaps-public-transport-1}
  \end{subfigure}
  \hfill
  \begin{subfigure}[b]{0.32\textwidth}
    \centering
    \includegraphics{Figures/public-transport-heaps-25.2}
    \caption{After dequeuing $A$.}
    \label{fig:heaps-public-transport-2}
  \end{subfigure}
  \hfill
  \begin{subfigure}[b]{0.32\textwidth}
    \centering
    \includegraphics{Figures/public-transport-heaps-25.3}
    \caption{After relaxing $(A, H)$.}
    \label{fig:heaps-public-transport-3}
  \end{subfigure}

  \begin{subfigure}[b]{0.32\textwidth}
    \centering
    \includegraphics{Figures/public-transport-heaps-25.4}
    \caption{After relaxing $(A, K)$.}
    \label{fig:heaps-public-transport-4}
  \end{subfigure}
  \hfill
  \begin{subfigure}[b]{0.32\textwidth}
    \centering
    \includegraphics{Figures/public-transport-heaps-25.5}
    \caption{After dequeuing $K$.}
    \label{fig:heaps-public-transport-5}
  \end{subfigure}
  \hfill
  \begin{subfigure}[b]{0.32\textwidth}
    \centering
    \includegraphics{Figures/public-transport-heaps-25.6}
    \caption{After relaxing $(K, H)$.}
    \label{fig:heaps-public-transport-6}
  \end{subfigure}

  \begin{subfigure}[b]{0.32\textwidth}
    \centering
    \includegraphics{Figures/public-transport-heaps-25.7}
    \caption{After relaxing $(K, V)$.}
    \label{fig:heaps-public-transport-7}
  \end{subfigure}
  \hfill
  \begin{subfigure}[b]{0.32\textwidth}
    \centering
    \includegraphics{Figures/public-transport-heaps-25.8}
%       \caption{After relaxing $(c,e)$.}
    \label{fig:heaps-public-transport-8}
  \end{subfigure}
  \hfill
  \begin{subfigure}[b]{0.32\textwidth}
    \centering
    \includegraphics{Figures/public-transport-heaps-25.8}
    \label{fig:heaps-public-transport-9}
  \end{subfigure}

  \caption{Heap configurations for priority queue in
    Problem~\ref{problem:shortest-path}.
  }
  \label{fig:heaps-public-transport}
\end{figure}



%     
%     We therefore run Dijkstra's algorithm on the graph in
%     \reffig{fig:public-transport-graph}.
%     The version of Dijkstra's algorithm that we have learned uses a
%     priority queue implemented with the help of a min-heap.
%   
  Following the instructions in the problem statement,
  when running Dijkstra's algorithm 
%     on the graph in
%     \reffig{fig:public-transport-graph}
  we illustrate in
  \reffig{fig:heaps-public-transport}
  how the heap used for the priority queue changes during 
%     the  execution of the algorithm.
  algorithm execution,
  using the
%     We use the 
  notation $v:k$
%     in the heap when vertex $v$ has key value~$k$.
  for a vertex $v$ with key value~$k$.
  At the outset, the vertex $A$
  corresponding to Kastrup Airport
  has key~$0$ and all other
  vertices have key~$\infty$ as in \reffig{fig:heaps-public-transport-1}.

  \begin{enumerate}
  \item 
    After vertex $A$ has been dequeued, vertex~$V$ is moved to the top of the heap
    and we have the configuration in
    \reffig{fig:heaps-public-transport-2}.
    Relaxing the edge $(A, H)$ gives value $20$ to~$H$ (\ie the weight
    of the edge), and 
    % so shifts $H$ to the top and pushes~$V$ down below it,
    and the min-heap property is restored by letting $H$ bubble up above~$V$, 
    yielding
    \reffig{fig:heaps-public-transport-3}.
    Relaxing~$(A, K)$
    decreases the key of~$K$ to~$13$ and
    makes $H$ bubble up above~$K$, yielding
    \reffig{fig:heaps-public-transport-4}.
%%
    These are all edges incident to~$A$ that we need to consider for
    relax operations.
    
  \item 
    Since vertex $K$ is at the top of the heap, it is dequeued
    next. This adds the edge~$(A,K)$ to the spanning tree,
    which we indicate in
    \reffig{fig:public-transport-spanning-tree}.
    When $K$ is removed, $U$~is moved to the top. This violates the
    min-heap property, since the key of~$U$ 
    is not smaller than or equal to those of its
    children.  Since $H$~has smaller key than~$V$, we swap $U$
    and~$H$. This restores the heap property
    (since the left subtree of the root was not changed, and~$U$~is
    now the root of a singleton subheap), and so the
    heap after removal of~$K$
    looks as in 
    \reffig{fig:heaps-public-transport-5}.    

    
\begin{figure}[t] %%  [tp]
  \centering
  \includegraphics{Figures/public-transport-graph-25.2}
%
  \caption{Shortest path tree for public transport from
    Kastrup Airport to DIKU in
    Problem~\ref{problem:shortest-path}.
  }
  \label{fig:public-transport-spanning-tree}
\end{figure}


%%% Local Variables:
%%% mode: latex
%%% TeX-master: "../IDMA_Exam_250402"
%%% End:

        
    Relaxing the edge~$(K, H)$ decreases the key value of~$H$ to~$17$.
    Since~$H$~is already the root, the heap does not
    change except for this key update and now looks like in
    \reffig{fig:heaps-public-transport-6}.    
%%
    Relaxing~$(K, V)$  decreases the key value of~$V$ to~$23$,
    but again does not lead to any structural changes in the heap
    since $V$ still has a larger key than its parent~$H$
    (see \reffig{fig:heaps-public-transport-7}).
    There are no further edges incident to~$K$ to relax.
%% 
    According to the instructions in the problem statement, we do not
    need to provide any further heap illustrations from this point on.
    

  \item
    Since $H$ is now the vertex with the smallest key value it is dequeued next,
    and the edge~$(K, H)$ is added to the spanning tree
    (since the latest update of the key value of~$H$ was when relaxing
    the edge~$(K, H)$).
%%
    When the edge $(H, N)$ is relaxed the key of~$N$ is updated to~$20$.

  \item
    Vertex~$N$ currently has the smallest key~$20$ and is dequeued next.
    This adds the edge~$(H, N)$ to the spanning paths tree, since
    the key of~$N$ was last updated when $(H, N)$ was relaxed.
    Relaxing~$(N,U)$
    updates the key of~$U$ to~$28$.

  \item
    Now vertex~$V$ has the smallest key~$23$ and so is dequeued,
    adding~$(K, V)$ to the spanning tree.
    When we relax~$(V,U)$
    nothing happens since $23 + 9 \geq 28$.

  \item
    Finally, vertex~$U$ is dequeued,
    adding the edge~$(N, U)$
    to the spanning tree. The queue is now empty, and there is nothing
    to relax.
  \end{enumerate}
%   
  The shortest path spanning tree computed as described above is indicated
  by the bold edges in 
  \reffig{fig:public-transport-spanning-tree}.
  From this tree we can read off that the fastest route for Jakob's
  guests is
  Kastrup Airport
  --
  Kongens Nytorv
  --
  København H
  --
  Nørreport
  --
  Universitetsparken/DIKU.
\end{solution}

\end{problem}



%%%
%%% GRAPH PROBLEM: IDENTIFY OUTPUT OF DIFFERENT ALGORITHMS
%%%
%   Give 10 points per correct answer.
%   Give 20 extra points for a overview of how and why the algorithms process the vertices in this order.
%   

\begin{problem}%
  \label{problem:graph-algorithm-outputs}%
  (120 p)
  Consider the graph in   
  \reffig{fig:graph-algorithm-outputs}
  and the following ordered sequences listing the vertices in this graph:
%%%  
%%% Random permutation: 3, 1, 4, 2
%%%  
  \begin{compactenum}
    % 3. Dijkstra from a
  \item%
    \label{sequence-dijkstra}%    
    $a$, $c$, $e$, $b$, $h$, $g$, $f$, $d$.

%       
\begin{figure}[t] %%  [tp]
  \centering
  \includegraphics{Figures/graph-algorithm-outputs-25.1}
%
  \caption{Graph for 
    Problem~\ref{problem:graph-algorithm-outputs}%
  }
  \label{fig:graph-algorithm-outputs}
\end{figure}

    
        
    % 1. BFS from a
  \item%
    \label{sequence-bfs}%    
    $a$, $b$, $c$, $e$, $d$, $h$, $g$, $f$.

    % 4. Spoiler        
  \item%
    \label{sequence-spoiler}%    
    $a$, $c$, $b$, $e$, $h$, $g$, $f$, $d$.
        
    % 2. DFS from a
  \item%
    \label{sequence-dfs}%
    $a$, $b$, $d$, $c$, $e$, $h$, $g$, $f$.
    
  \end{compactenum}
  %
  For each of the sequences above, determine whether it can be the result of:
  \begin{compactenum}[(a)]
  \item
    a breadth-first search with vertices listed in order of visits;
    
  \item
    a depth-first search with vertices listed in order of discovery;
    
  \item
    a shortest-path computation with vertices listed in the order they
    are removed from the priority queue.

  \end{compactenum}
%
  Each sequence above is the output of at most one of the algorithms,
  but since there are four sequences there could be a sequence that
  cannot be produced by any of the algorithms.

  Partial credit is given for matching algorithms and vertex sequences correctly.
%
  For full credit, you need to
  give an overview of how and why the
  algorithms process the vertices in the given order
  (such as explaining the order of recursive calls, edges processed,
  or similar),
  or why none of the proposed algorithms could yield the sequence in
  question,  
  but you do not have to provide detailed information about any
  auxiliary data structures used in the algorithms.

\begin{solution}
  Let us consider the different algorithms in the order listed and
  try to find sequences that would match their outputs.
  Since we are promised that each sequence is the output of at most
  one algorithm, we are done with each sequence as soon as we find a
  match.
  
  \textit{\textbf{Sequence~\ref{sequence-bfs}}}
  is the result of breadth-first search starting with vertex~$a$.
  To see this, consider how the different vertices are enqueued in and
  dequeued from the queue if we start the breadth-first search in
  vertex~$a$:

  \medskip
  \noindent
  \begin{tabular}[]{|l|l|l|}
    \hline
    \textbf{Dequeued vertex}
    &
      \textbf{Enqueued vertices}
    &
      \textbf{Queue}
    \\
    \hline
    ---
    &
      ---
    &
      $(a)$
    \\
    $a$
    &
      $\set{b, c, e}$
    &
      $(b, c, e)$
      \\
    $b$
    &
      $\set{d}$
    &
      $(c, e, d)$
    \\
    $c$
    &
      ---
    &
      $(e, d)$
    \\
    $e$
    &
      $\set{h}$
    &
      $(d, h)$
    \\
    $d$
    &
      ---
    &
      $(h)$
    \\
    $h$
    &
      $\set{g}$
    &
      $(g)$
    \\
    $g$
    &
      $\set{f}$
    &
      $(f)$
    \\
    $f$
    &
      ---
    &
      $()$
    \\
    \hline
  \end{tabular}

  \bigskip

  
  \textit{\textbf{Sequence~\ref{sequence-dfs}}}
  is the result of depth-first
  search starting with vertex~$a$.
  Below follows an illustration of in which orders vertices are
  discovered and finished: \\
  \vspace{-6.0mm}
  \begin{tabbing}
    Disco\={}ver  $a$ --- undiscovered neighbours $b$, $c$ and~$e$
    \\
    \> Disco\={}ver  $b$ --- undiscovered neighbour $d$
    \\
    \> \> Disco\={}ver  $d$  --- undiscovered neighbour $c$
    \\
    \> \> \> Disco\={}ver  $c$ --- no undiscovered neighbours
    \\
    \> \> \> Finish $c$, since no undiscovered neighbours
    \\
    \> \> Finish $d$, since no undiscovered neighbours left
    \\
    \> Finish $b$, since no undiscovered neighbours left
    \\
    \> Discover $e$ --- undiscovered neighbour $h$
    \\
    \> \> Discover $h$  --- undiscovered neighbour $g$
    \\
    \> \> \> Discover $g$  --- undiscovered neighbour $f$
    \\
    \> \> \> \> Discover $f$ --- no undiscovered neighbours
    \\
    \> \> \> \> Finish $f$, since no undiscovered neighbours 
    \\
    \> \> \> Finish $g$, since no undiscovered neighbours left
    \\
    \> \> Finish $h$, since no undiscovered neighbours left
    \\
    \> Finish $e$, since no undiscovered neighbours left
    \\
    Finish $a$, since no undiscovered neighbours left
  \end{tabbing}
  
  
  \textit{\textbf{Sequence~\ref{sequence-dijkstra}}}
  is the result of a call to Dijkstra's algorithm computing shortest
  paths from vertex~$a$.
  To see this, consider in which order the vertices are dequeued from
  the priority queue and how vertex key values are updated as edges
  are relaxed:

  \medskip
  \noindent
  \begin{tabular}[]{|l|l|l|}
    \hline
    \textbf{Dequeued}
    &
%         \textbf{Vertex updates}
      \textbf{Updates}
    &
      \textbf{Priority queue (after relaxations)}
    \\
    \hline
    ---
    &
      ---
    &
      $
      \set{
      (a: 0),
      (b: \infty),
      (c: \infty),
      (d: \infty),
      (e: \infty),
      (f: \infty),
      (g: \infty),
      (h: \infty)
      }
      $
    \\
    $a$
    &
      $\set{b, c, e}$
    &
      $
      \set{
      (c: 2),
      (e: 4),
      (b: 6),
      (d: \infty),
      (f: \infty),
      (g: \infty),
      (h: \infty)
      }
      $
    \\
    $c$
    &
      $\set{b}$
    &
      $
      \set{
      (e: 4),
      (b: 5),
      (d: \infty),
      (f: \infty),
      (g: \infty),
      (h: \infty)
      }
      $
    \\
    $e$
    &
      $\set{h}$
    &
      $
      \set{
      (b: 5),
      (h: 6),      
      (d: \infty),
      (f: \infty),
      (g: \infty)
      }
      $
    \\
    $b$
    &
      $\set{d}$
    &
      $
      \set{
      (h: 6),      
      (d: 12),
      (f: \infty),
      (g: \infty)
      }
      $
    \\
    $h$
    &
      $\set{g}$
    &
      $
      \set{
      (g: 7),      
      (d: 12),
      (f: \infty)
      }
      $
    \\
    $g$
    &
      $\set{f}$
    &
      $
      \set{
      (f: 9),      
      (d: 12)
      }
      $
    \\
    $f$
    &
      $\set{d}$
    &
      $
      \set{
      (d: 10)
      }
      $
    \\
    $d$
    &
      ---
    &
      $
      \set{
      }
      $
    \\
    \hline
  \end{tabular}

  \bigskip
  
  \textit{\textbf{Sequence~\ref{sequence-spoiler}}}, finally,
  is a spoiler sequence that cannot be the output of any of the
  algorithms listed. To see this, consider the following case
  analysis:
  \begin{compactitem}
  \item
    If the sequence were the result of breadth-first search,
    then it would list vertices in order of the number of directed
    edges required to get to the vertex     from
    the start vertex~$a$, but the first deviation from this is that
    $g$ is listed before~$d$.
    
  \item
    A depth-first search starting with $a$, $c$, $b$ would discover~$d$
    next and not~$e$.
        
  \item
    Finally, the sequence cannot be the output of Dijkstra's    algorithms,
    since the vertices are not listed in increasing order of distance
    from~$a$. The first deviation here is that $e$ should come
    before~$b$ in a listing in increasing order of distance.
    
  \end{compactitem}

  
\end{solution}

  
\begin{figure}[t] %%  [tp]
  \centering
  \includegraphics{Figures/graph-algorithm-outputs-25.1}
%
  \caption{Graph for 
    Problem~\ref{problem:graph-algorithm-outputs}%
  }
  \label{fig:graph-algorithm-outputs}
\end{figure}

    

  
\end{problem}




\end{document}



\documentclass[12pt,a4paper]{article}

% Packages
%\usepackage[english,danish]{babel} 
\usepackage[applemac]{inputenc}
\usepackage{amsmath,amscd}
\usepackage{amssymb}
\usepackage{amsthm}
\usepackage{enumerate}
\usepackage{graphicx}        
\usepackage{xcolor}            


\theoremstyle{plain}
\newtheorem{thm}{S�tning}
\newtheorem{prop}[thm]{Proposition}
\newtheorem{lem}[thm]{Lemma}
\newtheorem{cor}[thm]{Korollar}
\newtheorem{conj}[thm]{Formodning}
\theoremstyle{definition}
\newtheorem{exercise}{Opgave}
\newtheorem{definition}[thm]{Definition}
\newtheorem{prob}[thm]{Problem}
\newtheorem{remark}[thm]{Bem�rkning}
\newtheorem{example}[thm]{Eksempel}


% Blackboard bold
\newcommand{\NN}{\mathbb{N}}
\newcommand{\ZZ}{\mathbb{Z}}
\newcommand{\QQ}{\mathbb{Q}}
\newcommand{\RR}{\mathbb{R}}
\newcommand{\CC}{\mathbb{C}}


\newcommand{\GCD}{\operatorname{GCD}}
\newcommand{\LCM}{\operatorname{LCM}}


% Course name

% Intelligent spacing
\usepackage{xspace}

% IDMA course name
\newcommand{\coursename}{Introduction to Discrete Mathematics and Algorithms\xspace}
\newcommand{\coursenameshort}{IDMA 2026\xspace}




% Dates
\newcommand{\mondaylecturedate}{Monday Feb 10\xspace}
\newcommand{\wednesdaylecturedate}{Wednesday Feb 12\xspace}
\newcommand{\deadlinefirstpset}{Wednesday Feb 12 at 12:59 CET\xspace}

\begin{document}

\begin{center}
\textbf{\large \coursenameshort \\[0.5em] -- Ugeseddel 2 -- \\[0.5em]}
\end{center}
\noindent

\section*{General Plan}

This week we will
\textbf{move on to the more mathematical parts of the \coursenameshort course.}
We will start by drawing
%   We will draw 
upon your knowledge about functions like 
\[
2^x\qquad x^{10}\qquad \log_{10} x
\]
from your highschool curriculum
(which you were already reminded about in the notes on Absalon last week
about asymptotic analysis).
You will, however, soon experience
that the typical math tools within computer science are quite
different from the ones you 
%   already know. 
have dealt with before.
Formally speaking, the mathematics
you already know is mostly \textbf{continuous},
while in computer science we mostly care about
\textbf{discrete}
math.

This means that we start at the very beginning with many of the
topics. For this, we have chosen the book  
\begin{center}
\emph{B. Kolby, R.C. Busby, and S.C. Ross}\\
\emph{Discrete mathematical structures, 6th edition}
\end{center}
which we will refer to as \emph{KBR}. We will use a special edition of
the book in the course as you have probably already noticed.  
In addition to this, some of the mathematical contents will come from
the book
\begin{center}
\emph{T. H. Cormen, C. E. Leiserson, R. L. Rivest, and C. Stein}\\
\emph{Introduction to Algorithms, 4th edition}
\end{center}
which we will refer to as \emph{CLRS}. 


\emph{Note that problem set 1 is due on 
  \textbf{\deadlinefirstpset}.}


\section*{Lectures}

We will begin the week by
%   On Monday, we will start
talking about \textbf{sets, sequences and sums.}
In particular, when analyzing algorithms it will not seldom be the case that
the running time can be expressed as a sum, and so we will be interested in
\textbf{summation formulas}
when performing
\textbf{asymptotic analysis} (which we already started getting
acquainted with last week).
The asymptotic notation used in KBR differs from the one introduced in
CLRS. We will favour the CLRS notation as it is the one commonly used
in the context of computer science. The summation formulas are covered
later on in the KBR book and could thus be difficult to read.

We will
%   continue with sets, sequences and sums,
%   on  \textbf{Wednesday morning}, and
then switch to considering
\textbf{fundamental properties of integers} related to
division and discuss concepts such as divisor, multiple, prime,
greatest common divisor ($\GCD$), and least common multiple
($\LCM$). In particular, we will go over Euclid's algorithm, one of
the first algorithmic discoveries of mankind, which lets us calculate
$\GCD$ of two integers efficiently. We will also discuss different
ways to represent integers: first, in terms of their prime
factorization and, second, using the base-$b$ positional
notation. Note that positional notation with $b = 10$ is our standard
way of writing numbers while $b = 2$ is used to represent numbers on a
computer.  All of this is nicely covered in KBR Section 1.4, and quite
likely you are already familiar with most of it.

Towards the end of the week, 
%   On \textbf{Wednesday afternoon}
we will reach 
\emph{one of the most important topics in this whole course},
namely
\textbf{mathematical induction}. This will be our preferred method to prove
%   a sequence of
statements  such as  
\[
%   P(n) :
 \sum_{k=1}^nk=\frac{n(n + 1)}{2}
\]
(which is, strictly speaking,  a sequence of
statements $P(n)$ parameterized by the positive integer~$n$).
We will discuss how to set up and execute a proof by induction, and we
will see that this type of argument is well-suited for proving
statements about recursive algorithms.

Since mathematical induction is such an important element
of the course, wealso provide some
brief notes on mathematical induction on Absalon that 
might be an easier read than KBR.
The difference, however, is not that big, and so you should also read 
Section~2.4 in KBR.  
There might be some terminology in Section~2.4 that is not yet fully
familiar---if so, please just ignore this for now, and we will get to
it later
(next week if everything goes according to plan).



\section*{Reading Instructions}
\begin{itemize}  
\item
  \textbf{\textit{Recap:} CLRS Chapter 3:} 
  We talked about asymptotics already last week, but stressed the
  intuition that this is all about \emph{``focusing on the highest-order
  term and shaving away the constant factor in front.''} This is a good
  intuition, and will help you solve most problems you will encounter
  regarding asymptotic analysis, but of course we also need a firm,
  mathematical understanding of what it is we are doing. For this
  reason, do make sure to have read the discussion of asymptotic
  notation on  
%   pages 43--53.  %% 3rd edition
  pages 50--63.
  All of 
%     pages 43--53 
  this material
  is needed, although \mbox{little-o ($o$)} and \mbox{little-omega ($\omega$)}
  are  less   important   that the other pieces of notation.
%     Pages 53--58    %% 3rd edition
  Pages 63--68 
  discuss common functions and standard notation that you
  will want to become friends with.
  The more advanced material on 
%     pages 58--60    %% 3rd edition
  pages 68--70 
  about function
  iteration, iterated logarithms, and Fibonacci sequences is not so
  important for now 
  (but please note that for a basic course like this, pretty much
  anything you run into has a high chance/risk of appearing in later
  courses, so it might be a good time investment for the future to at
  least look at it anyway).


%   \item
%     \textbf{CLRS Chapter 10}
%     on elementary data structures such as stacks, queues, and linked lists.
%   

\item
  \textbf{KBR Sections 1.1--1.3:}
  We will need all of this material on 
  sets, sequences, and sums.



\item \textbf{KBR Section~1.4} on integers.   
  (Make sure you can follow the steps of the proof of Theorem~4, as
  this is a good example of a more advanced proof. 
%     
%     Expect that you
%     will need to pause and think after reading every sentence or
%     two. You will probably need to read this proof more than once to
%     understand it. 
%   
  In general, a good test for whether you have fully
  understood a proof is whether you can reproduce it---or at the main
  steps in it---without looking in the book.)

\item 
  \textbf{KBR Section~1.5} on matrices
  will be needed later in the course---why not read it right away?

\item KBR Section~1.6 is \textbf{not}    
  part of the required reading, but could be useful if you want to go
  deeper into the mathematics and think more abstractly about what is
  going on.

\item
  \textbf{KBR Section 2.4} on mathematical induction.
  Again, it cannot be stressed enough how important the concept of
  mathematical induction is, so this is something that you should
  really study, and study hard (not just for this course). 

\item
  \textbf{Notes:}
  There are also notes posted on Absalon which mostly overlap with
  the material above.  It is recommended to read these notes to see a
  second exposition of the same material, and thus double your chances
  of getting a deep understanding of it. ;-)
  
\end{itemize}
%
\emph{\textbf{Please note that all of the above is
    foundational material that we will use over
    and over again in the course}
  and that you will also be assumed to know well in later courses.
  Therefore, it is to do good service to yourself to read this carefully, 
  and do enough exercises
  (see below) to be sure that you understand 
  what is going on.}


\section*{Suggested Exercises}

\emph{How many exercises should you do?}
The short answer to this is that you should solve enough problems
to be sure that you know the material.
In general, we provide many exercises to work on, just to be sure that
you will not run out of entertainment, but there is
no need per se to solve absolutely all of these exercises.
What you need to do, though, is to make sure that you work on enough
exercises that you have fully digested the topics covered in the
lectures and the textbooks.


As already mentioned, you are also encouraged to work on the problem
set problems from previous years that are posted on Absalon, but it is
probably a good idea to do some of the exercises below first to make
sure that you have understood the course material.



\subsection*{Exercises from KBR on Integers and Induction}

At the end of the KBR book you can find solution to all odd-numbered
exercises. 
%   However, 
It goes without saying, though, that
it is advisable not to look at the solution until
after you have solved the exercise or if you are completely stuck.
Many of the early KBR exercises are testing basic understanding
and so should hopefully be fairly quick to solve---if not, then 
do make use of the fact that the TAs are there to help you!


\begin{enumerate}[(1)]

%   \begin{itemize}
\item 
KBR exercises
%   Do exercises KBR 
%   1.1.4, 
1.2.1, 
%   1.2.14, 
1.3.7, 
and 1.3.10
%   , and 1.3.38.


\item
  KBR exercises 1.4.10,  1.4.13, 
%   1.4.16, 
1.4.24, 
%    1.4.25,
  and 1.4.35. 
%   
%     Time permitting, instructor presents a sample proof of 1.4.24 at the
%     end of class.
%   

%   
%   \item 
%     Consider the algorithm from KBR for testing if an integer $n$ is a prime. Let's call this algorithm $\mathcal{A}$. 
%   \begin{enumerate}[(a)] 
%   \item  Argue that the running time of $\mathcal{A}$ is $O(\sqrt n)$.
%   \item  Argue that the running time of $\mathcal{A}$ is not $O(1)$\footnote{Recall that $O(1)$ means constant.}. 
%   \item {[*]} Let $t_n$ be the maximum running time of $\mathcal{A}$ over all inputs $k\in\ZZ$, where $1 \le k \le n$. Argue that  $t_n$ is $\Theta(\sqrt n)$.
%   \end{enumerate}
%   


\item 
  Solve KBR exercises
  1.4.43 and 
  1.4.44, and
\begin{enumerate}[(a)]
\item Argue that the running time of the algorithm given in KBR for finding the base $b$ expansion of $n\in\ZZ$ is $O(\log n)$. 
\item Why is it clear that the runtime of this algorithm must be $\Theta(\log n)$? 
\end{enumerate}
\item
  KBR exercises 2.4.3, 
%   2.4.4, 
%   2.4.8,
%   \item
%     Solve KBR exercises 
  2.4.6, 
%     2.4.8,
%     2.4.10,  
and 2.4.16.

\item
  If you read about matrices, then 
  solve (some selection of) 
  KBR exercises 1.5.5, 1.5.9, 1.5.12, 1.5.16, and 1.5.21
  to verify your basic understanding of  the relevant concepts
  (we will return to this later in the course).

\end{enumerate}



\subsection*{Some More Exercises on Asymptotic Analysis and Such}

Unless explicitly stated otherwise, you are free to use theorems from
the notes on Absalon without a proof, as well as make use of any
previously solved exercises.

\begin{enumerate}[(1)]

\item 
KBR exercises
%   as well as 
5.3.1, 
5.3.5, 
%   5.3.7, 
and 5.3.9. 
%   and 5.3.10.

%   \end{itemize}
%   


%   
%   \item Solve the equation (you are welcome to use a computer program)
%   \[
%   10\cdot N^{10}=2^N
%   \]
%   What does the solution mean in terms of the previous exercise? 
%   
%   
%   \item Team 1 and Team 2 have access to two different computers. It takes $c_1>0$ seconds to compute one step for Team 1 and $c_2>0$ seconds to compute a step for Team 2. One of the following statements must hold: 
%   \begin{enumerate}[(A')]
%   \item Team 1's algorithm finishes faster for all $N$.
%   \item Team 2's algorithm finishes faster for all $N$.
%   \item Team 1's algorithm runs finishes faster for some choices of $N$, while Team 2's algorithm finishes faster for some other choices of $N$. 
%   \end{enumerate}
%   Is it possible to dismiss one or more of the statements (A'), (B'), and (C') as being true without knowing the exact value of $c_1$ and $c_2$?
%\item Analyze, in a similar way, the situation where the number of steps for the two algorithms are
%\[
%g_1(N)=N^2-N
%\]
%and
%\[
%g_2(N)=3N^2+2N
%\]


\item Let $f,g:\RR^+ \to \RR$ be asymptotically positive functions. Prove the following:
\begin{itemize}
\item{} {\bf ($\Theta$-relation is symmetric)}  $f(x)$ is $\Theta(g(x))$ if and only if $g(x)$ is~$\Theta(f(x))$.
\item{} If $f(x)$ is $o(g(x))$ then $g(x)$ is \emph{not} $\Theta(f(x))$.
\end{itemize}

%   Time permitting, instructors provide sample proof of one or both of the directions at the end of the exercise session.
%   
%   \end{enumerate}
%   
%   \subsection*{Tuesday Sept.~17th, 15:15-17:00}
%   
%   Unless explicitly stated otherwise, you are free to use theorems from
%   this week's notes without a proof as well as make use of any
%   previously solved exercises.
%   
%   \begin{enumerate}
%   


\item Assume that $a,b,c>0$. Use the properties of the logarithms from
  last week's notes on functions and asymptotic analysis
  to show that
\begin{itemize}
\item  $\left(\log_{a}b\right)\left(\log_{c}d\right)=\left(\log_{a}d\right)\left(\log_cb\right)$.
\item $\log_b a = (\log_a b)^{-1}$
 \end{itemize}
where, in each equation above, logarithm bases are not~$1$.



%   
%   \end{enumerate}
%   
%   
%   \subsection*{Friday~Sept. 20th, 10:15-12:00}
%   
%   
%   Do exercises KBR 5.3.1, 5.3.5, 5.3.7, 5.3.9, 5.3.10 and the exercises
%   below. 
%   
%   Also, don't forget to complete this week's quiz to check your
%   understanding of this week's material. 
%   
%   \begin{enumerate}[(1)]\addtocounter{enumi}{6}
%   


\item Use summation formulas to calculate 
\[
\sum_{k=1}^{100}(k^2+k)+
\sum_{k=0}^{10}5^k
\]
%%%
%%% Why is this fun?
%%%
%   \item {[*]} 
%   \item {[Tricky]} 
%   Find an explicit expression for the series 
%   \[
%   \sum_{k=n}^{2n}(k+3)
%   \]
%%%
%%% I think this one is just too hard... -JN
%%%
%   \item {[**]} 
%   \item {[Even trickier]} 
%   Find an explicit expression for the series 
%   \[
%   \sum_{k=1}^{n}\frac{k}{2^k}
%   \]
\end{enumerate}

%   
%   \section*{Additional exercises}
%   \begin{enumerate}
%   \item Exercises 1.1.4, 1.2.1, 1.2.14, 1.3.7, 1.3.10 and 1.3.38 from KBR.
%   \end{enumerate}
%   




%   
%   \subsection*{Extra Exercises on Primality}
%   \begin{enumerate}[(1)]\addtocounter{enumi}{5}
%   \item
%   We want to find an efficient algorithm which can output a list of all primes between $2$ and $n$.
%   \begin{enumerate}[(a)]
%   \item {[*]} Argue that the algorithm, which checks all integers
%     between $2$ and~$n$ using the algorithm from KBR has running time
%     $\Theta(n^{3/2})$. 
%   \item {[*]} Read about the sieve of Erathosthenes online and produce
%     pseudocode for it. 
%   \item {[***]} Show that the sieve-algorithm is asymptotically
%     faster\footnote{More formally: we want to show that $f(n)$ is
%       $o(g(n))$, where $f(n)$ and $g(n)$ are the running times of the
%       sieve and naive algorithms respectively.} than the naive algorithm
%     discussed above. \textsl{[Hint: Show and then use that
%       $\sum_{k=1}^n\frac{n}{k}$ 
%    is $\Theta(n \log n)$.]}
%   \end{enumerate}
%   \end{enumerate}
%   
%   


\subsection*{Some Additional Exercises (Possibly for Later)}

Here are some extra exercises that might be useful if
you want to repeat this material when preparing for the exam.
%
\begin{enumerate}[(1)]

\item
KBR exercises
%   1.1.4, 
1.2.14
and 1.3.38.

\item
  KBR exercises 1.4.11--12, 
  1.4.16,
  and 
  1.4.25.

\item
  KBR exercise  1.4.41.

\item
  KBR exercises
  2.4.8,   2.4.10, 2.4.27,  
  and 2.4.29. 

\item
  KBR exercises
  5.3.7
   and 5.3.10.

\item Use the definition of big-$O$ to prove or disprove the following
\begin{itemize}
\item $2^{x+1}$ is $O(2^x)$
\item $2^{2x}$ is $O(2^x)$
\end{itemize}


\item \label{Props}  Let $f,g: \RR^+ \to \RR$ be asymptotically positive functions. Consider the following two statements:
\begin{enumerate}
\item $g(x)$ is $O(f(x))$ and $f(x)$ is $O(g(x))$. 
\item There exist constants $c_1,c_2>0$ and $x_0$ such that $c_1 g(x) \le f(x) \le  c_2 g(x) $ for all $x \ge x_0$.
\end{enumerate}
Show that (a) holds if and only if (b) holds thus establishing the equivalence of the two definition of big-$\Theta$ from
last week's notes on functions and asymptotic analysis.

\emph{Remember that proving an ``if and only if" statement involves
  showing two directions.}


\item Use the rules (\emph{B1})--(\emph{B6}), (\emph{L1})--(\emph{L6}), (\emph{M1})--(\emph{M4}) from
last week's notes on functions and asymptotic analysis
to show the following. 
\begin{itemize}
\item $10x^{10}$ is $O(2^x)$
%\item $2^x$ is not $O( 10x^{10})$
\item $10x^{10}$ is $o(2^x)$
\item $x^2-x$ is $O(3x^2+2x)$
\item $3x^2+2x$ is $O(x^2-x)$.
\item $x^2-x$ is $\Theta(3x^2+2x)$
\item $x^3$ is $o(x2^x)$
\item $\frac{\log_2 x}{x^2}$ is $O(1)$
\end{itemize}

\item Let $f,g:\RR^+ \to \RR$ be asymptotically positive functions. Prove rule~(\emph{L1}) from
  the notes on functions and asymptotic analysis.
In your proof you can use the rest of the rules and theorems from the notes as well as the following rule:

\emph{(L1')} If $f(x)$ is $o(g(x))$ then $g(x) \pm f(x)$ is~$\Theta(g(x))$.

\item Would rule (\emph{M2}) hold if we dropped the requirement that the constant~$c$ is positive? Justify your answer.

\item Let $f(n)=2^{\log_3(n)}$ and $g(n)=n$. Find $a$ such that $f(n)= n^a$. Then determine whether $f(n)$ is $O(g(n))$ and whether $g(n)$ is $O(f(n))$.
% grows faster or slower than $g(n)=n$.
\end{enumerate}


\end{document}

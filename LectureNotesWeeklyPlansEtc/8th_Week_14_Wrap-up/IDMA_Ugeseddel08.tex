\documentclass[12pt,a4paper]{article}

% Packages
%   \usepackage[english,danish]{babel}
\usepackage[english]{babel}
\usepackage[utf8]{inputenc}
\usepackage{amsmath,amscd}
\usepackage{amssymb}
\usepackage{amsthm}
\usepackage{enumerate}
\usepackage{graphicx}                    
\usepackage{framed}  
%\usepackage{multicols}

\theoremstyle{plain}
\newtheorem{thm}{S¾tning}
\newtheorem{prop}[thm]{Proposition}
\newtheorem{lem}[thm]{Lemma}
\newtheorem{cor}[thm]{Korollar}
\newtheorem{conj}[thm]{Formodning}
\theoremstyle{definition}
\newtheorem{exercise}{Opgave}
\newtheorem{definition}[thm]{Definition}
\newtheorem{prob}[thm]{Problem}
\newtheorem{remark}[thm]{Bem¾rkning}
\newtheorem{example}[thm]{Eksempel}


% Blackboard bold
\newcommand{\NN}{\mathbb{N}}
\newcommand{\ZZ}{\mathbb{Z}}
\newcommand{\QQ}{\mathbb{Q}}
\newcommand{\RR}{\mathbb{R}}
\newcommand{\CC}{\mathbb{C}}
\newcommand{\GCD}{\operatorname{GCD}}


% Course name
\usepackage{xspace}
\newcommand{\coursename}{Introduction to Discrete Mathematics and Algorithms\xspace}
\newcommand{\coursenameshort}{IDMA 2024\xspace}

\begin{document}

\begin{center}
\textbf{\large \coursenameshort \\[0.5em] -- Ugeseddel 8 -- \\[0.5em]}
\end{center}
\noindent

\section*{General Plan}
%   \section*{Work instructions}

As the final topic in the course curriculum, we will study how to compute shortest paths in graphs as in CLRS Chapter~22 on Monday.

The last items in the schedule
are a question-and-answer session on Wednesday March 26 at 13:15-15:00
followed by a final exercise session with your TAs.

Please think ahead of time if there is anything in particular that you
would like to hear about during the Q\&A session, and send a message
on Absalon if that is the case, so that I can plan accordingly.  In
the same way, do let your TAs know in advance if you have any specific
wishes for the final exercise session.

Otherwise, all that remains is to prepare for the exam
(and, of course, to hand in the final problem set in time, so that
you are sure to be eligible for the exam).


\mbox{ } \\
\noindent
I hope you have enjoyed this course.
\textbf{Good luck on the exam!}

\mbox{ } \\
\noindent
Jakob

\section*{Reading Instructions}
%   \section*{Assigned reading}

As usual, I will do my best to try to cover as much as possible of the
most important material in class, but it is very important that you
also read the textbook, which contains additional material.

\begin{itemize}

\item
  Parts of CLRS Chapter 22 including the introduction, Section 22.3,
  and Section 22.5. We will not talk about Sections 22.1--2, but they
  are warmly recommended reading and there is a high risk you will get
  exposed to this in your very next algorithms course. We will also
  not cover Section 22.4 (although it is a cool application).

\end{itemize}

\section*{Exercises}

\subsection*{CLRS Chapter 22:  Single-source shortest paths}

\begin{enumerate}
\item
  Draw some moderately sized directed graphs
  with weights on the edges and run 
  Dijkstra's algorithm on these graphs.
  Work out the details of how the priority queue heap changes during
  execution.

\item
  CLRS Section 22.3 exercises
  22.3-1,
  22.3-2,
%     22.3-3,
%     22.3-4,
  and
  22.3-6.

\item
  CLRS Section 22.5 exercises
  22.5-1,
  22.5-4,
  and
  22.5-7.

\item
  Let $T$ be a shortest path tree from a vertex~$s$ in a graph~$G$.
  Suppose that we add some constant~$c$ to all edge weights in~$G$. 
  Is $T$ still a shortest path tree? Prove that this is so or give a
  counter-example. 

\item{}[*]
  Let $G$ be a directed graph where all \emph{vertices} have weights,
  and where the weight of a path in~$G$ is the sum of the weights of
  all vertices on the path. Give an algorithm for computing the
  shortest path between two vertices~$s$ and~$t$ in~$G$.

\item{}[*]
  In some modern programming languages there are efficient
  implementations of priority queues in the standard libraries, but
  without support for 
  \textsc{Decrease-Key} (that is, once an element is inserted with a
  certain priority this cannot be changed).
  Can you modify Dijkstra's algorithm to run as efficiently as before
  (asymptotically)
  even with this kind of priority queue?



  
\end{enumerate}



\end{document}



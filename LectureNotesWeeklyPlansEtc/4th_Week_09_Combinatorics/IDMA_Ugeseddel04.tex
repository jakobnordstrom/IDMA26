\documentclass[12pt,a4paper]{article}

% Packages
%   \usepackage[english,danish]{babel}
\usepackage[english]{babel}
%   \usepackage[applemac]{inputenc}
\usepackage[utf8]{inputenc}
\usepackage{amsmath,amscd}
\usepackage{amssymb}
\usepackage{amsthm}
\usepackage{enumerate}
\usepackage{graphicx}                    
\usepackage{framed}  
\usepackage{xcolor}

\usepackage{multicol}

\theoremstyle{plain}
\newtheorem{thm}{Theorem}
\newtheorem{prop}[thm]{Proposition}
\newtheorem{lem}[thm]{Lemma}
\newtheorem{cor}[thm]{Corollary}
\newtheorem{conj}[thm]{Conjecture}
\theoremstyle{definition}
\newtheorem{exercise}{Exercise}
\newtheorem{definition}[thm]{Definition}
\newtheorem{prob}[thm]{Problem}
\newtheorem{remark}[thm]{Remark}
\newtheorem{example}[thm]{Example}


% Blackboard bold
\newcommand{\NN}{\mathbb{N}}
\newcommand{\ZZ}{\mathbb{Z}}
\newcommand{\QQ}{\mathbb{Q}}
\newcommand{\RR}{\mathbb{R}}
\newcommand{\CC}{\mathbb{C}}
\newcommand{\GCD}{\operatorname{GCD}}

% Course name

% Intelligent spacing
\usepackage{xspace}

% IDMA course name
\newcommand{\coursename}{Introduction to Discrete Mathematics and Algorithms\xspace}
\newcommand{\coursenameshort}{IDMA 2026\xspace}




\begin{document}

\begin{center}
\textbf{\large \coursenameshort \\[0.5em] -- Ugeseddel 4 -- \\[0.5em]}
\end{center}
\noindent


\section*{General Plan}


This week will be dedicated to 
\textbf{combinatorics}
as discussed in Chapter~3 of KBR.
In combinatorics we consider questions like:
\begin{itemize}
\item 
  In how many different ways can one draw a hand of 5 cards from a set of
  52~different cards?  
\item 
  What happens to the number if the order of the cards matter? 
\end{itemize}
%
One particularly important combinatorial principle that will come up
is the \textbf{pigeonhole principle}, which simply says that if you
put strictly more than $n$~objects (pigeons) in $n$~containers
(pigeonholes), then there will be more than one object in at least one
of the containers.  Although this is completely, blindingly obvious,
it can be a surprisingly useful fact.

When talking about combinatorics it is also natural to 
touch briefly on
\textbf{probability theory},  so that we can answer questions like 
\begin{itemize}
\item
  What is the probability that two of the cards that are drawn are spades? 
\end{itemize}
We will see that it is essential to differentiate between combinatoric
questions where the elements are ordered and questions where the order
does not matter.  

Just as a side note, an important application of probabilities in
computer science is when we go beyond worst-case analysis to perform
\emph{average-case running time analysis}, where one looks at how
efficient an algorithm is on average instead of focusing on the
worst-case performance.
Another application is in
\emph{randomized algorithms},
which can make random choices during execution to find the answer
faster (but might also commit errors with small probability).
Such things are beyond the scope of this course, however, so you will
have to wait until your next course on algorithms for this.

%We will also talk a bit about \textbf{matrices} and how they can be useful in
%discrete mathematics. Most (or all) of you would also be taking a
%separate course on linear algebra, but for many (or all) of you it
%comes after this course, and so cannot help us with the material we
%need.  Therefore, just to make sure we are on the same page, we will
%make sure to quickly cover the basics in a self-contained and
%elementary way.
%
%Depending on how things go, we might also be able to discuss
%\textbf{relations}
%on Wednesday, but most of this will be saved for next week.   


\section*{Reading Instructions}
%   \section*{Assigned reading}
\begin{itemize}
\item 
  Our lectures will cover
  KBR \mbox{Sections 3.1--3.4}.
  
\item
  \mbox{Section 3.5} is not included in the course requirements, 
  but is \emph{definitely} useful to read if you feel that you have already
  mastered the rest of the material.

%\item 
%  KBR Section 1.5 is about matrices.
%\item 
%  To learn about relations, we will read
%  KBR Chapter 4, except sections 4.6 and 4.8,
%  and
%  KBR Section 5.1,
%  but most or all of this will be covered next week.

\end{itemize}


\section*{Exercises}
%   \section*{Exercise plan}


%   \subsection*{Monday Oct. 28th, 10:15-12:00}


\begin{enumerate}
\item
  Here are two exercises 
  to understand the proof we did in class of how many different multisets of
  size~$r$ can be chosen from a universe of size~$n$.

  \begin{enumerate}
  \item  
  Do the translation from multisets and coloured boxes for the multiset
  $[2,3,5,5,8,8]$
  chosen from the universe
  $S = \{1,2,3,4,5,6,7,8\}$.
  Then take the row of coloured boxes that you get and translate them
  back to a  multiset, just to check that you get back the multiset
  with which you started. 

  (This example also illustrates an
  interesting corner case in the proof, namely when there are copies
  of the last element in the universe in the multiset. 
  It is hard to find time to talk properly about this during the lecture,
  but the details are ironed out in the lecture notes that are
  posted on Absalon.)

  \item
    With the same parameters $n=8$ and $r=6$,
    consider the row of boxes where all prime numbers between~$1$
    and~$13$ are coloured red. Which multiset does this correspond to?
    Take this multiset and run the colouring translation to check that
    you get the same coloured boxes back.

 \end{enumerate}

\item 
  To test your basic understanding,
  solve KBR exercises 3.1.1, 3.1.4, 3.1.8, 3.1.9

\item 
  For some more interesting problems,
  solve KBR exercises
  3.1.22 and 3.1.31.
   
\item 
  For some more interesting problems,
  solve KBR exercises 
  3.1.33 and 3.1.34.
%   
%     \begin{itemize}
%     \item{}[*] 
%       Write pseudocode for the procedure (algorithm) you came up
%       with in 3.1.34 for counting the number of zeroes at the end of $n!$ 
%     \item{} 
%       Analyze the runtime of your algorithm in terms of the input $n$ and
%       express it using the big-$O$ notation. 
%     \item{} [**] 
%       Can you find an algorithm that runs in time $O(\log(n))$?
%     \end{itemize}
%   
\item 
  For a mix of combinatorial problems,
  solve KBR exercises 
  3.2.1, 
%   3.2.2, 
  3.2.6,
  3.2.16,
%     3.2.17,
  3.2.25, 
%     3.2.26, 
  3.2.33.

\item 
  For a mix of combinatorial problems,
  solve KBR exercises 
  3.3.6, 3.3.23 

\item 
  To work on probability theory,
  solve KBR exercises 
%     3.4.1, 
  3.4.12, 3.4.20, 3.4.33, 3.4.34.

\item 
  For a challenging probability problem,
  solve KBR exercise
  KBR 3.4.36 (part (c) is [**]).

\item 
  Continue the discussion
  from the lecture 
%   regarding the fairness of the rules of 
  about
  poker 
%     (likely to happen on Wednesday)
  by calculating and comparing the associated probabilities 
  that we did not already compute in class. The hierarchy is as follows:
  % where the hierarchy is 
  \begin{itemize}
    \begin{multicols}{3}  
    \item[(1)] Straight flush
    \item[(2)] Four of a kind
    \item[(3)] Full house
    \item[(4)] Flush
    \item[(5)] Straight
    \item[(6)] Three of a kind
    \item[(7)] Two pairs
    \item[(8)] A pair
    \item[] {} \phantom{m}% High card
    \end{multicols}
  \end{itemize}%\vspace{-}

%   
%   \item
%     \emph{Recap problem:}
%     Prove that the sum of the first $n$ odd positive integers is~$n^2$.
%   
%   \item
%     \emph{Recap problem:}
%     Prove that there is some constant~$k$ such that
%     for \mbox{all $n \geq k$} it holds that
%     $n! > 2^n$.
%   
%   
%   \item
%     Finish any leftover exercises from the previous exercise classes 
%   %   of this week. 
%     so far,
%     if this would be helpful for you to understand what has been covered
%     in the course up to this point.
%   
\end{enumerate}


\end{document}


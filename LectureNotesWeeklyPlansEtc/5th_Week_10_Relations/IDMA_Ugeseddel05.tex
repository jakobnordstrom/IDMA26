\documentclass[12pt,a4paper]{article}

% Packages
%    \usepackage[english,danish]{babel}
\usepackage[english]{babel}
%    \usepackage[applemac]{inputenc}
\usepackage[utf8]{inputenc}
\usepackage{amsmath,amscd}
\usepackage{amssymb}
\usepackage{amsthm}
\usepackage{enumerate}
\usepackage{graphicx}                    
\usepackage{framed}  
\usepackage{multicol}

\theoremstyle{plain}
\newtheorem{thm}{Theorem}
\newtheorem{prop}[thm]{Proposition}
\newtheorem{lem}[thm]{Lemma}
\newtheorem{cor}[thm]{Corollary}
\newtheorem{conj}[thm]{Conjecture}
\theoremstyle{definition}
\newtheorem{exercise}{Exercise}
\newtheorem{definition}[thm]{Definition}
\newtheorem{prob}[thm]{Problem}
\newtheorem{remark}[thm]{Remark}
\newtheorem{example}[thm]{Example}


% Blackboard bold
\newcommand{\NN}{\mathbb{N}}
\newcommand{\ZZ}{\mathbb{Z}}
\newcommand{\QQ}{\mathbb{Q}}
\newcommand{\RR}{\mathbb{R}}
\newcommand{\CC}{\mathbb{C}}
\newcommand{\GCD}{\operatorname{GCD}}

% Domain and range
\newcommand{\ran}{\operatorname{Ran}}
\newcommand{\dom}{\operatorname{Dom}}

% Course name
\usepackage{xspace}
\newcommand{\coursename}{Introduction to Discrete Mathematics and Algorithms\xspace}
\newcommand{\coursenameshort}{IDMA 2025\xspace}

\begin{document}

\begin{center}
\textbf{\large \coursenameshort \\[0.5em] -- Ugeseddel 5 -- \\[0.5em]}
\end{center}
\noindent


\section*{General Plan}

We will start this week by talking a bit about \textbf{matrices} and how they can be useful in
discrete mathematics. Most (or all) of you would also be taking a
separate course on linear algebra, but for many (or all) of you it
comes after this course, and so cannot help us with the material we
need.  Therefore, just to make sure we are on the same page, we will
make sure to quickly cover the basics in a self-contained and
elementary way.

We then come to the topic of \textbf{relations}. As we all know, relations are some of the most important topics in life,
and that holds true also in mathematics, so we will focus on relations this week.
Only, in mathematics the word ``relation'' has a very precise meaning,
which might be slightly different from our every-day usage.
In fact, we have already been using mathematical relations, and
properties of them,  in several previous lectures, and so it is high
time that we pin down exactly what we mean.

Informally, relations between integers are things like
\emph{``$a$ is less than~$b$''} (which we usually denote $a < b$)
or
\emph{``$a$ divides~$b$''} (denoted $a | b$).
Other examples of relations are between people, such as
\emph{``A~is a friend of B''}
or
\emph{``A~is a parent of B.''}

Mathematically, 
relations can be viewed in  different ways: 
\begin{itemize}
\item
  as a logical predicate $P(a,b)$ regarding a pair of elements $a\in
  A$ and $b\in B$,  
\item 
  as a subset of the \emph{product set} $A\times B$,
\end{itemize}
and if $A$ and $B$ are finite sets:
\begin{itemize}
\item as a Boolean matrix, 
\item as a directed graph (digraph) when $A = B$. 
\end{itemize}
It might seem a little extreme to have four different descriptions of
the same concept. However, it is useful to be able to represent
relations in different ways, as then we have the freedom to choose the
description best suited for the task at hand.  

During our treatment of relations, we will also make a 
%   quick 
detour to
talk about \textbf{functions} to see that they are
%   
%   During our treatment of KBR Chapter 4, we will make a quick detour to
%   KBR 5.1 in order to see that functions are
%   
relations of a special form,
and then discuss the concepts of \emph{injective}, 
\emph{surjective}, \emph{bijective}, and \emph{inverse}
functions.


We will then talk about 
properties of relations
such as \emph{reflexivity},  \emph{symmetry}, and
\emph{transitivity}. These three properties together define an
important family of \textbf{equivalence relations}. 

Continuing our discussion of relations, we will
consider two important types: \textbf{order relations} and
\textbf{trees}. The former is relevant in computer science because
this type of relation is what we use when we sort data.  The latter
will lead us on to a more general discussion of \textbf{graphs}---which
will be the topic of the last two weeks of lectures on the course.


\section*{Reading Instructions}
%   \section*{Assigned reading}

\begin{itemize}
   \item 
     KBR Section 1.5 on matrices (Monday)
     \item 
  KBR Chapter 4 except sections 4.6 and 4.8 (Wednesday morning)
\item 
  KBR Section 5.1  (Wednesday morning)
  \item KBR Sections 6.1--6.2 (Wedesday afternoon)
\item KBR Sections 7.1 and 7.2 (Wednesday afternoon)
%     (except we will not talk about spanning trees quite yet)
\item KBR Section 8.1 (Wednesday afternoon)
\end{itemize}

Please note that there is quite a lot of material in Chapter~4, 
and a substantial part of it is introducing formal terminology for
different notions. We will not be able to cover all of this in detail
in class, so it is important (as is always the case, but even more so
here) that you also read through the material carefully on your own.

%\section*{Reading Instructions
%  Mostly for Wednesday
%}
%%   \section*{Assigned reading}
%
%As usual, I will try to cover the most important material during the
%lecture, but it is very important that you also read the textbook,
%which contains additional material. 
%\begin{itemize}
%\item KBR Sections 6.1--6.2
%\item KBR Sections 7.1 and 7.2
%%     (except we will not talk about spanning trees quite yet)
%\item KBR Section 8.1
%\end{itemize}



Although we will not talk about it this week, parts of Section~7.5
will be covered in a week or two.  The same goes for Section~7.3 with
traversals of trees. In other words, it does not hurt reading these
sections already now (and in any case they are part of your general
computer science education).

Also, in case you want to read on after Section~8.1, then in
Sections~8.2 and~8.3 you find a discussion of Eulerian and Hamiltonian
cycles, which you might remember is what we started talking about in
the very first lecture on the course. We probably will not have time
to talk about these sections, though (or at least that is not in the
plans as they are looking now).

%   
%   \emph{\textbf{Update:} It said Section~7.4 instead of Section~7.2 above
%     previously---my apologies for the typo.}
%   

\emph{Just to avoid confusion---in case you would start wondering why
  there is no mention of Section~7.5 in the course plan---the reason
  for this is that while we will cover this material, we will do so by
  reading CLRS instead.}




\section*{Exercises}

Note that as for previous weeks, there are quite a few exercises
suggested below. 
%   
%    One question that came up in connection with a
%   lectures is how many of these exercises you are supposed to do. The
%   answer is that this depends on how many exercises you yourself need to
%   solve in order to digest the material.
%   
If you feel that you understood what was covered in class and/or what
you read in the textbook, and if the exercises seem straightforward,
then it is fine to just do some of them (where I good idea would be to
try to focus on the one that seem hardest to you). If some of the
material seems harder, though, then it is a good idea to do more
exercises until you really understand what is going on.

Monday material: 

\begin{enumerate}
\item Solve KBR exercises 1.5.5, 1.5.9, 1.5.12, 1.5.16, and 1.5.21.
\item Solve KBR exercises 1.5.40, 1.5.42, 1.5.43.
\end{enumerate}

Wednesday material:

\begin{enumerate}
\item Solve KBR exercises 4.1.5, 4.1.10 %Cartesian products
\item Solve KBR exercises 4.2.4, 4.2.9, 4.2.10, 4.2.23, 4.2.25 
% The definition of a relation
% 4.2.29, 4.2.37 - remoed because they don't exist?

%   
%   \item A relation $R$ from $A$ to $B$ can be represented as a list of pairs from the product set $A \times B$.
%   % (this is actually how relations are defined on page 128 in KBR). 
%   Discuss the following two exercises among yourselves:
%   \begin{itemize}
%   \item Write a pseudocode that, given a list as described above, finds
%     the \emph{range} $\ran R$. 
%   \item In the same spirit, write a pseudocode that finds the
%     \emph{domain} $\dom R$. 
%   \end{itemize}
%   \item A relation $R$ from $A$ to $B$ can also be represented as a
%     Boolean matrix. How should the above algorithms be adapted to
%     accommodate a matrix as input instead of a list of elements in
%     $A\times B$? 
%   \item {[*]} \emph{(Proof exercise)} Solve KBR 4.2.20. Time permitting,
%     instructor presents a sample proof on the board. 
%   


\item Solve KBR exercises 4.3.1--2, 4.3.4--8, 4.3.19 %Basic definitions of digraphs
\item Solve KBR exercises 4.4.1, 4.4.13 
% Reflexive, symmetric, transitive etc.
% 4.4.31--32, 4.4.35 don't exist?
\item Solve KBR exercises 4.5.3, 4.5.4, 4.5.8, 4.5.12 %Equivalence relations
\item Solve KBR exercises 4.7.2, 4.7.7, 4.7.12, 4.7.16--17, 4.7.19 % Closures of relations
\item Solve KBR exercises 5.1.1, 5.1.11, 5.1.30 %Functions

%   
%   \item The instructor facilitates a discussion of \emph{injective},
%     \emph{surjective}, \emph{bijective}, and \emph{inverse} functions.
%   




% \item {[*]} \emph{(Proof exercise)} Solve KBR 4.5.23.  % Doesn't exist?
%   Time permitting, instructor presents a sample proof on the board at
%   the end of the exercise session. 

%   \item Finish any leftover exercises from Monday and Tuesday.


\item Solve KBR exercises 6.1.2, 6.1.6, 6.1.9, 6.1.13, 


\item Solve KBR exercises 6.2.1, 6.2.2, 6.2.9, 6.2.10, 6.2.13--14
%   \item The instructor introduces the concept of  ``lexicographical order''.
%   \item Solve KBR exercises 6.1.19, 6.1.20.

\item Solve KBR exercises  7.1.1--4, 7.1.9--13, 7.2.1--2, 7.1.24--26

%   
%   \item  Discuss the following exercise in class:
%     \begin{enumerate}
%     \item Write a pseudo code that determines if a relation, given as a subset of $A\times A$, is a tree.
%     \end{enumerate}
%   

%   
%   \item Solve KBR exercises 7.4.1, 7.4.2, 7.4.3.
%   %   \item The instructor introduces Prim's algorithm for finding spanning trees.
%    \item Solve KBR exercises 7.4.7, 7.4.8, 7.4.9
%   

\item Solve KBR exercises 8.1.1--2, 8.1.5, 8.1.17, 8.1.19, 8.1.20.
\end{enumerate}

More advanced exercises:

\begin{enumerate}
\item  Discuss the following among yourselves:
\begin{itemize}
\item Write pseudocode that determines whether a relation given as a list of elements from $A\times B$ is
\begin{itemize}
\item reflexive
\item irreflexive 
\item symmetric
\item asymmetric
\item antisymmetric
\end{itemize}
\item Write pseudocode that does the same for a relation given as a Boolean matrix.
\end{itemize}

\item Solve KBR exercises 6.1.19, 6.1.20.

   \item Solve KBR exercises 6.1.21, 6.1.22.

\item{\textit{(Partitioning a poset into linear orders)}} Solve parts
  (e) and (d) of \mbox{KBR~6.1.26--27}
\item {\textit{(Proof exercise)}} Solve KBR 6.1.28

\item Solve KBR exercises  6.2.17--18, 6.2.36
\item {\textit{(Proof exercise)}} KBR 6.2.20

\item {\textit{(Proof exercises)}} Solve KBR 7.1.29--30, 7.1.31

\end{enumerate}




\end{document}

%%% Local Variables:
%%% mode: latex
%%% TeX-master: t
%%% End:

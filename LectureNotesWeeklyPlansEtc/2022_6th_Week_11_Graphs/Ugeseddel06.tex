\documentclass[12pt,a4paper]{article}

% Packages
%   \usepackage[english,danish]{babel}
\usepackage[english]{babel}
\usepackage[utf8]{inputenc}
\usepackage{amsmath,amscd}
\usepackage{amssymb}
\usepackage{amsthm}
\usepackage{enumerate}
\usepackage{graphicx}                    
\usepackage{framed}  
%\usepackage{multicols}

\theoremstyle{plain}
\newtheorem{thm}{S¾tning}
\newtheorem{prop}[thm]{Proposition}
\newtheorem{lem}[thm]{Lemma}
\newtheorem{cor}[thm]{Korollar}
\newtheorem{conj}[thm]{Formodning}
\theoremstyle{definition}
\newtheorem{exercise}{Opgave}
\newtheorem{definition}[thm]{Definition}
\newtheorem{prob}[thm]{Problem}
\newtheorem{remark}[thm]{Bem¾rkning}
\newtheorem{example}[thm]{Eksempel}


% Blackboard bold
\newcommand{\NN}{\mathbb{N}}
\newcommand{\ZZ}{\mathbb{Z}}
\newcommand{\QQ}{\mathbb{Q}}
\newcommand{\RR}{\mathbb{R}}
\newcommand{\CC}{\mathbb{C}}
\newcommand{\GCD}{\operatorname{GCD}}

\begin{document}

\begin{center}
\textbf{\large DMFS 2022 \\[0.5em] -- Ugeseddel 6 -- \\[0.5em]}
\end{center}
\noindent

%   
%   \begin{center}
%   \textbf{\large 
%     DMFS 2021
%     \\[0.5em] 
%     -- 
%     Reading 
%   %     Instructions 
%     and Exercises
%     for 
%     the 
%     6th Week 
%     -- 
%     \\[0.5em]}
%   \end{center}
%   \noindent
%   

\section*{General Plan}
%   \section*{Work instructions}

This week we will continue talking about graphs and graph algorithms.
We will finish our discussion of minimum spanning trees in CLRS
Chapter 23, then make a detour to learn about the useful heap data
structure in CLRS Chapter 6, and finally study how to compute shortest
paths in graphs as in CLRS Chapter 24.

As already mentioned, graphs are a truly foundational topic in
computer science. They can be used to model all kinds of real-world
problems, and efficient graph algorithms therefore have applications
that are too numerous to list.  Since this is an introductory course
in discrete math and algorithms, though, we do not really have time to
get into any serious applications at this point (you will see them
later during your studies), but will focus on covering some of the
basic algorithms used to manipulate these objects.


\section*{Reading Instructions}
%   \section*{Assigned reading}

As usual, I will 
do my best to try to cover as much as possible of the most important
material in class,
%   try to cover the most important material during the lecture, 
but it is very important that you also read the textbook,
which contains additional material. 


\begin{itemize}
\item
  CLRS Chapter 23 about minimum spanning trees (all of it).


\item CLRS introduction to Part II (pages 147--150) (especially if you
  want to get a bit of an overview why we are so interested in sorting data).

\item 
  CLRS Chapter 6 about heaps (all of it).

\item
  Parts of CLRS Chapter 24 including the introduction, Section 24.3,
  and Section 24.5. We will not talk about Sections 24.1--2, but they
  are warmly recommended reading and there is a high risk you will get
  exposed to this in your very next algorithms course. We will also
  not cover Section 24.4 (although it is a cool application).

\end{itemize}


\section*{Exercises}

Some of the exercises below were listed also last week, 
but I am mentioning them here as well to give an
overview of good exercises concerning graphs.

\subsection*{CLRS Chapter 22: Elementary Graph Algorithms}

\begin{enumerate}
\item
  Draw some moderately sized directed graphs
  and make sure that you can run DFS and BFS on them.
%   
%     and make sure that you can run DFS, BFS, and Dijkstra's algorithm on
%     these graphs (where for Dijkstra's algorithm you will also need to
%     add weights on the edges).
%   

\item
  Run topological sort on the same algorithms. Check that
  the algorithm works precisely when your directed graphs are acyclic.

\item
  Use the algorithm in Sec 22.5 to compute strongly connected
  components of some of your directed graphs. 


\item
  CLRS Section 22.1 exercises
  22.1-3,
  22.1-4,
  22.1-5, and
  22.1-6.

\item
  CLRS Section 22.2 exercises
  22.2-2,
  22.2-4,
  22.2-7, and
  22.2-9.

\item
  CLRS Section 22.3 exercises
  22.3-2 and   22.3-7.

\item
  CLRS Section 22.4 exercise
  22.4-3.

\item
  CLRS exercise 22-3.

\end{enumerate}

\subsection*{CLRS Chapter 23:
  Minimum spanning trees}


\begin{enumerate}

 
\item
  Draw some moderately sized undirected graphs
  with edge weights (making sure to consider also graphs with several
  edges of the same weight) and run Prim's algorithm starting from
  different vertices. Can you find graphs where different starting
  points yield different MSTs?


\item
  Draw some moderately sized undirected graphs
  with edge weights (making sure to consider also graphs with several
  edges of the same weight) and run Kruskal's algorithm.
  Can you find graphs for which you get different MSTs depending on
  which order edges of the same weight happen to be sorted?


\item
  CLRS Section 23.1 exercises
  23.1-1,
  23.1-3--23.1-7,
  and
  23.1-9

\item
  CLRS Section 23.2 exercises
  23.2-1 and
  23.2-2

\item{}[*]
  CLRS  exercises
  23.2-7 and 23-4



\item CLRS Section 23.2 exercises 
  23.2-2, 


\item{}[*]
  In some modern programming languages there are efficient
  implementations of priority queues in the standard libraries, but
  without support for 
  \textsc{Decrease-Key} (that is, once an element is inserted with a
  certain priority this cannot be changed).
  Can you modify Dijkstra's algorithm to run as efficiently as before
  (asymptotically)
  even with this kind of priority queue?


\end{enumerate}


\subsection*{CLRS Chapter 6: Heaps}

\begin{enumerate}
\item
  Take some moderately sized arrays filled with numbers (or comparable
  keys of your choice) and build min-heaps and max-heaps from them.

\item
  CLRS Section 6.1 exercises
  6.1-1--6.1-7

\item
  CLRS Section 6.2 exercises
  6.2-1
  and
  6.2-3--6.2-6

\item
  CLRS Section 6.3 exercises
  6.3-1--6.3-3

\item
  CLRS Section 6.4 exercises
  6.4-1--6.4-4

\item
  CLRS Section 6.5 exercises
  6.5-1,
  6.5-2,
  6.5-5,
  and
  6.5-7--6.5-9

\item
  CLRS  exercise 6-1

\item
  Suppose that a weighted undirected graph~$G$ has unique heaviest
  edge~$e$.
  Could this edge ever be included in an MST?
  Give an example of how this can happen or prove that it is impossible.



\end{enumerate}

\subsection*{CLRS Chapter 24:  Single-source shortest paths}

\begin{enumerate}
\item
  Draw some moderately sized directed graphs
  with weights on the edges and run 
  Dijkstra's algorithm on these graphs.
  Work out the details of how the priority queue heap changes during
  execution.
  


\item
  CLRS Section 24.3 exercises
  24.3-1,
  24.3-2,
  24.3-3,
  24.3-4, and
  24.3-6.

\item
  CLRS Section 24.5 exercises
  24.5-1,
  24.5-4, and
  24.5-7.

\item
  Let $T$ be a shortest path tree from a vertex~$s$ in a graph~$G$.
  Suppose that we add some constant~$c$ to all edge weights in~$G$. 
  Is $T$ still a shortest path tree? Prove that this is so or give a
  counter-example. 

\item{}[*]
  Let $G$ be a directed graph where all \emph{vertices} have weights,
  and where the weight of a path in~$G$ is the sum of the weights of
  all vertices on the path. Give an algorithm for computing the
  shortest path between two vertices~$s$ and~$t$ in~$G$.

\end{enumerate}



\end{document}

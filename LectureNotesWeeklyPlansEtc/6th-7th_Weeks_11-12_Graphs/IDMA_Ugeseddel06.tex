\documentclass[12pt,a4paper]{article}

% Packages
%   \usepackage[english,danish]{babel}
\usepackage[english]{babel}
\usepackage[utf8]{inputenc}
\usepackage{amsmath,amscd}
\usepackage{amssymb}
\usepackage{amsthm}
\usepackage{enumerate}
\usepackage{graphicx}                    
\usepackage{framed}  
%\usepackage{multicols}

\theoremstyle{plain}
\newtheorem{thm}{S¾tning}
\newtheorem{prop}[thm]{Proposition}
\newtheorem{lem}[thm]{Lemma}
\newtheorem{cor}[thm]{Korollar}
\newtheorem{conj}[thm]{Formodning}
\theoremstyle{definition}
\newtheorem{exercise}{Opgave}
\newtheorem{definition}[thm]{Definition}
\newtheorem{prob}[thm]{Problem}
\newtheorem{remark}[thm]{Bem¾rkning}
\newtheorem{example}[thm]{Eksempel}


% Blackboard bold
\newcommand{\NN}{\mathbb{N}}
\newcommand{\ZZ}{\mathbb{Z}}
\newcommand{\QQ}{\mathbb{Q}}
\newcommand{\RR}{\mathbb{R}}
\newcommand{\CC}{\mathbb{C}}
\newcommand{\GCD}{\operatorname{GCD}}

% Course name
\usepackage{xspace}
\newcommand{\coursename}{Introduction to Discrete Mathematics and Algorithms\xspace}
\newcommand{\coursenameshort}{IDMA 2025\xspace}

\begin{document}

\begin{center}
\textbf{\large \coursenameshort \\[0.5em] -- Ugeseddel 6 -- \\[0.5em]}
\end{center}
\noindent

\section*{General Plan}

\emph{%
  I am taking the risk of publishing this ugeseddel mostly in
  Danish. Apologies in advance for any errors, which are the
  responsibility of your lecturer only\ldots 
}


Vi skal i denne uge og i uge 7 arbejde med grafalgoritmer. 
Vi har tidligere i kurset set grafer og eksempler på brug af grafer i
algoritmik. Som I måske har fået  fornemmelse af kan grafer ofte
bruges til at modellere strukturer i konkrete data: vi har blandt
andet talt om facebook-grafer og veje mellem byer som nogle få
eksempler på en meget lang række anvendelser af
grafer. Nerveforbindelser mellem hjernedele, design af computerchips,
datanetværk (som f.eks. internettet) og transportnetværk,
afhængigheder i statistik, og tildeling af variable til CPU-registre i
oversættere er nogle yderligere eksempler. Grafer er nogle af de mest
benyttede modelleringsstrukturer, og grafalgoritmer er derfor
tilsvarende vigtige.

Vi vil specielt se på algoritmer til at finde forskellige strukturer i
grafer, vise deres køretid, og bevise korrekthed af algoritmerne.

Uge 6 fokuserer på repræsentation af grafer i computere og strategier
for søgning i grafer. Søgning kan løst beskrives som at gå en tur i
grafen hvor alle knuder besøges. Her har vi to overordnede strategier:
bredde-først søgning og dybde-først søgning
(which in English is
breadth-first search, abbreviated BFS, and
depth-first search, abbreviated DFS, respectively).

%   
%   {\it Kursusevaluering af DMA foregår i ugerne 5 og 6. Der er sat tid
%     af til at lave kursusevaluering til øvelserne torsdag uge 6. Det er
%     meget vigtigt for os og for de studerende på kurset næste år at I
%     udfylder kursusevaluering senest til fristen søndag d. 13/1.}
%   

\section*{Program for forelæsninger}

Repræsentation af grafer (CLRS~20.1);
bredde-først søgning (CLRS~20.2)
og
dybde-først søgning  (CLRS~20.3);
%   Fortsættelse af
%   bredde-først søgning og
%   dybde-først søgning;
topologisk sortering  (CLRS~20.4);
og sammenhængskomponenter (CLRS~20.5).

%   og sammenhængskomponenter
%   CLRS~22.3 indtil 'Properties of depth-first search' 
%   \mbox{s. 606}, 22.4, 22.5).

%   
%   \subsection*{Onsdag}
%   %   Intro til m
%   Mindste udspændende træer
%   %   og Kruskals algoritme (CLRS~23 \mbox{s.~624-636})
%   og Prims og Kruskals algoritmer (CLRS~23).
%   

\section*{Program for øvelser}


Løs opgaverne
\begin{itemize}
\item CLRS~20.1-1,  20.1-3, 20.1-7.
\item CLRS 20.2-1, 20.2-2, 20.2-4
\end{itemize}

Løs udvalgte af opgaverne
\begin{itemize}
\item CLRS
  20.3-2, 20.3-11,  20.3-12

\item
  CLRS
  20.4-1, 20.2-5, 20.2-7
%     , 22.2-6

\end{itemize}

%%%
%%% Numbers as in CLRS 3rd edition below
%%%
%   Løs opgaverne
%   \begin{itemize}
%     \item håndkør Kruskals algoritme på en vægtet graf. Diskuter ved hver Union
%       hvorfor kanten er 'safe'.
%   \item CLRS~23.1-1 23.1-3 23.1-4 23.1-6 23.1-9 23.2-2 23.2-5
%   \end{itemize}
%   


%   \section*{Fordybelsesopgaver}
\subsection*{Fordybelsesopgaver}
 \begin{itemize}
  \item{[*]} CLRS~20.2-8 (diameter af træer)
 \end{itemize}

\end{document}

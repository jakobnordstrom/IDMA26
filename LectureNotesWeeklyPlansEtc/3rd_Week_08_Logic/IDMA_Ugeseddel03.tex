\documentclass[12pt,a4paper]{article}

% Packages
%   \usepackage[english,danish]{babel}
\usepackage[english]{babel}
%   \usepackage[applemac]{inputenc}
\usepackage[utf8]{inputenc}
\usepackage{amsmath,amscd}
\usepackage{amssymb}
\usepackage{amsthm}
\usepackage{enumerate}
\usepackage{graphicx}                    
\usepackage{framed}  
\usepackage{xcolor}

\usepackage{multicol}

\theoremstyle{plain}
\newtheorem{thm}{Theorem}
\newtheorem{prop}[thm]{Proposition}
\newtheorem{lem}[thm]{Lemma}
\newtheorem{cor}[thm]{Corollary}
\newtheorem{conj}[thm]{Conjecture}
\theoremstyle{definition}
\newtheorem{exercise}{Exercise}
\newtheorem{definition}[thm]{Definition}
\newtheorem{prob}[thm]{Problem}
\newtheorem{remark}[thm]{Remark}
\newtheorem{example}[thm]{Example}


% Blackboard bold
\newcommand{\NN}{\mathbb{N}}
\newcommand{\ZZ}{\mathbb{Z}}
\newcommand{\QQ}{\mathbb{Q}}
\newcommand{\RR}{\mathbb{R}}
\newcommand{\CC}{\mathbb{C}}
\newcommand{\GCD}{\operatorname{GCD}}


% Course name

% Intelligent spacing
\usepackage{xspace}

% IDMA course name
\newcommand{\coursename}{Introduction to Discrete Mathematics and Algorithms\xspace}
\newcommand{\coursenameshort}{IDMA 2026\xspace}




\begin{document}

\begin{center}
  \textbf{\large \coursenameshort \\[0.5em] -- Ugeseddel 3 -- \\[0.5em]}
\end{center}
\noindent


\section*{General Plan}
%   \section*{Work instructions}

In the first lecture of the week, we will wrap up the part of the course that deals with the important technique of \textbf{mathematical induction.}

Next up on our agenda is to 
%   talk about mathematical logic and proofs.
talk more formally about \textbf{mathematical logic} and \textbf{proofs}.
%
We will show how
propositional logic can be used to structure and analyse mathematical
arguments. Next, we will introduce predicate logic, which provides a
convenient language for formalizing properties of objects (like ``the
integer $n$ is a prime'' or ``the matrix $A$ is invertible'').  After this, we will
move on to a discussion of proof techniques such as direct
proofs, proofs by contraposition, and proof by contradiction. (There
is obviously lots of fancy terminology flying around here---the
important thing is not that you memorize all of it, but that you
understand what we are doing.)


Note that mathematical logic is nothing magical---it is just a way to
formalize well-known logical concept, and to describe in a clear,
structured way different ways of reasoning that we already know to be valid. 
%   
%   Mathematical logic can be viewed as a formalization of well-known
%   logical concepts.  Note that
%   
Indeed, in many cases common sense is sufficient to construct a valid
mathematical argument. However, when mathematical statements become
increasingly complicated it can be helpful to have a formal toolbox to
keep track of the reasoning and to be sure that everything is correct.
(As an illustration of this, already the proof we did of
Theorem~4 on page~22 of KBR 
during the second week
was not entirely trivial.)
%
%   it may also be necessary to use a formal toolbox to
%   evaluate the truth values of such statements.  
%

We will also introduce some pieces of mathematical notation that
allow us to make very precise statements in a brief, efficient way.
For instance, we can describe that an
%   
%   We will, e.g., introduce a notation suitable for describing what it
%   means that an
%   
asymptotically positive sequence $(a_n)$ is $O(1)$ by writing
\[
\exists c>0\; \exists k>0\; \forall n\geq k \;\; a_n \leq c
\ .
\]
In plain English, this  can be read as
\emph{``there exist 
$c$ greater than~$0$ 
and
$k$ greater than~$0$ 
such that for all~$n$ greater
than or equal to~$k$
it holds that $a_n$ is less than or equal to~$c$.''}
%such that if we want to know what it means that $(a_n)$ is \textbf{not} $O(1)$, we can with little effort describe this as 
%\[
%\forall c>0\forall k>0\exists n\geq k: |a_n|> c
%\]
%The same strategy is used when programming computers (which does not have sense of logic or common sense)  to process statements.
This is still slightly informal, in that we assume that the reader
understands from context that $c$ is any positive real number whereas
$k$ and $n$ are integers. If we wanted to be even more precise, then
we could write
\[
\exists c \in \RR^+\; \exists k \in \ZZ^+ \; 
  \forall n \in \ZZ^+ (n\geq k \Rightarrow a_n \leq c)
\ .
\]
However, this second, more detailed version adds a lot of clutter, and
sometimes it is preferable to be slightly more concise when the extra
details are understood from context.
There are no hard and fast rules here, but we will try during the
lectures to learn by example what a suitable level of detail is. Also,
when in doubt, it is never wrong to add extra details for clarity.
At the end of the day, the reason we use mathematical notation is in
order to communicate as clearly as possible.  

As most other material in this course, our discussion of logic and
proofs will involve fundamental topics that will reappear in later
courses, and so you will do yourself great service by making sure to
learn it inside out already now.

The course will cover KBR Sections 2.1--2.4 in detail. 
\textbf{Please make sure to read these sections before the lecture}, 
since much of it is better suited for self-study than for lecturing. 
Once we have covered all of it,
then read all of it again and check that you have digested
the material.
Section~2.5 is not as important, and Section~2.6 we will not touch upon at
all, but this is still useful reading.
%   
% Furthermore, we will (as an advanced example with computer science
% flavor) go through the argument about the runtime of Euklid's
% algorithm, which is found on pages 935--936 in CLRS.
%   

%   \section*{Work instructions}

\section*{Reading Instructions}
%   \section*{Assigned reading}
\begin{itemize}
\item KBR Sections~2.1--2.4 in depth.
%    Read this carefully. We will use (parts of) 2.1--2.2 on Monday and
%    cover the rest on Tuesday and Friday. 
\item KBR Section~2.5 is part of the course, but is not as important
  as  Sections~2.1--2.4 and we will probably not be able to cover it
  in class.

\item KBR Section~2.6 is not required to pass the course, but is
  recommended reading.

%   
%   \item KBR 2.4-2.6. 
%   It is enough to skim these sections. 
%   
\end{itemize}

%   
%   \section*{Lecture plan}
%   
%   \subsection*{Monday Oct.~21st 09:15-10:00}
%   Propositions, logical operations/connectives ($\wedge,\vee,\sim,\Rightarrow,\Leftrightarrow$), truth tables, tautologies, and absurdities.
%   
%   \subsection*{Tuesday Oct.~22nd, 13:15-14:00}
%   \begin{itemize}
%   \item Predicates, quantifiers ($\forall$ and $\exists$), negation of statements with quantifiers.
%   \item Proof methods.
%   \end{itemize}
%   
%   \subsection*{Friday Oct.~25th, 09:15-10:00}
%   Proof methods. Examples.
%   
%   

\section*{Exercises}

Note that as for previous weeks, there are quite a few exercises
suggested below.  One question that tends to come up is how many of
these exercises you are supposed to do. The answer is that this
depends on how many exercises you yourself need to solve in order to
digest the material.

If you feel that you understood what was covered in class and/or what
you read in the textbook, and if the exercises seem straightforward,
then it is fine to just do some of them (where I good idea would be to
try to focus on the one that seem hardest to you). If some of the
material seems harder, though, then it is a good idea to do more
exercises until you really understand what is going on.

At the obvious risk of repeating myself, you will do yourself a great service 
by learning the material on this course well, because much (or most?) of what
you will do in later courses will build on this material in one way or 
another.




\begin{enumerate}

\item More exercises on induction 2.4.10, 2.4.17, 2.4.22.

\item 
  If you wish to check your basic understanding,
  solve KBR exercises 2.1.1, 2.1.2, 2.1.8.

\item 
  If you wish to check your basic understanding,
  solve  KBR exercises 2.1.15, 2.1.16, 2.1.18.

\item 
  To check your basic understanding,
  solve KBR exercises 
  2.1.27, 2.1.28.

\item Check De-Morgan's laws by computing and comparing the truth tables of the left-hand-side and right-hand-side in each of the following
\begin{enumerate}
\item $\sim(p \vee q) \equiv (\sim p) \wedge (\sim q)$
\item $\sim(p \wedge q) \equiv (\sim p) \vee (\sim q)$
\end{enumerate}

%   \item Solve KBR exercises 2.1.37, 2.1.38.

\item
  To check your basic understanding,
 solve KBR exercises 2.2.10, 2.2.11.

\item 
  To get slightly more interesting challenges, 
  solve KBR exercises 2.2.13, 2.2.15.

 \item Let \texttt{xor}
   (also known as \emph{exclusive or})
   be the logical connective with the following truth table:\vspace{-.05in}
\begin{center}
\begin{tabular}{cc|c}
$P$ & $Q$ & $P$ \texttt{xor} $Q$\\
\hline
T & T & F \\
T & F & T \\
F & T & T \\
F & F & F 
\end{tabular}\vspace{-.05in}
\end{center}
Find an equivalent expression for 
\[P\; \texttt{xor} \;Q\] 
using only $\wedge$(and), $\vee$(or), and $\sim$(not). You can use $P$ and $Q$ any number of times and indicate the order of the operations using parentheses.
Verify your answers by computing the truth table of your expression. (See Example 5 in KBR 2.1 for an example.) 


\item  Let $f,g:\RR^+\to\RR$ be asymptotically positive functions. We
  can express the definition of ``$f(x)$ is $O(g(x))$''{} 
%   from Week 3 
using logical connectives and quantifiers as
\begin{equation}
  \exists c>0 \; \exists x_0\in \RR^+ \; \forall x\ge x_0 \;f(x) \le cg(x)
\end{equation} 
Recall 
%   from Week 3 
that we defined ``$f$ is $o(g)$''{} as ``for any constant $c>0$ we can find $x_0\in\RR^+$ such that
\[ f(x) <  c g(x)\]
for all $x\ge x_0$.''
\begin{enumerate}
\item Express the above definition of ``$f(x)$ is $o(g)$'' using logical connectives and quantifiers.
\item Write the negation of the proposition~from~the previous part and simplify it so that it does not contain the negation ($\sim$). \\
\textit{Hint: Theorem 3 from KBR 2.2 can be helpful here.}
\item Write a sentence in English that corresponds to your statement from the previous part.
\end{enumerate}

\item
  If you wish to solve some more exercises to test basic understanding,
 solve  KBR exercises 2.2.6, 2.2.21, 2.2.27.


\item 
  To develop your mastery of the proof techniques that we learned in  class,
  make sure to solve (and understand) a generous selection of 
  KBR exercises  %2.3.7, 2.3.8, 2.3.10
  2.3.18, 2.3.23, 2.3.24, 2.3.27, 2.3.31, 2.3.34.
%2.4.34, 2.4.35

%%%
%%% Similar to what is found in KBR
%%%
%   \item Let $k$ be a positive integer. Prove the following statement by
%     proving its contrapositive: If $k^3$ is odd then $k$ is odd. Make
%     sure to start by explicitly stating the contrapositive.
%   
%   
%   \item Solve any potentially leftover exercises from the previous
%     exercise classes. 
%   ;-)
%   
\end{enumerate}



\subsection*{A Couple of Extra Challenging Exercises}
 \begin{enumerate}[(1)]

 \item{[*]} 
   Recall that we have discussed in class the
   \emph{least number principle}, which can be formally stated as
   follows.

\begin{framed}
  If $S$ is a non-empty set of non-negative integers, then
  there is a minimum element
  $x \in S$ (i.e., such that for all $y\in S$ it holds that $y \geq x$).
\end{framed}


\begin{itemize}
\item 
  Note that it is essential that we are talking about
  (a) integers that are (b) non-negative.
  Give examples showing that the least-number principle does not hold if
  if $S$ is any non-empty set of integers, or if $S$ is any set of 
  non-negative rational numbers.

\item 
  Prove that  the least number principle implies the induction principle. 
  That is, we can prove a theorem saying that if the least number
  principle is true, then the induction principle (repeated below for
  your convenience) is also true.
  
  \begin{framed}
    If a collection of statements $P(n)$, where $n$ takes the values from 
    a set of non-negative integers
    $\{n_0,n_0+1,n_0+2,\dots\}$, 
    satisfy that 
    \[
      P(n_0)\text{ is true}
  %   \tag{S}
    \]
    and that for all $n\geq n_0$, we have that 
    \[
       \text{If }P(n)\text{ is true, then }P(n+1)\text{ is true}
  %   \tag{T}
     \]
     then $P(n)$ is true for all $n\geq n_0$. 
   \end{framed}
   
   \emph{Hint:} 
   Let $S$ be the set of all~$n$ for which $P(n)$ fails to hold.
   
 \item 
   Prove that the induction principle implies the least number
   principle.
   
   \emph{Hint:} Let $S$ be a set of non-negative integers without a
   minimum element. 
   Let $P(n)$ be the property that $m \notin S$ for all $m \leq n$
   and use  induction to prove that the set $S$ is empty.
%   
%     Let $P(n)$ be the property that $n \notin S$ and use strong
%     induction to prove that the set $S$ is empty.
%   

\end{itemize}

 \item{[**]} 
   Find the mistake in the proof given in KBR Example 2.4.7
   (page 73).
   \emph{[This is a tricky one, since what is claimed in the example is
   definitely true, but the proof is taking a shortcut.]}
\end{enumerate}

\end{document}


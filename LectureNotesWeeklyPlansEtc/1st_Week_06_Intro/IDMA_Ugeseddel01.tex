\documentclass[12pt,a4paper]{article}

% Packages
%   \usepackage[english,danish]{babel}
\usepackage[english]{babel}
\usepackage[utf8]{inputenc}
\usepackage{amsmath,amscd}
\usepackage{amssymb}
\usepackage{amsthm}
\usepackage{enumerate}
\usepackage{graphicx}                    
\usepackage{framed}  
%   \usepackage{multicols}
   
\theoremstyle{plain}
\newtheorem{thm}{S¾tning}
\newtheorem{prop}[thm]{Proposition}
\newtheorem{lem}[thm]{Lemma}
\newtheorem{cor}[thm]{Korollar}
\newtheorem{conj}[thm]{Formodning}
\theoremstyle{definition}
\newtheorem{exercise}{Opgave}
\newtheorem{definition}[thm]{Definition}
\newtheorem{prob}[thm]{Problem}
\newtheorem{remark}[thm]{Bem¾rkning}
\newtheorem{example}[thm]{Eksempel}


% Blackboard bold
\newcommand{\NN}{\mathbb{N}}
\newcommand{\ZZ}{\mathbb{Z}}
\newcommand{\QQ}{\mathbb{Q}}
\newcommand{\RR}{\mathbb{R}}
\newcommand{\CC}{\mathbb{C}}
\newcommand{\GCD}{\operatorname{GCD}}


% Course name

% Intelligent spacing
\usepackage{xspace}

% IDMA course name
\newcommand{\coursename}{Introduction to Discrete Mathematics and Algorithms\xspace}
\newcommand{\coursenameshort}{IDMA 2026\xspace}




% Dates
\newcommand{\mondaylecturedate}{Monday Feb 3\xspace}
\newcommand{\wednesdaylecturedate}{Wednesday Feb 5\xspace}
\newcommand{\deadlinefirstpset}{Wednesday Feb 12 at 12:59 CET\xspace}


\begin{document}
\begin{center}
  \textbf{\large \coursenameshort
    \\[0.5em] -- Ugeseddel 1 -- \\[0.5em]}
\end{center}
\noindent

\section*{Overview}

Welcome to the
\emph{\coursename{}}
course!
This first week we will talk about what discrete math and algorithms
and then get started on algorithms for \textbf{sorting} and
\textbf{searching} in data.
We will also discuss some simple data structures such as 
\textbf{stacks},
\textbf{queues}, and
\textbf{linked lists}.
Later in the course we will often assume that we have access to
efficient implementations of these kind of objects, but it is good to
understand how they can be constructed from first principles.
All of this  will also give us an opportunity to discuss what it means for an
algorithm to be \emph{efficient}, and how we can compare the
efficiency of different algorithms using
\textbf{asymptotic analysis}.   

{\it
  Please note that the first problem set to be handed in
%     has already been posted and  is due on
  will be posted in the coming days and will be due on
  \textbf{\deadlinefirstpset{}}.
  Please make sure to familiarize yourself with the problem set rules
  posted on Absalon.
  In particular, 
  you are welcome (and encouraged) to discuss the problems on the problem sets in
  groups of two or three persons, but your solutions should be written
  up individually, and you are not allowed to share any
  text, formulas, or code with others (or to copy such material from
  the internet or generate it via chatbots) when writing up your solution.
  \textbf{Passing all problem sets is a requirement to be eligible for the
    exam, so start working on the problems as soon as you can!}
}

\section*{Reading Material}

Cormen, Leiserson, Rivest, and Stein (CLRS): Chapters 1--3 and 10.

Please note that --- as a general rule --- 
\emph{the lectures will not cover everything in the textbooks},
so just attending the lecture without reading the assigned material in
the textbook is not sufficient.
%   
%   However, not everything in the textbook is equally important, and
%   during the lectures we will try to discuss how to prioritize.
%   If in doubt, please do not hesitate to ask. 
%   

\section*{Plan for Lectures}

\subsection*{\mondaylecturedate
  at 13:15-15:00}

\begin{itemize}
\item 
  Introduction
\item 
  RAM model (how do we think about what a computer is)
\item 
  Pseudocode
%   \item 
%     Searching
%   \item
%     Sorting
\end{itemize}


\subsection*{\wednesdaylecturedate
  at 10:15-12:00
  and 13:15-15:00
}

\begin{itemize}
\item 
  Searching
\item
  Sorting
\item
  Time complexity of algorithms
\item
  Stacks, queues, and lists
\end{itemize}


\section*{Software for Calculations and Visualisation}

The math part of the course will actually not contain that many
numerical calculations (just as the the algorithms part will not contain
that much actual coding).
Our main focus on this course is to make sure that you get a solid
\emph{theoretical, mathematical} understanding (after which
%   programming
coding things up
will become a breeze).
We will, however, occasionally 
%   ask 
encourage
you to calculate some function values
or draw a graph to visualize a function. For this purpose, you can use
whatever tools you are most comfortable with. It does not matter
whether this is an app on your phone or math software on your
computer.



\section*{Exercises}

\emph{Please note that the exercises below are
  \textbf{just suggestions for you.}
  At the end of the day, you have to read the material and work on it to
  make sure that you understand what is going on. This might involve
  solving fewer exercises than what is listed below, or might require
  solving more exercises. Only you can tell when you feel that you have
  mastered the material.}

You can of course also work on the problem set problems from previous years
that are posted on Absalon, and are highly encouraged to do so.
But it is probably a good idea to first do some of the exercises below
to make sure that you have understood the material.


Exercises that might be  slightly tricky are marked by stars
[*].

\subsection*{Exercises on Algorithm Analysis, Searching,
  and Sorting}

\begin{enumerate}
%   
%   \item 
%     Start on the problems in problem set 1 right away!
%     You can expect to get that type of problems on the exam.
%   

\item In the following snippet of code \verb+A+ and \verb+B+ are
  arrays indexed from $1$ to~$n$ and containing numbers.
%   B[i] = 1 / i \cdot \sum_{j=1}^{i} A[i]
\begin{verbatim}
for (i := 1 upto n)
    B[i] := A[1]
    for (j := 2 upto i)
        B[i] := B[i] + A[j]
    B[i] := B[i] / i
\end{verbatim}
  
\begin{enumerate}
\item 
  Explain in plain language what the algorithm above does.
  In particular, what are the values computed and stored in the 
  array~\verb+B+?

\item
  How does the running time of the function scale
  as a function of the array size~$n$?
  (Focus on the highest-order term and ignore constant factors.)

\item
    Can you improve the algorithm (i.e., change the pseudocode)
    to run asymptotically faster while
    retaining the same functionality?  
%   If so, how much faster can you get the algorithm to run?
%
    In case you can design an asymptotically faster algorithm,
    analyse the time complexity of your new algorithm. 
%
    For the best algorithm that you have---either your new one, or the
    old one---can you prove that the running time of this algorithm is
    asymptotically optimal?
\end{enumerate}

\item{} [*] 
  In the following snippet of code \verb+A+ is an array
  indexed from $1$ to~$n$
  containing elements that can be compared using operators~$>$, 
  $=$,  and~$>$.
\begin{verbatim}
flip1 := FALSE
flip2 := FALSE
for (i := 1 upto n-1)
    if (A[i] > A[i+1] and not(flip1))
        flip1 := TRUE
    else if (A[i] < A[i+1] and flip1)
        flip2 := TRUE
if (flip1 and not(flip2))
    return "success"
else
    return "failure"
\end{verbatim}
  
\begin{enumerate}
\item 
  Explain in plain language what the algorithm above does.
  (In particular, what does it mean that it achieves ``success''?)

\item
  How does the running time of the function scale
  as a function of the array size~$n$?
  (Focus on the highest-order term and ignore constant factors.)

\item
  Can you modify the algorithm so that it ``succeeds'' when 
  it encounters an odd number of ``flips''?
  What does the pseudo-code look like?
  Does the time complexity change? If so, how?

\end{enumerate}

\item
  Suppose we have the array
  $
  [
  1, 3, 4, 5, 8, 10, 11, 14, 15, 18, 20
  ]
  $.

  \begin{enumerate}

  \item
    Illustrate how the linear search algorithm searches for element
    $16$ in this array and for element $18$.
    
    
  \item
    Perform the same searches for elements
    $16$ and  $18$
    but using binary search.

%   
%     \item
%       Do the same exercise but with some small array that your study
%       buddy gives to you in exchange for an array you give to your study
%       buddy. 
%       
  \end{enumerate}
  
\item
  Suppose we have the array elements as above, but permuted to
  yield the array
  $
  [
  1, 3, 
  20,
  10,   8, 
  18,   15, 14,   11, 
  4, 5
  ]
  $.

  \begin{enumerate}

  \item
    Run the selection sort algorithm on
    this array and illustrate (iteration by iteration)
    how the array gets sorted.
    
  \item
    Run merge sort on the same array. Illustrate the recursive calls
    and how they are merged to form a sorted array.

  \item{} [*]
    Which algorithm of selection sort or merge sort is fastest on this
    particular array?
    
  \end{enumerate}

\item
  CLRS 2.1-1 

\item
  CLRS  2.2-2

\item
  CLRS
  2.3-1 
%   and   2.3-2

%   
%   \item
%     Can you modify the code for merge sort so that it
%     stops without making a recursive call when the size of the list to
%     sort is $\leq 2$?
%   

\item
  {}[*] CLRS 2-2 
  
\item
  CLRS 3.2-1

\end{enumerate}

\subsection*{More Exercises on Asymptotic Analysis}

Unless explicitly stated otherwise, you are free to use theorems from
the notes on Absalon without a proof, as well as make use of any
previously solved exercises.

\begin{enumerate}[(1)]

  \item Two teams of computer science students compete in writing the most efficient algorithm to solve a task on an array with $N$ entries. Team~1 has constructed an algorithm that runs in
\[
f_1(N)=10\cdot N^{10}
\]
steps. Team 2 has constructed an algorithm that runs in
\[
f_2(N)=2^N
\]
steps. One of the following statements must hold: 
\begin{enumerate}[(A)]
\item Team 1's algorithm runs with fewer steps for all $N$.
\item Team 2's algorithm runs with fewer steps for all $N$.
\item Team 1's algorithm runs with fewer steps for some choices of $N$, while Team 2's algorithm runs with fewer steps for some other choices of $N$. 
\end{enumerate}
Draw the graphs of the functions. Which of the statements (A), (B), and (C) are true? 


  \item Use the definition of big-$O$ to show that $x^3+12x-3$ is $O(x^3)$.

  \item Use the definition of big-$O$ to show that $x^3+12x-3$ is \emph{not} $O(x^2)$.

  \item Use the rules  (\emph{B1})--(\emph{B6}),
    (\emph{L1})--(\emph{L6}), (\emph{M1})--(\emph{M4})
    from the notes to determine which of the functions 
    \[
      x^2+2^x, (x+2)^2, (x+\log_{10}x)^2,x^3-x, x^2+\log_2 x
    \]
    are $\Theta(x^2)$. Keep in mind that you also need to provide a
    justification whenever you claim that $f(x)$ is \emph{not}
    $\Theta(x^2)$. 
     
  \item Sort the sequences
    \begin{eqnarray*}
      a_n&=&n2^n+n^7\\
      b_n&=&3^n+n^5\\
      c_n&=&\sum_{k=1}^n 2^k\\
      d_n&=&\log_2n+n^7
    \end{eqnarray*}
    in increasing asymptotic order using only the results from this week's notes. 
    
\end{enumerate}

\subsection*{Exercises on Stacks, Queues, and Lists}

See the separate document with exercises on stacks, queues, and lists
posted on Absalon.




\subsection*{Some Additional Exercises (Possibly for Later)}

Here are some extra exercises that might be useful if
you want to repeat this material when preparing for the exam.
%
\begin{enumerate}

\item
  CLRS 1-1 (a bit tedious, but useful)

\item
  CLRS 2.1-2

\item
  {}[*] CLRS  2-3
  
\item
  CLRS
  3.1-2
  
\item
  CLRS   3.2-2



\end{enumerate}


\end{document}

\documentclass[12pt,a4paper]{article}

% Packages
%   \usepackage[english,danish]{babel}
\usepackage[english]{babel}
\usepackage[utf8]{inputenc}
\usepackage{amsmath,amscd}
\usepackage{amssymb}
\usepackage{amsthm}
\usepackage{enumerate}
\usepackage{graphicx}                    
\usepackage{framed}  
%\usepackage{multicols}

\theoremstyle{plain}
\newtheorem{thm}{S¾tning}
\newtheorem{prop}[thm]{Proposition}
\newtheorem{lem}[thm]{Lemma}
\newtheorem{cor}[thm]{Korollar}
\newtheorem{conj}[thm]{Formodning}
\theoremstyle{definition}
\newtheorem{exercise}{Opgave}
\newtheorem{definition}[thm]{Definition}
\newtheorem{prob}[thm]{Problem}
\newtheorem{remark}[thm]{Bem¾rkning}
\newtheorem{example}[thm]{Eksempel}


% Blackboard bold
\newcommand{\NN}{\mathbb{N}}
\newcommand{\ZZ}{\mathbb{Z}}
\newcommand{\QQ}{\mathbb{Q}}
\newcommand{\RR}{\mathbb{R}}
\newcommand{\CC}{\mathbb{C}}
\newcommand{\GCD}{\operatorname{GCD}}

\begin{document}


\begin{center}
\textbf{\large DMFS 2022 \\[0.5em] -- Ugeseddel 5 -- \\[0.5em]}
\end{center}
\noindent

%   
%   \begin{center}
%   \textbf{\large 
%     DMFS 2021
%     \\[0.5em] 
%     -- 
%     Reading 
%   %     Instructions 
%     and Exercises
%     for 
%     the 
%     5th Week 
%     -- 
%     \\[0.5em]}
%   \end{center}
%   \noindent
%   

\section*{General Plan}
%   \section*{Arbejdsvejledning}

\emph{%
  I am taking the risk of publishing this ugeseddel mostly in
  Danish. Apologies in advance for any errors, which are the
  responsibility of your lecturer only\ldots 
}


Vi skal i denne uge og i uge 6 arbejde med grafalgoritmer. 
%   Før jul definerede vi grafer og forskellige egenskaber ved grafer
%   matematisk, og vi har
Vi har
tidligere i kurset set 
grafer og
eksempler
på brug af grafer i algoritmik. Som I måske har fået fået fornemmelse af kan
grafer ofte bruges til at modellere strukturer i konkrete data: vi har blandt
andet talt om facebook-grafer og veje mellem byer som 
nogle få eksempler på en meget lang
række anvendelser af grafer. Nerveforbindelser mellem hjernedele, design af
computerchips, datanetværk (som f.eks. internettet) og transportnetværk, afhængigheder i statistik,
og tildeling af variable til CPU-registre i oversættere er nogle yderligere eksempler. Grafer
er nogle af de mest benyttede modelleringsstrukturer, og grafalgoritmer er derfor
tilsvarende vigtige.

Vi vil specielt se på algoritmer til at finde forskellige strukturer i grafer,
vise deres køretid, og bevise korrekthed af algoritmerne.

Uge 5 fokuserer på repræsentation af grafer i computere og strategier for
søgning i grafer. Søgning kan løst beskrives som
at gå en tur i grafen hvor alle knuder besøges. Her har vi to overordnede
strategier: bredde-først søgning 
og dybde-først søgning.

%   
%   strategier: bredde-først søgning 
%   (breadth-first search, abbreviated BFS)
%   og dybde-først søgning
%   (depth-first search, abbreviated DFS).
%   

%   
%   {\it Kursusevaluering af DMA foregår i ugerne 5 og 6. Der er sat tid
%     af til at lave kursusevaluering til øvelserne torsdag uge 6. Det er
%     meget vigtigt for os og for de studerende på kurset næste år at I
%     udfylder kursusevaluering senest til fristen søndag d. 13/1.}
%   

\section*{Program for forelæsninger}

\subsection*{Mandag
%     080321, 
  13:15-15:00}
Repræsentation af grafer (CLRS~22.1);
bredde-først søgning (CLRS~22.2)
og
dybde-først søgning  (CLRS~22.3);
%   Fortsættelse af
%   bredde-først søgning og
%   dybde-først søgning;
topologisk sortering og sammenhængskomponenter
(CLRS~22.3 indtil 'Properties of depth-first search' 
\mbox{s. 606}, 22.4, 22.5).

\subsection*{Onsdag 
%     100321, 
  13:15-15:00}
%   Intro til m
Mindste udspændende træer
%   og Kruskals algoritme (CLRS~23 \mbox{s.~624-636})
og Prims og Kruskals algoritmer (CLRS~23).


\section*{Program for øvelser}


Løs opgaverne
\begin{itemize}
\item CLRS~22.1-1, 22.1-2, 22.1-3, 22.1-7, 22.2-1, 22.2-2, 22.2-3
\end{itemize}

Løs udvalgte af opgaverne
\begin{itemize}
\item CLRS~22.3-2, 22.3-1, 22.3-4, 22.3-9, 22.3-11, 22.2-4,
22.3-12, 22.4-1, 22.2-5, 22.2-7, 22.2-6
\end{itemize}


Løs opgaverne
\begin{itemize}
  \item håndkør Kruskals algoritme på en vægtet graf. Diskuter ved hver Union
    hvorfor kanten er 'safe'.
\item CLRS~23.1-1 23.1-3 23.1-4 23.1-6 23.1-9 23.2-2 23.2-5
\end{itemize}



%   \section*{Fordybelsesopgaver}
\subsection*{Fordybelsesopgaver}
 \begin{enumerate}[(1)]
  \item{[*]} CLRS~22.2-8 (diameter af træer)
 \end{enumerate}

\end{document}

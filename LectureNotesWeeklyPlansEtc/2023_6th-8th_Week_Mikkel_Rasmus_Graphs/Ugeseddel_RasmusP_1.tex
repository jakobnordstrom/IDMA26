\documentclass[11pt,a4paper]{article}

% Packages
%   \usepackage[english,danish]{babel}
\usepackage[english]{babel}
\usepackage[utf8]{inputenc}
\usepackage{amsmath,amscd}
\usepackage{amssymb}
\usepackage{amsthm}
\usepackage{enumerate}
\usepackage{graphicx}                    
\usepackage{framed}
\usepackage{hyperref}
\usepackage{fullpage}
%\usepackage{multicols}

\theoremstyle{plain}
\newtheorem{thm}{S¾tning}
\newtheorem{prop}[thm]{Proposition}
\newtheorem{lem}[thm]{Lemma}
\newtheorem{cor}[thm]{Korollar}
\newtheorem{conj}[thm]{Formodning}
\theoremstyle{definition}
\newtheorem{exercise}{Opgave}
\newtheorem{definition}[thm]{Definition}
\newtheorem{prob}[thm]{Problem}
\newtheorem{remark}[thm]{Bem¾rkning}
\newtheorem{example}[thm]{Eksempel}


% Blackboard bold
\newcommand{\NN}{\mathbb{N}}
\newcommand{\ZZ}{\mathbb{Z}}
\newcommand{\QQ}{\mathbb{Q}}
\newcommand{\RR}{\mathbb{R}}
\newcommand{\CC}{\mathbb{C}}
\newcommand{\GCD}{\operatorname{GCD}}
	
\title{DMFS 2023 Lectures on Graphs by Rasmus Pagh, Part 1}
%   \author{Rasmus Pagh}
\date{}

\begin{document}
	\maketitle


\section*{Litteratur}

\begin{itemize}
\item[-] \emph{\href{https://absalon.ku.dk/courses/60107/files/6656269}{Noter om grafer}} 
%   (alt indhold bortset fra dybde-først søgning dækkes i denne uge)
\item[-] CLRS 20.1, 20.2 (dækker emner fra noten, men med flere detaljer)
\item[-] CLRS 22.0, 22.3 (dækker emner fra noten, men med flere detaljer)
\end{itemize}


\section*{Mål 
%   for ugen
}


\begin{itemize}
\item[-] Kendskab til terminologi for grafer
\item[-] Kendskab til repræsentationer af grafer (tætte og tynde)
\item[-] Forståelse af bredde-først søgning og Dijkstra's algoritme
\end{itemize}



\section*{Plan
%    for ugen
}

\begin{itemize}
\item[-]
%   Mandag: 
Repræsentation af grafer (Noter om grafer / CLRS~20.1), bredde-først søgning (Noter om grafer / CLRS~20.2)

\item[-]
%   Tirsdag: 
Korteste vej, Dijkstra's algoritme (Noter om grafer / CLRS 22.0, 22.3)

%   \item[-]
%   Fredag: Ingen forelæsning, juleferie!

\end{itemize}



\section*{Opgaver}

For opgaver markeret med $\dagger$ kan du finde svar i \href{https://mitp-content-server.mit.edu/books/content/sectbyfn/books_pres_0/11599/selected-solutions.pdf}{CLRS løsningsmanual}, men det er vigtigt at du forsøger at løse opgaven, evt.~med hjælp fra din instruktor, inden du kigger dér.

%   \subsection*{Mandag}

\begin{itemize}
\item CLRS~20.1-1, 20.1-2, 20.2-1, 20.2-2, 20.2-4, 22.2-5$\dagger$, 20.1-7*$\dagger$

%   
%   \end{itemize}
%   
%   \subsection*{Tirsdag}
%   
%   
%   \begin{itemize}
%   

\item CLRS 22.3-1, 22.3-3$\dagger$ 
\item Løs CLRS 20.2-1 igen, men erstat BFS med Dijkstra's algoritme hvor alle vægte er~1.
\item CLRS 22.3-8, 22.3-2*, 22.3-11*
\end{itemize}

%   
%   \subsection*{Fredag}
%   \begin{itemize}
%   \item Find den korteste vej til julehyggen!
%   \end{itemize}
%   
\end{document}


























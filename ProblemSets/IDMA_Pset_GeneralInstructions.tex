

\begin{abstract}
  \noindent
  \textbf{Due:} \duedate.

  \noindent
  \textbf{Submission:}
  Please submit your solutions
  via \emph{Absalon}
  as a PDF file.
  State your name and e-mail address 
  close to the top   of the first page.
  Solutions should be written in \LaTeX{} or some other math-aware
  typesetting system with reasonable margins on all sides (at least 2.5~cm).
  Please try to be precise and to the point in your solutions and
  refrain from vague statements.
  % Make sure to explain your reasoning.
  Never, ever just state the answer, but always
  make sure to explain your reasoning.
  \emph{Write so that a fellow student of yours can read, understand, and
    verify your solutions.}
  In addition to what is stated below, the general rules 
  for problem sets stated on \emph{Absalon} always apply.

%    
%      \noindent
%      \textbf{Hints:}
%      For most or all problems, ``hints'' can be purchased at a cost of 
%      \mbox{5--10~points}. In this way, you can configure yourself whether
%      you want the problems to be more creative and open-ended, where
%      sometimes a lot can depend on finding the right idea, or whether you
%      want them to be more of guided exercises providing a useful work-out
%      on the concepts of proof complexity. If you do not solve a problem,
%      there is no charge for the hint (i.e., it is not deducted from the
%      score on other problems).  
%    

  \noindent
  \textbf{Collaboration:}
  Discussions of ideas in groups of 
  two to three people 
  are allowed---and indeed, encouraged---but 
  you should always write up your solutions completely on your own,
  from start to finish, and you should understand all aspects of them
  fully. It is not allowed to compose draft solutions together and
  then continue editing individually, or to share any text, formulas,
  or pseudocode. Also, no such material may be downloaded from or
  generated via the internet to be used in draft or final solutions.
  Submitted solutions will be checked for plagiarism.

% 
%     You should also clearly acknowledge any collaboration.
%     State   close to the top 
%     of the first page of your problem set
%     solutions if you have been collaborating with someone and if so with
%     whom.  
%     \emph{Note that collaboration is on a per problem set basis,
%       so you should not discuss different problems on the same problem
%       set with different people.}
%    
%    
%   
%     \noindent
%     \textbf{Reference material:} 
%     Some of the problems are ``classic'' and hence it might be easy to
%     find solutions on the Internet, in textbooks or in research
%     papers. It is not allowed to use such material in any way unless
%     explicitly stated otherwise. Anything said during the lectures or in
%     the lecture notes 
%   %      , or which can be found in chapters of Arora-Barak covered in
%   %      the course,  
%     should be fair game, though, unless you are specifically asked to
%     show something that we claimed without proof in class. All
%     definitions 
%   %      used 
%     should be as given in class
%     or in Arora-Barak 
%     and cannot be substituted by 
%   %      definitions
%     versions
%     from other sources.  It is
%     hard to pin down 100\% watertight formal rules on what all of this
%     means---when in doubt, ask the main instructor.
%     

  \noindent
  \textbf{Grading:}
  A score of 
  \thresholdforpass
  is guaranteed to be enough to pass this problem set.


  \noindent
  \textbf{Questions:}
  Please do not hesitate to ask the instructor or TAs if any problem
  statement is unclear, but please make sure to send private
  messages---sometimes specific enough questions could give away the
  solution to your fellow students, and we want all of you to benefit
  from working on, and learning from, the problems.
%   
  Good luck!
\end{abstract}



\documentclass{jn-pset}
%   \documentclass[solutions]{jn-pset}

\usepackage{ifthen}
\newboolean{versionwithsolutions}
% Uncomment to insert text that should only be there in version WITH solutions
\setboolean{versionwithsolutions}{true}
% Uncomment to insert text that should only be there in version WITHOUT solutions
\setboolean{versionwithsolutions}{false}


%%%%
%%%% QUICK INSTRUCTIONS FOR FORMATTING OF PROBLEMS
%%%%
%    
% Code a problem by
%    \begin{problem}
%    \end{problem}
%
% Code a subproblem inside a problem by (note the percent signs!)
%    \begin{subproblem}%
%        \label{problem:labelhere}%
%        Text here
%    \end{subproblem}
%
% Get a small vertical space by issuing the command
%    \smallskip
%
% To give a hint to a problem (after having completed the problem statement),
% use the following LaTeX code for formatting consistency:
%
%    \smallskip
%    \noindent
%    \emph{Hint:}
%    Consider the following super-useful hint for this particular problem...
%

% PROBLEM-SET-SPECIFIC MACROS (UPDATE FOR EACH PROBLEM SET)

\newcommand{\psetno}{1}
\newcommand{\duedate}{Wednesday February 12 at 12:59 CET}

\newcommand{\thresholdforpass}{$120$~points\xspace}

% COURSE-SPECIFIC MACROS FOR IDMA 2025

\newcommand{\coursenameabbrev}{IDMA}
\newcommand{\coursenameshort}{Introduktion til diskret matematik og algoritmer}
\newcommand{\coursenamelong}{NDAB23002U Introduktion til diskret matematik og algoritmer}
%    
\newcommand{\submissionemail}{jn@di.ku.dk\xspace}
%   \newcommand{\courseinstructor}{Jakob Nordstr\"om}
\newcommand{\courseinstructor}
        {Jakob Nordstr\"om and Srikanth Srinivasan}
\newcommand{\courseperiod}{2024/2025}

% PACKAGES, MACROS, ET CETERA

\usepackage[T1]{fontenc}
\usepackage[utf8]{inputenc}
%%% Apparently a newer version of babel doesn't play well with nada-ten
%    \usepackage[english]{babel}

\usepackage{hyperref}

\usepackage{amsmath}
\usepackage{amssymb}
\usepackage{amsfonts}
\usepackage{mathtools}

% Provide calligraphic \mathscr font
\usepackage{mathrsfs}

% Enable use of MetaPost generated PostScript files
\usepackage{ifpdf}
\usepackage{graphicx}  
\ifpdf         
\DeclareGraphicsRule{*}{mps}{*}{}
\fi            

% For getting subfigures 1(a), 1(b) etc
% The package "subfigure" is obsolete, so switch to subcaption
%    \usepackage[sf,SF]{subfigure}
\usepackage{subcaption} 

% Sam Buss's package for formatting proofs
\usepackage{bussproofs}

% Extensions to verbatim commands
\usepackage{verbatim}

% Smiley
\usepackage{wasysym}

% To choose how to enumerate lists
\usepackage{enumerate}

%%%%
%%%% THIS FILE IS INTENDED TO BE READ-ONLY --- PLEASE DO NOT EDIT.
%%%% PLEASE CONTACT JAKOB NORDSTRÖM AT jn@di.ku.dk REGARDING ANY ISSUES.
%%%%

%
% DETECTION OF DOCUMENT TYPE
%----------------------------
%
% Version date: September 24, 2022
%
% First versions by Jakob Nordström <jn@di.ku.dk>
% Cleaned up version by Marc Vinyals <vinyals@kth.se>
% Minor later additions by Jakob Nordström and Susanna F. de Rezende
%
% This file needs to be included or input(ted) so that the conditional
% macro definitions in other LaTeX files will not generate compilation
% errors. Most document classes are detected using \@ifclassloaded.
%
% Use the file 'testdoctypedetection.tex' to double-check that these
% Boolean detectors work.
%
%
%%% The report class is not detected correctly in view of later
%%% updates, but this should be easy to fix when needed. 
%%% [Jakob Nordström, May 16, 2016]
% detectedReport is set to true if none of detectedArticle, detectedThesis,
% detectedSTOC, detectedFOCS, detectedSIAM, detectedIEEE, detectedICS,
% or detectedPoster is true.
%


\usepackage{ifthen}

\provideboolean{detectedSTOC}
\provideboolean{detectedFOCS}
\provideboolean{detectedElsevier}
\provideboolean{detectedNOW}
\provideboolean{detectedLMCS}
\provideboolean{detectedIEEE}
\provideboolean{detectedPoster}
\provideboolean{detectedSIAM}
\provideboolean{detectedLNCS}
\provideboolean{detectedACM}
\provideboolean{detectedACMconf}
\provideboolean{detectedSigplanconf}
\provideboolean{detectedToC}
\provideboolean{detectedLIPIcs}
\provideboolean{detectedAAAI}
\provideboolean{detectedIJCAI}
\provideboolean{detectedCompCplx}
\provideboolean{detectedEasyChair}
\provideboolean{detectedJAIR}
\provideboolean{detectedArticle}
\provideboolean{detectedReport}
\provideboolean{detectedThesis}

\makeatletter

\@ifclassloaded{sig-alternate}
{\setboolean{detectedSTOC}{true}}
{\setboolean{detectedSTOC}{false}}

\@ifclassloaded{elsarticle}
{\setboolean{detectedElsevier}{true}}
{\setboolean{detectedElsevier}{false}}

\@ifclassloaded{now}
{\setboolean{detectedNOW}{true}}
{\setboolean{detectedNOW}{false}}

\@ifclassloaded{lmcs}
{\setboolean{detectedLMCS}{true}}
{\setboolean{detectedLMCS}{false}}

\@ifclassloaded{IEEEtran} {
  \setboolean{detectedIEEE}{true}
  \ifCLASSOPTIONconference {
    \setboolean{detectedFOCS}{true}
  }
  \else {
    \setboolean{detectedFOCS}{false}
  }
  \fi
}
{
  \setboolean{detectedFOCS}{false}
  \setboolean{detectedIEEE}{false}
}

%%% Obsolete SIAM class file
%   \@ifclassloaded{siamltex1213} 
%   {\setboolean{detectedSIAM}{true}}
%   {\setboolean{detectedSIAM}{false}}
\@ifclassloaded{siamart171218}
{\setboolean{detectedSIAM}{true}}
{\setboolean{detectedSIAM}{false}}

\@ifclassloaded{llncs}
{\setboolean{detectedLNCS}{true}}
{\setboolean{detectedLNCS}{false}}

\@ifclassloaded{acmsmall}
{\setboolean{detectedACM}{true}}
{\setboolean{detectedACM}{false}}

\@ifclassloaded{acmart}
{\setboolean{detectedACMconf}{true}
 \setboolean{detectedACM}{true}}
{\setboolean{detectedACMconf}{false}}

\@ifclassloaded{sigplanconf}
{\setboolean{detectedSigplanconf}{true}}
{\setboolean{detectedSigplanconf}{false}}

\@ifclassloaded{toc}
{\setboolean{detectedToC}{true}}
{\setboolean{detectedToC}{false}}

\@ifclassloaded{lipics}
{\setboolean{detectedLIPIcs}{true}}
{\@ifclassloaded{lipics-v2019}
  {\setboolean{detectedLIPIcs}{true}}
  {\@ifclassloaded{oasics-v2019}
    {\setboolean{detectedLIPIcs}{true}}
    {\setboolean{detectedLIPIcs}{false}}
  }
}

%   
%   \@ifclassloaded{lipics}
%   {\setboolean{detectedLIPIcs}{true}}
%   {\@ifclassloaded{lipics-v2016}
%     {\setboolean{detectedLIPIcs}{true}}
%     {\@ifclassloaded{oasics-v2016}
%       {\setboolean{detectedLIPIcs}{true}}
%       {\setboolean{detectedLIPIcs}{false}}
%     }
%   }
%   

\@ifclassloaded{cc}
{\setboolean{detectedCompCplx}{true}}
{\setboolean{detectedCompCplx}{false}}

\@ifclassloaded{easychair}
{\setboolean{detectedEasyChair}{true}}
{\setboolean{detectedEasyChair}{false}}

%%%
%%% For JAIR, detect that the "jair" package is being used
%%%
\@ifpackageloaded{jair}
{\setboolean{detectedJAIR}{true}}        
{\setboolean{detectedJAIR}{false}}        

%%%
%%% AAAI and IJCAI have special style files that needs to be detected.
%%% It seems they update the name with the year of the conference also.
%%%
\@ifpackageloaded{aaai}
{\setboolean{detectedAAAI}{true}}        
{\@ifpackageloaded{aaai18}
  {\setboolean{detectedAAAI}{true}}
  {\@ifpackageloaded{aaai20}       
    {\setboolean{detectedAAAI}{true}}
    {\setboolean{detectedAAAI}{false}}}}

\@ifpackageloaded{ijcai18}
{\setboolean{detectedIJCAI}{true}}        
{\@ifpackageloaded{ijcai19}
  {\setboolean{detectedIJCAI}{true}}        
  {\setboolean{detectedIJCAI}{false}}}

\@ifclassloaded{sciposter}
{\setboolean{detectedPoster}{true}}
{\setboolean{detectedPoster}{false}}

\@ifclassloaded{article}
{\setboolean{detectedArticle}{true}}
{\setboolean{detectedArticle}{false}}

\makeatother

\ifthenelse{\not \isundefined{\examen} 
  \and \not \isundefined{\disputationsdatum} 
  \and \not \isundefined{\disputationslokal}}   
  {\setboolean{detectedThesis}{true}}
  {\setboolean{detectedThesis}{false}}

%%%
%%% Not entirely sure whether detectedReport is set correctly in view
%%% of later updates [Jakob Nordström, May 16, 2016]
%%%
           
\ifthenelse{\boolean{detectedArticle} \or \boolean{detectedThesis}
  \or \boolean{detectedSTOC}    \or \boolean{detectedFOCS}
  \or \boolean{detectedSIAM}    \or \boolean{detectedIEEE}
  \or \boolean{detectedACMconf} \or \boolean{detectedACM}
  \or \boolean{detectedPoster}}
{\setboolean{detectedReport}{false}}
{\setboolean{detectedReport}{true}}



%%%%
%%%% THIS FILE IS INTENDED TO BE READ-ONLY --- PLEASE DO NOT EDIT.
%%%% PLEASE CONTACT JAKOB NORDSTRÖM AT jakobn@kth.se REGARDING ANY ISSUES.
%%%%

% GENERAL MACROS TO USE IN LaTeX-FILES
%======================================
%
% AUTHOR
%   Jakob Nordström <jakobn@kth.se>
%   Some improvements added by Marc Vinyals <vinyals@kth.se>
%
% VERSION
%   Last updated June 7, 2019
%    
% KNOWN ISSUES:
%   References with page numbers such as \refsecP, \refthP, etc will not
%   work with the Elsevier and SIAM document classes. No work-arounds
%   have been added, so compilation will fail if these macros are used
%   for Elsevier or SIAM articles.

%
% MACRO NAMING CONVENTION FOR MATHEMATICAL MACROS WITH DELIMITERS 
%-----------------------------------------------------------------
%
% For mathematical macros with delimiters there are usually
% three different flavours corresponding to different sizes of 
% the delimiters as follows:
%
%    \newcommand{\mycommand}[1]{<command> ( {#1} )}
%    \newcommand{\Mycommand}[1]{<command> \bigl( {#1} \bigr)}
%    \newcommand{\MYCOMMAND}[1]{<command> \left( {#1} \right)}
%

%
% REQUIRED PACKAGES
%-------------------
%

\usepackage{ifthen}
\usepackage{xspace}
% varioref does not seem to mix well with the Elsevier document class
\ifthenelse
{\boolean{detectedElsevier} \or \boolean{detectedSIAM} 
  \or \boolean{detectedLIPIcs}}
{}
{\usepackage{varioref}}


%
% M I S C E L L A N E O U S 
%---------------------------
%

\DeclareMathAlphabet{\mathsfsl}{OT1}{cmss}{m}{sl}


%
% G E N E R A L   F O R M A T T I N G   R U L E S 
%-------------------------------------------------
%
% To achieve some kind of consistency in the notation
%

% Format of functions to integers or real numbers
\newcommand{\formatfunctiontonumbers}[1]{\mathrm{#1}}

% Format of functions to sets
\newcommand{\formatfunctiontoset}[1]{\mathit{#1}}

% Dots in x_1 \lor ... \lor x_n and the like
% (make a generic macro that can be changed according to publisher
% requirements)  
\newcommand{\formuladots}{\cdots}


%
% B I G - O H   N O T A T I O N 
%-------------------------------
%

\newcommand{\BIGOH}[1]{\mathrm{O} \left( #1 \right)}
\newcommand{\Bigoh}[1]{\mathrm{O} \bigl( #1 \bigr)}
\newcommand{\bigoh}[1]{\mathrm{O} ( #1 )}
\newcommand{\LITTLEOH}[1]{\mathrm{o} \left( #1 \right)}
\newcommand{\Littleoh}[1]{\mathrm{o} \bigl( #1 \bigr)}
\newcommand{\littleoh}[1]{\mathrm{o} ( #1 )}
\newcommand{\BIGTHETA}[1]{\Theta \left( #1 \right)}
\newcommand{\Bigtheta}[1]{\Theta \bigl( #1 \bigr)}
\newcommand{\bigtheta}[1]{\Theta ( #1 )}
\newcommand{\BIGOMEGA}[1]{\Omega \left( #1 \right)}
\newcommand{\Bigomega}[1]{\Omega \bigl( #1 \bigr)}
\newcommand{\bigomega}[1]{\Omega ( #1 )}
\newcommand{\LITTLEOMEGA}[1]{\omega \left( #1 \right)}
\newcommand{\Littleomega}[1]{\omega \bigl( #1 \bigr)}
\newcommand{\littleomega}[1]{\omega ( #1 )}
\newcommand{\POLYBOUND}[1]{\mathrm{poly} \left( #1 \right)}
\newcommand{\Polybound}[1]{\mathrm{poly} \bigl( #1 \bigr)}
\newcommand{\polybound}[1]{\mathrm{poly} ( #1 )}
\newcommand{\POLYLOGBOUND}[1]{\mathrm{polylog} \left( #1 \right)}
\newcommand{\Polylogbound}[1]{\mathrm{polylog} \bigl( #1 \bigr)}
\newcommand{\polylogbound}[1]{\mathrm{polylog} ( #1 )}
            
\DeclareMathOperator{\polylog}{polylog}


%
% G E N E R A L  M A T H E M A T I C A L   N O T A T I O N
%----------------------------------------------------------
%

% N, Z, Q, R as symbols for classes of numbers
\ifthenelse{\boolean{detectedToC}}{}
{
  \newcommand{\Q}         {\mathbb{Q}}
  \newcommand{\R}         {\mathbb{R}}
  \newcommand{\Rplus}     {\mathbb{R}^{+}}
  \newcommand{\N}         {\mathbb{N}}
  \newcommand{\Nplus}     {\mathbb{N}^{+}}
  \newcommand{\Nzero}     {\mathbb{N}_{0}}
  \newcommand{\Z}         {\mathbb{Z}}
}

% Sigma sum sign with indices to the right, not below and above
\newcommand{\sumnodisplay}{{\textstyle \sum}}

% Absolute value and norm
\providecommand{\abs}[1]{\lvert#1\rvert}
\providecommand{\Abs}[1]{\bigl\lvert#1\bigr\rvert}
\providecommand{\ABS}[1]{\left\lvert#1\right\rvert}
\providecommand{\norm}[1]{\lVert#1\rVert}
\providecommand{\Norm}[1]{\bigl\lVert#1\bigr\rVert}
\providecommand{\NORM}[1]{\left\lVert#1\right\rVert}

% Exists unique
\newcommand{\existsunique}{\exists!}

% Rounding
\newcommand{\ceiling}[1]{\lceil #1 \rceil}
\newcommand{\Ceiling}[1]{\bigl \lceil #1 \bigr \rceil}
\newcommand{\CEILING}[1]{\left \lceil #1 \right \rceil}

\newcommand{\floor}[1]{\lfloor #1 \rfloor}
\newcommand{\Floor}[1]{\bigl \lfloor #1 \bigr \rfloor}
\newcommand{\FLOOR}[1]{\left \lfloor #1 \right \rfloor}

\newcommand{\intpart}[1]{\lceil #1 \rfloor}
\newcommand{\Intpart}[1]{\bigl \lceil #1 \bigr \rfloor}
\newcommand{\INTPART}[1]{\left \lceil #1 \right \rfloor}

% Max and min
% Don't use \maxof and \minof to avoid conflict with calc package
\newcommand{\MAXOFEXPR}[2][]{\max_{#1} \left\{ #2 \right\}}
\newcommand{\MINOFEXPR}[2][]{\min_{#1} \left\{ #2 \right\}}
\newcommand{\Maxofexpr}[2][]{\max_{#1} \bigl\{ #2 \bigr\}}
\newcommand{\Minofexpr}[2][]{\min_{#1} \bigl\{ #2 \bigr\}}
\newcommand{\maxofexpr}[2][]{\max_{#1} \{ #2 \}}
\newcommand{\minofexpr}[2][]{\min_{#1} \{ #2 \}}

\newcommand{\maxofset}[3][:]{\max \{ #2 #1 #3 \}}
\newcommand{\minofset}[3][:]{\min \{ #2 #1 #3 \}}
 
\newcommand{\MAXOFSET}[3][:]%
     {\ifthenelse{\equal{#1}{;}}%
     {\MAXOFEXPR{ #2 \,;\, #3 }}
     {\ifthenelse{\equal{#1}{:}}%
     {\MAXOFEXPR{ #2 \,:\, #3 }}
     {\max \twincommandJN{\left\{}{#2}{\left#1}{\right}{\,#3}{\right\}}}}}
\newcommand{\MINOFSET}[3][:]%
     {\ifthenelse{\equal{#1}{;}}%
     {\MINOFEXPR{ #2 \,;\, #3 }}
     {\ifthenelse{\equal{#1}{:}}%
     {\MINOFEXPR{ #2 \,:\, #3 }}
     {\min \twincommandJN{\left\{}{#2}{\left#1}{\right}{\,#3}{\right\}}}}}

\newcommand{\Maxofset}[3][:]%
     {\ifthenelse{\equal{#1}{;}}%
     {\Maxofexpr{ #2 \,;\, #3 }}
     {\ifthenelse{\equal{#1}{:}}%
     {\Maxofexpr{ #2 \,:\, #3 }}
     {\max \twincommandJN{\bigl\{}{#2}{\bigl#1}{\bigr}{\,#3}{\bigr\}}}}}
\newcommand{\Minofset}[3][:]%
     {\ifthenelse{\equal{#1}{;}}%
     {\Minofexpr{ #2 \,;\, #3 }}
     {\ifthenelse{\equal{#1}{:}}%
     {\Minofexpr{ #2 \,:\, #3 }}
     {\min \twincommandJN{\bigl\{}{#2}{\bigl#1}{\bigr}{\,#3}{\bigr\}}}}}


%
% A L G E B R A
%---------------
%

% Some linear algebra
\newcommand{\transpose}[1]{\ensuremath{#1^{\top}}}
\newcommand{\innerproduct}[2]{\langle #1, #2 \rangle}
\newcommand{\Innerproduct}[2]{\bigl\langle #1, #2 \bigr\rangle}
\newcommand{\INNERPRODUCT}[2]{\left\langle #1, #2 \right\rangle}

% Generic field
\newcommand{\fieldstd}{\mathbb{F}}
\newcommand{\fieldf}{\mathbb{F}}
\newcommand{\F}{\mathbb{F}}

% Finite fields
\newcommand{\GF}[1]{\mathrm{GF} ( #1 )}
\newcommand{\gf}[1]{\mathrm{GF} ( #1 )}
\newcommand{\GFmul}[1]{\mathrm{GF} ( #1 )^{*}}
\newcommand{\gfmul}[1]{\mathrm{GF} ( #1 )^{*}}


%
% P R O B A B I L I T Y   T H E O R Y 
%-------------------------------------

% AMS-TeX defines an operator name \Pr
\DeclareMathOperator{\Expop}{E}
\DeclareMathOperator{\Varianceop}{Var}

% Probability
\newcommand{\PROB}[2][]{\Pr_{#1} \left[ #2 \right]}
\newcommand{\Prob}[2][]{\Pr_{#1} \bigl[ #2 \bigr]}
\ifthenelse{\boolean{detectedLMCS}}
{\renewcommand{\prob}[2][]{\Pr_{#1} [ #2 ]}}
{\newcommand{\prob}[2][]{\Pr_{#1} [ #2 ]}}

% Expectation
\newcommand{\EXPECTATION}[2][]{\Expop_{#1} \left[ #2 \right]}
\newcommand{\Expectation}[2][]{\Expop_{#1} \bigl[ #2 \bigr]}
\newcommand{\expectation}[2][]{\Expop_{#1} [ #2 ]}
\newcommand{\VARIANCE}[1]{\Varianceop \left( #1 \right)}
\newcommand{\Variance}[1]{\Varianceop \bigl( #1 \bigr)}
\newcommand{\variance}[1]{\Varianceop ( #1 )}

% 
% INTERLUDE: MATCHING MIDDLE SEPARATORS (FROM THE UK TeX FAQ)
% 
% 
% One of the few glaring omissions from TeX's mathematical typesetting
% capabilities is a means of setting separators in the middle of
% mathematical expressions. In all sorts of mathematical enterprises one
% may find oneself needing a \middle command, to be used in expressions
% like \left\{ x \in \mathbb{N} \middle| x \mbox{ even} \right\} to
% specify the set of even natural numbers. The e-TeX system defines just
% such a command, but users of Knuth's original need some support.
% Donald Arseneau's braket package provides commands for set
% specifications (as above) and for Dirac brackets (and bras and kets).
% The package uses the e-TeX built-in command if it finds itself running
% under e-TeX.
% 
% See ftp://cam.ctan.org/tex-archive/macros/latex/contrib/misc/braket.sty .
% 
% Or one can do as below.
%

\newcommand{\twincommandJN}[6]%
    {#1#2#3\vphantom{#2#5}\mspace{-2.05mu}#4.#5#6}

% Perhaps this is superfluous---in text mode there is no need for measuring
% with \vphantom, I think, since \bigl[ and \bigr] are what they are 
% independent of what is inside (are they not?).
%
% The length -2.25mu probably should be set instead by doing sth like
%    
%    \newlength{\lengthJN}
%    \settowidth{\lengthJN}{$\left.\right.$}
%    \setlength{\lengthJN}{0.5\lengthJN}
%
% and then using \mspace{-\lengthJN}, but the difference appears to be
% very small so I have not implemented this.


% CONDITIONAL EXPECTATION
\newcommand{\condexp}[2]{\Expop{#1  \mid  #2}}
\newcommand{\CondExp}[2]%
    {\Expop\twincommandJN{\bigl[}{#1}{\bigl|}{\bigr}{\,#2}{\bigr]}}
\newcommand{\CONDEXP}[2]%
     {\Expop\twincommandJN{\left[}{#1}{\left|}{\right}{\,#2}{\right]}}

% CONDITIONAL PROBABILITY
\newcommand{\condprob}[3][]{\prob[#1]{#2  \mid  #3}}
\newcommand{\Condprob}[3][]%
    {\Pr_{#1}\twincommandJN{\bigl[}{#2}{\bigl|}{\bigr}{\,#3}{\bigr]}}
\newcommand{\CONDPROB}[3][]%
    {\Pr_{#1}\twincommandJN{\left[}{#2}{\left|}{\right}{\,#3}{\right]}}

%
% Example code:
%    
%    \begin{displaymath}
%     \CONDEXP{\sum_{i=1}^kX_i}{Z}\quad\mbox{and}\quad%
%     \CONDPROB{B\land C}{\bigwedge_{i\in S}A_i}
%    \end{displaymath}
%    
%    $\CondExp{\sum_{i=1}^k X_i}{Z}$  
%    and
%    $\CondProb{B\land C}{\bigwedge_{i\in S}A_i}$
%    


%
% F U N C T I O N S
%-------------------
%

% DESCRIPTION OF FUNCTION
\newcommand{\funcdescr}[3]{\ensuremath{ #1 : #2 \to #3}}

% DOMAIN
\newcommand{\domainof}[1]{\ensuremath{\mathrm{dom} ( #1 )}}
\newcommand{\Domainof}[1]{\ensuremath{\mathrm{dom}\bigl( #1 \bigr)}}

% INVERSE IMAGE
\newcommand{\invimageof}[2]{{\ensuremath{{#1}^{-1} \left( #2 \right)}}}

%
% G R A P H S
%-------------
%

\newcommand{\edges}[1]{E( #1 )}
\newcommand{\Edges}[1]{E\bigl( #1 \bigr)}
\newcommand{\vertices}[1]{V( #1 )}
\newcommand{\Vertices}[1]{V\bigl( #1 \bigr)}

\newcommand{\vdegree}[2][]{\mathrm{deg}_{#1}(#2)}
\newcommand{\Vdegree}[2][]{\mathrm{deg}_{#1}\bigl(#2\bigr)}
\newcommand{\vneighbour}[2][]{N_{#1}({#2})}
\newcommand{\Vneighbour}[2][]{N_{#1}\bigl({#2}\bigr)}

% Boundary
\newcommand{\boundary}[1]{\ensuremath{\partial #1}}

\newcommand{\pathstd}{\ensuremath{P}}
\newcommand{\pathalt}{\ensuremath{Q}}
\newcommand{\pathfromto}[3]{#1 : #2 \rightsquigarrow #3}


%
% S E T S   A N D   T U P L E S
%-------------------------------
%

\newcommand{\set}[1]{\{ #1 \}}
\newcommand{\Set}[1]{\bigl\{ #1 \bigr\}}
\newcommand{\SET}[1]{\left\{ #1 \right\}}

\newcommand{\setdescr}[3][\mid]{\set{ #2 #1 #3 }}
\newcommand{\Setdescr}[3][|]%
     {\ifthenelse{\equal{#1}{;}}%
     {\Set{ #2 \,;\, #3 }}
     {\ifthenelse{\equal{#1}{:}}%
     {\Set{ #2 \,:\, #3 }}
     {\twincommandJN{\bigl\{}{#2\,}{\bigl#1}{\bigr}{\,#3}{\bigr\}}}}}
\newcommand{\SETDESCR}[3][|]%
     {\twincommandJN{\left\{}{#2\,}{\left#1}{\right}{\,#3}{\right\}}}

\newcommand{\setbrackets}[1]{[ #1 ]}
\newcommand{\Setbrackets}[1]{\bigl[ #1 \bigr]}
\newcommand{\SETBRACKETS}[1]{\left[ #1 \right]}

\newcommand{\setdescrbrackets}[3][\mid]{{\setbrackets{ #2 #1 #3 }}}
\newcommand{\Setdescrbrackets}[3][|]%
     {\twincommandJN{\bigl[}{#2}{\bigl#1}{\bigr}{\,#3}{\bigr]}}
\newcommand{\SETDESCRBRACKETS}[3][|]%
     {\twincommandJN{\left[}{#2}{\left#1}{\right}{\,#3}{\right]}}

\newcommand{\SETSIZE}[1]{\left\lvert#1\right\rvert}
\newcommand{\Setsize}[1]{\bigl\lvert#1\bigr\rvert}
\newcommand{\setsize}[1]{\lvert#1\rvert}

% Set complement
\newcommand{\setcompl}[1]{\overline{#1}}

% Intersection and union
\newcommand{\intersection}{\cap}
\newcommand{\Intersection}{\bigcap}
\newcommand{\Intersectionnodisplay}{\textstyle \bigcap}

\newcommand{\union}{\cup}
\newcommand{\Union}{\bigcup}
\newcommand{\Unionnodisplay}{\textstyle \bigcup}

% Intersection and union with some space
\newcommand{\unionSP}{\, \union \, }
\newcommand{\intersectionSP}{\, \intersection \, }

% Disjoint union 
%
% Can see no difference between below definition and \mathbin{\dot{\cup}}
\newcommand{\disjointunion}{\overset{.}{\cup}}
\newcommand{\disjointunionSP}{\disjointunion}
\newcommand{\Disjointunion}{\overset{.}{\bigcup}}
\newcommand{\disjunion}{\disjointunion}
\newcommand{\Disjunion}{\Disjointunion}

% First n positive integers
\newcommand{\nset}[1]{[{#1}]}
\newcommand{\Nset}[1]{\bigl[ {#1} \bigr]}


%
% L O G I C
%-----------
%

%
% Logic connectives
%
% Logic or is \lor. Logic and is \land. Logic not is \lnot.
% They can be used in math mode only.
\newcommand{\Lor}{\bigvee}
\newcommand{\Land}{\bigwedge}

% nodisplay = indices to the right, not below and above
\newcommand{\Lornodisplay}{{\textstyle \bigvee}}
\newcommand{\Landnodisplay}{{\textstyle \bigwedge}}

\newcommand{\limpl}{\rightarrow}
\newcommand{\lequiv}{\leftrightarrow}

% Prefixed NOT (pfnot):  \lnot x
% Overlined NOT (olnot): \overline{x}
% \stdnot{lit} is the standard NOT notation for variables and literals

\newcommand{\pfnot}[1]{\lnot #1}
\newcommand{\olnot}[1]{\overline{#1}}
\newcommand{\stdnot}[1]{\olnot{#1}}
    
% syntactic equivalence
\newcommand{\synteq}{\doteq}

% constants "true" and "false"
\newcommand{\FALSE}{\mathit{FALSE}}
\newcommand{\TRUE}{\mathit{TRUE}}

\newcommand{\false}{\bot}
\newcommand{\true}{\top}

\newcommand{\falsenum}{0}
\newcommand{\truenum}{1}

%
% Notation and terms for CNF/DNF formulas
%
% Standard notation for parameters in k-CNF formulas
\newcommand{\nvar}{n}
\newcommand{\nvars}{\nvar}
\newcommand{\nclause}{m}
\newcommand{\nclauses}{\nclause}
\newcommand{\clwidth}{k}
\newcommand{\density}{\Delta}

% Formatting of k-CNF/k-DNF in running text
\newcommand{\xdnf}[1]{\mbox{\ensuremath{#1}-DNF}\xspace}
\newcommand{\kdnf}{\xdnf{\clwidth}}
\newcommand{\xdnfform}[1]{\mbox{\ensuremath{#1}-DNF} formula\xspace}
\newcommand{\kdnfform}{\xdnfform{\clwidth}}

\newcommand{\xcnf}[1]{\mbox{\ensuremath{#1}-CNF}\xspace}
\newcommand{\kcnf}{\xcnf{\clwidth}}
\newcommand{\xcnfform}[1]{\mbox{\ensuremath{#1}-CNF} formula\xspace}
\newcommand{\kcnfform}{\xcnfform{\clwidth}}
\newcommand{\xclause}[1]{\mbox{\ensuremath{#1}-clause}\xspace}
\newcommand{\kclause}{\xclause{\clwidth}}

\newcommand{\Excnf}[1]{\mbox{\ensuremath{\mathrm{E}#1}-CNF}\xspace}
\newcommand{\Ekcnf}{\Excnf{\clwidth}}
\newcommand{\Excnfform}[1]{\mbox{\ensuremath{\mathrm{E}#1}-CNF} formula\xspace}
\newcommand{\Ekcnfform}{\Excnfform{\clwidth}}

\newcommand{\xterm}[1]{\mbox{\ensuremath{#1}-term}\xspace}
\newcommand{\kterm}{\xterm{\clwidth}}
\newcommand{\Exterm}[1]{\mbox{\ensuremath{\mathrm{E}#1}-term}\xspace}
\newcommand{\Ekterm}{\Exterm{\clwidth}}

%
% Distributions for random k-CNF formulas
%
% fixed # random clauses chosen with replacement
% \randkcnfnclwrepl{clause width}{#variables}{#clauses}
%
\newcommand{\randkcnfnclwrepl}[3][\clwidth]%
        {\ensuremath{\mathcal{F}^{#2, #3}_{#1}}}
\newcommand{\randkcnfnclwreplstd}% 
        {\randkcnfnclwrepl{\clwidth}{\nvar}{\nclause}}


%
% C O M P L E X I T Y   T H E O R Y 
%-----------------------------------
%

% NOTATION FOR LANGUAGES/PROBLEMS
\newcommand{\problemlanguageformat}[1]{\textsc{#1}\xspace}
\newcommand{\problemlanguageformatnospace}[1]{\textsc{#1}}
\newcommand{\langstd}{\ensuremath{L}}
\newcommand{\langcompl}[1]{\ensuremath{\overline{#1}}}

\newcommand{\PROP}{\problemlanguageformat{PROP}}
\newcommand{\TAUT}{\problemlanguageformat{TAUT}}
\newcommand{\TAUTOLOGY}{\problemlanguageformat{Tautology}}
\newcommand{\SAT}{\problemlanguageformat{Sat}}
\newcommand{\CNFSAT}{\problemlanguageformat{CnfSat}}
\newcommand{\SATISFIABILITY}{\problemlanguageformat{Satisfiability}}
\newcommand{\THREESAT}{\text{$3$-}\problemlanguageformat{Sat}}
\newcommand{\TWOSAT}{\text{$2$-}\problemlanguageformat{Sat}}
\newcommand{\ONEINTHREESAT}{\text{$1$-}\problemlanguageformatnospace{In}%
        \text{-$3$-}\problemlanguageformat{Sat}}
\newcommand{\MAXSAT}{\problemlanguageformat{MaxSat}}
\newcommand{\MAXTWOSAT}{\problemlanguageformatnospace{Max}%
        \text{-$2$-}\problemlanguageformat{Sat}}
\newcommand{\CIRCUITSAT}{\problemlanguageformat{CircuitSat}}
\newcommand{\NAESAT}{\problemlanguageformat{NotAllEqualSat}}
\newcommand{\DOMINATINGSET}{\problemlanguageformat{DominatingSet}}
\newcommand{\VERTEXCOVER}{\problemlanguageformat{VertexCover}}
\newcommand{\MAXCUT}{\problemlanguageformat{MaxCut}}
\newcommand{\DOMATICNUMBER}{\problemlanguageformat{DomaticNumber}}
\newcommand{\MONOCHROMTRI}{\problemlanguageformat{MonochromaticTriangle}}
\newcommand{\CLIQUECOVER}{\problemlanguageformat{CliqueCover}}
\newcommand{\INDSET}{\problemlanguageformat{IndependentSet}}
\newcommand{\GCLIQUE}{\problemlanguageformat{Clique}}
\newcommand{\GCOLOURING}{\problemlanguageformat{Colouring}}
\newcommand{\SUBGRAPHISO}{\problemlanguageformat{Subgraph\-Isomorphism}}
\newcommand{\GKERNEL}{\problemlanguageformat{Kernel}}
\newcommand{\MINMAXMATCHING}{\problemlanguageformat{MinMaxMatching}}
\newcommand{\CUBICSUBGRAPH}{\problemlanguageformat{CubicSubgraph}}

% NOTATION FOR COMPLEXITY CLASSES
\newcommand{\complclassformat}[1]%
        {\textrm{\upshape{\textsf{#1}}}\xspace}
\newcommand{\cocomplclass}[1]%
        {\textrm{\upshape{\textsf{co#1}}}\xspace}

\newcommand{\TIMEclass}[1]{\ensuremath{\complclassformat{TIME}\bigl(#1\bigr)}}
\newcommand{\SPACEclass}[1]{\ensuremath{\complclassformat{SPACE}\bigl(#1\bigr)}}
\newcommand{\DTIMEclass}[1]{\ensuremath{\complclassformat{DTIME}\bigl(#1\bigr)}}
\newcommand{\DTIMEadviceclass}[2]%
    {\ensuremath{\complclassformat{DTIME}\bigl(#1\bigr)/{#2}}}
\newcommand{\Pclass}{\complclassformat{P}}
\newcommand{\NP}{\complclassformat{NP}}
\newcommand{\NPclass}{\NP}
\newcommand{\coNP}{\cocomplclass{NP}}
\newcommand{\coNPclass}{\coNP}
\newcommand{\CoNP}{\coNP}
\newcommand{\Ppoly}{\complclassformat{P/poly}}
\newcommand{\Pslashpoly}{\Ppoly}
\newcommand{\Pslashpolyclass}{\Pslashpoly}
\newcommand{\PSPACE}{\complclassformat{PSPACE}}
\newcommand{\PSPACEclass}{\PSPACE}
\newcommand{\NPSPACE}{\complclassformat{NPSPACE}}
\newcommand{\NPSPACEclass}{\NPSPACE}
\newcommand{\pspace}{\PSPACE}
\newcommand{\EXPTIME}{\complclassformat{EXPTIME}}
\newcommand{\exptime}{\complclassformat{EXPTIME}}
\newcommand{\EXPSPACE}{\complclassformat{EXPSPACE}}
\newcommand{\expspace}{\complclassformat{EXPSPACE}}
\newcommand{\EXP}{\complclassformat{EXP}}
\newcommand{\NEXP}{\complclassformat{NEXP}}
\newcommand{\coNEXP}{\cocomplclass{NEXP}}
\newcommand{\Lclass}{\complclassformat{L}}
\newcommand{\LOGSPACE}{\complclassformat{L}}
\newcommand{\NL}{\complclassformat{NL}}
\newcommand{\NLclass}{\NL}
\newcommand{\coNL}{\cocomplclass{NL}}
\newcommand{\NCclass}{\complclassformat{NC}}
\newcommand{\DP}{\complclassformat{DP}}
\newcommand{\DPclass}{\DP}
\newcommand{\SIGMACLASS}[1]{\ensuremath{\Sigma^p_{#1}}}
\newcommand{\Sigmaclass}[1]{\SIGMACLASS{#1}}
\newcommand{\PICLASS}[1]{\ensuremath{\Pi^p_{#1}}}
\newcommand{\Piclass}[1]{\PICLASS{#1}}
\newcommand{\PH}{\complclassformat{PH}}
\newcommand{\EXPEXP}{\complclassformat{EXPEXP}}
\newcommand{\NEXPEXP}{\complclassformat{NEXPEXP}}
\newcommand{\coNEXPEXP}{\cocomplclass{NEXPEXP}}        
\newcommand{\BPP}{\complclassformat{BPP}}
\newcommand{\ZPP}{\complclassformat{ZPP}}
\newcommand{\RP}{\complclassformat{RP}}
\newcommand{\coRP}{\cocomplclass{RP}}
\newcommand{\MIParg}[1]{\ensuremath{\complclassformat{MIP}[#1]}}
\newcommand{\MIP}{\complclassformat{MIP}}
\newcommand{\IP}{\complclassformat{IP}}
\newcommand{\PCPnot}{\complclassformat{PCP}}
% PCP{completeness}{soundness}{randomness}{queries}{alphabet size}
\newcommand{\PCPalph}[5]%
    {\ensuremath{\complclassformat{PCP}_{{#1},{#2}}[{#3}, {#4}, {#5}]}}
\newcommand{\PCP}[4]%
    {\ensuremath{\complclassformat{PCP}_{{#1},{#2}}[{#3}, {#4}]}}

\newcommand{\SAC}{\complclassformat{SAC}}
\newcommand{\NC}{\complclassformat{NC}}
\newcommand{\ACzero}{\ensuremath{\complclassformat{AC}^{0}}}
\newcommand{\SC}{\complclassformat{SC}}
\newcommand{\TISP}{\complclassformat{TISP}}
\newcommand{\LOGCFL}{\complclassformat{LOGCFL}}


%
% T E X T   F O R M A T T I N G
%-------------------------------
%

% introducing a new term or mentioning a familiar one for the first time
%\newcommand{\introduceterm}[1]{{\textsl{#1}}}
\newcommand{\introduceterm}[1]{{\emph{#1}}}

% Spacing before punctuation in displayed equations according to LLNCS
% (But SIAM disagrees, so it is convenient to have a macro for this.)
\newcommand{\eqperiod}{\enspace .}
\newcommand{\eqcomma}{\enspace ,}

% For description lists with items in italics (to avoid confusion
% with theorem headers, for instance)
\newcommand{\italicitem}[1][]{\item[\textit{#1}]}


%
% A B B R E V I A T I O N S   O F   F R E Q U E N T     E X P R E S S I O N S
%-----------------------------------------------------------------------------
%

% GENERAL EXPRESSIONS
\newcommand{\wrt}{with respect to\xspace}
\newcommand{\wrtabbrev}{w.r.t.\ }
\ifthenelse{\boolean{detectedToC}}{}
  {\newcommand{\eg}{for instance\xspace} % should be surrounded by commas 
    \newcommand{\Eg}{For instance\xspace}}
\ifthenelse{\boolean{detectedToC}}{}{
\newcommand{\ie}{i.e.,\ }
\newcommand{\Ie}{I.e.,\ }
}
\newcommand{\ieComma}{i.e.,\ }  %%%% Special hack to use when adapting to ToC
\newcommand{\ieNoComma}{i.e.\ }
\ifthenelse{\boolean{detectedLIPIcs} \or \boolean{detectedIJCAI}}
{\renewcommand{\st}{\errmessage{Please do not use st}}}
{\ifthenelse{\isundefined{\st}}
  {\newcommand{\st}{such that\xspace}}
  {}}      
%   \newcommand{\st}{such that\xspace}}

\newcommand{\etal}{et al.\@\xspace}
\newcommand{\etalS}{et al\@. }

\newcommand{\vs}{vs.\ }

% TYPICAL MATHEMATICAL EXPRESSIONS
\newcommand{\ifaoif}{if and only if\xspace}
\newcommand{\wolog}{without loss of generality\xspace}
\newcommand{\Wolog}{Without loss of generality\xspace}
\newcommand{\iid}{independently and identically distributed\xspace}
\ifthenelse{\boolean{detectedIEEE}}{}{\newcommand{\QED}{Q.E.D.}}
\newcommand{\qedlong}{which was to be proved\xspace}

\newcommand{\whp}{with high probability\xspace}
\newcommand{\Whp}{With high probability\xspace}
\newcommand{\aas}{asymptotically almost surely\xspace}
\newcommand{\Aas}{Asymptotically almost surely\xspace}


%
% R E F E R E N C E S 
%---------------------
%
% Macros with capital initial letter intended for use at start of sentence.
%

%
% REFERENCES TO SECTIONAL UNITS AND PAGE INTERVALS
%
% The LLNCS document class and instructions says that in the running text, 
% but not at the beginning of a sentence, Sect., Chap. and Fig. should be 
% abbreviated.
%
% LLNCS wants, e.g., "Section 4.3" with capital S, SIAM doesn't.
%

% Sections
\newcommand{\refsec}[1]{Section~\ref{#1}}
\newcommand{\refsecP}[1]{Section~\vref{#1}}
\newcommand{\Refsec}[1]{Section~\ref{#1}}
\newcommand{\RefsecP}[1]{Section~\vref{#1}}
\newcommand{\reftwosecs}[2]{Sections~\ref{#1} and~\ref{#2}}
\newcommand{\refthreesecs}[3]{Sections~\ref{#1}, \ref{#2}, and~\ref{#3}}

% Chapters
\newcommand{\refch}[1]{Chapter~\ref{#1}}
\newcommand{\Refch}[1]{Chapter~\ref{#1}}
\newcommand{\reftwochs}[2]{Chapters~\ref{#1} and~\ref{#2}}
\newcommand{\refthreechs}[3]{Chapters~\ref{#1}, \ref{#2} and~\ref{#3}}

% Appendices
\newcommand{\refapp}[1]{Appendix~\ref{#1}}
\newcommand{\Refapp}[1]{Appendix~\ref{#1}}
\newcommand{\reftwoapps}[2]{Appendices~\ref{#1} and~\ref{#2}}

% Figures
\newcommand{\reffig}[1]{Figure~\ref{#1}}
\newcommand{\reffigP}[1]{Figure~\vref{#1}}
\newcommand{\Reffig}[1]{Figure~\ref{#1}}
\newcommand{\ReffigP}[1]{Figure~\vref{#1}}
\newcommand{\reftwofigs}[2]{Figures~\ref{#1} and~\ref{#2}}
\newcommand{\refthreefigs}[3]{Figures~\ref{#1}, \ref{#2}, and~\ref{#3}}


%
% REFERENCES TO THEOREM-LIKE ENVIRONMENTS
%
% Requires \usepackage{varioref}
%
% Adapted to LLNCS document class and instructions. Definition, Theorem etc
% should be capitalized when followed by a number.
%

% Theorems
\newcommand{\refth}[1]{Theorem~\ref{#1}}
\newcommand{\reftwoths}[2]{Theorems~\ref{#1} and~\ref{#2}}
\newcommand{\refthreeths}[4][and]{Theorems~\ref{#2}, \ref{#3}, {#1}~\ref{#4}}
\newcommand{\refthm}[1]{Theorem~\ref{#1}}
\newcommand{\reftwothms}[2]{Theorems~\ref{#1} and~\ref{#2}}
\newcommand{\refthreethms}[4][and]{Theorems~\ref{#2}, \ref{#3}, {#1}~\ref{#4}}

% Lemmas
\newcommand{\reflem}[1]{Lemma~\ref{#1}}
\newcommand{\reftwolems}[2]{Lemmas~\ref{#1} and~\ref{#2}}
\newcommand{\refthreelems}[4][and]{Lemmas~\ref{#2}, \ref{#3}, {#1}~\ref{#4}}

% Propositions
\newcommand{\refpr}[1]{Proposition~\ref{#1}}
\newcommand{\reftwoprs}[2]{Propositions~\ref{#1} and~\ref{#2}}

% Corollaries
\newcommand{\refcor}[1]{Corollary~\ref{#1}}
\newcommand{\reftwocors}[2]{Corollaries~\ref{#1} and~\ref{#2}}

% Definitions
\newcommand{\refdef}[1]{Definition~\ref{#1}}
\newcommand{\reftwodefs}[2]{Definitions~\ref{#1} and~\ref{#2}}
\newcommand{\refthreedefs}[3]{Definitions~\ref{#1}, \ref{#2}, and~\ref{#3}}

% Remarks
\newcommand{\refrem}[1]{Remark~\ref{#1}}
\newcommand{\reftworems}[2]{Remarks~\ref{#1} and~\ref{#2}}

% Observations
\newcommand{\refobs}[1]{Observation~\ref{#1}}
\newcommand{\reftwoobs}[2]{Observations~\ref{#1} and~\ref{#2}}

% Facts
\newcommand{\reffact}[1]{Fact~\ref{#1}}

% Conjectures
\newcommand{\refconj}[1]{Conjecture~\ref{#1}}
\newcommand{\reftwoconjs}[2]{Conjectures~\ref{#1} and~\ref{#2}}

% Examples
\newcommand{\refex}[1]{Example~\ref{#1}}
\newcommand{\reftwoexs}[2]{Examples~\ref{#1} and~\ref{#2}}

% Properties
\newcommand{\refproperty}[1]{Property~\ref{#1}}
\newcommand{\reftwoproperties}[2]{Properties~\ref{#1} and~\ref{#2}}
 
% Claims
\newcommand{\refclaim}[1]{Claim~\ref{#1}}
\newcommand{\reftwoclaims}[2]{Claims~\ref{#1} and~\ref{#2}}

% References at start of sentence
\newcommand{\Refth}[1]{Theorem~\ref{#1}}
\newcommand{\Reflem}[1]{Lemma~\ref{#1}}
\newcommand{\Refpr}[1]{Proposition~\ref{#1}}
\newcommand{\Refcor}[1]{Corollary~\ref{#1}}
\newcommand{\Refdef}[1]{Definition~\ref{#1}}
\newcommand{\Refrem}[1]{Remark~\ref{#1}}
\newcommand{\Refobs}[1]{Observation~\ref{#1}}
\newcommand{\Refconj}[1]{Conjecture~\ref{#1}}
\newcommand{\Refex}[1]{Example~\ref{#1}}
\newcommand{\Refclaim}[1]{Claim~\ref{#1}}

% References with page numbers
\newcommand{\refthP}[1]{Theorem~\vref{#1}}
\newcommand{\reflemP}[1]{Lemma~\vref{#1}}
\newcommand{\refprP}[1]{Proposition~\vref{#1}}
\newcommand{\refcorP}[1]{Corollary~\vref{#1}}
\newcommand{\refdefP}[1]{Definition~\vref{#1}}
\newcommand{\refremP}[1]{Remark~\vref{#1}}
\newcommand{\refobsP}[1]{Observation~\vref{#1}}
\newcommand{\refconjP}[1]{Conjecture~\vref{#1}}
\newcommand{\refexP}[1]{Example~\vref{#1}}
\newcommand{\refpropertyP}[1]{Property~\vref{#1}}

% Some more references
\newcommand{\refrule}[1]{rule~\ref{#1}}
\newcommand{\reftworules}[2]{rules~\ref{#1} and~\ref{#2}}

\newcommand{\refpart}[1]{part~\ref{#1}}
\newcommand{\Refpart}[1]{Part~\ref{#1}}
\newcommand{\reftwoparts}[2]{parts~\ref{#1} and~\ref{#2}}
\newcommand{\Reftwoparts}[2]{Parts~\ref{#1} and~\ref{#2}}

\newcommand{\refitem}[1]{item~\ref{#1}}
\newcommand{\Refitem}[1]{Item~\ref{#1}}
\newcommand{\reftwoitems}[2]{items~\ref{#1} and~\ref{#2}}
\newcommand{\Reftwoitems}[2]{Items~\ref{#1} and~\ref{#2}}

\newcommand{\refcase}[1]{case~\ref{#1}}
\newcommand{\Refcase}[1]{Case~\ref{#1}}
\newcommand{\reftwocases}[2]{cases~\ref{#1} and~\ref{#2}}
\newcommand{\Reftwocases}[2]{Cases~\ref{#1} and~\ref{#2}}

% References to equations (alias for \eqref just for simplicity)
%
% The definition of \refeq overwrites a command in the mathtools package
% if this package has been loaded

\ifthenelse
{\isundefined{\refeq}}
{\newcommand{\refeq}[1]{\eqref{#1}}}
{\renewcommand{\refeq}[1]{\eqref{#1}}}
\newcommand{\refeqP}[1]{\eqref{#1} on page~\pageref{#1}}




%%%
%%% SOME LOCAL MACROS FOR THE IDMA COURSE
%%%

% GCD
%   \DeclareMathOperator{\gcd}{gcd}

% Two-norm
\newcommand{\twonorm}[1]{\lVert#1\rVert_2}
\newcommand{\Twonorm}[1]{\bigl\lVert#1\bigr\rVert_2}
\newcommand{\TWONORM}[1]{\left\lVert#1\right\rVert_2}

% Formal language grammars
%   \newcommand{\produces}{\rightarrow}
%   \newcommand{\terminalf}[1]{\mathtt{#1}}
%   \newcommand{\tokenf}[1]{\text{\textbf{#1}}}
%   \newcommand{\emptystring}{\varepsilon}
%   \newcommand{\numtoken}{\tokenf{num}}

% For checkmark and xmark
\usepackage{pifont}
\newcommand{\xmark}{\ding{55}}

% For formatting induction proofs
\newcommand{\indproofstep}[1]%
{
  \smallskip
  \noindent
  \textbf{\textit{#1:}}
}

\newcommand{\indbase}[1]{\indproofstep{Base case (#1)}}
\newcommand{\indstep}{\indproofstep{Induction step}}
\newcommand{\indclaim}{\indproofstep{Claim}}





% For getting watermark "DRAFT" across all pages (for instance, 
% when posting preliminary version of problem set)
%    \usepackage{draftwatermark}
%    % \SetWatermarkFontSize{20 cm}
%    \SetWatermarkScale{5}

% For METAPOST logo as \hologo{METAPOST}
%   \usepackage{hologo}

% For TikZ
%   \input{Figures/tikz-packages.tex}

%%%
%%% TITLE
%%%

\author{\courseinstructor}
\course{\coursenamelong{}}
\semester{\courseperiod}
\title{\coursenameshort: Problem Set \psetno}

\begin{document}

\maketitle



\begin{abstract}
  \noindent
  \textbf{Due:} \duedate.

  \noindent
  \textbf{Submission:}
  Please submit your solutions
  via \emph{Absalon}
  as a PDF file.
  State your name and e-mail address 
  close to the top   of the first page.
  Solutions should be written in \LaTeX{} or some other math-aware
  typesetting system with reasonable margins on all sides (at least 2.5~cm).
  Please try to be precise and to the point in your solutions and
  refrain from vague statements.
  % Make sure to explain your reasoning.
  Never, ever just state the answer, but always
  make sure to explain your reasoning.
  \emph{Write so that a fellow student of yours can read, understand, and
    verify your solutions.}
  In addition to what is stated below, the general rules 
  for problem sets stated on \emph{Absalon} always apply.

%    
%      \noindent
%      \textbf{Hints:}
%      For most or all problems, ``hints'' can be purchased at a cost of 
%      \mbox{5--10~points}. In this way, you can configure yourself whether
%      you want the problems to be more creative and open-ended, where
%      sometimes a lot can depend on finding the right idea, or whether you
%      want them to be more of guided exercises providing a useful work-out
%      on the concepts of proof complexity. If you do not solve a problem,
%      there is no charge for the hint (i.e., it is not deducted from the
%      score on other problems).  
%    

  \noindent
  \textbf{Collaboration:}
  Discussions of ideas in groups of 
  two to three people 
  are allowed---and indeed, encouraged---but 
  you should always write up your solutions completely on your own,
  from start to finish, and you should understand all aspects of them
  fully. It is not allowed to compose draft solutions together and
  then continue editing individually, or to share any text, formulas,
  or pseudocode. Also, no such material may be downloaded from or
  generated via the internet to be used in draft or final solutions.
  Submitted solutions will be checked for plagiarism.

% 
%     You should also clearly acknowledge any collaboration.
%     State   close to the top 
%     of the first page of your problem set
%     solutions if you have been collaborating with someone and if so with
%     whom.  
%     \emph{Note that collaboration is on a per problem set basis,
%       so you should not discuss different problems on the same problem
%       set with different people.}
%    
%    
%   
%     \noindent
%     \textbf{Reference material:} 
%     Some of the problems are ``classic'' and hence it might be easy to
%     find solutions on the Internet, in textbooks or in research
%     papers. It is not allowed to use such material in any way unless
%     explicitly stated otherwise. Anything said during the lectures or in
%     the lecture notes 
%   %      , or which can be found in chapters of Arora-Barak covered in
%   %      the course,  
%     should be fair game, though, unless you are specifically asked to
%     show something that we claimed without proof in class. All
%     definitions 
%   %      used 
%     should be as given in class
%     or in Arora-Barak 
%     and cannot be substituted by 
%   %      definitions
%     versions
%     from other sources.  It is
%     hard to pin down 100\% watertight formal rules on what all of this
%     means---when in doubt, ask the main instructor.
%     

  \noindent
  \textbf{Grading:}
  A score of 
  \thresholdforpass
  is guaranteed to be enough to pass this problem set.


  \noindent
  \textbf{Questions:}
  Please do not hesitate to ask the instructor or TAs if any problem
  statement is unclear, but please make sure to send private
  messages---sometimes specific enough questions could give away the
  solution to your fellow students, and we want all of you to benefit
  from working on, and learning from, the problems.
%   
  Good luck!
\end{abstract}




%%%
%%% DMFS 2022 pset 1:1
%%%
%%% a: 30, b: 10, c: 20
%%%

\begin{problem}%
  \label{problem:code-snippet-1}%
  (60 p)
  In the following snippet of code \verb+A+ and \verb+B+ are arrays
  indexed from~$1$ to~$n$ that contain numbers.
\begin{verbatim}
for i := 1 upto n {
    B[i] := 1
    for j := 1 upto i {
        B[i] := B[i] * A[j]
    }
}
\end{verbatim}

  \begin{subproblem}%
    \ifthenelse{\boolean{versionwithsolutions}}
    {(30 p)}
    {\ignorespaces}
  Explain in plain language what the algorithm above does.
  In particular, what is the meaning of the entries \verb+B[i]+
  after the algorithm has terminated?
\end{subproblem}

\begin{solution}
  At termination we will have
  $
  B[i] = \prod_{j=1}^{i} A[i]
  $,
  i.e., 
  $B[i]$ is the product of all entries
  $A[1], \, A[2], \, \ldots, \, A[i-1], \, A[i]$.
\end{solution}


\begin{subproblem}%
    \ifthenelse{\boolean{versionwithsolutions}}
    {(10 p)}
    {\ignorespaces}
  Provide an asymptotic analysis of the running time
  as a function of the array size~$n$.
  (That is, state how the worst-case running time scales with~$n$,
  focusing only on the highest-order term, and ignoring the constant
  factor in front of this term.)
\end{subproblem}

\begin{solution}
  The algorithm has two nested for loops. 
  The outer loop runs for $i$ from $1$ to $n$
  and the inner loop runs from $1$ to $i$.
  Inside the innermost loop a constant number of operations are performed.
  The asymptotic time complexity is therefore determined by the total
  number of times the inner loop is executed.

  It follows from what is written above that the inner loop will run a
  total of
  $
  \sum_{i=1}^{n} i
  $
  times, which is $\Bigtheta{n^2}$.
  If we want to prove this from first principles, then we can observe that
  $
  \sum_{i=1}^{n} i
  \leq
  \sum_{i=1}^{n} n
  = n^2
  $,
  which shows that the running time is $\Bigoh{n^2}$, 
  and also that 
  $
  \sum_{i=1}^{n} i
  \geq
  \sum_{i=n/2}^{n} n/2
  = n^2 / 4
  $,
  which shows that the running time is $\Bigomega{n^2}$.

  In general, for this introductory course we do not care so much
  about the distinction between big-oh and big-theta, so correct
  answers with only big-oh bounds will also be acceptable unless
  stated otherwise (and as long as these bounds are tight).
\end{solution}


\begin{subproblem}%
  \ifthenelse{\boolean{versionwithsolutions}}
  {(20 p)}
  {\ignorespaces}
  Can you improve the code to run faster while retaining the same
  functionality?
%     Improve the code to run faster while retaining the same
%     functionality.
  How much faster can you get the algorithm to run?
  Analyse the time complexity of your new algorithm.
  Can you prove that it is asymptotically optimal?
  (That is, that no algorithm solving this problem can run faster
  except possibly for a constant factor in the highest-order term or
  improvements in lower-order terms.)
\end{subproblem}

\begin{solution}
  From our analysis of the algorithm above, it should be clear that
  the snippet of code below has the same functionality.
\begin{verbatim}
B[1] := A[1]
for i := 2 upto n {
    B[i] := A[i] * B[i-1]
}
\end{verbatim}  
It goes through the array only once, thus having a running time of
$\bigoh{n}$.
Since we need to store $n$ values in the array~$B$,
it is clear that linear time is optimal. 
\end{solution}

\end{problem}


%%%
%%% NEW PROBLEM
%%%
%%% a: 30, b: 10, c: 20
%%%

\begin{problem}%
  \label{problem:code-snippet-1}%
  (60 p)
  In the following snippet of code \verb+A+ is an array
  indexed from~$1$ to~$n$ that contain elements that
  can be compared
\begin{verbatim}
j    := n
good := TRUE
while (j > 1 and good)
    i := j - 1
    while (i >= 1 and good)
        if (A[i] > A[j])
            good := FALSE
        i := i - 1
    j := j - 1
if (good)
    return "success"
else
    return "failure"
\end{verbatim}

  \begin{subproblem}%
    \ifthenelse{\boolean{versionwithsolutions}}
    {(30 p)}
    {\ignorespaces}
    Explain in plain language what the algorithm above does.
    In particular, what do we know about the array
    \verb+A+ when
    \verb+"success"+
    or
    \verb+"failure"+
    is returned, respectively?
\end{subproblem}

\begin{solution}
  The two nested while loops in the algorithm
  will compare all elements
  $A[i]$ and~$A[j]$
  for $1 \leq i < j \leq n$,
  and if it ever happens that
  $A[i] > A[j]$
  the algorithm will set the
  Boolean flag \verb+good+ to false,
  exit the while loops, and return failure.
  Otherwise it will return success.
  That is, success is returned if and only if the elements in the
  array are sorted in increasing order
  $
  A[1] \leq
  A[2] \leq
  A[3] \leq
  \cdots \leq
  A[n]
  $.
\end{solution}


\begin{subproblem}%
    \ifthenelse{\boolean{versionwithsolutions}}
    {(10 p)}
    {\ignorespaces}
  Provide an asymptotic analysis of the running time
  as a function of the array size~$n$.
  (That is, state how the worst-case running time scales with~$n$,
  focusing only on the highest-order term, and ignoring the constant
  factor in front of this term.)
\end{subproblem}

\begin{solution}
  The algorithm has two nested while  loops. 
  The outer loop runs for $j$ from $n$ down to $2$
  and the inner loop runs from $j-1$ to $1$
  in the worst case (which is when the array is in fact sorted).
  Inside the loops a constant number of operations are performed.
  The asymptotic time complexity is therefore determined by the total
  number of times the inner loop is executed.

  It follows from what is written above that the inner loop will run a
  total of
  $
  \sum_{i=1}^{n-1} i
  $
  times, which is $\Bigtheta{n^2}$.
%%%
  Again, for this introductory course we do not care so much
  about the distinction between big-oh and big-theta, so correct
  answers with only big-oh bounds will also be acceptable unless
  stated otherwise (and as long as these bounds are tight).
\end{solution}


\begin{subproblem}%
  \ifthenelse{\boolean{versionwithsolutions}}
  {(20 p)}
  {\ignorespaces}
  Can you improve the code to run faster while retaining the same
  functionality?
%     Improve the code to run faster while retaining the same
%     functionality.
  How much faster can you get the algorithm to run?
  Analyse the time complexity of your new algorithm.
  Can you prove that it is asymptotically optimal?
  (That is, that no algorithm solving this problem can run faster
  except possibly for a constant factor in the highest-order term or
  improvements in lower-order terms.)
\end{subproblem}

\begin{solution}
  From our analysis of the algorithm above, it should be clear that
  the snippet of code below checks that the array is sorted in increasing
  order, and so has the same functionality.
\begin{verbatim}
for i := 1 upto n-1 {
    if (A[i] > A[i+1])
        return "failure"
}
return "success"
\end{verbatim}
  This algorithm runs in linear time, which is optimal since
  we have to look at all elements in an array to be able to determine
  whether the array is sorted or not.
\end{solution}

\end{problem}


%%%
%%% DMFS 2022 pset 1:2
%%%
%%% a: 40, b: 20, c: 20
%%%

\begin{problem}
  (80 p)
In the following snippet of code \verb+A+ is an array
indexed from $1$ to~$n$
that contains integers,
and \verb+B+~is an auxiliary array, also 
indexed from $1$ to~$n$,
that is meant to contain Boolean values.
\begin{verbatim}
for i := 1 upto n {
    if (A[i] < 1 or A[i] > n)
        return "failure"
}
i     := 1
found := -1
while (i <= n and found < 0) {
    for j := 1 upto n {
        B[j] := false
    }
    j := i
    while (B[j] == false) {
        B[j] := true
        j := A[j]
    }
    if (A[A[j]] == j)
        found := j
    i := i + 1
}
return found
\end{verbatim}

\begin{subproblem}%
  \ifthenelse{\boolean{versionwithsolutions}}
  {(40 p)}
  {\ignorespaces}
  Explain in plain language what the algorithm above does.
  In particular, when does it return a positive value, and, 
  if it does, what is the meaning of this value?
\end{subproblem}

\begin{solution}
  First, the algorithm checks that all array entries
  $A[i]$ for $1 \leq i \leq n$
  are between~$1$ and~$n$.
  It then does the following, starting from all positions $i = 1, 2,
  \ldots, n$:
  \begin{itemize}
  \item 
    Visit all positions 
    $j = i, \, A[i],\, A[A[i]], \,  A[A[A[i]]],\, \ldots$
    in the array, stopping as soon as we see a number we have seen
    before (which we keep track of using the array $B$).
    Note  that we will never get problems with array indices being out
    of bounds, thanks to the checks in the first for loop.
  \item 
    As soon as $A[j] = k$ for some $k$ already seen before,
    check if
    $A[k] = j$, 
    i.e., if the two entries $A[j]$ and  $A[k]$ ``point at each other''.
  \item 
    If this is the case, then terminate and return $j$, otherwise 
    increment~$i$ and continue.
  \end{itemize}
  If a positive value~$j$ is returned, then this is an array index
  such that $A[A[j]] = j$.
\end{solution}

\begin{subproblem}%
  \ifthenelse{\boolean{versionwithsolutions}}
  {(20 p)}
  {\ignorespaces}
  Provide an asymptotic analysis of the running time
  as a function of the array size~$n$.
  (That is, state how the worst-case running time scales with~$n$,
  focusing only on the highest-order term, and ignoring the constant
  factor in front of this term.)
\end{subproblem}

\begin{solution}
  The initial for loop takes linear time.
  In the main part of the algorithm we have a while loop that will
  require $n$~iterations in the worst case, and a nested for loop
  (resetting the array~$B$) that requires $n$~steps.
  The nested while loop checking entries $B[j]$ will also run for at most
  $n$~iterations (and every iteration involves a constant amount of
  work), since every iteration sets a new entry $B[j]$ to 
  true and there are $n$~entries all in all.
  Therefore, the total time inside the outermost while loop 
%     is~$\bigtheta{n}$,
  is~$\bigoh{n}$,
  which means that the time complexity of the
  whole algorithm
%     is~$\Bigtheta{n^2}$.
  is~$\Bigoh{n^2}$.
  (In fact, it is not hard to argue that the above analysis is tight,
  so that the bound  
  is~$\Bigtheta{n^2}$,
  but correct bounds stated in big-oh notation is sufficient
  for a full score on this course unless explicitly stated otherwise.)
\end{solution}

\begin{subproblem}%
  \ifthenelse{\boolean{versionwithsolutions}}
  {(20 p)}
  {\ignorespaces}
  Can you improve the code to run faster while retaining the same
  functionality?
  How much faster can you get the algorithm to run?
  Analyse the time complexity of your new algorithm.
  Can you prove that it is asymptotically optimal?
\end{subproblem}


\begin{solution}
  From our analysis of the algorithm above, it is clear that the code
  does
  the following
  two things:
  \begin{enumerate}
  \item 
    It first checks that all array entries
    $A[i]$ for $1 \leq i \leq n$
    are between~$1$ and~$n$.

  \item 
    It then tries to find an~$i$ for which $A[A[i]] = i$.
  \end{enumerate}
  We get exactly the same functionality
%     with
  by instead coding up an algorithm that uses
  the following snippet of code:
\begin{verbatim}
for i := 1 upto n {
    if (A[i] < 1 or A[i] > n)
        return "failure"
}
i     := 1
found := -1
while (i <= n and found < 0) {
    if (A[A[i]] == i)
        found := i
    i := i + 1
}
return found
\end{verbatim}
This algorithm runs in linear time, which is optimal --- if, 
for instance, we have
$A[i] = i+1$ for $1 \leq i \leq n-1 $ 
and 
$A[n] = 1$,
then the algorithm will have to look at the whole array in order to
determine that there is no solution.

A slightly shorter solution that might be worth discussing is as follows:
\begin{verbatim}
i     := 1
found := -1
while (i <= n and found < 0) {
    j := A[i]
    if (j < 1 or j > n)
        return "failure"
    else if (A[j] == i)
        found := i
    i := i + 1
}
return found
\end{verbatim}
Note that this solution is in fact not fully equivalent, since it
could be that we find and return an~$i$ such that
$A[A[i]] = i$
before detecting that there are out-of-bounds entries in the array.
However, for an introductory course like this we will give a full
score also to solutions that do not give exactly the same behaviour
for corner cases like the
\verb+"failure"+ case in this algorithm, which 
are anyway somewhat tangential to the main point of the problem at hand.
\end{solution}

\end{problem}



%%%
%%% DMFS 2022 pset 1:3
%%%
%%% a: 40, b: 20, c: 20
%%%
%%%

\begin{problem}
  (80 p)
In the following snippet of code \verb+A+ is an array indexed from $1$
to~$n \geq 2$
containing integers.
\begin{verbatim}
search (A, lo, hi)
    if (A[lo] >= A[hi])
        return "failure"
    else if (lo + 1 == hi)
        return lo
    else
        mid = floor ((lo + hi) / 2))
        if (A[mid] > A[lo])
            search (A, lo, mid)
        else
            search (A, mid, hi)
\end{verbatim}
The first call to the algorithm is
\verb+search (A, 1, n)+,
where $n$ is whatever size (at least~$2$) the array has.


\begin{subproblem}%
  \label{problem:first-subproblem}%
  \ifthenelse{\boolean{versionwithsolutions}}
  {(40 p)}
  {\ignorespaces}
  Explain in plain language what the algorithm above does.
  If the algorithm returns something other than 
  \verb+"failure"+, 
  then what is the meaning of the value returned?
%     Could it be the case that recursive calls of the algorithm also
%     return
%     \verb+"failure"+, 
%     or would it be sufficient to check just once before making the first
%     recursive call?
\end{subproblem}


\begin{solution}
  The algorithm first checks that the left endpoint of the array
  \verb+A[lo]+
  is strictly smaller than the right endpoint 
  \verb+A[hi]+.
  Once we know this, it will be the case for any other array entry 
  \verb+A[i]+
  that
%%% TYPO
%     $\verb+A[i]+<\verb+A[lo]+$
  $\verb+A[i]+ > \verb+A[lo]+$
  or
%%% TYPO
%     $\verb+A[i]+ > \verb+A[hi]+$
  $\verb+A[i]+ < \verb+A[hi]+$
  (or both), and since
  \verb+mid+
  is chosen so as to maintain the invariant that the left endpoint of the
  array is strictly smaller than the right endpoint, 
  recursive calls can never return
  \verb+"failure"+.
  This answers part of 
  Problem~\ref{problem:third-subproblem}
  below.

  Given that we do not have a failure for the first recursive call,
  the algorithm will maintain the invariant that
  $\verb+A[lo]+ < \verb+A[hi]+$
  and the difference
  $\verb+hi+ - \verb+lo+$ will decrease as long as
  $\verb+hi+ - \verb+lo+ \geq 2$.
%   
  When
  $\verb+lo+ = \verb+hi+ + 1 $
  the value
  \verb+lo+
  will be returned, and this is an array index such that
  $\verb+A[lo]+ < \verb-A[lo+1]- $.
\end{solution}


\begin{subproblem}%
  \ifthenelse{\boolean{versionwithsolutions}}
  {(20 p)}
  {\ignorespaces}
  Provide an asymptotic analysis of the running time
  as a function of the array size~$n$.
\end{subproblem}


\begin{solution}
  The amount of work per recursive call of the \verb+search+ algorithm
  is constant, so the running time is determined by the number of
  recursive calls.

  If the size of the array $m$ before a recursive call is
  even $m = 2k$,
  then the recursive call will be made either on an array or size~$k$
  or on one of size~$k+1$.
  If the size
  odd $m = 2k+1$ is odd,
  then the recursive call will be made on an array of size~$k+1$.
  We see that (apart from an annoying additive~$1$)
  the array size will halve in every recursive call,
  and this means that after a logarithmic number of recursive calls
  the algorithm will terminate.
  The time complexity of the algorithm is therefore
  $\bigoh{\log n}$
  (or
  $\bigtheta{\log n}$,
  if we wish to be fully precise).

  Just to be clear, the above argument is a bit handwavy, but
  it will be sufficient for a full score on an introductory course
  like this.
  If we wanted to be more formal, then we could argue, e.g.,  along the
  following lines:
  \begin{itemize}
  \item 
    For arrays of size less than~$12$, the algorithm will take
    constant time for some constant~$K$
    (just take the max of the worst-case running times for all arrays of sizes
    $2,\, 3,\, 4, \, \ldots, \, 11$).
  \item 
    For arrays of size at least~$12$, the array size will shrink
    by at least a factor $3/2$
    for each recursive call.
  \end{itemize}
  This means that the running time will be proportional to 
  $\log_{3/2} n + K$,
  which is $\bigoh{\log n}$.
\end{solution}



\begin{subproblem}%
  \label{problem:third-subproblem}%
  \ifthenelse{\boolean{versionwithsolutions}}
  {(20 p)}
  {\ignorespaces}
  Could it be the case that recursive calls of the algorithm also
  return
  \verb+"failure"+, 
  or would it be sufficient to check just once before making the first
  recursive call?
  If we get the additional guarantee that all elements in the array
  are distinct, 
  could we remove the 
  \verb+"failure"+
  check completely, since we would be guaranteed to never have this answer
  returned anyway?
  What about if we get the additional guarantee that the array is
  sorted in increasing order? 
  What if both of these extra guarantees apply?
\end{subproblem}


\begin{solution}
  The answer
  \verb+"failure"+
  can only be returned for the very first recursive call, as explained
  in the solution to 
  Problem~\ref{problem:first-subproblem}.
  In order for this never to happen, we need to know that
  $\verb+A[1]+ < \verb+A[n]+$.
  This, in turn, is certainly guaranteed if the array is sorted in
  increasing order and  all elements are distinct, 
  but it does not hold if we only have one of the guarantees in the
  problem statement above.
\end{solution}


\end{problem}



\end{document}



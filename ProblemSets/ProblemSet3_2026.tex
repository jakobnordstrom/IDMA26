\documentclass{jn-pset}
%   \documentclass[solutions]{jn-pset}

\usepackage{ifthen}
\newboolean{versionwithsolutions}
% Uncomment to insert text that should only be there in version WITH solutions
\setboolean{versionwithsolutions}{true}
% Uncomment to insert text that should only be there in version WITHOUT solutions
\setboolean{versionwithsolutions}{false}


%%%%
%%%% QUICK INSTRUCTIONS FOR FORMATTING OF PROBLEMS
%%%%
%    
% Code a problem by
%    \begin{problem}
%    \end{problem}
%
% Code a subproblem inside a problem by (note the percent signs!)
%    \begin{subproblem}%
%        \label{problem:labelhere}%
%        Text here
%    \end{subproblem}
%
% Get a small vertical space by issuing the command
%    \smallskip
%
% To give a hint to a problem (after having completed the problem statement),
% use the following LaTeX code for formatting consistency:
%
%    \smallskip
%    \noindent
%    \emph{Hint:}
%    Consider the following super-useful hint for this particular problem...
%

% PROBLEM-SET-SPECIFIC MACROS (UPDATE FOR EACH PROBLEM SET)

\newcommand{\psetno}{3}
\newcommand{\duedate}{Wednesday March 11 at 12:59 CET}

\newcommand{\thresholdforpass}{$120$~points\xspace}

% Compact lists
\usepackage{paralist}

% COURSE-SPECIFIC MACROS FOR IDMA 2025

\newcommand{\coursenameabbrev}{IDMA}
\newcommand{\coursenameshort}{Introduktion til diskret matematik og algoritmer}
\newcommand{\coursenamelong}{NDAB23002U Introduktion til diskret matematik og algoritmer}
%    
\newcommand{\submissionemail}{jn@di.ku.dk\xspace}
%   \newcommand{\courseinstructor}{Jakob Nordstr\"om}
\newcommand{\courseinstructor}
        {Jakob Nordstr\"om and Srikanth Srinivasan}
\newcommand{\courseperiod}{2024/2025}

% PACKAGES, MACROS, ET CETERA

\usepackage[T1]{fontenc}
\usepackage[utf8]{inputenc}
%%% Apparently a newer version of babel doesn't play well with nada-ten
%    \usepackage[english]{babel}

\usepackage{hyperref}

\usepackage{amsmath}
\usepackage{amssymb}
\usepackage{amsfonts}
\usepackage{mathtools}

% Provide calligraphic \mathscr font
\usepackage{mathrsfs}

% Enable use of MetaPost generated PostScript files
\usepackage{ifpdf}
\usepackage{graphicx}  
\ifpdf         
\DeclareGraphicsRule{*}{mps}{*}{}
\fi            

% For getting subfigures 1(a), 1(b) etc
% The package "subfigure" is obsolete, so switch to subcaption
%    \usepackage[sf,SF]{subfigure}
\usepackage{subcaption} 

% Sam Buss's package for formatting proofs
\usepackage{bussproofs}

% Extensions to verbatim commands
\usepackage{verbatim}

% Smiley
\usepackage{wasysym}

% To choose how to enumerate lists
\usepackage{enumerate}

%%%%
%%%% THIS FILE IS INTENDED TO BE READ-ONLY --- PLEASE DO NOT EDIT.
%%%% PLEASE CONTACT JAKOB NORDSTRÖM AT jn@di.ku.dk REGARDING ANY ISSUES.
%%%%

%
% DETECTION OF DOCUMENT TYPE
%----------------------------
%
% Version date: September 24, 2022
%
% First versions by Jakob Nordström <jn@di.ku.dk>
% Cleaned up version by Marc Vinyals <vinyals@kth.se>
% Minor later additions by Jakob Nordström and Susanna F. de Rezende
%
% This file needs to be included or input(ted) so that the conditional
% macro definitions in other LaTeX files will not generate compilation
% errors. Most document classes are detected using \@ifclassloaded.
%
% Use the file 'testdoctypedetection.tex' to double-check that these
% Boolean detectors work.
%
%
%%% The report class is not detected correctly in view of later
%%% updates, but this should be easy to fix when needed. 
%%% [Jakob Nordström, May 16, 2016]
% detectedReport is set to true if none of detectedArticle, detectedThesis,
% detectedSTOC, detectedFOCS, detectedSIAM, detectedIEEE, detectedICS,
% or detectedPoster is true.
%


\usepackage{ifthen}

\provideboolean{detectedSTOC}
\provideboolean{detectedFOCS}
\provideboolean{detectedElsevier}
\provideboolean{detectedNOW}
\provideboolean{detectedLMCS}
\provideboolean{detectedIEEE}
\provideboolean{detectedPoster}
\provideboolean{detectedSIAM}
\provideboolean{detectedLNCS}
\provideboolean{detectedACM}
\provideboolean{detectedACMconf}
\provideboolean{detectedSigplanconf}
\provideboolean{detectedToC}
\provideboolean{detectedLIPIcs}
\provideboolean{detectedAAAI}
\provideboolean{detectedIJCAI}
\provideboolean{detectedCompCplx}
\provideboolean{detectedEasyChair}
\provideboolean{detectedJAIR}
\provideboolean{detectedArticle}
\provideboolean{detectedReport}
\provideboolean{detectedThesis}

\makeatletter

\@ifclassloaded{sig-alternate}
{\setboolean{detectedSTOC}{true}}
{\setboolean{detectedSTOC}{false}}

\@ifclassloaded{elsarticle}
{\setboolean{detectedElsevier}{true}}
{\setboolean{detectedElsevier}{false}}

\@ifclassloaded{now}
{\setboolean{detectedNOW}{true}}
{\setboolean{detectedNOW}{false}}

\@ifclassloaded{lmcs}
{\setboolean{detectedLMCS}{true}}
{\setboolean{detectedLMCS}{false}}

\@ifclassloaded{IEEEtran} {
  \setboolean{detectedIEEE}{true}
  \ifCLASSOPTIONconference {
    \setboolean{detectedFOCS}{true}
  }
  \else {
    \setboolean{detectedFOCS}{false}
  }
  \fi
}
{
  \setboolean{detectedFOCS}{false}
  \setboolean{detectedIEEE}{false}
}

%%% Obsolete SIAM class file
%   \@ifclassloaded{siamltex1213} 
%   {\setboolean{detectedSIAM}{true}}
%   {\setboolean{detectedSIAM}{false}}
\@ifclassloaded{siamart171218}
{\setboolean{detectedSIAM}{true}}
{\setboolean{detectedSIAM}{false}}

\@ifclassloaded{llncs}
{\setboolean{detectedLNCS}{true}}
{\setboolean{detectedLNCS}{false}}

\@ifclassloaded{acmsmall}
{\setboolean{detectedACM}{true}}
{\setboolean{detectedACM}{false}}

\@ifclassloaded{acmart}
{\setboolean{detectedACMconf}{true}
 \setboolean{detectedACM}{true}}
{\setboolean{detectedACMconf}{false}}

\@ifclassloaded{sigplanconf}
{\setboolean{detectedSigplanconf}{true}}
{\setboolean{detectedSigplanconf}{false}}

\@ifclassloaded{toc}
{\setboolean{detectedToC}{true}}
{\setboolean{detectedToC}{false}}

\@ifclassloaded{lipics}
{\setboolean{detectedLIPIcs}{true}}
{\@ifclassloaded{lipics-v2019}
  {\setboolean{detectedLIPIcs}{true}}
  {\@ifclassloaded{oasics-v2019}
    {\setboolean{detectedLIPIcs}{true}}
    {\setboolean{detectedLIPIcs}{false}}
  }
}

%   
%   \@ifclassloaded{lipics}
%   {\setboolean{detectedLIPIcs}{true}}
%   {\@ifclassloaded{lipics-v2016}
%     {\setboolean{detectedLIPIcs}{true}}
%     {\@ifclassloaded{oasics-v2016}
%       {\setboolean{detectedLIPIcs}{true}}
%       {\setboolean{detectedLIPIcs}{false}}
%     }
%   }
%   

\@ifclassloaded{cc}
{\setboolean{detectedCompCplx}{true}}
{\setboolean{detectedCompCplx}{false}}

\@ifclassloaded{easychair}
{\setboolean{detectedEasyChair}{true}}
{\setboolean{detectedEasyChair}{false}}

%%%
%%% For JAIR, detect that the "jair" package is being used
%%%
\@ifpackageloaded{jair}
{\setboolean{detectedJAIR}{true}}        
{\setboolean{detectedJAIR}{false}}        

%%%
%%% AAAI and IJCAI have special style files that needs to be detected.
%%% It seems they update the name with the year of the conference also.
%%%
\@ifpackageloaded{aaai}
{\setboolean{detectedAAAI}{true}}        
{\@ifpackageloaded{aaai18}
  {\setboolean{detectedAAAI}{true}}
  {\@ifpackageloaded{aaai20}       
    {\setboolean{detectedAAAI}{true}}
    {\setboolean{detectedAAAI}{false}}}}

\@ifpackageloaded{ijcai18}
{\setboolean{detectedIJCAI}{true}}        
{\@ifpackageloaded{ijcai19}
  {\setboolean{detectedIJCAI}{true}}        
  {\setboolean{detectedIJCAI}{false}}}

\@ifclassloaded{sciposter}
{\setboolean{detectedPoster}{true}}
{\setboolean{detectedPoster}{false}}

\@ifclassloaded{article}
{\setboolean{detectedArticle}{true}}
{\setboolean{detectedArticle}{false}}

\makeatother

\ifthenelse{\not \isundefined{\examen} 
  \and \not \isundefined{\disputationsdatum} 
  \and \not \isundefined{\disputationslokal}}   
  {\setboolean{detectedThesis}{true}}
  {\setboolean{detectedThesis}{false}}

%%%
%%% Not entirely sure whether detectedReport is set correctly in view
%%% of later updates [Jakob Nordström, May 16, 2016]
%%%
           
\ifthenelse{\boolean{detectedArticle} \or \boolean{detectedThesis}
  \or \boolean{detectedSTOC}    \or \boolean{detectedFOCS}
  \or \boolean{detectedSIAM}    \or \boolean{detectedIEEE}
  \or \boolean{detectedACMconf} \or \boolean{detectedACM}
  \or \boolean{detectedPoster}}
{\setboolean{detectedReport}{false}}
{\setboolean{detectedReport}{true}}



%%%%
%%%% THIS FILE IS INTENDED TO BE READ-ONLY --- PLEASE DO NOT EDIT.
%%%% PLEASE CONTACT JAKOB NORDSTRÖM AT jakobn@kth.se REGARDING ANY ISSUES.
%%%%

% GENERAL MACROS TO USE IN LaTeX-FILES
%======================================
%
% AUTHOR
%   Jakob Nordström <jakobn@kth.se>
%   Some improvements added by Marc Vinyals <vinyals@kth.se>
%
% VERSION
%   Last updated June 7, 2019
%    
% KNOWN ISSUES:
%   References with page numbers such as \refsecP, \refthP, etc will not
%   work with the Elsevier and SIAM document classes. No work-arounds
%   have been added, so compilation will fail if these macros are used
%   for Elsevier or SIAM articles.

%
% MACRO NAMING CONVENTION FOR MATHEMATICAL MACROS WITH DELIMITERS 
%-----------------------------------------------------------------
%
% For mathematical macros with delimiters there are usually
% three different flavours corresponding to different sizes of 
% the delimiters as follows:
%
%    \newcommand{\mycommand}[1]{<command> ( {#1} )}
%    \newcommand{\Mycommand}[1]{<command> \bigl( {#1} \bigr)}
%    \newcommand{\MYCOMMAND}[1]{<command> \left( {#1} \right)}
%

%
% REQUIRED PACKAGES
%-------------------
%

\usepackage{ifthen}
\usepackage{xspace}
% varioref does not seem to mix well with the Elsevier document class
\ifthenelse
{\boolean{detectedElsevier} \or \boolean{detectedSIAM} 
  \or \boolean{detectedLIPIcs}}
{}
{\usepackage{varioref}}


%
% M I S C E L L A N E O U S 
%---------------------------
%

\DeclareMathAlphabet{\mathsfsl}{OT1}{cmss}{m}{sl}


%
% G E N E R A L   F O R M A T T I N G   R U L E S 
%-------------------------------------------------
%
% To achieve some kind of consistency in the notation
%

% Format of functions to integers or real numbers
\newcommand{\formatfunctiontonumbers}[1]{\mathrm{#1}}

% Format of functions to sets
\newcommand{\formatfunctiontoset}[1]{\mathit{#1}}

% Dots in x_1 \lor ... \lor x_n and the like
% (make a generic macro that can be changed according to publisher
% requirements)  
\newcommand{\formuladots}{\cdots}


%
% B I G - O H   N O T A T I O N 
%-------------------------------
%

\newcommand{\BIGOH}[1]{\mathrm{O} \left( #1 \right)}
\newcommand{\Bigoh}[1]{\mathrm{O} \bigl( #1 \bigr)}
\newcommand{\bigoh}[1]{\mathrm{O} ( #1 )}
\newcommand{\LITTLEOH}[1]{\mathrm{o} \left( #1 \right)}
\newcommand{\Littleoh}[1]{\mathrm{o} \bigl( #1 \bigr)}
\newcommand{\littleoh}[1]{\mathrm{o} ( #1 )}
\newcommand{\BIGTHETA}[1]{\Theta \left( #1 \right)}
\newcommand{\Bigtheta}[1]{\Theta \bigl( #1 \bigr)}
\newcommand{\bigtheta}[1]{\Theta ( #1 )}
\newcommand{\BIGOMEGA}[1]{\Omega \left( #1 \right)}
\newcommand{\Bigomega}[1]{\Omega \bigl( #1 \bigr)}
\newcommand{\bigomega}[1]{\Omega ( #1 )}
\newcommand{\LITTLEOMEGA}[1]{\omega \left( #1 \right)}
\newcommand{\Littleomega}[1]{\omega \bigl( #1 \bigr)}
\newcommand{\littleomega}[1]{\omega ( #1 )}
\newcommand{\POLYBOUND}[1]{\mathrm{poly} \left( #1 \right)}
\newcommand{\Polybound}[1]{\mathrm{poly} \bigl( #1 \bigr)}
\newcommand{\polybound}[1]{\mathrm{poly} ( #1 )}
\newcommand{\POLYLOGBOUND}[1]{\mathrm{polylog} \left( #1 \right)}
\newcommand{\Polylogbound}[1]{\mathrm{polylog} \bigl( #1 \bigr)}
\newcommand{\polylogbound}[1]{\mathrm{polylog} ( #1 )}
            
\DeclareMathOperator{\polylog}{polylog}


%
% G E N E R A L  M A T H E M A T I C A L   N O T A T I O N
%----------------------------------------------------------
%

% N, Z, Q, R as symbols for classes of numbers
\ifthenelse{\boolean{detectedToC}}{}
{
  \newcommand{\Q}         {\mathbb{Q}}
  \newcommand{\R}         {\mathbb{R}}
  \newcommand{\Rplus}     {\mathbb{R}^{+}}
  \newcommand{\N}         {\mathbb{N}}
  \newcommand{\Nplus}     {\mathbb{N}^{+}}
  \newcommand{\Nzero}     {\mathbb{N}_{0}}
  \newcommand{\Z}         {\mathbb{Z}}
}

% Sigma sum sign with indices to the right, not below and above
\newcommand{\sumnodisplay}{{\textstyle \sum}}

% Absolute value and norm
\providecommand{\abs}[1]{\lvert#1\rvert}
\providecommand{\Abs}[1]{\bigl\lvert#1\bigr\rvert}
\providecommand{\ABS}[1]{\left\lvert#1\right\rvert}
\providecommand{\norm}[1]{\lVert#1\rVert}
\providecommand{\Norm}[1]{\bigl\lVert#1\bigr\rVert}
\providecommand{\NORM}[1]{\left\lVert#1\right\rVert}

% Exists unique
\newcommand{\existsunique}{\exists!}

% Rounding
\newcommand{\ceiling}[1]{\lceil #1 \rceil}
\newcommand{\Ceiling}[1]{\bigl \lceil #1 \bigr \rceil}
\newcommand{\CEILING}[1]{\left \lceil #1 \right \rceil}

\newcommand{\floor}[1]{\lfloor #1 \rfloor}
\newcommand{\Floor}[1]{\bigl \lfloor #1 \bigr \rfloor}
\newcommand{\FLOOR}[1]{\left \lfloor #1 \right \rfloor}

\newcommand{\intpart}[1]{\lceil #1 \rfloor}
\newcommand{\Intpart}[1]{\bigl \lceil #1 \bigr \rfloor}
\newcommand{\INTPART}[1]{\left \lceil #1 \right \rfloor}

% Max and min
% Don't use \maxof and \minof to avoid conflict with calc package
\newcommand{\MAXOFEXPR}[2][]{\max_{#1} \left\{ #2 \right\}}
\newcommand{\MINOFEXPR}[2][]{\min_{#1} \left\{ #2 \right\}}
\newcommand{\Maxofexpr}[2][]{\max_{#1} \bigl\{ #2 \bigr\}}
\newcommand{\Minofexpr}[2][]{\min_{#1} \bigl\{ #2 \bigr\}}
\newcommand{\maxofexpr}[2][]{\max_{#1} \{ #2 \}}
\newcommand{\minofexpr}[2][]{\min_{#1} \{ #2 \}}

\newcommand{\maxofset}[3][:]{\max \{ #2 #1 #3 \}}
\newcommand{\minofset}[3][:]{\min \{ #2 #1 #3 \}}
 
\newcommand{\MAXOFSET}[3][:]%
     {\ifthenelse{\equal{#1}{;}}%
     {\MAXOFEXPR{ #2 \,;\, #3 }}
     {\ifthenelse{\equal{#1}{:}}%
     {\MAXOFEXPR{ #2 \,:\, #3 }}
     {\max \twincommandJN{\left\{}{#2}{\left#1}{\right}{\,#3}{\right\}}}}}
\newcommand{\MINOFSET}[3][:]%
     {\ifthenelse{\equal{#1}{;}}%
     {\MINOFEXPR{ #2 \,;\, #3 }}
     {\ifthenelse{\equal{#1}{:}}%
     {\MINOFEXPR{ #2 \,:\, #3 }}
     {\min \twincommandJN{\left\{}{#2}{\left#1}{\right}{\,#3}{\right\}}}}}

\newcommand{\Maxofset}[3][:]%
     {\ifthenelse{\equal{#1}{;}}%
     {\Maxofexpr{ #2 \,;\, #3 }}
     {\ifthenelse{\equal{#1}{:}}%
     {\Maxofexpr{ #2 \,:\, #3 }}
     {\max \twincommandJN{\bigl\{}{#2}{\bigl#1}{\bigr}{\,#3}{\bigr\}}}}}
\newcommand{\Minofset}[3][:]%
     {\ifthenelse{\equal{#1}{;}}%
     {\Minofexpr{ #2 \,;\, #3 }}
     {\ifthenelse{\equal{#1}{:}}%
     {\Minofexpr{ #2 \,:\, #3 }}
     {\min \twincommandJN{\bigl\{}{#2}{\bigl#1}{\bigr}{\,#3}{\bigr\}}}}}


%
% A L G E B R A
%---------------
%

% Some linear algebra
\newcommand{\transpose}[1]{\ensuremath{#1^{\top}}}
\newcommand{\innerproduct}[2]{\langle #1, #2 \rangle}
\newcommand{\Innerproduct}[2]{\bigl\langle #1, #2 \bigr\rangle}
\newcommand{\INNERPRODUCT}[2]{\left\langle #1, #2 \right\rangle}

% Generic field
\newcommand{\fieldstd}{\mathbb{F}}
\newcommand{\fieldf}{\mathbb{F}}
\newcommand{\F}{\mathbb{F}}

% Finite fields
\newcommand{\GF}[1]{\mathrm{GF} ( #1 )}
\newcommand{\gf}[1]{\mathrm{GF} ( #1 )}
\newcommand{\GFmul}[1]{\mathrm{GF} ( #1 )^{*}}
\newcommand{\gfmul}[1]{\mathrm{GF} ( #1 )^{*}}


%
% P R O B A B I L I T Y   T H E O R Y 
%-------------------------------------

% AMS-TeX defines an operator name \Pr
\DeclareMathOperator{\Expop}{E}
\DeclareMathOperator{\Varianceop}{Var}

% Probability
\newcommand{\PROB}[2][]{\Pr_{#1} \left[ #2 \right]}
\newcommand{\Prob}[2][]{\Pr_{#1} \bigl[ #2 \bigr]}
\ifthenelse{\boolean{detectedLMCS}}
{\renewcommand{\prob}[2][]{\Pr_{#1} [ #2 ]}}
{\newcommand{\prob}[2][]{\Pr_{#1} [ #2 ]}}

% Expectation
\newcommand{\EXPECTATION}[2][]{\Expop_{#1} \left[ #2 \right]}
\newcommand{\Expectation}[2][]{\Expop_{#1} \bigl[ #2 \bigr]}
\newcommand{\expectation}[2][]{\Expop_{#1} [ #2 ]}
\newcommand{\VARIANCE}[1]{\Varianceop \left( #1 \right)}
\newcommand{\Variance}[1]{\Varianceop \bigl( #1 \bigr)}
\newcommand{\variance}[1]{\Varianceop ( #1 )}

% 
% INTERLUDE: MATCHING MIDDLE SEPARATORS (FROM THE UK TeX FAQ)
% 
% 
% One of the few glaring omissions from TeX's mathematical typesetting
% capabilities is a means of setting separators in the middle of
% mathematical expressions. In all sorts of mathematical enterprises one
% may find oneself needing a \middle command, to be used in expressions
% like \left\{ x \in \mathbb{N} \middle| x \mbox{ even} \right\} to
% specify the set of even natural numbers. The e-TeX system defines just
% such a command, but users of Knuth's original need some support.
% Donald Arseneau's braket package provides commands for set
% specifications (as above) and for Dirac brackets (and bras and kets).
% The package uses the e-TeX built-in command if it finds itself running
% under e-TeX.
% 
% See ftp://cam.ctan.org/tex-archive/macros/latex/contrib/misc/braket.sty .
% 
% Or one can do as below.
%

\newcommand{\twincommandJN}[6]%
    {#1#2#3\vphantom{#2#5}\mspace{-2.05mu}#4.#5#6}

% Perhaps this is superfluous---in text mode there is no need for measuring
% with \vphantom, I think, since \bigl[ and \bigr] are what they are 
% independent of what is inside (are they not?).
%
% The length -2.25mu probably should be set instead by doing sth like
%    
%    \newlength{\lengthJN}
%    \settowidth{\lengthJN}{$\left.\right.$}
%    \setlength{\lengthJN}{0.5\lengthJN}
%
% and then using \mspace{-\lengthJN}, but the difference appears to be
% very small so I have not implemented this.


% CONDITIONAL EXPECTATION
\newcommand{\condexp}[2]{\Expop{#1  \mid  #2}}
\newcommand{\CondExp}[2]%
    {\Expop\twincommandJN{\bigl[}{#1}{\bigl|}{\bigr}{\,#2}{\bigr]}}
\newcommand{\CONDEXP}[2]%
     {\Expop\twincommandJN{\left[}{#1}{\left|}{\right}{\,#2}{\right]}}

% CONDITIONAL PROBABILITY
\newcommand{\condprob}[3][]{\prob[#1]{#2  \mid  #3}}
\newcommand{\Condprob}[3][]%
    {\Pr_{#1}\twincommandJN{\bigl[}{#2}{\bigl|}{\bigr}{\,#3}{\bigr]}}
\newcommand{\CONDPROB}[3][]%
    {\Pr_{#1}\twincommandJN{\left[}{#2}{\left|}{\right}{\,#3}{\right]}}

%
% Example code:
%    
%    \begin{displaymath}
%     \CONDEXP{\sum_{i=1}^kX_i}{Z}\quad\mbox{and}\quad%
%     \CONDPROB{B\land C}{\bigwedge_{i\in S}A_i}
%    \end{displaymath}
%    
%    $\CondExp{\sum_{i=1}^k X_i}{Z}$  
%    and
%    $\CondProb{B\land C}{\bigwedge_{i\in S}A_i}$
%    


%
% F U N C T I O N S
%-------------------
%

% DESCRIPTION OF FUNCTION
\newcommand{\funcdescr}[3]{\ensuremath{ #1 : #2 \to #3}}

% DOMAIN
\newcommand{\domainof}[1]{\ensuremath{\mathrm{dom} ( #1 )}}
\newcommand{\Domainof}[1]{\ensuremath{\mathrm{dom}\bigl( #1 \bigr)}}

% INVERSE IMAGE
\newcommand{\invimageof}[2]{{\ensuremath{{#1}^{-1} \left( #2 \right)}}}

%
% G R A P H S
%-------------
%

\newcommand{\edges}[1]{E( #1 )}
\newcommand{\Edges}[1]{E\bigl( #1 \bigr)}
\newcommand{\vertices}[1]{V( #1 )}
\newcommand{\Vertices}[1]{V\bigl( #1 \bigr)}

\newcommand{\vdegree}[2][]{\mathrm{deg}_{#1}(#2)}
\newcommand{\Vdegree}[2][]{\mathrm{deg}_{#1}\bigl(#2\bigr)}
\newcommand{\vneighbour}[2][]{N_{#1}({#2})}
\newcommand{\Vneighbour}[2][]{N_{#1}\bigl({#2}\bigr)}

% Boundary
\newcommand{\boundary}[1]{\ensuremath{\partial #1}}

\newcommand{\pathstd}{\ensuremath{P}}
\newcommand{\pathalt}{\ensuremath{Q}}
\newcommand{\pathfromto}[3]{#1 : #2 \rightsquigarrow #3}


%
% S E T S   A N D   T U P L E S
%-------------------------------
%

\newcommand{\set}[1]{\{ #1 \}}
\newcommand{\Set}[1]{\bigl\{ #1 \bigr\}}
\newcommand{\SET}[1]{\left\{ #1 \right\}}

\newcommand{\setdescr}[3][\mid]{\set{ #2 #1 #3 }}
\newcommand{\Setdescr}[3][|]%
     {\ifthenelse{\equal{#1}{;}}%
     {\Set{ #2 \,;\, #3 }}
     {\ifthenelse{\equal{#1}{:}}%
     {\Set{ #2 \,:\, #3 }}
     {\twincommandJN{\bigl\{}{#2\,}{\bigl#1}{\bigr}{\,#3}{\bigr\}}}}}
\newcommand{\SETDESCR}[3][|]%
     {\twincommandJN{\left\{}{#2\,}{\left#1}{\right}{\,#3}{\right\}}}

\newcommand{\setbrackets}[1]{[ #1 ]}
\newcommand{\Setbrackets}[1]{\bigl[ #1 \bigr]}
\newcommand{\SETBRACKETS}[1]{\left[ #1 \right]}

\newcommand{\setdescrbrackets}[3][\mid]{{\setbrackets{ #2 #1 #3 }}}
\newcommand{\Setdescrbrackets}[3][|]%
     {\twincommandJN{\bigl[}{#2}{\bigl#1}{\bigr}{\,#3}{\bigr]}}
\newcommand{\SETDESCRBRACKETS}[3][|]%
     {\twincommandJN{\left[}{#2}{\left#1}{\right}{\,#3}{\right]}}

\newcommand{\SETSIZE}[1]{\left\lvert#1\right\rvert}
\newcommand{\Setsize}[1]{\bigl\lvert#1\bigr\rvert}
\newcommand{\setsize}[1]{\lvert#1\rvert}

% Set complement
\newcommand{\setcompl}[1]{\overline{#1}}

% Intersection and union
\newcommand{\intersection}{\cap}
\newcommand{\Intersection}{\bigcap}
\newcommand{\Intersectionnodisplay}{\textstyle \bigcap}

\newcommand{\union}{\cup}
\newcommand{\Union}{\bigcup}
\newcommand{\Unionnodisplay}{\textstyle \bigcup}

% Intersection and union with some space
\newcommand{\unionSP}{\, \union \, }
\newcommand{\intersectionSP}{\, \intersection \, }

% Disjoint union 
%
% Can see no difference between below definition and \mathbin{\dot{\cup}}
\newcommand{\disjointunion}{\overset{.}{\cup}}
\newcommand{\disjointunionSP}{\disjointunion}
\newcommand{\Disjointunion}{\overset{.}{\bigcup}}
\newcommand{\disjunion}{\disjointunion}
\newcommand{\Disjunion}{\Disjointunion}

% First n positive integers
\newcommand{\nset}[1]{[{#1}]}
\newcommand{\Nset}[1]{\bigl[ {#1} \bigr]}


%
% L O G I C
%-----------
%

%
% Logic connectives
%
% Logic or is \lor. Logic and is \land. Logic not is \lnot.
% They can be used in math mode only.
\newcommand{\Lor}{\bigvee}
\newcommand{\Land}{\bigwedge}

% nodisplay = indices to the right, not below and above
\newcommand{\Lornodisplay}{{\textstyle \bigvee}}
\newcommand{\Landnodisplay}{{\textstyle \bigwedge}}

\newcommand{\limpl}{\rightarrow}
\newcommand{\lequiv}{\leftrightarrow}

% Prefixed NOT (pfnot):  \lnot x
% Overlined NOT (olnot): \overline{x}
% \stdnot{lit} is the standard NOT notation for variables and literals

\newcommand{\pfnot}[1]{\lnot #1}
\newcommand{\olnot}[1]{\overline{#1}}
\newcommand{\stdnot}[1]{\olnot{#1}}
    
% syntactic equivalence
\newcommand{\synteq}{\doteq}

% constants "true" and "false"
\newcommand{\FALSE}{\mathit{FALSE}}
\newcommand{\TRUE}{\mathit{TRUE}}

\newcommand{\false}{\bot}
\newcommand{\true}{\top}

\newcommand{\falsenum}{0}
\newcommand{\truenum}{1}

%
% Notation and terms for CNF/DNF formulas
%
% Standard notation for parameters in k-CNF formulas
\newcommand{\nvar}{n}
\newcommand{\nvars}{\nvar}
\newcommand{\nclause}{m}
\newcommand{\nclauses}{\nclause}
\newcommand{\clwidth}{k}
\newcommand{\density}{\Delta}

% Formatting of k-CNF/k-DNF in running text
\newcommand{\xdnf}[1]{\mbox{\ensuremath{#1}-DNF}\xspace}
\newcommand{\kdnf}{\xdnf{\clwidth}}
\newcommand{\xdnfform}[1]{\mbox{\ensuremath{#1}-DNF} formula\xspace}
\newcommand{\kdnfform}{\xdnfform{\clwidth}}

\newcommand{\xcnf}[1]{\mbox{\ensuremath{#1}-CNF}\xspace}
\newcommand{\kcnf}{\xcnf{\clwidth}}
\newcommand{\xcnfform}[1]{\mbox{\ensuremath{#1}-CNF} formula\xspace}
\newcommand{\kcnfform}{\xcnfform{\clwidth}}
\newcommand{\xclause}[1]{\mbox{\ensuremath{#1}-clause}\xspace}
\newcommand{\kclause}{\xclause{\clwidth}}

\newcommand{\Excnf}[1]{\mbox{\ensuremath{\mathrm{E}#1}-CNF}\xspace}
\newcommand{\Ekcnf}{\Excnf{\clwidth}}
\newcommand{\Excnfform}[1]{\mbox{\ensuremath{\mathrm{E}#1}-CNF} formula\xspace}
\newcommand{\Ekcnfform}{\Excnfform{\clwidth}}

\newcommand{\xterm}[1]{\mbox{\ensuremath{#1}-term}\xspace}
\newcommand{\kterm}{\xterm{\clwidth}}
\newcommand{\Exterm}[1]{\mbox{\ensuremath{\mathrm{E}#1}-term}\xspace}
\newcommand{\Ekterm}{\Exterm{\clwidth}}

%
% Distributions for random k-CNF formulas
%
% fixed # random clauses chosen with replacement
% \randkcnfnclwrepl{clause width}{#variables}{#clauses}
%
\newcommand{\randkcnfnclwrepl}[3][\clwidth]%
        {\ensuremath{\mathcal{F}^{#2, #3}_{#1}}}
\newcommand{\randkcnfnclwreplstd}% 
        {\randkcnfnclwrepl{\clwidth}{\nvar}{\nclause}}


%
% C O M P L E X I T Y   T H E O R Y 
%-----------------------------------
%

% NOTATION FOR LANGUAGES/PROBLEMS
\newcommand{\problemlanguageformat}[1]{\textsc{#1}\xspace}
\newcommand{\problemlanguageformatnospace}[1]{\textsc{#1}}
\newcommand{\langstd}{\ensuremath{L}}
\newcommand{\langcompl}[1]{\ensuremath{\overline{#1}}}

\newcommand{\PROP}{\problemlanguageformat{PROP}}
\newcommand{\TAUT}{\problemlanguageformat{TAUT}}
\newcommand{\TAUTOLOGY}{\problemlanguageformat{Tautology}}
\newcommand{\SAT}{\problemlanguageformat{Sat}}
\newcommand{\CNFSAT}{\problemlanguageformat{CnfSat}}
\newcommand{\SATISFIABILITY}{\problemlanguageformat{Satisfiability}}
\newcommand{\THREESAT}{\text{$3$-}\problemlanguageformat{Sat}}
\newcommand{\TWOSAT}{\text{$2$-}\problemlanguageformat{Sat}}
\newcommand{\ONEINTHREESAT}{\text{$1$-}\problemlanguageformatnospace{In}%
        \text{-$3$-}\problemlanguageformat{Sat}}
\newcommand{\MAXSAT}{\problemlanguageformat{MaxSat}}
\newcommand{\MAXTWOSAT}{\problemlanguageformatnospace{Max}%
        \text{-$2$-}\problemlanguageformat{Sat}}
\newcommand{\CIRCUITSAT}{\problemlanguageformat{CircuitSat}}
\newcommand{\NAESAT}{\problemlanguageformat{NotAllEqualSat}}
\newcommand{\DOMINATINGSET}{\problemlanguageformat{DominatingSet}}
\newcommand{\VERTEXCOVER}{\problemlanguageformat{VertexCover}}
\newcommand{\MAXCUT}{\problemlanguageformat{MaxCut}}
\newcommand{\DOMATICNUMBER}{\problemlanguageformat{DomaticNumber}}
\newcommand{\MONOCHROMTRI}{\problemlanguageformat{MonochromaticTriangle}}
\newcommand{\CLIQUECOVER}{\problemlanguageformat{CliqueCover}}
\newcommand{\INDSET}{\problemlanguageformat{IndependentSet}}
\newcommand{\GCLIQUE}{\problemlanguageformat{Clique}}
\newcommand{\GCOLOURING}{\problemlanguageformat{Colouring}}
\newcommand{\SUBGRAPHISO}{\problemlanguageformat{Subgraph\-Isomorphism}}
\newcommand{\GKERNEL}{\problemlanguageformat{Kernel}}
\newcommand{\MINMAXMATCHING}{\problemlanguageformat{MinMaxMatching}}
\newcommand{\CUBICSUBGRAPH}{\problemlanguageformat{CubicSubgraph}}

% NOTATION FOR COMPLEXITY CLASSES
\newcommand{\complclassformat}[1]%
        {\textrm{\upshape{\textsf{#1}}}\xspace}
\newcommand{\cocomplclass}[1]%
        {\textrm{\upshape{\textsf{co#1}}}\xspace}

\newcommand{\TIMEclass}[1]{\ensuremath{\complclassformat{TIME}\bigl(#1\bigr)}}
\newcommand{\SPACEclass}[1]{\ensuremath{\complclassformat{SPACE}\bigl(#1\bigr)}}
\newcommand{\DTIMEclass}[1]{\ensuremath{\complclassformat{DTIME}\bigl(#1\bigr)}}
\newcommand{\DTIMEadviceclass}[2]%
    {\ensuremath{\complclassformat{DTIME}\bigl(#1\bigr)/{#2}}}
\newcommand{\Pclass}{\complclassformat{P}}
\newcommand{\NP}{\complclassformat{NP}}
\newcommand{\NPclass}{\NP}
\newcommand{\coNP}{\cocomplclass{NP}}
\newcommand{\coNPclass}{\coNP}
\newcommand{\CoNP}{\coNP}
\newcommand{\Ppoly}{\complclassformat{P/poly}}
\newcommand{\Pslashpoly}{\Ppoly}
\newcommand{\Pslashpolyclass}{\Pslashpoly}
\newcommand{\PSPACE}{\complclassformat{PSPACE}}
\newcommand{\PSPACEclass}{\PSPACE}
\newcommand{\NPSPACE}{\complclassformat{NPSPACE}}
\newcommand{\NPSPACEclass}{\NPSPACE}
\newcommand{\pspace}{\PSPACE}
\newcommand{\EXPTIME}{\complclassformat{EXPTIME}}
\newcommand{\exptime}{\complclassformat{EXPTIME}}
\newcommand{\EXPSPACE}{\complclassformat{EXPSPACE}}
\newcommand{\expspace}{\complclassformat{EXPSPACE}}
\newcommand{\EXP}{\complclassformat{EXP}}
\newcommand{\NEXP}{\complclassformat{NEXP}}
\newcommand{\coNEXP}{\cocomplclass{NEXP}}
\newcommand{\Lclass}{\complclassformat{L}}
\newcommand{\LOGSPACE}{\complclassformat{L}}
\newcommand{\NL}{\complclassformat{NL}}
\newcommand{\NLclass}{\NL}
\newcommand{\coNL}{\cocomplclass{NL}}
\newcommand{\NCclass}{\complclassformat{NC}}
\newcommand{\DP}{\complclassformat{DP}}
\newcommand{\DPclass}{\DP}
\newcommand{\SIGMACLASS}[1]{\ensuremath{\Sigma^p_{#1}}}
\newcommand{\Sigmaclass}[1]{\SIGMACLASS{#1}}
\newcommand{\PICLASS}[1]{\ensuremath{\Pi^p_{#1}}}
\newcommand{\Piclass}[1]{\PICLASS{#1}}
\newcommand{\PH}{\complclassformat{PH}}
\newcommand{\EXPEXP}{\complclassformat{EXPEXP}}
\newcommand{\NEXPEXP}{\complclassformat{NEXPEXP}}
\newcommand{\coNEXPEXP}{\cocomplclass{NEXPEXP}}        
\newcommand{\BPP}{\complclassformat{BPP}}
\newcommand{\ZPP}{\complclassformat{ZPP}}
\newcommand{\RP}{\complclassformat{RP}}
\newcommand{\coRP}{\cocomplclass{RP}}
\newcommand{\MIParg}[1]{\ensuremath{\complclassformat{MIP}[#1]}}
\newcommand{\MIP}{\complclassformat{MIP}}
\newcommand{\IP}{\complclassformat{IP}}
\newcommand{\PCPnot}{\complclassformat{PCP}}
% PCP{completeness}{soundness}{randomness}{queries}{alphabet size}
\newcommand{\PCPalph}[5]%
    {\ensuremath{\complclassformat{PCP}_{{#1},{#2}}[{#3}, {#4}, {#5}]}}
\newcommand{\PCP}[4]%
    {\ensuremath{\complclassformat{PCP}_{{#1},{#2}}[{#3}, {#4}]}}

\newcommand{\SAC}{\complclassformat{SAC}}
\newcommand{\NC}{\complclassformat{NC}}
\newcommand{\ACzero}{\ensuremath{\complclassformat{AC}^{0}}}
\newcommand{\SC}{\complclassformat{SC}}
\newcommand{\TISP}{\complclassformat{TISP}}
\newcommand{\LOGCFL}{\complclassformat{LOGCFL}}


%
% T E X T   F O R M A T T I N G
%-------------------------------
%

% introducing a new term or mentioning a familiar one for the first time
%\newcommand{\introduceterm}[1]{{\textsl{#1}}}
\newcommand{\introduceterm}[1]{{\emph{#1}}}

% Spacing before punctuation in displayed equations according to LLNCS
% (But SIAM disagrees, so it is convenient to have a macro for this.)
\newcommand{\eqperiod}{\enspace .}
\newcommand{\eqcomma}{\enspace ,}

% For description lists with items in italics (to avoid confusion
% with theorem headers, for instance)
\newcommand{\italicitem}[1][]{\item[\textit{#1}]}


%
% A B B R E V I A T I O N S   O F   F R E Q U E N T     E X P R E S S I O N S
%-----------------------------------------------------------------------------
%

% GENERAL EXPRESSIONS
\newcommand{\wrt}{with respect to\xspace}
\newcommand{\wrtabbrev}{w.r.t.\ }
\ifthenelse{\boolean{detectedToC}}{}
  {\newcommand{\eg}{for instance\xspace} % should be surrounded by commas 
    \newcommand{\Eg}{For instance\xspace}}
\ifthenelse{\boolean{detectedToC}}{}{
\newcommand{\ie}{i.e.,\ }
\newcommand{\Ie}{I.e.,\ }
}
\newcommand{\ieComma}{i.e.,\ }  %%%% Special hack to use when adapting to ToC
\newcommand{\ieNoComma}{i.e.\ }
\ifthenelse{\boolean{detectedLIPIcs} \or \boolean{detectedIJCAI}}
{\renewcommand{\st}{\errmessage{Please do not use st}}}
{\ifthenelse{\isundefined{\st}}
  {\newcommand{\st}{such that\xspace}}
  {}}      
%   \newcommand{\st}{such that\xspace}}

\newcommand{\etal}{et al.\@\xspace}
\newcommand{\etalS}{et al\@. }

\newcommand{\vs}{vs.\ }

% TYPICAL MATHEMATICAL EXPRESSIONS
\newcommand{\ifaoif}{if and only if\xspace}
\newcommand{\wolog}{without loss of generality\xspace}
\newcommand{\Wolog}{Without loss of generality\xspace}
\newcommand{\iid}{independently and identically distributed\xspace}
\ifthenelse{\boolean{detectedIEEE}}{}{\newcommand{\QED}{Q.E.D.}}
\newcommand{\qedlong}{which was to be proved\xspace}

\newcommand{\whp}{with high probability\xspace}
\newcommand{\Whp}{With high probability\xspace}
\newcommand{\aas}{asymptotically almost surely\xspace}
\newcommand{\Aas}{Asymptotically almost surely\xspace}


%
% R E F E R E N C E S 
%---------------------
%
% Macros with capital initial letter intended for use at start of sentence.
%

%
% REFERENCES TO SECTIONAL UNITS AND PAGE INTERVALS
%
% The LLNCS document class and instructions says that in the running text, 
% but not at the beginning of a sentence, Sect., Chap. and Fig. should be 
% abbreviated.
%
% LLNCS wants, e.g., "Section 4.3" with capital S, SIAM doesn't.
%

% Sections
\newcommand{\refsec}[1]{Section~\ref{#1}}
\newcommand{\refsecP}[1]{Section~\vref{#1}}
\newcommand{\Refsec}[1]{Section~\ref{#1}}
\newcommand{\RefsecP}[1]{Section~\vref{#1}}
\newcommand{\reftwosecs}[2]{Sections~\ref{#1} and~\ref{#2}}
\newcommand{\refthreesecs}[3]{Sections~\ref{#1}, \ref{#2}, and~\ref{#3}}

% Chapters
\newcommand{\refch}[1]{Chapter~\ref{#1}}
\newcommand{\Refch}[1]{Chapter~\ref{#1}}
\newcommand{\reftwochs}[2]{Chapters~\ref{#1} and~\ref{#2}}
\newcommand{\refthreechs}[3]{Chapters~\ref{#1}, \ref{#2} and~\ref{#3}}

% Appendices
\newcommand{\refapp}[1]{Appendix~\ref{#1}}
\newcommand{\Refapp}[1]{Appendix~\ref{#1}}
\newcommand{\reftwoapps}[2]{Appendices~\ref{#1} and~\ref{#2}}

% Figures
\newcommand{\reffig}[1]{Figure~\ref{#1}}
\newcommand{\reffigP}[1]{Figure~\vref{#1}}
\newcommand{\Reffig}[1]{Figure~\ref{#1}}
\newcommand{\ReffigP}[1]{Figure~\vref{#1}}
\newcommand{\reftwofigs}[2]{Figures~\ref{#1} and~\ref{#2}}
\newcommand{\refthreefigs}[3]{Figures~\ref{#1}, \ref{#2}, and~\ref{#3}}


%
% REFERENCES TO THEOREM-LIKE ENVIRONMENTS
%
% Requires \usepackage{varioref}
%
% Adapted to LLNCS document class and instructions. Definition, Theorem etc
% should be capitalized when followed by a number.
%

% Theorems
\newcommand{\refth}[1]{Theorem~\ref{#1}}
\newcommand{\reftwoths}[2]{Theorems~\ref{#1} and~\ref{#2}}
\newcommand{\refthreeths}[4][and]{Theorems~\ref{#2}, \ref{#3}, {#1}~\ref{#4}}
\newcommand{\refthm}[1]{Theorem~\ref{#1}}
\newcommand{\reftwothms}[2]{Theorems~\ref{#1} and~\ref{#2}}
\newcommand{\refthreethms}[4][and]{Theorems~\ref{#2}, \ref{#3}, {#1}~\ref{#4}}

% Lemmas
\newcommand{\reflem}[1]{Lemma~\ref{#1}}
\newcommand{\reftwolems}[2]{Lemmas~\ref{#1} and~\ref{#2}}
\newcommand{\refthreelems}[4][and]{Lemmas~\ref{#2}, \ref{#3}, {#1}~\ref{#4}}

% Propositions
\newcommand{\refpr}[1]{Proposition~\ref{#1}}
\newcommand{\reftwoprs}[2]{Propositions~\ref{#1} and~\ref{#2}}

% Corollaries
\newcommand{\refcor}[1]{Corollary~\ref{#1}}
\newcommand{\reftwocors}[2]{Corollaries~\ref{#1} and~\ref{#2}}

% Definitions
\newcommand{\refdef}[1]{Definition~\ref{#1}}
\newcommand{\reftwodefs}[2]{Definitions~\ref{#1} and~\ref{#2}}
\newcommand{\refthreedefs}[3]{Definitions~\ref{#1}, \ref{#2}, and~\ref{#3}}

% Remarks
\newcommand{\refrem}[1]{Remark~\ref{#1}}
\newcommand{\reftworems}[2]{Remarks~\ref{#1} and~\ref{#2}}

% Observations
\newcommand{\refobs}[1]{Observation~\ref{#1}}
\newcommand{\reftwoobs}[2]{Observations~\ref{#1} and~\ref{#2}}

% Facts
\newcommand{\reffact}[1]{Fact~\ref{#1}}

% Conjectures
\newcommand{\refconj}[1]{Conjecture~\ref{#1}}
\newcommand{\reftwoconjs}[2]{Conjectures~\ref{#1} and~\ref{#2}}

% Examples
\newcommand{\refex}[1]{Example~\ref{#1}}
\newcommand{\reftwoexs}[2]{Examples~\ref{#1} and~\ref{#2}}

% Properties
\newcommand{\refproperty}[1]{Property~\ref{#1}}
\newcommand{\reftwoproperties}[2]{Properties~\ref{#1} and~\ref{#2}}
 
% Claims
\newcommand{\refclaim}[1]{Claim~\ref{#1}}
\newcommand{\reftwoclaims}[2]{Claims~\ref{#1} and~\ref{#2}}

% References at start of sentence
\newcommand{\Refth}[1]{Theorem~\ref{#1}}
\newcommand{\Reflem}[1]{Lemma~\ref{#1}}
\newcommand{\Refpr}[1]{Proposition~\ref{#1}}
\newcommand{\Refcor}[1]{Corollary~\ref{#1}}
\newcommand{\Refdef}[1]{Definition~\ref{#1}}
\newcommand{\Refrem}[1]{Remark~\ref{#1}}
\newcommand{\Refobs}[1]{Observation~\ref{#1}}
\newcommand{\Refconj}[1]{Conjecture~\ref{#1}}
\newcommand{\Refex}[1]{Example~\ref{#1}}
\newcommand{\Refclaim}[1]{Claim~\ref{#1}}

% References with page numbers
\newcommand{\refthP}[1]{Theorem~\vref{#1}}
\newcommand{\reflemP}[1]{Lemma~\vref{#1}}
\newcommand{\refprP}[1]{Proposition~\vref{#1}}
\newcommand{\refcorP}[1]{Corollary~\vref{#1}}
\newcommand{\refdefP}[1]{Definition~\vref{#1}}
\newcommand{\refremP}[1]{Remark~\vref{#1}}
\newcommand{\refobsP}[1]{Observation~\vref{#1}}
\newcommand{\refconjP}[1]{Conjecture~\vref{#1}}
\newcommand{\refexP}[1]{Example~\vref{#1}}
\newcommand{\refpropertyP}[1]{Property~\vref{#1}}

% Some more references
\newcommand{\refrule}[1]{rule~\ref{#1}}
\newcommand{\reftworules}[2]{rules~\ref{#1} and~\ref{#2}}

\newcommand{\refpart}[1]{part~\ref{#1}}
\newcommand{\Refpart}[1]{Part~\ref{#1}}
\newcommand{\reftwoparts}[2]{parts~\ref{#1} and~\ref{#2}}
\newcommand{\Reftwoparts}[2]{Parts~\ref{#1} and~\ref{#2}}

\newcommand{\refitem}[1]{item~\ref{#1}}
\newcommand{\Refitem}[1]{Item~\ref{#1}}
\newcommand{\reftwoitems}[2]{items~\ref{#1} and~\ref{#2}}
\newcommand{\Reftwoitems}[2]{Items~\ref{#1} and~\ref{#2}}

\newcommand{\refcase}[1]{case~\ref{#1}}
\newcommand{\Refcase}[1]{Case~\ref{#1}}
\newcommand{\reftwocases}[2]{cases~\ref{#1} and~\ref{#2}}
\newcommand{\Reftwocases}[2]{Cases~\ref{#1} and~\ref{#2}}

% References to equations (alias for \eqref just for simplicity)
%
% The definition of \refeq overwrites a command in the mathtools package
% if this package has been loaded

\ifthenelse
{\isundefined{\refeq}}
{\newcommand{\refeq}[1]{\eqref{#1}}}
{\renewcommand{\refeq}[1]{\eqref{#1}}}
\newcommand{\refeqP}[1]{\eqref{#1} on page~\pageref{#1}}




%%%
%%% SOME LOCAL MACROS FOR THE IDMA COURSE
%%%

% GCD
%   \DeclareMathOperator{\gcd}{gcd}

% Two-norm
\newcommand{\twonorm}[1]{\lVert#1\rVert_2}
\newcommand{\Twonorm}[1]{\bigl\lVert#1\bigr\rVert_2}
\newcommand{\TWONORM}[1]{\left\lVert#1\right\rVert_2}

% Formal language grammars
%   \newcommand{\produces}{\rightarrow}
%   \newcommand{\terminalf}[1]{\mathtt{#1}}
%   \newcommand{\tokenf}[1]{\text{\textbf{#1}}}
%   \newcommand{\emptystring}{\varepsilon}
%   \newcommand{\numtoken}{\tokenf{num}}

% For checkmark and xmark
\usepackage{pifont}
\newcommand{\xmark}{\ding{55}}

% For formatting induction proofs
\newcommand{\indproofstep}[1]%
{
  \smallskip
  \noindent
  \textbf{\textit{#1:}}
}

\newcommand{\indbase}[1]{\indproofstep{Base case (#1)}}
\newcommand{\indstep}{\indproofstep{Induction step}}
\newcommand{\indclaim}{\indproofstep{Claim}}





% For getting watermark "DRAFT" across all pages (for instance, 
% when posting preliminary version of problem set)
%   \usepackage{draftwatermark}
%   % \SetWatermarkFontSize{20 cm}
%   \SetWatermarkScale{5}

% For METAPOST logo as \hologo{METAPOST}
%   \usepackage{hologo}

% For TikZ
%   \input{Figures/tikz-packages.tex}

%%%
%%% TITLE
%%%

\author{\courseinstructor}
\course{\coursenamelong{}}
\semester{\courseperiod}
\title{\coursenameshort: Problem Set \psetno}

\begin{document}

\maketitle



\begin{abstract}
  \noindent
  \textbf{Due:} \duedate.

  \noindent
  \textbf{Submission:}
  Please submit your solutions
  via \emph{Absalon}
  as a PDF file.
  State your name and e-mail address 
  close to the top   of the first page.
  Solutions should be written in \LaTeX{} or some other math-aware
  typesetting system with reasonable margins on all sides (at least 2.5~cm).
  Please try to be precise and to the point in your solutions and
  refrain from vague statements.
  % Make sure to explain your reasoning.
  Never, ever just state the answer, but always
  make sure to explain your reasoning.
  \emph{Write so that a fellow student of yours can read, understand, and
    verify your solutions.}
  In addition to what is stated below, the general rules 
  for problem sets stated on \emph{Absalon} always apply.

%    
%      \noindent
%      \textbf{Hints:}
%      For most or all problems, ``hints'' can be purchased at a cost of 
%      \mbox{5--10~points}. In this way, you can configure yourself whether
%      you want the problems to be more creative and open-ended, where
%      sometimes a lot can depend on finding the right idea, or whether you
%      want them to be more of guided exercises providing a useful work-out
%      on the concepts of proof complexity. If you do not solve a problem,
%      there is no charge for the hint (i.e., it is not deducted from the
%      score on other problems).  
%    

  \noindent
  \textbf{Collaboration:}
  Discussions of ideas in groups of 
  two to three people 
  are allowed---and indeed, encouraged---but 
  you should always write up your solutions completely on your own,
  from start to finish, and you should understand all aspects of them
  fully. It is not allowed to compose draft solutions together and
  then continue editing individually, or to share any text, formulas,
  or pseudocode. Also, no such material may be downloaded from or
  generated via the internet to be used in draft or final solutions.
  Submitted solutions will be checked for plagiarism.

% 
%     You should also clearly acknowledge any collaboration.
%     State   close to the top 
%     of the first page of your problem set
%     solutions if you have been collaborating with someone and if so with
%     whom.  
%     \emph{Note that collaboration is on a per problem set basis,
%       so you should not discuss different problems on the same problem
%       set with different people.}
%    
%    
%   
%     \noindent
%     \textbf{Reference material:} 
%     Some of the problems are ``classic'' and hence it might be easy to
%     find solutions on the Internet, in textbooks or in research
%     papers. It is not allowed to use such material in any way unless
%     explicitly stated otherwise. Anything said during the lectures or in
%     the lecture notes 
%   %      , or which can be found in chapters of Arora-Barak covered in
%   %      the course,  
%     should be fair game, though, unless you are specifically asked to
%     show something that we claimed without proof in class. All
%     definitions 
%   %      used 
%     should be as given in class
%     or in Arora-Barak 
%     and cannot be substituted by 
%   %      definitions
%     versions
%     from other sources.  It is
%     hard to pin down 100\% watertight formal rules on what all of this
%     means---when in doubt, ask the main instructor.
%     

  \noindent
  \textbf{Grading:}
  A score of 
  \thresholdforpass
  is guaranteed to be enough to pass this problem set.


  \noindent
  \textbf{Questions:}
  Please do not hesitate to ask the instructor or TAs if any problem
  statement is unclear, but please make sure to send private
  messages---sometimes specific enough questions could give away the
  solution to your fellow students, and we want all of you to benefit
  from working on, and learning from, the problems.
%   
  Good luck!
\end{abstract}





%%%
%%% IDMA EXAM 2026: PROBLEM 4
%%% RELATIONS
%%%

\begin{problem}%
  \label{problem:relations}%
  (90 p)  
  Let
  $A = \set{1, 2, 3, 4}$
  and consider the following binary relations on $A$:
  \begin{align*}
    R &= \set{ (2,1), (3,1), (3,2), (4,1), (4,2), (4,3) } \\
    S &= \set{ (1,1), (2,2), (3,3), (4,4) } \\
    T &= \set{ (1,1), (1, 4), (2,2), (2,3), (3,2), (3,3), (4,1), (4,4) }
  \end{align*}

%%% 5 points per correctly identified property
  \begin{subproblem}
    \ifthenelse{\boolean{versionwithsolutions}}
    {(60 p)}
    {\ignorespaces}
    For each of the relations above, determine whether it is
    \begin{compactenum}
    \item
      reflexive,      
    \item
      symmetric,
    \item
      antisymmetric,
    \item
      transitive.
    \end{compactenum}
    Please make sure to explain, briefly but clearly, what these properties
    mean and why they are satisfied for a relation when
    they are. For any relation that
    fails to satisfy a property, make sure to provide a specific
    counterexample.

  \end{subproblem}

  \begin{solution}
    Recall that a relation $R$ is:
    \begin{itemize}
    \item
      \emph{reflexive} if for all $x$ it holds that $(x,x) \in R$;
      
    \item
      \emph{symmetric} if whenever
      $(x,y) \in R$
      it also holds that
      $(y,x) \in R$;

    \item
      \emph{anti-symmetric} if 
      $(x,y) \in R$
      and
      $(y,x) \in R$ implies
      $x=y$;
      
    \item
      \emph{transitive} if whenever
      $(x,y) \in R$
      and
      $(y,z) \in R$
      it also holds that
      $(x,z) \in R$.      
    \end{itemize}

    The relation~$R$ specified in the problem statement
    is anti-symmetric
    (simply since there is no pair $(x, y)$
    such that
    $(x, y) \in R$
    and
    $(y, x) \in R$)
    and transitive
    (which can be verified by case analysis, or by observing that
    $R$ is the greater-than relation).
    It is not reflexive since, e.g., 
    $(1, 1) \notin R$,
    and it is not symmetric since, e.g.,
    $(2, 1) \in R$
    but
    $(1, 2) \notin R$.

    The relation~$S$ is the identity relation,
    which vacuously satisfies all the properties listed.

    The relation~$T$ can be verified to be 
    reflexive, symmetric, and transitive.
    It is not anti-symmetric since
    $(2,3) \in T$
    and
    $(3,2) \in T$
    but
    $2 \neq 3$.    
\end{solution}

%%% 10 points per correctly identified relation
  \begin{subproblem}
    \ifthenelse{\boolean{versionwithsolutions}}
    {(30 p)}
    {\ignorespaces}
    Which of the relations above, if any, are equivalence relations or
    partial orders? Please make sure to justify your answers.
  \end{subproblem}

\begin{solution}
  An \emph{equivalence relation} is a relation that is
  reflexive, symmetric, and transitive.
  The relations $S$ and~$T$ satisfy these conditions
  as argued above.

  A \emph{partial order}
  is a reflexive, anti-symmetric, and transitive relation.
  The identity relation~$S$ is formally speaking also a partial order,
  since it satisfies all the  required properties
  (but it is of course a very boring partial order). 

  It might be worth pointing out that
  the relation~$R$ is \emph{not} a partial order, since it is not reflexive.
  This is just a special case of the general fact that
  non-strict order relations define partial orders but strict order relations
  do not.
\end{solution}
\end{problem}



%%%
%%% IDMA EXAM 2026: PROBLEM 6
%%% TRICKY INDUCTION PROBLEM: GCD OF FIBONACCI NUMBERS
%%%

\begin{problem}%
  (80 p)
  Recall that the Fibonacci numbers are defined as
  \begin{align*}
    F_1 &= 1 \\
    F_2 &= 1 \\
    F_n &= F_{n - 1} +  F_{n - 2} &&\text{for $n \geq 3$.}
  \end{align*}
  %
  Prove that consecutive Fibonacci numbers
  $F_{n+1}$
  and
  $F_{n}$
  are relatively prime, and show that for  $n \geq 2$
  the  Euclidean algorithm when run on $F_{n+1}$ and $F_n$
  makes exactly $n-1$~function calls to determine that this is so
  (i.e., it reaches remainder~$0$ after exactly $n-1$
  function calls).

\begin{solution}
  We prove that two consecutive Fibonacci numbers are relatively
  prime, i.e., that they have greatest common divisor~$1$, by using
  the Euclidean algorithm. While doing so, we count the number of
  function calls, i.e., the number of times the relation
  $
  \gcd (m, n) 
  =
  \gcd (n, m \bmod n) 
  $
  is applied before reaching the trivial base case where $n$
  divides~$m$
  and the remainder is~$0$.

  \begin{description}
  \item[\emph{Base case ($n=2$):}]
    For
    $F_3 = 2$
    and
    $F_2 = 1$,
    we clearly have
    $
    \gcd ( 2, 1 ) = 1
    $.
    If we run the Euclidean algorithm, we get that
    $
    F_3 = 2 \cdot F_2 + 0
    $,
    and so we reach remainder~$0$ after a single step.
    
  \item[\emph{Induction step:}]
    Suppose that for $n-1$ it holds that
    $
    \gcd ( F_{n - 1} , F_{n - 2}) = 1
    $
    and that the Euclidean algorithm reaches remainder~$0$ after
    $(n-1) - 1 = n-2 $ function calls.

    As a first step when computing
    $
    \gcd ( F_{n} ,  F_{n - 1})
    $,
    the Euclidean algorithm divides
    $F_{n}$ by  $F_{n - 1}$
    to compute the remainder.
    This remainder is~$F_{n-2}$, since by the definition of Fibonacci
    numbers we have that
    $F_n = 1 \cdot F_{n - 1} +  F_{n - 2}$,
    and so the equalities
    \begin{equation}
      \gcd ( F_{n} ,  F_{n - 1})
      =
      \gcd ( F_{n-1} , F_n \bmod F_{n - 1})
      =
      \gcd ( F_{n-1} ,  F_{n - 2})
    \end{equation}
    hold.
    By the induction hypothesis,  the Euclidean algorithm
    computes
    $\gcd ( F_{n-1} ,  F_{n - 2}) = 1  $
    with $n-2$ additional recursive calls. 
    Hence, we conclude that the Euclidean algorithm will
    determine that
    $F_n$ and~$F_{n-1}$
    are relatively prime after $n-1$ function calls.
  \end{description}
  The claim in the problem statement now follows by the induction principle.
\end{solution}


\end{problem}



%%%
%%% IDMA EXAM 2026: PROBLEM 7
%%% COMBINATORICS
%%%

\begin{problem}%
  \label{problem:combinatorics}%
  (90 p)
  For a few years now the Copenhagen metropolitan area 
  (including Lund) has had an unusually large number of researchers in 
  computational complexity theory, and a team of such
  researchers have decided to submit a joint grant application  to
  create the
  \emph{Copenhagen Computational Complexity Centre}
  focusing on research in this scientific field.
  Since gender balance is a serious issue in computer science, 
  a noteworthy aspect of the team of co-applicants is that
  the male professors
  Amir, Jakob, and Srikanth at the University of Copenhagen
  are balanced by the female professors
  Nutan and Paloma at the IT University of Copenhagen and
  Susanna at Lund University.

  For the subproblems below, please make sure to answer not just with
  numbers but with more combinatorial-looking expressions, and to expand these
  expressions out to show that you understand the meaning of any
  notation used. Also make sure to explain how you reason to reach
  your answers.

  \begin{subproblem}
    \ifthenelse{\boolean{versionwithsolutions}}
    {(50 p)}
    {\ignorespaces}
    Together with the application documents, the co-applicants are
    planning to enclose a group photo, and much thought has gone into
    how to choose the seating arrangement.
%       for this photo.
    All the researchers will be placed in a single row, but they have
    agreed that a great way to highlight the gender
    balance would be to make sure that male and female researchers
    alternate, so that every second person in the row
    is male or female, respectively.
    In how many different ways can the $6$~researchers be arranged on
    the photo to satisfy this constraint?
  \end{subproblem}

\begin{solution}
  The seating arrangement is uniquely specified by determining whether
  the leftmost person on the photo is male or female, and then by
  specifying the internal order of the female and male researchers,
  respectively.
  This gives us:
  \begin{itemize}
  \item
    $2$ choices for a male or female researcher at the leftmost position;
  \item
    $3! = 6$ ways of arranging the $3$~female researchers
    from left to right;
  \item
    likewise
    $3! = 6$ ways of arranging the $3$~male researchers
    from left to right;
  \end{itemize}
  for a total of
  $
  2 \cdot 3! \cdot 3!
  =
  2 \cdot 6 \cdot 6
  =
  72
  $
  different arrangements.
\end{solution}
  
  \begin{subproblem}
    \ifthenelse{\boolean{versionwithsolutions}}
    {(40 p)}
    {\ignorespaces}
    Any serious research centre application these days should also
    identify a steering committee for the centre.
    After long deliberations, the co-applicants have decided that this
    committee should:
    \begin{compactitem}
    \item
      consist of $4$ persons all in all;

    \item
      include co-applicants representing all $3$ partner
      institutions, i.e, 
      the University of Copenhagen,
      the IT University of Copenhagen,
      and
      Lund University;

    \item
      have perfect gender balance, i.e., two male and two female members.
      
    \end{compactitem}
    %
    In how many different ways can the steering committee be composed?
    
  \end{subproblem}

\begin{solution}
  Since Susanna is the only co-applicant from Lund University, she has
  to be on the steering committee.

  This means that we need exactly one more female committee member,
  who will be from ITU,
  and this gives us $2$~choices for either Nutan or Paloma.

  Finally, we need two male members, who will both have to come from
  the University of Copenhagen.
  We can  think of choosing either two persons among Amir, Jakob, and
  Srikanth in
  $\binom{3}{2} = 3$ ways,
  or choosing one person to leave out in
  $\binom{3}{1} = 3$ ways.

  Summing up (or, rather, multiplying together), we see that the
  steering committee can be composed in
  $
  1 \cdot 2 \cdot \binom{3}{2}
  =
  1 \cdot 2  \cdot 3 = 6
  $
  ways.  
\end{solution}
  
  
\end{problem}



%%%
%%% DMFS 2023 problem set 4 problem 3
%%% Matrix multiplication and eigen vectors (really hard)
%%%


\providecommand{\vecx}{\vec{x}}
\providecommand{\vecy}{\vec{y}}

\begin{problem}%
  \label{problem:graph-eigen}%
  (150 p)
  We have learned in class about matrix multiplication, 
  but there is also a way of multiplying matrices (or vectors) by just
  a number, which is called  \emph{scalar multiplication}.
  To multiply a matrix~$A$ by a number~$c$,
  we multiply each entry~$a_{i,j}$ in the matrix
  with the number~$c$ so that
  \begin{equation*}
    c \cdot A
    =
    c \cdot
    \begin{pmatrix}
      a_{1,1} & a_{1,2} & \cdots & a_{1, n} \\
      a_{2,1} & a_{2,2} & \cdots & a_{2, n} \\
      \vdots & \vdots & \ddots & \vdots \\
      a_{m,1} & a_{m,2} & \cdots & a_{m, n} \\
    \end{pmatrix}    
    =
    \begin{pmatrix}
      c \cdot a_{1,1} & c \cdot a_{1,2} & \cdots & c \cdot a_{1, n} \\
      c \cdot a_{2,1} & c \cdot a_{2,2} & \cdots & c \cdot a_{2, n} \\
      \vdots & \vdots & \ddots & \vdots \\
      c \cdot a_{m,1} & c \cdot a_{m,2} & \cdots & c \cdot a_{m, n} \\
    \end{pmatrix}    
  \end{equation*}
  is the result of the scalar multiplication.

  An intriguing phenomenon that sometimes arises  is that for some pairs of
  matrices and vectors   matrix multiplication and scalar
  multiplication give the same   result. 
  As an example of this, we have
  \begin{equation}
    \label{eq:graph-4-vertices}
    \begin{pmatrix}
      0 & 1 & 0 & 1 \\
      1 & 0 & 1 & 0 \\
      0 & 1 & 0 & 1 \\
      1 & 0 & 1 & 0 
    \end{pmatrix}
    \cdot
    \begin{pmatrix}
      1 \\
      1 \\
      1 \\
      1
    \end{pmatrix}
    =
    \begin{pmatrix}
      2 \\
      2 \\
      2 \\
      2
    \end{pmatrix}
    =
    \ 2 \cdot 
    \begin{pmatrix}
      1 \\
      1 \\
      1 \\
      1
    \end{pmatrix}
  \end{equation}
  and another example is
  \begin{equation}
    \label{eq:graph-6-vertices}
    \begin{pmatrix}
      0 & 1 & 0 & 0 & 1 & 1 \\
      1 & 0 & 1 & 1 & 0 & 0 \\
      0 & 1 & 0 & 1 & 0 & 1 \\
      0 & 1 & 1 & 0 & 1 & 0 \\ 
      1 & 0 & 0 & 1 & 0 & 1 \\ 
      1 & 0 & 1 & 0 & 1 & 0 
    \end{pmatrix}
    \cdot
    \begin{pmatrix}
      1 \\
      1 \\
      1 \\
      1 \\
      1 \\
      1
    \end{pmatrix}
    =
    \begin{pmatrix}
      3 \\
      3 \\
      3 \\
      3 \\
      3 \\
      3
    \end{pmatrix}
    =
    \ 3 \cdot
    \begin{pmatrix}
      1 \\
      1 \\
      1 \\
      1 \\
      1 \\
      1
    \end{pmatrix}
    \eqperiod
  \end{equation}
  % 
  When for a square matrix~$A$ there is a vector~$\vecx$
  (with not all entries equal to zero)
  and a number~$\lambda$ such that
  \begin{equation}
    A \cdot \vecx = \lambda \cdot \vecx
    \eqcomma
  \end{equation}
  %     as in \refeq{eq:graph-4-vertices}--\refeq{eq:graph-6-vertices}
  %     above, 
  such a vector~$\vecx$ is called an \emph{eigenvector}
  of the matrix~$A$  with  \emph{eigenvalue}~$\lambda$.
  We see that the matrix in~\refeq{eq:graph-4-vertices}
  has the all-ones vector as eigenvector with eigenvalue~$2$,
  and the matrix in~\refeq{eq:graph-6-vertices}
  also has the all-ones vector as eigenvector but with eigenvalue~$3$.
  %%
  %%
  In this problem, we want to develop our skills of matrix
  multiplication by studying such eigenvalues and eigenvectors,
  and also to establish some non-obvious connections between
  eigenvalues and -vectors on the one hand and graphs on the other
  hand.

  
\begin{figure}[t] %%  [tp]
  \begin{subfigure}[b]{0.47\textwidth}
    \centering
    \includegraphics{Figures/graphs-exam22.10}
    \caption{Graph with adjacency matrix as in
      \refeq{eq:graph-4-vertices}. }
    \label{fig:graph-eigen-1}
  \end{subfigure}
  \hfill
  \begin{subfigure}[b]{0.47\textwidth}
    \centering
    \includegraphics{Figures/graphs-exam22.11}
    \caption{Graph with adjacency matrix as in
      \refeq{eq:graph-6-vertices}.}
    \label{fig:graph-eigen-2}
  \end{subfigure}


  \caption{Two example regular graphs in 
    Problem~\ref{problem:graph-eigen}.
  }
    \label{fig:graph-eigen}
\end{figure}



  We say that an undirected, simple graph is \emph{$d$-regular} if
  every vertex is incident to exactly $d$~edges, or, in other words,
  has exactly~$d$ neighbours.
  For two illustrations of this, the graph in
  \reffig{fig:graph-eigen-1}
  is \mbox{$2$-regular}
  and the graph in
  \reffig{fig:graph-eigen-2}
  is \mbox{$3$-regular}.
  Now, a fun fact is that the 
  \mbox{$2$-regular} graph in 
  \reffig{fig:graph-eigen-1}
  has the adjacency matrix in~\refeq{eq:graph-4-vertices}, 
  which has eigenvalue~$2$,
  and the 
  \mbox{$3$-regular} graph in 
  \reffig{fig:graph-eigen-2}
  has the adjacency matrix in~\refeq{eq:graph-6-vertices}
  with eigenvalue~$3$.
  Your task is to show that this is not a coincidence,
  and to derive some other interesting connections between
  \mbox{$d$-regular} graphs
  and the eigenvalues and -vectors of their  adjacency matrices.

  %   
  %     \reffig{fig:graph-eigen-1}
  %     \refeq{eq:graph-4-vertices}
  %     
  %     \reffig{fig:graph-eigen-2}
  %     \refeq{eq:graph-6-vertices}
  %   

  \begin{subproblem}
    (10 p)
    If $\vecx$ is an eigenvector of~$A$
    corresponding to some eigenvalue~$\lambda$, then so
    is $c \cdot \vecx$ for any $c \neq 0$.
    Explain why this is so.

    \smallskip
    \noindent
    \emph{Hint:}
    This should be easy---just use the definitions above.
  \end{subproblem}

\begin{solution}
    Suppose that $A$ is an $n \times n$ matrix, just to fix the dimension.
    By definition, coordinate~$i$ in the vector
    $A \vecx$
    is
    $\sum_{j=1}^{n} a_{i,j} x_j$.
    If $\lambda$ is an eigenvalue with eigenvector~$\vecx$, it holds that
    $
    \sum_{j=1}^{n} a_{i,j} x_j
    =
    \lambda x_i
    $.
    But then for the vector $c \vecx$ we get that coordinate~$i$ in the vector
    $A (c \cdot \vecx)$
    is
    $
    \sum_{j=1}^{n} a_{i,j} (c x_j)
    =
    c \sum_{j=1}^{n} a_{i,j} x_j
    =
    c \cdot (\lambda x_i)
    =
    \lambda (c \cdot x_i)
    $,
    so clearly
    $c \cdot \vecx$
    is also an eigenvector.

    Just explaining briefly that
    $
    A (c \vecx)
    =
    c (A \vecx)
    =
    c \cdot (\lambda \cdot \vecx)
    =
    \lambda (c \cdot x_i)
    $
    is also fine.  
\end{solution}

  \begin{subproblem}
    (20 p)
    Show that if $G$ is a  
    \mbox{$d$-regular} graph, 
    then its adjacency matrix~$A_G$
    always has $d$ as an eigenvalue.    
  \end{subproblem}

  \newcommand{\vecone}{\boldsymbol{1}}

\begin{solution}
    For this and the following problems, an important
    first observation is that if we identify the vertices
    $v_1, v_2, \ldots, v_n$ of~$G$ with the integers
    $1, 2, \ldots, n$
    (as in \reffig{fig:graph-eigen}),
    then the $i$th entry of
    $A_G \vecx$ 
    is
    \begin{equation}
      \label{eq:neighbour-sum}
      \bigl ( A_G \vecx \bigr)_i
      =
      \sum_{j=1}^{n} a_{i,j} x_j
      =
      \sum_{j \in N(i)}  x_j
      \eqcomma    
    \end{equation}
    where as usual $N(\cdot)$ denotes the set of neighbours of a vertex.
    This is so since $a_{i,j}$ is~$1$ when vertices~$i$ and~$j$ are
    neighbours and is~$0$  otherwise.

    If we follow the examples in the problem statement and let 
    $\vecx = \vecone$
    be the all-ones vector,
    then we see from~\refeq{eq:neighbour-sum} that
    $
    \bigl ( A_G \vecone) \bigr)_i
    =
    \sum_{j \in N(i)}  1
    $
    will simply count the number of neighbours of vertex~$i$.
    Since all vertices have $d$~neighbours in a $d$-regular graph,
    we have
    $
    \bigl ( A_G \vecone \bigr)_i
    = d = d \cdot 1
    $,
    and so it follows that $\vecone$ is an eigenvector with eigenvalue~$d$.
\end{solution}

  \begin{subproblem}
    (20 p)
    Show that if $G$ is a  
    \mbox{$d$-regular} graph, 
    then its adjacency matrix~$A_G$
    can never have an eigenvalue~$\lambda$ such that
    $\abs{\lambda} > d $.

    \smallskip
    \noindent
    \emph{Hint:}
    Suppose that there is an eigenvector~$\vecx$ with
    eigenvalue~$\lambda$ such that
    $\abs{\lambda} > d $.
    Rescale the entries in~$\vecx$ by some~$c \neq 0$ so that the largest
    entry has value~$1$ and all other entries have absolute value at
    most~$1$. Consider the product $A_G \cdot \vecx$, focus on a
    largest entry in~$\vecx$, and argue by contradiction.
  \end{subproblem}

\begin{solution}
    Follow the hint and rescale the vector~$\vecx$, so that
    $x_i = 1$ for the largest entry~$i$
    and
    $\abs{x_j} \leq 1$ for all~$j$. Then we get 
    \begin{equation}
      \label{eq:d-is-max}
      \Abs{\bigl ( A_G \vecx \bigr)_i}
      =
      \ABS{\sum_{j \in N(i)}  x_j}
      \leq
      \sum_{j \in N(i)}  \abs{x_j}
      \leq
      \sum_{j \in N(i)} \!\! 1 
      \leq d
      <
      \abs{\lambda}
      \eqcomma
    \end{equation}
    and so $\abs{\lambda}$ is just too large to possibly be the absolute
    value of an eigenvalue of~$A_G$.
\end{solution}

  \begin{subproblem}%
    \label{problem:connected-implies-mult-1}%
    (30 p)
    Show that if the
    \mbox{$d$-regular} graph
    $G$ is connected,
    so that there is a path between any two vertices $u$ and~$v$
    in~$V(G)$,
    then any eigenvector~$\vecx$ of the
    adjacency matrix~$A_G$
    with eigenvalue~$d$
    must have all entries equal
    (\ie $\vecx$ is the all-ones vector or some multiple of the all-ones vector).

    \smallskip
    \noindent
    \emph{Hint:}
    Suppose $\vecx$ is an eigenvector of~$A_G$
    with eigenvalue~$d$ in which not all entries are equal.
    Rescale the entries in~$\vecx$ by some~$c \neq 0$ so that the largest
    entry has value~$1$ and all other entries have absolute value at
    most~$1$, and consider the product $A_G \cdot \vecx$.
  \end{subproblem}

\begin{solution}
    Again we rescale the vector~$\vecx$ so that
    the largest entry has value~$1$ and so that
    $\abs{x_j} \leq 1$ holds for all~$j$. 
    Suppose that $i^*$ is a coordinate such that
    $x_{i^*} = 1$  but there is a neighbour~$j^*$ of~$i^*$ such that $x_{j^*} < 1$.
    (If there are no such $i^*$ and~$j^*$, then all entries in the
    vector are the same---note that we are using here the assumption
    that the graph is connected.)  
    We now have
    \begin{equation}
      \label{eq:all-entries-equal}
      \bigl ( A_G \vecx \bigr)_{i^*}
      =
      \sum_{j \in N(i^*)}  x_j
      <
      \sum_{j \in N(i^*)} \!\! 1 
      = d
      \eqcomma
    \end{equation}
    where we get a strict inequality  since
    $x_j \leq 1$ holds for all~$j$ and for the particular
    neighbour~$j^*$ we have  strict inequality
    $x_{j^*} < 1$. This shows that $\vecx$ is not an eigenvector
    of~$A_G$ with eigenvalue~$d$.
\end{solution}



  \begin{subproblem}%
    \label{problem:mult-1implies-connected}%
    (30 p)
    Show that if the
    \mbox{$d$-regular} graph
    $G$ is \emph{not} connected,
    so that there exist two vertices
    $u$ and~$v$ in~$V(G)$ with no path between them,
    then there is in fact is an eigenvector~$\vecx$ of the
    adjacency matrix~$A_G$
    with eigenvalue~$d$
    in which not all entries are equal.

    \smallskip
    \noindent
    \emph{Hint:}
    Consider the different connected components of~$G$ and
    use them to define interesting vectors for which $d$~is an eigenvalue.
  \end{subproblem}
  
\begin{solution}
  Fix some connected component~$C$ and define a
  vector~$\vecx$ such that $x_i = 1$ for $i \in C$
  and $x_i = 0$ for $i \notin C$.
%%
  We know from 
  \refeq{eq:neighbour-sum}
  that
  $  
  \bigl ( A_G \vecx \bigr)_i
  =
  \sum_{j \in N(i)}  x_j
  $.
  Since for every vertex~$i$ it holds that
  $i$ and~$N(i)$ lie in the same connected component, we get for all
  $i' \notin C$ that
  $  
  \bigl ( A_G \vecx \bigr)_{i'}
  =
  \sum_{j \in N(i')}  x_j
  =
  \sum_{j \in N(i')}  0
  = 0 = d \cdot x_{i'}
  $.
  For vertices $i'' \in C$ we get that $C$ must contain all the
  $d$~neighbours, and so in this case it holds that
  $  
  \bigl ( A_G \vecx \bigr)_{i''}
  =
  \sum_{j \in N(i'')}  x_j
  =
  \sum_{j \in N(i')}  1
  = d =  d \cdot x_{i'}
  $.
  Hence, $\vecx$ is an eigenvector with eigenvalue~$d$.  
\end{solution}

  \begin{subproblem}%
    \label{problem:bipartite-eigen}%
    (40 p)
    We say that an undirected graph~$G = (V,E)$ is
    \emph{bipartite}
    if there is a bipartition
    $V = V_1 \disjunion V_2$
    (where     $\disjunion$ denotes \emph{disjoint union},
    so that 
    $V_1 \union V_2 = V$
    but
    $V_1 \intersection V_2 = \emptyset$)
    such that any edge in~$G$ has one endpoint in~$V_1$ and one
    endpoint in~$V_2$, but there are no edges connecting vertices 
    in~$V_1$ to each other or vertices in~$V_2$ to each other.
    (Just to give examples for this definition,
    it is not hard to verify that the graph in
    \reffig{fig:graph-eigen-1}
    is bipartite but that the graph in
    \reffig{fig:graph-eigen-2}
    is not.)

    Show that if the \mbox{$d$-regular} graph $G$ is bipartite and 
    the adjacency matrix~$A_G$     has
    an eigenvector~$\vecx$ with eigenvalue~$\lambda$,
    then $-\lambda$ is also an eigenvalue for~$A_G$.

    \smallskip
    \noindent
    \emph{Hint:}
    Consider the bipartition
    $V = V_1 \disjunion V_2$
    and use it to modify the eigenvector~$\vecx$
    in some interesting way.
%%    
    (This connection between bipartiteness and negated eigenvalues is
    actually an if and only if---it holds that
    $-\lambda$ is also an eigenvalue only if $G$ is bipartite---but
    you definitely do not need to prove this.)
  \end{subproblem}
%%
%%
\begin{solution}
  Suppose that $G$ is a bipartite \mbox{$d$-regular} graph
  with bipartition
  $V = V_1 \disjunion V_2$, and that
  $\vecx$~is an eigenvector  with eigenvalue~$\lambda$.
  Define the vector~$\vecy$ by
  \begin{equation}
    \label{eq:def-y}
    y_i =
    \begin{cases}
      x_i & \text{if $i \in V_1$,} \\
      -x_i & \text{if $i \in V_2$.} \\
    \end{cases}
  \end{equation}
  %%
  We are again going to use that
  $  
  \bigl ( A_G \vecy \bigr)_i
  =
  \sum_{j \in N(i)}  y_j
  $
  according to~\refeq{eq:neighbour-sum}.

  Consider first a vertex $i \in V_1$.
  Note that for all its neighbours
  $j \in N(i)$
  it holds that $j \in V_2$ because of bipartiteness, and so
%     for all $j \in N(i)$ we have
  $y_j = -x_j$ by our construction of~$\vecy$ in~\refeq{eq:def-y}.  
  This means that
  \begin{equation}
    \bigl ( A_G \vecy \bigr)_i
    =
    \sum_{j \in N(i)}  y_j
    =
    \sum_{j \in N(i)}  -x_j
    =
    -\lambda x_i
    =
    (-\lambda) \cdot  y_i
  \end{equation}
  (where we used that
  $
  \sum_{j \in N(i)}  x_j
  =
  \lambda x_i
  $
  by assumption, since $\vecx$ is an eigenvector with eigenvalue~$\lambda$).

  In exactly the same way, for $i \in V_2$ we have that
  $y_i = -x_i$
  but that the neighbours
  $j \in N(i)$
  have vector entries
  $y_j = x_i$
  since $j \in V_1$. The calculation
  \begin{equation}
    \bigl ( A_G \vecy \bigr)_i
    =
    \sum_{j \in N(i)}  y_j
    =
    \sum_{j \in N(i)}  x_j
    =
    \lambda x_i
    =
    (-\lambda) \cdot y_i
  \end{equation}
  completes the proof that $\vecy$ is an eigenvector with
  eigenvalue~$-\lambda$. 

\bigskip

\end{solution}
%%
%%
  Note that
  Problems~\ref{problem:connected-implies-mult-1}
  and~\ref{problem:mult-1implies-connected}
  say that a 
  \mbox{$d$-regular} graph~$G$ is connected if and only if
  the only eigenvectors of the adjacency matrix~$A_G$
  with eigenvalue~$d$ are multiples of the
  all-ones vector, and 
  Problem~\ref{problem:bipartite-eigen}
  says that $G$ is bipartite only if the 
  eigenvalues of~$A_G$ are symmetric \wrt~$0$.
  We are in fact only scratching the surface here, in that the
  eigenvalues of~$A_G$ can tell us much more about the properties
  of~$G$.
  Perhaps the most important connection is that
  the second largest eigenvalue~$\lambda_2$ in absolute value
%     tells  us
  is a measure of
  how well-connected the graph is---if the gap between $d$
  and $\abs{\lambda_2}$ is large, then $G$ is an \emph{expander graph}
  in which  there are short paths between
  any two vertices. These and other highly nontrivial
  facts are further studied in
  \emph{spectral graph theory}.
  
  


\end{problem}


\end{document}



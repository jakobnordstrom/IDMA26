\documentclass{jn-pset}
%   \documentclass[solutions]{jn-pset}

\usepackage{ifthen}
\newboolean{versionwithsolutions}
% Uncomment to insert text that should only be there in version WITH solutions
\setboolean{versionwithsolutions}{true}
% Uncomment to insert text that should only be there in version WITHOUT solutions
\setboolean{versionwithsolutions}{false}


%%%%
%%%% QUICK INSTRUCTIONS FOR FORMATTING OF PROBLEMS
%%%%
%    
% Code a problem by
%    \begin{problem}
%    \end{problem}
%
% Code a subproblem inside a problem by (note the percent signs!)
%    \begin{subproblem}%
%        \label{problem:labelhere}%
%        Text here
%    \end{subproblem}
%
% Get a small vertical space by issuing the command
%    \smallskip
%
% For instance, to give a hint to a problem (after having completed
% the problem statement) code in the following way:
%
%    \smallskip
%     \noindent
%     \emph{Hint:}
%     Consider the following super-useful hint for this particular problem...
%

% PROBLEM-SET-SPECIFIC MACROS (UPDATE FOR EACH PROBLEM SET)

\newcommand{\psetno}{2}
\newcommand{\duedate}{Wedneday February 26 at 12:59 CET}

\newcommand{\thresholdforpass}{$120$~points\xspace}

% COURSE-SPECIFIC MACROS FOR IDMA 2025

\newcommand{\coursenameabbrev}{IDMA}
\newcommand{\coursenameshort}{Introduktion til diskret matematik og algoritmer}
\newcommand{\coursenamelong}{NDAB23002U Introduktion til diskret matematik og algoritmer}
%    
\newcommand{\submissionemail}{jn@di.ku.dk\xspace}
%   \newcommand{\courseinstructor}{Jakob Nordstr\"om}
\newcommand{\courseinstructor}
        {Jakob Nordstr\"om and Srikanth Srinivasan}
\newcommand{\courseperiod}{2024/2025}

% PACKAGES, MACROS, ET CETERA

\usepackage[T1]{fontenc}
\usepackage[utf8]{inputenc}
%%% Apparently a newer version of babel doesn't play well with nada-ten
%    \usepackage[english]{babel}

\usepackage{hyperref}

\usepackage{amsmath}
\usepackage{amssymb}
\usepackage{amsfonts}
\usepackage{mathtools}

% Provide calligraphic \mathscr font
\usepackage{mathrsfs}

% Enable use of MetaPost generated PostScript files
\usepackage{ifpdf}
\usepackage{graphicx}  
\ifpdf         
\DeclareGraphicsRule{*}{mps}{*}{}
\fi            

% For getting subfigures 1(a), 1(b) etc
% The package "subfigure" is obsolete, so switch to subcaption
%    \usepackage[sf,SF]{subfigure}
\usepackage{subcaption} 

% Sam Buss's package for formatting proofs
\usepackage{bussproofs}

% Extensions to verbatim commands
\usepackage{verbatim}

% Smiley
\usepackage{wasysym}

% To choose how to enumerate lists
\usepackage{enumerate}

%%%%
%%%% THIS FILE IS INTENDED TO BE READ-ONLY --- PLEASE DO NOT EDIT.
%%%% PLEASE CONTACT JAKOB NORDSTRÖM AT jn@di.ku.dk REGARDING ANY ISSUES.
%%%%

%
% DETECTION OF DOCUMENT TYPE
%----------------------------
%
% Version date: September 24, 2022
%
% First versions by Jakob Nordström <jn@di.ku.dk>
% Cleaned up version by Marc Vinyals <vinyals@kth.se>
% Minor later additions by Jakob Nordström and Susanna F. de Rezende
%
% This file needs to be included or input(ted) so that the conditional
% macro definitions in other LaTeX files will not generate compilation
% errors. Most document classes are detected using \@ifclassloaded.
%
% Use the file 'testdoctypedetection.tex' to double-check that these
% Boolean detectors work.
%
%
%%% The report class is not detected correctly in view of later
%%% updates, but this should be easy to fix when needed. 
%%% [Jakob Nordström, May 16, 2016]
% detectedReport is set to true if none of detectedArticle, detectedThesis,
% detectedSTOC, detectedFOCS, detectedSIAM, detectedIEEE, detectedICS,
% or detectedPoster is true.
%


\usepackage{ifthen}

\provideboolean{detectedSTOC}
\provideboolean{detectedFOCS}
\provideboolean{detectedElsevier}
\provideboolean{detectedNOW}
\provideboolean{detectedLMCS}
\provideboolean{detectedIEEE}
\provideboolean{detectedPoster}
\provideboolean{detectedSIAM}
\provideboolean{detectedLNCS}
\provideboolean{detectedACM}
\provideboolean{detectedACMconf}
\provideboolean{detectedSigplanconf}
\provideboolean{detectedToC}
\provideboolean{detectedLIPIcs}
\provideboolean{detectedAAAI}
\provideboolean{detectedIJCAI}
\provideboolean{detectedCompCplx}
\provideboolean{detectedEasyChair}
\provideboolean{detectedJAIR}
\provideboolean{detectedArticle}
\provideboolean{detectedReport}
\provideboolean{detectedThesis}

\makeatletter

\@ifclassloaded{sig-alternate}
{\setboolean{detectedSTOC}{true}}
{\setboolean{detectedSTOC}{false}}

\@ifclassloaded{elsarticle}
{\setboolean{detectedElsevier}{true}}
{\setboolean{detectedElsevier}{false}}

\@ifclassloaded{now}
{\setboolean{detectedNOW}{true}}
{\setboolean{detectedNOW}{false}}

\@ifclassloaded{lmcs}
{\setboolean{detectedLMCS}{true}}
{\setboolean{detectedLMCS}{false}}

\@ifclassloaded{IEEEtran} {
  \setboolean{detectedIEEE}{true}
  \ifCLASSOPTIONconference {
    \setboolean{detectedFOCS}{true}
  }
  \else {
    \setboolean{detectedFOCS}{false}
  }
  \fi
}
{
  \setboolean{detectedFOCS}{false}
  \setboolean{detectedIEEE}{false}
}

%%% Obsolete SIAM class file
%   \@ifclassloaded{siamltex1213} 
%   {\setboolean{detectedSIAM}{true}}
%   {\setboolean{detectedSIAM}{false}}
\@ifclassloaded{siamart171218}
{\setboolean{detectedSIAM}{true}}
{\setboolean{detectedSIAM}{false}}

\@ifclassloaded{llncs}
{\setboolean{detectedLNCS}{true}}
{\setboolean{detectedLNCS}{false}}

\@ifclassloaded{acmsmall}
{\setboolean{detectedACM}{true}}
{\setboolean{detectedACM}{false}}

\@ifclassloaded{acmart}
{\setboolean{detectedACMconf}{true}
 \setboolean{detectedACM}{true}}
{\setboolean{detectedACMconf}{false}}

\@ifclassloaded{sigplanconf}
{\setboolean{detectedSigplanconf}{true}}
{\setboolean{detectedSigplanconf}{false}}

\@ifclassloaded{toc}
{\setboolean{detectedToC}{true}}
{\setboolean{detectedToC}{false}}

\@ifclassloaded{lipics}
{\setboolean{detectedLIPIcs}{true}}
{\@ifclassloaded{lipics-v2019}
  {\setboolean{detectedLIPIcs}{true}}
  {\@ifclassloaded{oasics-v2019}
    {\setboolean{detectedLIPIcs}{true}}
    {\setboolean{detectedLIPIcs}{false}}
  }
}

%   
%   \@ifclassloaded{lipics}
%   {\setboolean{detectedLIPIcs}{true}}
%   {\@ifclassloaded{lipics-v2016}
%     {\setboolean{detectedLIPIcs}{true}}
%     {\@ifclassloaded{oasics-v2016}
%       {\setboolean{detectedLIPIcs}{true}}
%       {\setboolean{detectedLIPIcs}{false}}
%     }
%   }
%   

\@ifclassloaded{cc}
{\setboolean{detectedCompCplx}{true}}
{\setboolean{detectedCompCplx}{false}}

\@ifclassloaded{easychair}
{\setboolean{detectedEasyChair}{true}}
{\setboolean{detectedEasyChair}{false}}

%%%
%%% For JAIR, detect that the "jair" package is being used
%%%
\@ifpackageloaded{jair}
{\setboolean{detectedJAIR}{true}}        
{\setboolean{detectedJAIR}{false}}        

%%%
%%% AAAI and IJCAI have special style files that needs to be detected.
%%% It seems they update the name with the year of the conference also.
%%%
\@ifpackageloaded{aaai}
{\setboolean{detectedAAAI}{true}}        
{\@ifpackageloaded{aaai18}
  {\setboolean{detectedAAAI}{true}}
  {\@ifpackageloaded{aaai20}       
    {\setboolean{detectedAAAI}{true}}
    {\setboolean{detectedAAAI}{false}}}}

\@ifpackageloaded{ijcai18}
{\setboolean{detectedIJCAI}{true}}        
{\@ifpackageloaded{ijcai19}
  {\setboolean{detectedIJCAI}{true}}        
  {\setboolean{detectedIJCAI}{false}}}

\@ifclassloaded{sciposter}
{\setboolean{detectedPoster}{true}}
{\setboolean{detectedPoster}{false}}

\@ifclassloaded{article}
{\setboolean{detectedArticle}{true}}
{\setboolean{detectedArticle}{false}}

\makeatother

\ifthenelse{\not \isundefined{\examen} 
  \and \not \isundefined{\disputationsdatum} 
  \and \not \isundefined{\disputationslokal}}   
  {\setboolean{detectedThesis}{true}}
  {\setboolean{detectedThesis}{false}}

%%%
%%% Not entirely sure whether detectedReport is set correctly in view
%%% of later updates [Jakob Nordström, May 16, 2016]
%%%
           
\ifthenelse{\boolean{detectedArticle} \or \boolean{detectedThesis}
  \or \boolean{detectedSTOC}    \or \boolean{detectedFOCS}
  \or \boolean{detectedSIAM}    \or \boolean{detectedIEEE}
  \or \boolean{detectedACMconf} \or \boolean{detectedACM}
  \or \boolean{detectedPoster}}
{\setboolean{detectedReport}{false}}
{\setboolean{detectedReport}{true}}



%%%%
%%%% THIS FILE IS INTENDED TO BE READ-ONLY --- PLEASE DO NOT EDIT.
%%%% PLEASE CONTACT JAKOB NORDSTRÖM AT jakobn@kth.se REGARDING ANY ISSUES.
%%%%

% GENERAL MACROS TO USE IN LaTeX-FILES
%======================================
%
% AUTHOR
%   Jakob Nordström <jakobn@kth.se>
%   Some improvements added by Marc Vinyals <vinyals@kth.se>
%
% VERSION
%   Last updated June 7, 2019
%    
% KNOWN ISSUES:
%   References with page numbers such as \refsecP, \refthP, etc will not
%   work with the Elsevier and SIAM document classes. No work-arounds
%   have been added, so compilation will fail if these macros are used
%   for Elsevier or SIAM articles.

%
% MACRO NAMING CONVENTION FOR MATHEMATICAL MACROS WITH DELIMITERS 
%-----------------------------------------------------------------
%
% For mathematical macros with delimiters there are usually
% three different flavours corresponding to different sizes of 
% the delimiters as follows:
%
%    \newcommand{\mycommand}[1]{<command> ( {#1} )}
%    \newcommand{\Mycommand}[1]{<command> \bigl( {#1} \bigr)}
%    \newcommand{\MYCOMMAND}[1]{<command> \left( {#1} \right)}
%

%
% REQUIRED PACKAGES
%-------------------
%

\usepackage{ifthen}
\usepackage{xspace}
% varioref does not seem to mix well with the Elsevier document class
\ifthenelse
{\boolean{detectedElsevier} \or \boolean{detectedSIAM} 
  \or \boolean{detectedLIPIcs}}
{}
{\usepackage{varioref}}


%
% M I S C E L L A N E O U S 
%---------------------------
%

\DeclareMathAlphabet{\mathsfsl}{OT1}{cmss}{m}{sl}


%
% G E N E R A L   F O R M A T T I N G   R U L E S 
%-------------------------------------------------
%
% To achieve some kind of consistency in the notation
%

% Format of functions to integers or real numbers
\newcommand{\formatfunctiontonumbers}[1]{\mathrm{#1}}

% Format of functions to sets
\newcommand{\formatfunctiontoset}[1]{\mathit{#1}}

% Dots in x_1 \lor ... \lor x_n and the like
% (make a generic macro that can be changed according to publisher
% requirements)  
\newcommand{\formuladots}{\cdots}


%
% B I G - O H   N O T A T I O N 
%-------------------------------
%

\newcommand{\BIGOH}[1]{\mathrm{O} \left( #1 \right)}
\newcommand{\Bigoh}[1]{\mathrm{O} \bigl( #1 \bigr)}
\newcommand{\bigoh}[1]{\mathrm{O} ( #1 )}
\newcommand{\LITTLEOH}[1]{\mathrm{o} \left( #1 \right)}
\newcommand{\Littleoh}[1]{\mathrm{o} \bigl( #1 \bigr)}
\newcommand{\littleoh}[1]{\mathrm{o} ( #1 )}
\newcommand{\BIGTHETA}[1]{\Theta \left( #1 \right)}
\newcommand{\Bigtheta}[1]{\Theta \bigl( #1 \bigr)}
\newcommand{\bigtheta}[1]{\Theta ( #1 )}
\newcommand{\BIGOMEGA}[1]{\Omega \left( #1 \right)}
\newcommand{\Bigomega}[1]{\Omega \bigl( #1 \bigr)}
\newcommand{\bigomega}[1]{\Omega ( #1 )}
\newcommand{\LITTLEOMEGA}[1]{\omega \left( #1 \right)}
\newcommand{\Littleomega}[1]{\omega \bigl( #1 \bigr)}
\newcommand{\littleomega}[1]{\omega ( #1 )}
\newcommand{\POLYBOUND}[1]{\mathrm{poly} \left( #1 \right)}
\newcommand{\Polybound}[1]{\mathrm{poly} \bigl( #1 \bigr)}
\newcommand{\polybound}[1]{\mathrm{poly} ( #1 )}
\newcommand{\POLYLOGBOUND}[1]{\mathrm{polylog} \left( #1 \right)}
\newcommand{\Polylogbound}[1]{\mathrm{polylog} \bigl( #1 \bigr)}
\newcommand{\polylogbound}[1]{\mathrm{polylog} ( #1 )}
            
\DeclareMathOperator{\polylog}{polylog}


%
% G E N E R A L  M A T H E M A T I C A L   N O T A T I O N
%----------------------------------------------------------
%

% N, Z, Q, R as symbols for classes of numbers
\ifthenelse{\boolean{detectedToC}}{}
{
  \newcommand{\Q}         {\mathbb{Q}}
  \newcommand{\R}         {\mathbb{R}}
  \newcommand{\Rplus}     {\mathbb{R}^{+}}
  \newcommand{\N}         {\mathbb{N}}
  \newcommand{\Nplus}     {\mathbb{N}^{+}}
  \newcommand{\Nzero}     {\mathbb{N}_{0}}
  \newcommand{\Z}         {\mathbb{Z}}
}

% Sigma sum sign with indices to the right, not below and above
\newcommand{\sumnodisplay}{{\textstyle \sum}}

% Absolute value and norm
\providecommand{\abs}[1]{\lvert#1\rvert}
\providecommand{\Abs}[1]{\bigl\lvert#1\bigr\rvert}
\providecommand{\ABS}[1]{\left\lvert#1\right\rvert}
\providecommand{\norm}[1]{\lVert#1\rVert}
\providecommand{\Norm}[1]{\bigl\lVert#1\bigr\rVert}
\providecommand{\NORM}[1]{\left\lVert#1\right\rVert}

% Exists unique
\newcommand{\existsunique}{\exists!}

% Rounding
\newcommand{\ceiling}[1]{\lceil #1 \rceil}
\newcommand{\Ceiling}[1]{\bigl \lceil #1 \bigr \rceil}
\newcommand{\CEILING}[1]{\left \lceil #1 \right \rceil}

\newcommand{\floor}[1]{\lfloor #1 \rfloor}
\newcommand{\Floor}[1]{\bigl \lfloor #1 \bigr \rfloor}
\newcommand{\FLOOR}[1]{\left \lfloor #1 \right \rfloor}

\newcommand{\intpart}[1]{\lceil #1 \rfloor}
\newcommand{\Intpart}[1]{\bigl \lceil #1 \bigr \rfloor}
\newcommand{\INTPART}[1]{\left \lceil #1 \right \rfloor}

% Max and min
% Don't use \maxof and \minof to avoid conflict with calc package
\newcommand{\MAXOFEXPR}[2][]{\max_{#1} \left\{ #2 \right\}}
\newcommand{\MINOFEXPR}[2][]{\min_{#1} \left\{ #2 \right\}}
\newcommand{\Maxofexpr}[2][]{\max_{#1} \bigl\{ #2 \bigr\}}
\newcommand{\Minofexpr}[2][]{\min_{#1} \bigl\{ #2 \bigr\}}
\newcommand{\maxofexpr}[2][]{\max_{#1} \{ #2 \}}
\newcommand{\minofexpr}[2][]{\min_{#1} \{ #2 \}}

\newcommand{\maxofset}[3][:]{\max \{ #2 #1 #3 \}}
\newcommand{\minofset}[3][:]{\min \{ #2 #1 #3 \}}
 
\newcommand{\MAXOFSET}[3][:]%
     {\ifthenelse{\equal{#1}{;}}%
     {\MAXOFEXPR{ #2 \,;\, #3 }}
     {\ifthenelse{\equal{#1}{:}}%
     {\MAXOFEXPR{ #2 \,:\, #3 }}
     {\max \twincommandJN{\left\{}{#2}{\left#1}{\right}{\,#3}{\right\}}}}}
\newcommand{\MINOFSET}[3][:]%
     {\ifthenelse{\equal{#1}{;}}%
     {\MINOFEXPR{ #2 \,;\, #3 }}
     {\ifthenelse{\equal{#1}{:}}%
     {\MINOFEXPR{ #2 \,:\, #3 }}
     {\min \twincommandJN{\left\{}{#2}{\left#1}{\right}{\,#3}{\right\}}}}}

\newcommand{\Maxofset}[3][:]%
     {\ifthenelse{\equal{#1}{;}}%
     {\Maxofexpr{ #2 \,;\, #3 }}
     {\ifthenelse{\equal{#1}{:}}%
     {\Maxofexpr{ #2 \,:\, #3 }}
     {\max \twincommandJN{\bigl\{}{#2}{\bigl#1}{\bigr}{\,#3}{\bigr\}}}}}
\newcommand{\Minofset}[3][:]%
     {\ifthenelse{\equal{#1}{;}}%
     {\Minofexpr{ #2 \,;\, #3 }}
     {\ifthenelse{\equal{#1}{:}}%
     {\Minofexpr{ #2 \,:\, #3 }}
     {\min \twincommandJN{\bigl\{}{#2}{\bigl#1}{\bigr}{\,#3}{\bigr\}}}}}


%
% A L G E B R A
%---------------
%

% Some linear algebra
\newcommand{\transpose}[1]{\ensuremath{#1^{\top}}}
\newcommand{\innerproduct}[2]{\langle #1, #2 \rangle}
\newcommand{\Innerproduct}[2]{\bigl\langle #1, #2 \bigr\rangle}
\newcommand{\INNERPRODUCT}[2]{\left\langle #1, #2 \right\rangle}

% Generic field
\newcommand{\fieldstd}{\mathbb{F}}
\newcommand{\fieldf}{\mathbb{F}}
\newcommand{\F}{\mathbb{F}}

% Finite fields
\newcommand{\GF}[1]{\mathrm{GF} ( #1 )}
\newcommand{\gf}[1]{\mathrm{GF} ( #1 )}
\newcommand{\GFmul}[1]{\mathrm{GF} ( #1 )^{*}}
\newcommand{\gfmul}[1]{\mathrm{GF} ( #1 )^{*}}


%
% P R O B A B I L I T Y   T H E O R Y 
%-------------------------------------

% AMS-TeX defines an operator name \Pr
\DeclareMathOperator{\Expop}{E}
\DeclareMathOperator{\Varianceop}{Var}

% Probability
\newcommand{\PROB}[2][]{\Pr_{#1} \left[ #2 \right]}
\newcommand{\Prob}[2][]{\Pr_{#1} \bigl[ #2 \bigr]}
\ifthenelse{\boolean{detectedLMCS}}
{\renewcommand{\prob}[2][]{\Pr_{#1} [ #2 ]}}
{\newcommand{\prob}[2][]{\Pr_{#1} [ #2 ]}}

% Expectation
\newcommand{\EXPECTATION}[2][]{\Expop_{#1} \left[ #2 \right]}
\newcommand{\Expectation}[2][]{\Expop_{#1} \bigl[ #2 \bigr]}
\newcommand{\expectation}[2][]{\Expop_{#1} [ #2 ]}
\newcommand{\VARIANCE}[1]{\Varianceop \left( #1 \right)}
\newcommand{\Variance}[1]{\Varianceop \bigl( #1 \bigr)}
\newcommand{\variance}[1]{\Varianceop ( #1 )}

% 
% INTERLUDE: MATCHING MIDDLE SEPARATORS (FROM THE UK TeX FAQ)
% 
% 
% One of the few glaring omissions from TeX's mathematical typesetting
% capabilities is a means of setting separators in the middle of
% mathematical expressions. In all sorts of mathematical enterprises one
% may find oneself needing a \middle command, to be used in expressions
% like \left\{ x \in \mathbb{N} \middle| x \mbox{ even} \right\} to
% specify the set of even natural numbers. The e-TeX system defines just
% such a command, but users of Knuth's original need some support.
% Donald Arseneau's braket package provides commands for set
% specifications (as above) and for Dirac brackets (and bras and kets).
% The package uses the e-TeX built-in command if it finds itself running
% under e-TeX.
% 
% See ftp://cam.ctan.org/tex-archive/macros/latex/contrib/misc/braket.sty .
% 
% Or one can do as below.
%

\newcommand{\twincommandJN}[6]%
    {#1#2#3\vphantom{#2#5}\mspace{-2.05mu}#4.#5#6}

% Perhaps this is superfluous---in text mode there is no need for measuring
% with \vphantom, I think, since \bigl[ and \bigr] are what they are 
% independent of what is inside (are they not?).
%
% The length -2.25mu probably should be set instead by doing sth like
%    
%    \newlength{\lengthJN}
%    \settowidth{\lengthJN}{$\left.\right.$}
%    \setlength{\lengthJN}{0.5\lengthJN}
%
% and then using \mspace{-\lengthJN}, but the difference appears to be
% very small so I have not implemented this.


% CONDITIONAL EXPECTATION
\newcommand{\condexp}[2]{\Expop{#1  \mid  #2}}
\newcommand{\CondExp}[2]%
    {\Expop\twincommandJN{\bigl[}{#1}{\bigl|}{\bigr}{\,#2}{\bigr]}}
\newcommand{\CONDEXP}[2]%
     {\Expop\twincommandJN{\left[}{#1}{\left|}{\right}{\,#2}{\right]}}

% CONDITIONAL PROBABILITY
\newcommand{\condprob}[3][]{\prob[#1]{#2  \mid  #3}}
\newcommand{\Condprob}[3][]%
    {\Pr_{#1}\twincommandJN{\bigl[}{#2}{\bigl|}{\bigr}{\,#3}{\bigr]}}
\newcommand{\CONDPROB}[3][]%
    {\Pr_{#1}\twincommandJN{\left[}{#2}{\left|}{\right}{\,#3}{\right]}}

%
% Example code:
%    
%    \begin{displaymath}
%     \CONDEXP{\sum_{i=1}^kX_i}{Z}\quad\mbox{and}\quad%
%     \CONDPROB{B\land C}{\bigwedge_{i\in S}A_i}
%    \end{displaymath}
%    
%    $\CondExp{\sum_{i=1}^k X_i}{Z}$  
%    and
%    $\CondProb{B\land C}{\bigwedge_{i\in S}A_i}$
%    


%
% F U N C T I O N S
%-------------------
%

% DESCRIPTION OF FUNCTION
\newcommand{\funcdescr}[3]{\ensuremath{ #1 : #2 \to #3}}

% DOMAIN
\newcommand{\domainof}[1]{\ensuremath{\mathrm{dom} ( #1 )}}
\newcommand{\Domainof}[1]{\ensuremath{\mathrm{dom}\bigl( #1 \bigr)}}

% INVERSE IMAGE
\newcommand{\invimageof}[2]{{\ensuremath{{#1}^{-1} \left( #2 \right)}}}

%
% G R A P H S
%-------------
%

\newcommand{\edges}[1]{E( #1 )}
\newcommand{\Edges}[1]{E\bigl( #1 \bigr)}
\newcommand{\vertices}[1]{V( #1 )}
\newcommand{\Vertices}[1]{V\bigl( #1 \bigr)}

\newcommand{\vdegree}[2][]{\mathrm{deg}_{#1}(#2)}
\newcommand{\Vdegree}[2][]{\mathrm{deg}_{#1}\bigl(#2\bigr)}
\newcommand{\vneighbour}[2][]{N_{#1}({#2})}
\newcommand{\Vneighbour}[2][]{N_{#1}\bigl({#2}\bigr)}

% Boundary
\newcommand{\boundary}[1]{\ensuremath{\partial #1}}

\newcommand{\pathstd}{\ensuremath{P}}
\newcommand{\pathalt}{\ensuremath{Q}}
\newcommand{\pathfromto}[3]{#1 : #2 \rightsquigarrow #3}


%
% S E T S   A N D   T U P L E S
%-------------------------------
%

\newcommand{\set}[1]{\{ #1 \}}
\newcommand{\Set}[1]{\bigl\{ #1 \bigr\}}
\newcommand{\SET}[1]{\left\{ #1 \right\}}

\newcommand{\setdescr}[3][\mid]{\set{ #2 #1 #3 }}
\newcommand{\Setdescr}[3][|]%
     {\ifthenelse{\equal{#1}{;}}%
     {\Set{ #2 \,;\, #3 }}
     {\ifthenelse{\equal{#1}{:}}%
     {\Set{ #2 \,:\, #3 }}
     {\twincommandJN{\bigl\{}{#2\,}{\bigl#1}{\bigr}{\,#3}{\bigr\}}}}}
\newcommand{\SETDESCR}[3][|]%
     {\twincommandJN{\left\{}{#2\,}{\left#1}{\right}{\,#3}{\right\}}}

\newcommand{\setbrackets}[1]{[ #1 ]}
\newcommand{\Setbrackets}[1]{\bigl[ #1 \bigr]}
\newcommand{\SETBRACKETS}[1]{\left[ #1 \right]}

\newcommand{\setdescrbrackets}[3][\mid]{{\setbrackets{ #2 #1 #3 }}}
\newcommand{\Setdescrbrackets}[3][|]%
     {\twincommandJN{\bigl[}{#2}{\bigl#1}{\bigr}{\,#3}{\bigr]}}
\newcommand{\SETDESCRBRACKETS}[3][|]%
     {\twincommandJN{\left[}{#2}{\left#1}{\right}{\,#3}{\right]}}

\newcommand{\SETSIZE}[1]{\left\lvert#1\right\rvert}
\newcommand{\Setsize}[1]{\bigl\lvert#1\bigr\rvert}
\newcommand{\setsize}[1]{\lvert#1\rvert}

% Set complement
\newcommand{\setcompl}[1]{\overline{#1}}

% Intersection and union
\newcommand{\intersection}{\cap}
\newcommand{\Intersection}{\bigcap}
\newcommand{\Intersectionnodisplay}{\textstyle \bigcap}

\newcommand{\union}{\cup}
\newcommand{\Union}{\bigcup}
\newcommand{\Unionnodisplay}{\textstyle \bigcup}

% Intersection and union with some space
\newcommand{\unionSP}{\, \union \, }
\newcommand{\intersectionSP}{\, \intersection \, }

% Disjoint union 
%
% Can see no difference between below definition and \mathbin{\dot{\cup}}
\newcommand{\disjointunion}{\overset{.}{\cup}}
\newcommand{\disjointunionSP}{\disjointunion}
\newcommand{\Disjointunion}{\overset{.}{\bigcup}}
\newcommand{\disjunion}{\disjointunion}
\newcommand{\Disjunion}{\Disjointunion}

% First n positive integers
\newcommand{\nset}[1]{[{#1}]}
\newcommand{\Nset}[1]{\bigl[ {#1} \bigr]}


%
% L O G I C
%-----------
%

%
% Logic connectives
%
% Logic or is \lor. Logic and is \land. Logic not is \lnot.
% They can be used in math mode only.
\newcommand{\Lor}{\bigvee}
\newcommand{\Land}{\bigwedge}

% nodisplay = indices to the right, not below and above
\newcommand{\Lornodisplay}{{\textstyle \bigvee}}
\newcommand{\Landnodisplay}{{\textstyle \bigwedge}}

\newcommand{\limpl}{\rightarrow}
\newcommand{\lequiv}{\leftrightarrow}

% Prefixed NOT (pfnot):  \lnot x
% Overlined NOT (olnot): \overline{x}
% \stdnot{lit} is the standard NOT notation for variables and literals

\newcommand{\pfnot}[1]{\lnot #1}
\newcommand{\olnot}[1]{\overline{#1}}
\newcommand{\stdnot}[1]{\olnot{#1}}
    
% syntactic equivalence
\newcommand{\synteq}{\doteq}

% constants "true" and "false"
\newcommand{\FALSE}{\mathit{FALSE}}
\newcommand{\TRUE}{\mathit{TRUE}}

\newcommand{\false}{\bot}
\newcommand{\true}{\top}

\newcommand{\falsenum}{0}
\newcommand{\truenum}{1}

%
% Notation and terms for CNF/DNF formulas
%
% Standard notation for parameters in k-CNF formulas
\newcommand{\nvar}{n}
\newcommand{\nvars}{\nvar}
\newcommand{\nclause}{m}
\newcommand{\nclauses}{\nclause}
\newcommand{\clwidth}{k}
\newcommand{\density}{\Delta}

% Formatting of k-CNF/k-DNF in running text
\newcommand{\xdnf}[1]{\mbox{\ensuremath{#1}-DNF}\xspace}
\newcommand{\kdnf}{\xdnf{\clwidth}}
\newcommand{\xdnfform}[1]{\mbox{\ensuremath{#1}-DNF} formula\xspace}
\newcommand{\kdnfform}{\xdnfform{\clwidth}}

\newcommand{\xcnf}[1]{\mbox{\ensuremath{#1}-CNF}\xspace}
\newcommand{\kcnf}{\xcnf{\clwidth}}
\newcommand{\xcnfform}[1]{\mbox{\ensuremath{#1}-CNF} formula\xspace}
\newcommand{\kcnfform}{\xcnfform{\clwidth}}
\newcommand{\xclause}[1]{\mbox{\ensuremath{#1}-clause}\xspace}
\newcommand{\kclause}{\xclause{\clwidth}}

\newcommand{\Excnf}[1]{\mbox{\ensuremath{\mathrm{E}#1}-CNF}\xspace}
\newcommand{\Ekcnf}{\Excnf{\clwidth}}
\newcommand{\Excnfform}[1]{\mbox{\ensuremath{\mathrm{E}#1}-CNF} formula\xspace}
\newcommand{\Ekcnfform}{\Excnfform{\clwidth}}

\newcommand{\xterm}[1]{\mbox{\ensuremath{#1}-term}\xspace}
\newcommand{\kterm}{\xterm{\clwidth}}
\newcommand{\Exterm}[1]{\mbox{\ensuremath{\mathrm{E}#1}-term}\xspace}
\newcommand{\Ekterm}{\Exterm{\clwidth}}

%
% Distributions for random k-CNF formulas
%
% fixed # random clauses chosen with replacement
% \randkcnfnclwrepl{clause width}{#variables}{#clauses}
%
\newcommand{\randkcnfnclwrepl}[3][\clwidth]%
        {\ensuremath{\mathcal{F}^{#2, #3}_{#1}}}
\newcommand{\randkcnfnclwreplstd}% 
        {\randkcnfnclwrepl{\clwidth}{\nvar}{\nclause}}


%
% C O M P L E X I T Y   T H E O R Y 
%-----------------------------------
%

% NOTATION FOR LANGUAGES/PROBLEMS
\newcommand{\problemlanguageformat}[1]{\textsc{#1}\xspace}
\newcommand{\problemlanguageformatnospace}[1]{\textsc{#1}}
\newcommand{\langstd}{\ensuremath{L}}
\newcommand{\langcompl}[1]{\ensuremath{\overline{#1}}}

\newcommand{\PROP}{\problemlanguageformat{PROP}}
\newcommand{\TAUT}{\problemlanguageformat{TAUT}}
\newcommand{\TAUTOLOGY}{\problemlanguageformat{Tautology}}
\newcommand{\SAT}{\problemlanguageformat{Sat}}
\newcommand{\CNFSAT}{\problemlanguageformat{CnfSat}}
\newcommand{\SATISFIABILITY}{\problemlanguageformat{Satisfiability}}
\newcommand{\THREESAT}{\text{$3$-}\problemlanguageformat{Sat}}
\newcommand{\TWOSAT}{\text{$2$-}\problemlanguageformat{Sat}}
\newcommand{\ONEINTHREESAT}{\text{$1$-}\problemlanguageformatnospace{In}%
        \text{-$3$-}\problemlanguageformat{Sat}}
\newcommand{\MAXSAT}{\problemlanguageformat{MaxSat}}
\newcommand{\MAXTWOSAT}{\problemlanguageformatnospace{Max}%
        \text{-$2$-}\problemlanguageformat{Sat}}
\newcommand{\CIRCUITSAT}{\problemlanguageformat{CircuitSat}}
\newcommand{\NAESAT}{\problemlanguageformat{NotAllEqualSat}}
\newcommand{\DOMINATINGSET}{\problemlanguageformat{DominatingSet}}
\newcommand{\VERTEXCOVER}{\problemlanguageformat{VertexCover}}
\newcommand{\MAXCUT}{\problemlanguageformat{MaxCut}}
\newcommand{\DOMATICNUMBER}{\problemlanguageformat{DomaticNumber}}
\newcommand{\MONOCHROMTRI}{\problemlanguageformat{MonochromaticTriangle}}
\newcommand{\CLIQUECOVER}{\problemlanguageformat{CliqueCover}}
\newcommand{\INDSET}{\problemlanguageformat{IndependentSet}}
\newcommand{\GCLIQUE}{\problemlanguageformat{Clique}}
\newcommand{\GCOLOURING}{\problemlanguageformat{Colouring}}
\newcommand{\SUBGRAPHISO}{\problemlanguageformat{Subgraph\-Isomorphism}}
\newcommand{\GKERNEL}{\problemlanguageformat{Kernel}}
\newcommand{\MINMAXMATCHING}{\problemlanguageformat{MinMaxMatching}}
\newcommand{\CUBICSUBGRAPH}{\problemlanguageformat{CubicSubgraph}}

% NOTATION FOR COMPLEXITY CLASSES
\newcommand{\complclassformat}[1]%
        {\textrm{\upshape{\textsf{#1}}}\xspace}
\newcommand{\cocomplclass}[1]%
        {\textrm{\upshape{\textsf{co#1}}}\xspace}

\newcommand{\TIMEclass}[1]{\ensuremath{\complclassformat{TIME}\bigl(#1\bigr)}}
\newcommand{\SPACEclass}[1]{\ensuremath{\complclassformat{SPACE}\bigl(#1\bigr)}}
\newcommand{\DTIMEclass}[1]{\ensuremath{\complclassformat{DTIME}\bigl(#1\bigr)}}
\newcommand{\DTIMEadviceclass}[2]%
    {\ensuremath{\complclassformat{DTIME}\bigl(#1\bigr)/{#2}}}
\newcommand{\Pclass}{\complclassformat{P}}
\newcommand{\NP}{\complclassformat{NP}}
\newcommand{\NPclass}{\NP}
\newcommand{\coNP}{\cocomplclass{NP}}
\newcommand{\coNPclass}{\coNP}
\newcommand{\CoNP}{\coNP}
\newcommand{\Ppoly}{\complclassformat{P/poly}}
\newcommand{\Pslashpoly}{\Ppoly}
\newcommand{\Pslashpolyclass}{\Pslashpoly}
\newcommand{\PSPACE}{\complclassformat{PSPACE}}
\newcommand{\PSPACEclass}{\PSPACE}
\newcommand{\NPSPACE}{\complclassformat{NPSPACE}}
\newcommand{\NPSPACEclass}{\NPSPACE}
\newcommand{\pspace}{\PSPACE}
\newcommand{\EXPTIME}{\complclassformat{EXPTIME}}
\newcommand{\exptime}{\complclassformat{EXPTIME}}
\newcommand{\EXPSPACE}{\complclassformat{EXPSPACE}}
\newcommand{\expspace}{\complclassformat{EXPSPACE}}
\newcommand{\EXP}{\complclassformat{EXP}}
\newcommand{\NEXP}{\complclassformat{NEXP}}
\newcommand{\coNEXP}{\cocomplclass{NEXP}}
\newcommand{\Lclass}{\complclassformat{L}}
\newcommand{\LOGSPACE}{\complclassformat{L}}
\newcommand{\NL}{\complclassformat{NL}}
\newcommand{\NLclass}{\NL}
\newcommand{\coNL}{\cocomplclass{NL}}
\newcommand{\NCclass}{\complclassformat{NC}}
\newcommand{\DP}{\complclassformat{DP}}
\newcommand{\DPclass}{\DP}
\newcommand{\SIGMACLASS}[1]{\ensuremath{\Sigma^p_{#1}}}
\newcommand{\Sigmaclass}[1]{\SIGMACLASS{#1}}
\newcommand{\PICLASS}[1]{\ensuremath{\Pi^p_{#1}}}
\newcommand{\Piclass}[1]{\PICLASS{#1}}
\newcommand{\PH}{\complclassformat{PH}}
\newcommand{\EXPEXP}{\complclassformat{EXPEXP}}
\newcommand{\NEXPEXP}{\complclassformat{NEXPEXP}}
\newcommand{\coNEXPEXP}{\cocomplclass{NEXPEXP}}        
\newcommand{\BPP}{\complclassformat{BPP}}
\newcommand{\ZPP}{\complclassformat{ZPP}}
\newcommand{\RP}{\complclassformat{RP}}
\newcommand{\coRP}{\cocomplclass{RP}}
\newcommand{\MIParg}[1]{\ensuremath{\complclassformat{MIP}[#1]}}
\newcommand{\MIP}{\complclassformat{MIP}}
\newcommand{\IP}{\complclassformat{IP}}
\newcommand{\PCPnot}{\complclassformat{PCP}}
% PCP{completeness}{soundness}{randomness}{queries}{alphabet size}
\newcommand{\PCPalph}[5]%
    {\ensuremath{\complclassformat{PCP}_{{#1},{#2}}[{#3}, {#4}, {#5}]}}
\newcommand{\PCP}[4]%
    {\ensuremath{\complclassformat{PCP}_{{#1},{#2}}[{#3}, {#4}]}}

\newcommand{\SAC}{\complclassformat{SAC}}
\newcommand{\NC}{\complclassformat{NC}}
\newcommand{\ACzero}{\ensuremath{\complclassformat{AC}^{0}}}
\newcommand{\SC}{\complclassformat{SC}}
\newcommand{\TISP}{\complclassformat{TISP}}
\newcommand{\LOGCFL}{\complclassformat{LOGCFL}}


%
% T E X T   F O R M A T T I N G
%-------------------------------
%

% introducing a new term or mentioning a familiar one for the first time
%\newcommand{\introduceterm}[1]{{\textsl{#1}}}
\newcommand{\introduceterm}[1]{{\emph{#1}}}

% Spacing before punctuation in displayed equations according to LLNCS
% (But SIAM disagrees, so it is convenient to have a macro for this.)
\newcommand{\eqperiod}{\enspace .}
\newcommand{\eqcomma}{\enspace ,}

% For description lists with items in italics (to avoid confusion
% with theorem headers, for instance)
\newcommand{\italicitem}[1][]{\item[\textit{#1}]}


%
% A B B R E V I A T I O N S   O F   F R E Q U E N T     E X P R E S S I O N S
%-----------------------------------------------------------------------------
%

% GENERAL EXPRESSIONS
\newcommand{\wrt}{with respect to\xspace}
\newcommand{\wrtabbrev}{w.r.t.\ }
\ifthenelse{\boolean{detectedToC}}{}
  {\newcommand{\eg}{for instance\xspace} % should be surrounded by commas 
    \newcommand{\Eg}{For instance\xspace}}
\ifthenelse{\boolean{detectedToC}}{}{
\newcommand{\ie}{i.e.,\ }
\newcommand{\Ie}{I.e.,\ }
}
\newcommand{\ieComma}{i.e.,\ }  %%%% Special hack to use when adapting to ToC
\newcommand{\ieNoComma}{i.e.\ }
\ifthenelse{\boolean{detectedLIPIcs} \or \boolean{detectedIJCAI}}
{\renewcommand{\st}{\errmessage{Please do not use st}}}
{\ifthenelse{\isundefined{\st}}
  {\newcommand{\st}{such that\xspace}}
  {}}      
%   \newcommand{\st}{such that\xspace}}

\newcommand{\etal}{et al.\@\xspace}
\newcommand{\etalS}{et al\@. }

\newcommand{\vs}{vs.\ }

% TYPICAL MATHEMATICAL EXPRESSIONS
\newcommand{\ifaoif}{if and only if\xspace}
\newcommand{\wolog}{without loss of generality\xspace}
\newcommand{\Wolog}{Without loss of generality\xspace}
\newcommand{\iid}{independently and identically distributed\xspace}
\ifthenelse{\boolean{detectedIEEE}}{}{\newcommand{\QED}{Q.E.D.}}
\newcommand{\qedlong}{which was to be proved\xspace}

\newcommand{\whp}{with high probability\xspace}
\newcommand{\Whp}{With high probability\xspace}
\newcommand{\aas}{asymptotically almost surely\xspace}
\newcommand{\Aas}{Asymptotically almost surely\xspace}


%
% R E F E R E N C E S 
%---------------------
%
% Macros with capital initial letter intended for use at start of sentence.
%

%
% REFERENCES TO SECTIONAL UNITS AND PAGE INTERVALS
%
% The LLNCS document class and instructions says that in the running text, 
% but not at the beginning of a sentence, Sect., Chap. and Fig. should be 
% abbreviated.
%
% LLNCS wants, e.g., "Section 4.3" with capital S, SIAM doesn't.
%

% Sections
\newcommand{\refsec}[1]{Section~\ref{#1}}
\newcommand{\refsecP}[1]{Section~\vref{#1}}
\newcommand{\Refsec}[1]{Section~\ref{#1}}
\newcommand{\RefsecP}[1]{Section~\vref{#1}}
\newcommand{\reftwosecs}[2]{Sections~\ref{#1} and~\ref{#2}}
\newcommand{\refthreesecs}[3]{Sections~\ref{#1}, \ref{#2}, and~\ref{#3}}

% Chapters
\newcommand{\refch}[1]{Chapter~\ref{#1}}
\newcommand{\Refch}[1]{Chapter~\ref{#1}}
\newcommand{\reftwochs}[2]{Chapters~\ref{#1} and~\ref{#2}}
\newcommand{\refthreechs}[3]{Chapters~\ref{#1}, \ref{#2} and~\ref{#3}}

% Appendices
\newcommand{\refapp}[1]{Appendix~\ref{#1}}
\newcommand{\Refapp}[1]{Appendix~\ref{#1}}
\newcommand{\reftwoapps}[2]{Appendices~\ref{#1} and~\ref{#2}}

% Figures
\newcommand{\reffig}[1]{Figure~\ref{#1}}
\newcommand{\reffigP}[1]{Figure~\vref{#1}}
\newcommand{\Reffig}[1]{Figure~\ref{#1}}
\newcommand{\ReffigP}[1]{Figure~\vref{#1}}
\newcommand{\reftwofigs}[2]{Figures~\ref{#1} and~\ref{#2}}
\newcommand{\refthreefigs}[3]{Figures~\ref{#1}, \ref{#2}, and~\ref{#3}}


%
% REFERENCES TO THEOREM-LIKE ENVIRONMENTS
%
% Requires \usepackage{varioref}
%
% Adapted to LLNCS document class and instructions. Definition, Theorem etc
% should be capitalized when followed by a number.
%

% Theorems
\newcommand{\refth}[1]{Theorem~\ref{#1}}
\newcommand{\reftwoths}[2]{Theorems~\ref{#1} and~\ref{#2}}
\newcommand{\refthreeths}[4][and]{Theorems~\ref{#2}, \ref{#3}, {#1}~\ref{#4}}
\newcommand{\refthm}[1]{Theorem~\ref{#1}}
\newcommand{\reftwothms}[2]{Theorems~\ref{#1} and~\ref{#2}}
\newcommand{\refthreethms}[4][and]{Theorems~\ref{#2}, \ref{#3}, {#1}~\ref{#4}}

% Lemmas
\newcommand{\reflem}[1]{Lemma~\ref{#1}}
\newcommand{\reftwolems}[2]{Lemmas~\ref{#1} and~\ref{#2}}
\newcommand{\refthreelems}[4][and]{Lemmas~\ref{#2}, \ref{#3}, {#1}~\ref{#4}}

% Propositions
\newcommand{\refpr}[1]{Proposition~\ref{#1}}
\newcommand{\reftwoprs}[2]{Propositions~\ref{#1} and~\ref{#2}}

% Corollaries
\newcommand{\refcor}[1]{Corollary~\ref{#1}}
\newcommand{\reftwocors}[2]{Corollaries~\ref{#1} and~\ref{#2}}

% Definitions
\newcommand{\refdef}[1]{Definition~\ref{#1}}
\newcommand{\reftwodefs}[2]{Definitions~\ref{#1} and~\ref{#2}}
\newcommand{\refthreedefs}[3]{Definitions~\ref{#1}, \ref{#2}, and~\ref{#3}}

% Remarks
\newcommand{\refrem}[1]{Remark~\ref{#1}}
\newcommand{\reftworems}[2]{Remarks~\ref{#1} and~\ref{#2}}

% Observations
\newcommand{\refobs}[1]{Observation~\ref{#1}}
\newcommand{\reftwoobs}[2]{Observations~\ref{#1} and~\ref{#2}}

% Facts
\newcommand{\reffact}[1]{Fact~\ref{#1}}

% Conjectures
\newcommand{\refconj}[1]{Conjecture~\ref{#1}}
\newcommand{\reftwoconjs}[2]{Conjectures~\ref{#1} and~\ref{#2}}

% Examples
\newcommand{\refex}[1]{Example~\ref{#1}}
\newcommand{\reftwoexs}[2]{Examples~\ref{#1} and~\ref{#2}}

% Properties
\newcommand{\refproperty}[1]{Property~\ref{#1}}
\newcommand{\reftwoproperties}[2]{Properties~\ref{#1} and~\ref{#2}}
 
% Claims
\newcommand{\refclaim}[1]{Claim~\ref{#1}}
\newcommand{\reftwoclaims}[2]{Claims~\ref{#1} and~\ref{#2}}

% References at start of sentence
\newcommand{\Refth}[1]{Theorem~\ref{#1}}
\newcommand{\Reflem}[1]{Lemma~\ref{#1}}
\newcommand{\Refpr}[1]{Proposition~\ref{#1}}
\newcommand{\Refcor}[1]{Corollary~\ref{#1}}
\newcommand{\Refdef}[1]{Definition~\ref{#1}}
\newcommand{\Refrem}[1]{Remark~\ref{#1}}
\newcommand{\Refobs}[1]{Observation~\ref{#1}}
\newcommand{\Refconj}[1]{Conjecture~\ref{#1}}
\newcommand{\Refex}[1]{Example~\ref{#1}}
\newcommand{\Refclaim}[1]{Claim~\ref{#1}}

% References with page numbers
\newcommand{\refthP}[1]{Theorem~\vref{#1}}
\newcommand{\reflemP}[1]{Lemma~\vref{#1}}
\newcommand{\refprP}[1]{Proposition~\vref{#1}}
\newcommand{\refcorP}[1]{Corollary~\vref{#1}}
\newcommand{\refdefP}[1]{Definition~\vref{#1}}
\newcommand{\refremP}[1]{Remark~\vref{#1}}
\newcommand{\refobsP}[1]{Observation~\vref{#1}}
\newcommand{\refconjP}[1]{Conjecture~\vref{#1}}
\newcommand{\refexP}[1]{Example~\vref{#1}}
\newcommand{\refpropertyP}[1]{Property~\vref{#1}}

% Some more references
\newcommand{\refrule}[1]{rule~\ref{#1}}
\newcommand{\reftworules}[2]{rules~\ref{#1} and~\ref{#2}}

\newcommand{\refpart}[1]{part~\ref{#1}}
\newcommand{\Refpart}[1]{Part~\ref{#1}}
\newcommand{\reftwoparts}[2]{parts~\ref{#1} and~\ref{#2}}
\newcommand{\Reftwoparts}[2]{Parts~\ref{#1} and~\ref{#2}}

\newcommand{\refitem}[1]{item~\ref{#1}}
\newcommand{\Refitem}[1]{Item~\ref{#1}}
\newcommand{\reftwoitems}[2]{items~\ref{#1} and~\ref{#2}}
\newcommand{\Reftwoitems}[2]{Items~\ref{#1} and~\ref{#2}}

\newcommand{\refcase}[1]{case~\ref{#1}}
\newcommand{\Refcase}[1]{Case~\ref{#1}}
\newcommand{\reftwocases}[2]{cases~\ref{#1} and~\ref{#2}}
\newcommand{\Reftwocases}[2]{Cases~\ref{#1} and~\ref{#2}}

% References to equations (alias for \eqref just for simplicity)
%
% The definition of \refeq overwrites a command in the mathtools package
% if this package has been loaded

\ifthenelse
{\isundefined{\refeq}}
{\newcommand{\refeq}[1]{\eqref{#1}}}
{\renewcommand{\refeq}[1]{\eqref{#1}}}
\newcommand{\refeqP}[1]{\eqref{#1} on page~\pageref{#1}}




%%%
%%% SOME LOCAL MACROS FOR THE IDMA COURSE
%%%

% GCD
%   \DeclareMathOperator{\gcd}{gcd}

% Two-norm
\newcommand{\twonorm}[1]{\lVert#1\rVert_2}
\newcommand{\Twonorm}[1]{\bigl\lVert#1\bigr\rVert_2}
\newcommand{\TWONORM}[1]{\left\lVert#1\right\rVert_2}

% Formal language grammars
%   \newcommand{\produces}{\rightarrow}
%   \newcommand{\terminalf}[1]{\mathtt{#1}}
%   \newcommand{\tokenf}[1]{\text{\textbf{#1}}}
%   \newcommand{\emptystring}{\varepsilon}
%   \newcommand{\numtoken}{\tokenf{num}}

% For checkmark and xmark
\usepackage{pifont}
\newcommand{\xmark}{\ding{55}}

% For formatting induction proofs
\newcommand{\indproofstep}[1]%
{
  \smallskip
  \noindent
  \textbf{\textit{#1:}}
}

\newcommand{\indbase}[1]{\indproofstep{Base case (#1)}}
\newcommand{\indstep}{\indproofstep{Induction step}}
\newcommand{\indclaim}{\indproofstep{Claim}}





% For getting watermark "DRAFT" across all pages (for instance, 
% when posting preliminary version of problem set)
%    \usepackage{draftwatermark}
%    % \SetWatermarkFontSize{20 cm}
%    \SetWatermarkScale{5}

% For METAPOST logo as \hologo{METAPOST}
%   \usepackage{hologo}

% For TikZ
%   \input{Figures/tikz-packages.tex}

%%%
%%% TITLE
%%%

\author{\courseinstructor}
\course{\coursenamelong{}}
\semester{\courseperiod}
\title{\coursenameshort: Problem Set \psetno}

\begin{document}

\maketitle



\begin{abstract}
  \noindent
  \textbf{Due:} \duedate.

  \noindent
  \textbf{Submission:}
  Please submit your solutions
  via \emph{Absalon}
  as a PDF file.
  State your name and e-mail address 
  close to the top   of the first page.
  Solutions should be written in \LaTeX{} or some other math-aware
  typesetting system with reasonable margins on all sides (at least 2.5~cm).
  Please try to be precise and to the point in your solutions and
  refrain from vague statements.
  % Make sure to explain your reasoning.
  Never, ever just state the answer, but always
  make sure to explain your reasoning.
  \emph{Write so that a fellow student of yours can read, understand, and
    verify your solutions.}
  In addition to what is stated below, the general rules 
  for problem sets stated on \emph{Absalon} always apply.

%    
%      \noindent
%      \textbf{Hints:}
%      For most or all problems, ``hints'' can be purchased at a cost of 
%      \mbox{5--10~points}. In this way, you can configure yourself whether
%      you want the problems to be more creative and open-ended, where
%      sometimes a lot can depend on finding the right idea, or whether you
%      want them to be more of guided exercises providing a useful work-out
%      on the concepts of proof complexity. If you do not solve a problem,
%      there is no charge for the hint (i.e., it is not deducted from the
%      score on other problems).  
%    

  \noindent
  \textbf{Collaboration:}
  Discussions of ideas in groups of 
  two to three people 
  are allowed---and indeed, encouraged---but 
  you should always write up your solutions completely on your own,
  from start to finish, and you should understand all aspects of them
  fully. It is not allowed to compose draft solutions together and
  then continue editing individually, or to share any text, formulas,
  or pseudocode. Also, no such material may be downloaded from or
  generated via the internet to be used in draft or final solutions.
  Submitted solutions will be checked for plagiarism.

% 
%     You should also clearly acknowledge any collaboration.
%     State   close to the top 
%     of the first page of your problem set
%     solutions if you have been collaborating with someone and if so with
%     whom.  
%     \emph{Note that collaboration is on a per problem set basis,
%       so you should not discuss different problems on the same problem
%       set with different people.}
%    
%    
%   
%     \noindent
%     \textbf{Reference material:} 
%     Some of the problems are ``classic'' and hence it might be easy to
%     find solutions on the Internet, in textbooks or in research
%     papers. It is not allowed to use such material in any way unless
%     explicitly stated otherwise. Anything said during the lectures or in
%     the lecture notes 
%   %      , or which can be found in chapters of Arora-Barak covered in
%   %      the course,  
%     should be fair game, though, unless you are specifically asked to
%     show something that we claimed without proof in class. All
%     definitions 
%   %      used 
%     should be as given in class
%     or in Arora-Barak 
%     and cannot be substituted by 
%   %      definitions
%     versions
%     from other sources.  It is
%     hard to pin down 100\% watertight formal rules on what all of this
%     means---when in doubt, ask the main instructor.
%     

  \noindent
  \textbf{Grading:}
  A score of 
  \thresholdforpass
  is guaranteed to be enough to pass this problem set.


  \noindent
  \textbf{Questions:}
  Please do not hesitate to ask the instructor or TAs if any problem
  statement is unclear, but please make sure to send private
  messages---sometimes specific enough questions could give away the
  solution to your fellow students, and we want all of you to benefit
  from working on, and learning from, the problems.
%   
  Good luck!
\end{abstract}





\begin{problem}
  In this problem we wish to compare different sorting algorithms.
%   
%     In this problem we will perform dry-runs of different sorting
%     algorithms and compare them to each other.
%   
%   Suppose that we are given an array
%   $A = [5, 2, 19, 7, 6, 12, 10, 17, 13, 14]$  
%   to be sorted in increasing order.
%   

\begin{subproblem}%
  (40 p)
  The exercises in Chapter 2 of CLRS mention the \emph{bubblesort}
  algorithm, which can be further optimized as follows:
\begin{verbatim}
OptimizedBubbleSort (A)
    i       := 1
    swapped := true
    while (i <= size(A) and swapped) {
        swapped := false
        for j := 1 upto size(A) - i {
            if (A[j] > A[j + 1]) {
                tmp      := A[j]
                A[j]     := A[j + 1]
                A[j + 1] := tmp
                swapped  := true
            }
        }
        i := i + 1
    }
\end{verbatim}
Run the optimized bubblesort algorithm by hand on the array
\begin{equation}
  \label{eq:array-A}
  A = [5, 2, 19, 7, 6, 12, 10, 17, 13, 14]  
\end{equation}
and show how the elements in the array are moved
(similarly to what was done for insertion sort in class).
Argue formally why this algorithm is guaranteed to always sort 
an array correctly. Analyse the time complexity of the algorithm.

\smallskip
\noindent
\emph{Hint:}
Try to find a nice invariant for the inner while loop
%   that will help you to argue 
to help you argue
correctness.

\begin{solution}
  For the dry-run of the algorithm, let us look at the iterations of
  the while loop, which we enter for the first time after having set
  $i=1$ and \verb+swapped+ to true. 
  \begin{enumerate}
  \item 
    For $i=1$, 
    we first flip \verb+swapped+ to false and then run the inner for
    loop for $j$ from~$1$ to~$9$.
    We  first compare
    $A[1] = 5$
    with
    $A[2] = 2$
    and swap them, meaning that \verb+swapped+ is flipped back to true.
    For the next comparison we have
    $A[2] = 5 < A[3]= 19$, so no swap is made.
    From this point onward,
    $A[3] = 19$ will be compared to all other elements stored at
    higher indices in the array, and will be shifted step by step
    until it reaches index~$10$, since $19$ is the largest element in
    the array.
    Once we are done with the for loop, we increment $i$ to~$2$.
    At the end of the first iteration the array looks like
    \begin{equation}
      \label{eq:array-A-1st-iter}
      A^{(1)} 
      = 
      [2, 5, 7, 6, 12, 10, 17, 13, 14, 19]  
      \eqperiod
    \end{equation}

    
  \item 
    Since \verb+swapped+ is true and 
    $i=2$ is less than the length of the array, 
    we enter a second iteration of the while loop in which the inner
    for loop runs for $j$ from~$1$ to~$8$
    (after we have assigned \verb+swapped+ to false again).
    In this iteration, we swap places 
    of $A[3] = 7$ and  $A[4] = 6$
    and
    of $A[5] = 12$ and  $A[6] = 10$
    (meaning that \verb+swapped+ is also set to true).
    Then 
    $A[7] = 17$ 
    is compared to  $A[8] $ and  $A[9] $
    and is shifted into position~$9$, since $17$ is the next largest element
    in the array,
    after which $i$ is incremented to~$3$.
    At the end of the second iteration the array looks like
    \begin{equation}
      \label{eq:array-A-2nd-iter}
      A^{(2)} 
      = 
      [2, 5, 6, 7, 10, 12, 13, 14, 17, 19]  
      \eqperiod
    \end{equation}


  \item 
    Since \verb+swapped+ is true and 
    $i=3$ is less than the length of the array, 
    we enter a third iteration of the while loop in which the inner
    for loop runs for $j$ from~$1$ to~$7$.
    This time, however, we have that 
    $A[j] \leq A[j+1]$
    for all pairwise comparisons, and so \verb+swapped+ 
    is never flipped back to true again but stays false.
    Therefore, the whole algorithm terminates after this iteration.
  \end{enumerate}

  From this dry-run, we can see that for an array of size~$n$, in the
  first iteration the largest element in the array is guaranteed to
  ``bubble up'' to index~$n$, in the second iteration the
  next-to-largest element bubbles to index~$n-1$, et cetera. Guided
  by this, we propose the following invariant:
%
  \begin{quotation}
    \noindent
    \emph{At the start of each iteration of the while loop,
      the subarray from $A[n-i+2]$ to~$A[n]$ contains the $i-1$
      largest elements in~$A$  sorted in correct order.}
  \end{quotation}
%
  Note that this invariant is vacuously true in the first iteration
  for $i=1$
  (since it says that the zero largest elements can be found in an
  empty part of the array).

  In what follows, let us assume for simplicity that all elements in
  the array are all different, so that there is a unique $i$th largest
  element for each $i= 1, \ldots n$. Once we are done with the
  analysis, it should be clear that this assumption is not needed, and
  that all of our argument will go through even if there are duplicates.

  Suppose that the invariant is true right when the $i$th
  iteration is about to start.
  Then the $i-1$ largest element are already placed in order in
  correct position, and the $i$th largest element
  resides in some position between~$1$ and $n-i+1$.
  During the $i$th iteration the algorithm will look at all elements
  $A[1],\, A[2],\, \ldots, \, A[n-i], \, A[n-i+1]$,
  and as soon as the algorithm encounters the $i$th largest element 
  we see that this element
  will ``win'' all comparisons it takes part in and will be shifted
  all the way to the right, ending up on position~$n-i+1$.
  But this shows that the invariant will also hold at the start of the
  next iteration.
  (And if there are duplicates, then one of the largest remaining elements will 
  bubble to position~$n-i+1$, and we do not care for which copy this happens.)

  
  It remains to analyse the situation when the algorithm terminates.
  This can happen either because $i = n+1$ or because \verb+swapped+ is
  false.
  In the former case, we know from the invariant that
  the subarray from $A[n-(n+1)+2]$ to~$A[n]$ now contains the $(n+1)-1$
  largest elements sorted in correct order,
  which is a somewhat complicated way of saying that the whole array
  is sorted.
  In the latter case, we know from the invariant that
  the subarray from $A[n-i+2]$ to~$A[n]$ contains the $i-1$
  largest elements sorted in correct order. Moreover, 
  since \verb+swapped+ was never set to true during the last
  iteration, it must be the case that all elements
  $
  A[1],\ A[2],\ \ldots, \,  A[n-i+1],\ A[n-i+2]
  $
  appear in increasing order
  $
  A[1] \leq A[2] \leq \ldots \leq  A[n-i+1] \leq A[n-i+2]
  $.
  (Note that we are using here that $i$ was incremented after the last
  iteration, so we know that in the last iteration all elements up to~$A[n-i+2]$
  were looked at). 
  Hence, all of the array must be sorted. 

  Finally, as to time complexity, the outer while loop
  can run for at most~$n$ times for
  $i= 1, 2, \ldots, n$, and the inner for loop has $i$~iterations
  every time. 
  It is not hard to see that this worst-case scenario can arise, for instance,
  for an array sorted in reverse order.
  There is a constant number of work being done inside the innermost
  for loop, and at all other places in the algorithm, so the worst-case running
  time scales like
  $
  \sum_{i=1}^{n}i 
  = 
  \frac{n(n+1)}{2}
  =
  \Bigtheta{n^2}
  $.
\end{solution}

\end{subproblem}

\begin{subproblem}%
  (30 p)
  Run merge sort by hand on the array in   \refeq{eq:array-A}
  (as in the notes for the lectures).
  Show in every step of the algorithm what recursive calls are made
  and how the results from these recursive calls are combined,
  and make sure to explain the final (and most interesting) merge step carefully.
  (Any clear way of explaining is fine---you do not have to learn how to draw
  pictures in \LaTeX{} if you do not want to.)
%   
%     (Making legible hand-drawing and appending scanned pictures to your 
%     typed up solution is fine if you do not what to typeset pictures in
%     \LaTeX.)
\end{subproblem}


\begin{solution}
  See
%     \reffig{fig:merge-sort-example}.
  \reffig{fig:merge-sort-recursion-tree} for an illustration of what
  happens in the algorithm.

  We first recursively split the list in two part, where the first
  part is one element larger if the list size is odd, until all lists
%     have sizes at most~$2$.
  have size~$1$.
  This leads to the split lists in the leaves
  of the recursion tree in
  \reffig{fig:merge-sort-recursion-tree-down}.
%   
%     We are allowing ourselves a small optimization here compared to the
%     pseudocode that was presented in class, in that we terminate the
%     recursion when the lists have size at most~$2$. As we have discussed
%     before, lists consisting of single elements are already sorted, 
%     and a list consisting of a pair of element can be sorted most simply
%     be swapping places for the elements if needed, rather than by
%     powering up the merge algorithm. If the lecturer would have been a
%     little bit less lazy, then he would have drawn a figure with an
%     extra layer where pairs had been split to two singletons, but
%     instead he is using this exercise to illustrate that small,
%     inconsequential, changes  like this can be perfectly in order as
%     long as you take care to explain clearly what you are doing.
%     In any case, once  we have be base cases in the leaves of
%     \reffig{fig:merge-sort-recursion-tree-down},
%     we easily obtain sorted versions of these lists as shown in the
%     leaves of
%     \reffig{fig:merge-sort-recursion-tree-up}.
%   

  In the merge phase we work bottom up from the level above the
  leaves. Every parent node~$P$ takes its two children lists $C_1$ and~$C_2$, 
  which have previously been sorted, and merges them.
  We do so by placing to pointers $e_1$ and~$e_2$ at the start of
  $C_1$ and~$C_2$, respectively,
  and then adding the smallest of these elements to the list under
  construction after which the corresponding pointer is advanced.

  Let us give some examples of how this works, 
  but ignoring the merging of lists of size~$1$ which is not so exciting
  (since either the elements are in order or we should just swap them---and,
  in fact, a more reasonable way to implement merge sort would be to have 
  size~$\leq 2$ as base case and then swap elements if needed for
  lists of size~$2$ rather than making a call to the merge method).
  For instance, for the two leftmost
%     leaves 
  vertices three levels down
  we have
  $
  e_1 = 2 < 19 = e_2
  $,
  so $2$ is added first and $e_1$ advances to~$5$. Since
  $5 < 19$ we add $5$ to the output also. Now~$C_1$ has been emptied,
  and so we just append~$C_2$ to obtain the list
  t $[2, 5, 19]$.
  When
  $[2, 5, 19]$
  is merged with
  $[6,7]$,
  then $2$ is the smallest element and goes first, and the next
  element $5$ is also taken from this list.
  But then $6$ and~$7$ are both smaller than~$19$, which goes last,
and as a result we  get the
  list
  $[2, 5, 6, 7, 19]$.
  The recursive calls in the right subtree of the root are dealt with
  similarly (and this   should be described in the solutions,
  although it is perfectly fine to be a bit brief once you are
  explaining the merge steps for the third time or so).

  In the final step, we need to merge
  $[2, 5, 6, 7, 19]$
  and
  $[10, 12, 13, 14, 17]$---note that we are specifically
  asked in the problem statement to explain this merge step carefully,
  and so we will make sure to do so.
  Here we start with
  $
  e_1 = 2 < 10 = e_2
  $,
  so $2$ goes first, updating $e_1$ to~$5$.
  Now we still have
  $
  e_1 = 5 < 10 = e_2
  $,
  so we add $5$ to the list under construction, updating $e_1$ to~$6$.
  In the same way we add $6$ and~$7$ from the left-hand side.
  After this we reach $e_1 = 19$, however,
  which leads to the whole list of the right-hand side being added,
  since all elements in this list are smaller than~$19$.
  Once the list on the right-hand side has been emptied, $19$~is added
  at the end.
  This produces the sorted list at the root of the tree in
  \reffig{fig:merge-sort-recursion-tree-up}.

%   \begin{figure}[tp]
  \begin{subfigure}[b]{0.48\textwidth}
    \centering
    \includegraphics[scale=.65]{Figures/merge-sort-recursion-tree.1}
    \caption{Tree of recursive calls splitting list.}
    \label{fig:merge-sort-recursion-tree-down}
  \end{subfigure}
  \begin{subfigure}[b]{0.48\textwidth}
    \centering
    \includegraphics[scale=.65]{Figures/merge-sort-recursion-tree.2}
    \caption{Bottom-up sorting of split lists.}
    \label{fig:merge-sort-recursion-tree-up}
  \end{subfigure}
    \caption{Merge sort
%         by hand 
      of array $
      A = [8, 4, 2, 9 , 3, 1, 5, 10, 7, 6]
      $.}
    \label{fig:merge-sort-recursion-tree}
\end{figure}


%%% Local Variables:
%%% mode: latex
%%% TeX-master: "ProblemSet2_2022"
%%% End:

\begin{figure}[tp]
  \begin{subfigure}[b]{0.48\textwidth}
    \centering
    \includegraphics[scale=.65]{Figures/merge-sort-recursion-tree.1}
    \caption{Tree of recursive calls splitting list.}
    \label{fig:merge-sort-recursion-tree-down}
  \end{subfigure}
  \begin{subfigure}[b]{0.48\textwidth}
    \centering
    \includegraphics[scale=.65]{Figures/merge-sort-recursion-tree.2}
    \caption{Bottom-up sorting of split lists.}
    \label{fig:merge-sort-recursion-tree-up}
  \end{subfigure}
    \caption{Merge sort
%         by hand 
      of array $
      A = [8, 4, 2, 9 , 3, 1, 5, 10, 7, 6]
      $.}
    \label{fig:merge-sort-recursion-tree}
\end{figure}


%%% Local Variables:
%%% mode: latex
%%% TeX-master: "ProblemSet2_2022"
%%% End:


\end{solution}

\begin{subproblem}%
  (20 p)
  Suppose that we are given another array~$B$ 
  of size~$n$
  that is already sorted
  in increasing order. 
  How fast do the merge sort and optimized bubblesort algorithms run
  in this case? 
  Is any of them asymptotically faster 
  than the other
  as the size of the array~$B$ grows?
\end{subproblem}

\begin{solution}
  Merge sort will be the same,
  i.e.,   $O(n \log n)$. 
  The algorithm will always split the input into lists of constant
  size, which requires a logarithmic number of recursive calls, 
  and merging will always take a linear amount of work per recursion
  level even if the lists are sorted. (Why?)
  Another matter is that this might seem overly stupid, and, indeed, 
  there is a \emph{natural merge sort} version of the algorithm
  that identifies sorted runs in the input and does not split already
  sorted sublists further. But even without this optimization merge
  sort is a very efficient algorithm.
  
  Bubblesort  just verifies 
  in a single pass through the array 
  in linear time that the array is
  already sorted, and so will be significantly faster if the input is
  an array that is already sorted in the correct order.
\end{solution}

\begin{subproblem}%
  (20 p)
  Suppose that we are given a third array~$C$   of size~$n$  that is
  sorted in \emph{decreasing} order, so that it needs to be reversed
  to be sorted in the order that we prefer, namely increasing.
  How fast do the merge sort and optimized bubblesort algorithms run
  in this case?
  Is any of them asymptotically faster than the other
  as the size of the array~$C$ grows?
\end{subproblem}

\begin{solution}
  Merge sort will be the same,
  i.e.,   $O(n \log n)$. 

  Bubblesort will require quadratic time, since the smallest element in
  the array will only be shifted down by one step in every iteration
  until it finally reaches the correct index~$1$ in the very last
  iteration, 
  and every iteration takes linear time as per the analysis presented above.
\end{solution}


%   
%   \begin{subproblem}
%     Suppose that we build a max-heap from an array~$A$. Is it true that
%     once the array~$A$ has been converted to a correct max-heap, then
%     considering the elements in reverse order
%     (\ie 
%     $
%     A[n],
%     A[n-1],
%     \ldots,
%     A[2],
%     A[1]
%     $)
%     yields a correct min-heap? Prove this or give a simple counter-example.
%   \end{subproblem}
%   

\end{problem}

\begin{problem}
%     Last year
  In 2021
  DIKU celebrated its 50th anniversary with a lot of pomp,
  although slightly less emphasis was given to the fact that it was
  done one year late due to the Covid pandemic.
%   
  Even less publicity was given to the public outreach day
  organized in F{\ae}lledsparken for school children by the Algorithms
  and Complexity Section as part of the anniversary, for reasons that
  might become clearer after you have studied the problems below.

  \begin{subproblem}%
    \label{problem:balls}%
    (30 p) 
    In one of the events of the AC Section outreach day, Jakob had
    arranged so that  $51$~children%
    \footnote{Well, because it was a $51$st anniversary, after all.} 
    were given brightly coloured balls, 
%       and were instructed to     position themselves 
    and were positioned
    in a field in such a way that all the pairwise
    distances between the children were distinct. The children were
    then asked to identify which other child was closest to them and, at a
    given signal, to throw their ball to this child (and hopefully also 
    receive an incoming    ball from somewhere).

    This turned out to be a public relations catastrophe.  However
    the children were positioned as described above,
    every time at least one child ended up without a ball
    (but instead with tears in the eyes). This did not at all generate
    the goodwill DIKU was hoping for.

    What went wrong? Was Jakob just immensely unlucky? Or can you
    prove mathematically that 
%       the way he had set up this event it had    to be the case 
    it was unavoidable
    that at least one child would end up without a ball?
    Would this had been different if Jakob had not insisted on
    $51$~children, but had accepted the proposal by his colleagues to
    have $50$~children? Or if not all distances would have had to be different?
  \end{subproblem}

\begin{solution}
  Firstly, we note that if the distances would not all have had to be
  different, the children could have arranged themselves at equal
  distance along a circle and could have thrown their balls
  clockwise, making everybody happy.
  Also, if we would have had an even number of children, then
  they could have been placed in pairs close to each other but far
  from all other pairs of children, and then each pair of children
  would just have exchanged balls.

  However, if all distances have to be different, and if the total
  number of children is odd, then there is no way to make everybody
  happy. 
  Let us prove this by induction for any odd number~$n = 2k+1$~of children.
  \begin{description}
  \item[\emph{Base case $(n=3)$:}]
    All pairwise distances between the children are different.
    The two children with the shorted pairwise distance between
    them---let us call them Anders and Betina---will throw
    their balls to each other.
    The third child, whom we call Christian,
    will throw his ball to either Anders or Betina, but will not get a
    ball in return.
    
  \item[\emph{Induction step:}]
    Our induction hypothesis is that for $n= 2k-1$ children 
    there is no way to make all children receive a ball.

    Suppose now that we have $n+2 = 2k+1$ children and consider again
    the two children Anders and Betina with the shorted pairwise distance,
    who will throw their balls to each other.
    We now have two cases:
    \begin{enumerate}
    \item 
      Some other child throws their ball to Anders and Betina.
      If so, we are done. Clearly, there are now too few balls left
      for the other children to get one ball each.

    \item 
      Nobody throws a ball to Anders or Betina.
      If so, this means that we can completely remove Anders and
      Betina from the picture when considering the rest of the
      children,
      meaning that we have exactly the same problem but with
      $n=2k-1$ children.
      We know by the induction hypothesis that there is no way
      all of these children will get balls.
    \end{enumerate}
  \end{description}
    % 
    It now follows by the induction principle that
    whenever Jakob's game is played with an odd number of
    children, at least one child will be left without a ball.

    In particular, it was unavoidable that at least one of the
    $51$~children at the DIKU outreach day would be left without a ball.
\end{solution}

  \begin{subproblem}%
    \label{problem:cars}%
    (30 p) 
    In another event, Jakob built a $5$-kilometre 
    car track across the park. On this track, $51$~electric cars were
    placed at random locations 
    (but all pointing in the same direction clockwise around the circuit).
%       
    One car battery was sufficient for exactly one full lap
    if charged to 100\% capacity. However, instead
    the batteries of all the cars were charged partially in such a way that
    the total charge of all the batteries together was sufficient for one
    car to travel exactly the full distance of $5$~kilometres.
    After this, the batteries were distributed to the cars in some random way.

    The children were given the challenge to start driving one car in
    such a way that one full lap of the track would covered. The
    rules were that if one car travelled far enough to bump into the
    rear of the car in front, then this next car could continue, and
    also the battery from the car behind could be shifted to the car
    in front so that the front car could use any remaining charge
    (and similarly for any other batteries picked up along
    the way). If, however, a car would run out of batteries before
    reaching the car in front, or before the full 
    lap
%       track 
    was completed by all the cars together, 
%       then 
    this was a failure.

    This event went slightly better, in that the children were able to
    figure out a solution to the challenge most of the time. 
    Jakob prided himself with that this was thanks to the fact that he
    had given the friendly advice, to avoid more
    embarrassment as in Problem~\ref{problem:balls}, that a good
    strategy was to start with the car with the most charge in its battery.
    Were the children just lucky this time, or can you prove that
    there is always a solution to this challenge?
    And is it a good idea to follow Jakob's advice, or does it seem
    more likely that the children figured out something smarter?

    \smallskip
    \noindent
%       \emph{Illustration:}    
    \emph{Example:}
    If we for simplicity consider $3$ cars on a perfectly circular track, with
%       To give a concrete example, albeit with only $3$~cars for
%       simplicity, if 
    Car~$1$ 
%       is 
    placed at $12$~o'clock with a $10$\% charge,
    Car~$2$ 
%       is 
    placed at $3$~o'clock with a $60$\% charge,
    and
    Car~$3$ 
%       is 
    placed at $9$~o'clock with a $30$\% charge,
    then starting with Car~$2$ is a winning strategy whereas starting
    with Car~$1$ or Car~$3$ leads to failure.


  \end{subproblem}
   
\begin{solution}
Jakob is wrong---it is not necessarily a good idea to start with the car
with the highest charge.
Take the example above, but move Car~$3$ from
$9$~o'clock to $1$~o'clock.
Then starting with Car~$2$ is a failing strategy.

However, it is indeed the case that there is always a solution to this
challenge. We will prove this by induction over the number of cars,
but we will first make a crucial observation.
%
  \begin{quotation}
    \noindent
    \emph{There has to exist at least one car
      that can drive far enough to reach the next car.}
  \end{quotation}
%
Suppose that this is not the case.
Go over the cars along the track in clockwise order,
and sum up their fractional charges.
Since no car can reach the next, each fractional charge
is strictly smaller than the distance to the next car measured as a
fraction of the total length of the track.
This means that the sum of all fractional charges is strictly smaller
than the sum of all fractional distances along the track. But this
latter number is clearly~$1$, also known as~$100$\%,
and the total charge of all batteries is supposed to sum to~$100$\%
according to the problem description, and so it is impossible that we
would have a strict inequality. This proves the claim in our
observation.


We proceed to our proof by induction over the number~$n$~of cars.
\begin{description}
\item[\emph{Base case ($n=1$):}]
  If we have just one car, 
  then this car has to have a $100$\% charge and so can drive around
  the whole track.


\item[\emph{Possible second base case ($n=2$):}]
  Just in case $n=1$ feels like a weird base case, you can also start
  with $n=2$. If so, we note that by our observation above one of the
  cars can reach the other. Since the sum of the charges of the two
  batteries is $100$\%, the second car will be able to complete the
  whole track.

    
  \item[\emph{Induction step:}]
    Our induction hypothesis is that the problem is possible to solve
    for $n$~cars.

    Suppose now that we have $n+1$~cars. By our observation, one of
    these cars---say car number~$i$---can reach the next car~$i+1$.

    But if so, we can mentally consider cars~$i$ and~$i+1$ together to
    be a new super-car with a battery that contains the sum of the
    charges of the batteries in the
    two cars and that is placed at the same position as car~$i$. It should be
    clear that if we can solve the problem with this new super-car,
    then we can also solve the original problem.
    
    But we see that our new problem is just the problem with exactly
    the right conditions for~$n$~cars, and our induction hypothesis
    says that there is a solution for any problem with~$n$ cars.
  \end{description}
  It follows by the principle of mathematical induction that the
  problems is always solvable for any number of cars~$n$ (and, in
  particular, for $n=51$).
\end{solution}


%       
\begin{figure}[t]
  \begin{subfigure}[b]{0.48\textwidth}
    \centering
    \includegraphics{Figures/tiling-L.1}
    \caption{Punctured square grid.}
    \label{fig:tiling-grid}
  \end{subfigure}
  \begin{subfigure}[b]{0.48\textwidth}
    \centering
    \includegraphics{Figures/tiling-L.2}
    \caption{Tiling of punctured square grid.}
    \label{fig:completed-tiling}
  \end{subfigure}
    \caption{Tiling a punctured
      $2^n$-by-$2^n$ grid with L-shaped tiles
      (for $n=3$).}
    \label{fig:tiling-problem}
\end{figure}



    
\begin{figure}[t]
  \begin{subfigure}[b]{0.48\textwidth}
    \centering
    \includegraphics{Figures/tiling-L.1}
    \caption{Punctured square grid.}
    \label{fig:tiling-grid}
  \end{subfigure}
  \begin{subfigure}[b]{0.48\textwidth}
    \centering
    \includegraphics{Figures/tiling-L.2}
    \caption{Tiling of punctured square grid.}
    \label{fig:completed-tiling}
  \end{subfigure}
    \caption{Tiling a punctured
      $2^n$-by-$2^n$ grid with L-shaped tiles
      (for $n=3$).}
    \label{fig:tiling-problem}
\end{figure}




  \begin{subproblem}%
    \label{problem:tiling}%
    (30 p) 
    In the final event of the day,
    a big $2^n$-by-$2^n$ grid was constructed,
    after which one cell in the grid was removed
    by placing a black square on it
    as illustrated in
    \reffig{fig:tiling-grid}.
    The children were then given the
    task to cover all the other cells in the grid by placing L-shaped
    tiles in such a way that every cell was covered exactly once,
    and nothing outside of the grid was covered,    as shown in
    \reffig{fig:completed-tiling}.

    By now the children were fairly fed up with these strange games,
    however, and Jakob's colleagues also started getting a bit
    annoyed, wondering if the strange shape of the tiles
    was somehow a not-so-subtle attempt to push for a
%       competing foreign university
    competing foreign university at the other side of the
    \O{}resund strait
    instead, and the day did not end on a festive
    note at all.
    Disregarding this unfortunate turn of events, 
    can you prove that it is actually true that for  any
    $2^n$-by-$2^n$ grid, regardless of how it is punctured by removing
    a cell, it is always possible to tile the rest of the grid with L-shaped
    tiles?
  \end{subproblem}

\begin{solution}
  Let us prove 
  by induction over~$n$
  that it is always possible
  to tile a punctured 
  $2^n$-by-$2^n$ grid with L-shaped tiles, regardless of how the
  puncturing is made.

  \begin{description}
  \item[\emph{Base case ($n=1$):}]
    For a $2$-by-$2$ tile it is clear that removing any of the 
    $4$~cells will make the remaining $3$~cells form an~L.

%           
\begin{figure}[t]
    \centering
    \includegraphics{Figures/tiling-L.3}
    \caption{Inductive step in tiling problem.}
    \label{fig:tiling-problem-induction}
\end{figure}



        
\begin{figure}[t]
    \centering
    \includegraphics{Figures/tiling-L.3}
    \caption{Inductive step in tiling problem.}
    \label{fig:tiling-problem-induction}
\end{figure}



    
  \item[\emph{Induction step:}]
    Our induction hypothesis is that any punctured
    $2^n$-by-$2^n$ grid can be tiled with L-shaped tiles.

    Consider a grid of size $2^{n+1}$-by-$2^{n+1}$, and
    suppose without loss of generality (because of rotational
    symmetry) that some cell in the bottom leftmost half of the grid is
    removed as in
    \reffig{fig:tiling-problem-induction}.
    Divide the grid up in $4$~equal-sized quarters
    by drawing a horizontal and vertical line
    after the $2^n$ first rows and columns, respectively.
    Place an L-shaped tile where these lines intersect
    so that the tile touches the $3$ non-punctured quarters of the square grid.
    Now we have $4$~punctured subgrids of size $2^n$-by-$2^n$.
    By the inductive hypothesis, all of these 
    subgrids can be tiled with L-shaped tiles.
  \end{description}
  % 
  It follows by the induction principle that 
  it holds for all positive integers~$n$ that 
  any punctured
  $2^n$-by-$2^n$ grid can be tiled with L-shaped tiles.
\end{solution}

\end{problem}




\end{document}



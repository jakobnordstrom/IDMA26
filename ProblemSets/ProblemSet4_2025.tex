\documentclass{jn-pset}
%   \documentclass[solutions]{jn-pset}

\usepackage{ifthen}
\newboolean{versionwithsolutions}
% Uncomment to insert text that should only be there in version WITH solutions
\setboolean{versionwithsolutions}{true}
% Uncomment to insert text that should only be there in version WITHOUT solutions
\setboolean{versionwithsolutions}{false}


%%%%
%%%% QUICK INSTRUCTIONS FOR FORMATTING OF PROBLEMS
%%%%
%    
% Code a problem by
%    \begin{problem}
%    \end{problem}
%
% Code a subproblem inside a problem by (note the percent signs!)
%    \begin{subproblem}%
%        \label{problem:labelhere}%
%        Text here
%    \end{subproblem}
%
% Get a small vertical space by issuing the command
%    \smallskip
%
% For instance, to give a hint to a problem (after having completed
% the problem statement) code in the following way:
%
%    \smallskip
%     \noindent
%     \emph{Hint:}
%     Consider the following super-useful hint for this particular problem...
%

% PROBLEM-SET-SPECIFIC MACROS (UPDATE FOR EACH PROBLEM SET)

\newcommand{\psetno}{4}
\newcommand{\duedate}{Wednesday March 26 at 17:15 CET}

\newcommand{\thresholdforpass}{$120$~points\xspace}

% COURSE-SPECIFIC MACROS FOR IDMA 2025

\newcommand{\coursenameabbrev}{IDMA}
\newcommand{\coursenameshort}{Introduktion til diskret matematik og algoritmer}
\newcommand{\coursenamelong}{NDAB23002U Introduktion til diskret matematik og algoritmer}
%    
\newcommand{\submissionemail}{jn@di.ku.dk\xspace}
%   \newcommand{\courseinstructor}{Jakob Nordstr\"om}
\newcommand{\courseinstructor}
        {Jakob Nordstr\"om and Srikanth Srinivasan}
\newcommand{\courseperiod}{2024/2025}

% PACKAGES, MACROS, ET CETERA

\usepackage[T1]{fontenc}
\usepackage[utf8]{inputenc}
%%% Apparently a newer version of babel doesn't play well with nada-ten
%    \usepackage[english]{babel}

\usepackage{hyperref}

\usepackage{amsmath}
\usepackage{amssymb}
\usepackage{amsfonts}
\usepackage{mathtools}

% Provide calligraphic \mathscr font
\usepackage{mathrsfs}

% Enable use of MetaPost generated PostScript files
\usepackage{ifpdf}
\usepackage{graphicx}  
\ifpdf         
\DeclareGraphicsRule{*}{mps}{*}{}
\fi            

% For getting subfigures 1(a), 1(b) etc
% The package "subfigure" is obsolete, so switch to subcaption
%    \usepackage[sf,SF]{subfigure}
\usepackage{subcaption} 

% Sam Buss's package for formatting proofs
\usepackage{bussproofs}

% Extensions to verbatim commands
\usepackage{verbatim}

% Smiley
\usepackage{wasysym}

% To choose how to enumerate lists
\usepackage{enumerate}

%%%%
%%%% THIS FILE IS INTENDED TO BE READ-ONLY --- PLEASE DO NOT EDIT.
%%%% PLEASE CONTACT JAKOB NORDSTRÖM AT jn@di.ku.dk REGARDING ANY ISSUES.
%%%%

%
% DETECTION OF DOCUMENT TYPE
%----------------------------
%
% Version date: September 24, 2022
%
% First versions by Jakob Nordström <jn@di.ku.dk>
% Cleaned up version by Marc Vinyals <vinyals@kth.se>
% Minor later additions by Jakob Nordström and Susanna F. de Rezende
%
% This file needs to be included or input(ted) so that the conditional
% macro definitions in other LaTeX files will not generate compilation
% errors. Most document classes are detected using \@ifclassloaded.
%
% Use the file 'testdoctypedetection.tex' to double-check that these
% Boolean detectors work.
%
%
%%% The report class is not detected correctly in view of later
%%% updates, but this should be easy to fix when needed. 
%%% [Jakob Nordström, May 16, 2016]
% detectedReport is set to true if none of detectedArticle, detectedThesis,
% detectedSTOC, detectedFOCS, detectedSIAM, detectedIEEE, detectedICS,
% or detectedPoster is true.
%


\usepackage{ifthen}

\provideboolean{detectedSTOC}
\provideboolean{detectedFOCS}
\provideboolean{detectedElsevier}
\provideboolean{detectedNOW}
\provideboolean{detectedLMCS}
\provideboolean{detectedIEEE}
\provideboolean{detectedPoster}
\provideboolean{detectedSIAM}
\provideboolean{detectedLNCS}
\provideboolean{detectedACM}
\provideboolean{detectedACMconf}
\provideboolean{detectedSigplanconf}
\provideboolean{detectedToC}
\provideboolean{detectedLIPIcs}
\provideboolean{detectedAAAI}
\provideboolean{detectedIJCAI}
\provideboolean{detectedCompCplx}
\provideboolean{detectedEasyChair}
\provideboolean{detectedJAIR}
\provideboolean{detectedArticle}
\provideboolean{detectedReport}
\provideboolean{detectedThesis}

\makeatletter

\@ifclassloaded{sig-alternate}
{\setboolean{detectedSTOC}{true}}
{\setboolean{detectedSTOC}{false}}

\@ifclassloaded{elsarticle}
{\setboolean{detectedElsevier}{true}}
{\setboolean{detectedElsevier}{false}}

\@ifclassloaded{now}
{\setboolean{detectedNOW}{true}}
{\setboolean{detectedNOW}{false}}

\@ifclassloaded{lmcs}
{\setboolean{detectedLMCS}{true}}
{\setboolean{detectedLMCS}{false}}

\@ifclassloaded{IEEEtran} {
  \setboolean{detectedIEEE}{true}
  \ifCLASSOPTIONconference {
    \setboolean{detectedFOCS}{true}
  }
  \else {
    \setboolean{detectedFOCS}{false}
  }
  \fi
}
{
  \setboolean{detectedFOCS}{false}
  \setboolean{detectedIEEE}{false}
}

%%% Obsolete SIAM class file
%   \@ifclassloaded{siamltex1213} 
%   {\setboolean{detectedSIAM}{true}}
%   {\setboolean{detectedSIAM}{false}}
\@ifclassloaded{siamart171218}
{\setboolean{detectedSIAM}{true}}
{\setboolean{detectedSIAM}{false}}

\@ifclassloaded{llncs}
{\setboolean{detectedLNCS}{true}}
{\setboolean{detectedLNCS}{false}}

\@ifclassloaded{acmsmall}
{\setboolean{detectedACM}{true}}
{\setboolean{detectedACM}{false}}

\@ifclassloaded{acmart}
{\setboolean{detectedACMconf}{true}
 \setboolean{detectedACM}{true}}
{\setboolean{detectedACMconf}{false}}

\@ifclassloaded{sigplanconf}
{\setboolean{detectedSigplanconf}{true}}
{\setboolean{detectedSigplanconf}{false}}

\@ifclassloaded{toc}
{\setboolean{detectedToC}{true}}
{\setboolean{detectedToC}{false}}

\@ifclassloaded{lipics}
{\setboolean{detectedLIPIcs}{true}}
{\@ifclassloaded{lipics-v2019}
  {\setboolean{detectedLIPIcs}{true}}
  {\@ifclassloaded{oasics-v2019}
    {\setboolean{detectedLIPIcs}{true}}
    {\setboolean{detectedLIPIcs}{false}}
  }
}

%   
%   \@ifclassloaded{lipics}
%   {\setboolean{detectedLIPIcs}{true}}
%   {\@ifclassloaded{lipics-v2016}
%     {\setboolean{detectedLIPIcs}{true}}
%     {\@ifclassloaded{oasics-v2016}
%       {\setboolean{detectedLIPIcs}{true}}
%       {\setboolean{detectedLIPIcs}{false}}
%     }
%   }
%   

\@ifclassloaded{cc}
{\setboolean{detectedCompCplx}{true}}
{\setboolean{detectedCompCplx}{false}}

\@ifclassloaded{easychair}
{\setboolean{detectedEasyChair}{true}}
{\setboolean{detectedEasyChair}{false}}

%%%
%%% For JAIR, detect that the "jair" package is being used
%%%
\@ifpackageloaded{jair}
{\setboolean{detectedJAIR}{true}}        
{\setboolean{detectedJAIR}{false}}        

%%%
%%% AAAI and IJCAI have special style files that needs to be detected.
%%% It seems they update the name with the year of the conference also.
%%%
\@ifpackageloaded{aaai}
{\setboolean{detectedAAAI}{true}}        
{\@ifpackageloaded{aaai18}
  {\setboolean{detectedAAAI}{true}}
  {\@ifpackageloaded{aaai20}       
    {\setboolean{detectedAAAI}{true}}
    {\setboolean{detectedAAAI}{false}}}}

\@ifpackageloaded{ijcai18}
{\setboolean{detectedIJCAI}{true}}        
{\@ifpackageloaded{ijcai19}
  {\setboolean{detectedIJCAI}{true}}        
  {\setboolean{detectedIJCAI}{false}}}

\@ifclassloaded{sciposter}
{\setboolean{detectedPoster}{true}}
{\setboolean{detectedPoster}{false}}

\@ifclassloaded{article}
{\setboolean{detectedArticle}{true}}
{\setboolean{detectedArticle}{false}}

\makeatother

\ifthenelse{\not \isundefined{\examen} 
  \and \not \isundefined{\disputationsdatum} 
  \and \not \isundefined{\disputationslokal}}   
  {\setboolean{detectedThesis}{true}}
  {\setboolean{detectedThesis}{false}}

%%%
%%% Not entirely sure whether detectedReport is set correctly in view
%%% of later updates [Jakob Nordström, May 16, 2016]
%%%
           
\ifthenelse{\boolean{detectedArticle} \or \boolean{detectedThesis}
  \or \boolean{detectedSTOC}    \or \boolean{detectedFOCS}
  \or \boolean{detectedSIAM}    \or \boolean{detectedIEEE}
  \or \boolean{detectedACMconf} \or \boolean{detectedACM}
  \or \boolean{detectedPoster}}
{\setboolean{detectedReport}{false}}
{\setboolean{detectedReport}{true}}



%%%%
%%%% THIS FILE IS INTENDED TO BE READ-ONLY --- PLEASE DO NOT EDIT.
%%%% PLEASE CONTACT JAKOB NORDSTRÖM AT jakobn@kth.se REGARDING ANY ISSUES.
%%%%

% GENERAL MACROS TO USE IN LaTeX-FILES
%======================================
%
% AUTHOR
%   Jakob Nordström <jakobn@kth.se>
%   Some improvements added by Marc Vinyals <vinyals@kth.se>
%
% VERSION
%   Last updated June 7, 2019
%    
% KNOWN ISSUES:
%   References with page numbers such as \refsecP, \refthP, etc will not
%   work with the Elsevier and SIAM document classes. No work-arounds
%   have been added, so compilation will fail if these macros are used
%   for Elsevier or SIAM articles.

%
% MACRO NAMING CONVENTION FOR MATHEMATICAL MACROS WITH DELIMITERS 
%-----------------------------------------------------------------
%
% For mathematical macros with delimiters there are usually
% three different flavours corresponding to different sizes of 
% the delimiters as follows:
%
%    \newcommand{\mycommand}[1]{<command> ( {#1} )}
%    \newcommand{\Mycommand}[1]{<command> \bigl( {#1} \bigr)}
%    \newcommand{\MYCOMMAND}[1]{<command> \left( {#1} \right)}
%

%
% REQUIRED PACKAGES
%-------------------
%

\usepackage{ifthen}
\usepackage{xspace}
% varioref does not seem to mix well with the Elsevier document class
\ifthenelse
{\boolean{detectedElsevier} \or \boolean{detectedSIAM} 
  \or \boolean{detectedLIPIcs}}
{}
{\usepackage{varioref}}


%
% M I S C E L L A N E O U S 
%---------------------------
%

\DeclareMathAlphabet{\mathsfsl}{OT1}{cmss}{m}{sl}


%
% G E N E R A L   F O R M A T T I N G   R U L E S 
%-------------------------------------------------
%
% To achieve some kind of consistency in the notation
%

% Format of functions to integers or real numbers
\newcommand{\formatfunctiontonumbers}[1]{\mathrm{#1}}

% Format of functions to sets
\newcommand{\formatfunctiontoset}[1]{\mathit{#1}}

% Dots in x_1 \lor ... \lor x_n and the like
% (make a generic macro that can be changed according to publisher
% requirements)  
\newcommand{\formuladots}{\cdots}


%
% B I G - O H   N O T A T I O N 
%-------------------------------
%

\newcommand{\BIGOH}[1]{\mathrm{O} \left( #1 \right)}
\newcommand{\Bigoh}[1]{\mathrm{O} \bigl( #1 \bigr)}
\newcommand{\bigoh}[1]{\mathrm{O} ( #1 )}
\newcommand{\LITTLEOH}[1]{\mathrm{o} \left( #1 \right)}
\newcommand{\Littleoh}[1]{\mathrm{o} \bigl( #1 \bigr)}
\newcommand{\littleoh}[1]{\mathrm{o} ( #1 )}
\newcommand{\BIGTHETA}[1]{\Theta \left( #1 \right)}
\newcommand{\Bigtheta}[1]{\Theta \bigl( #1 \bigr)}
\newcommand{\bigtheta}[1]{\Theta ( #1 )}
\newcommand{\BIGOMEGA}[1]{\Omega \left( #1 \right)}
\newcommand{\Bigomega}[1]{\Omega \bigl( #1 \bigr)}
\newcommand{\bigomega}[1]{\Omega ( #1 )}
\newcommand{\LITTLEOMEGA}[1]{\omega \left( #1 \right)}
\newcommand{\Littleomega}[1]{\omega \bigl( #1 \bigr)}
\newcommand{\littleomega}[1]{\omega ( #1 )}
\newcommand{\POLYBOUND}[1]{\mathrm{poly} \left( #1 \right)}
\newcommand{\Polybound}[1]{\mathrm{poly} \bigl( #1 \bigr)}
\newcommand{\polybound}[1]{\mathrm{poly} ( #1 )}
\newcommand{\POLYLOGBOUND}[1]{\mathrm{polylog} \left( #1 \right)}
\newcommand{\Polylogbound}[1]{\mathrm{polylog} \bigl( #1 \bigr)}
\newcommand{\polylogbound}[1]{\mathrm{polylog} ( #1 )}
            
\DeclareMathOperator{\polylog}{polylog}


%
% G E N E R A L  M A T H E M A T I C A L   N O T A T I O N
%----------------------------------------------------------
%

% N, Z, Q, R as symbols for classes of numbers
\ifthenelse{\boolean{detectedToC}}{}
{
  \newcommand{\Q}         {\mathbb{Q}}
  \newcommand{\R}         {\mathbb{R}}
  \newcommand{\Rplus}     {\mathbb{R}^{+}}
  \newcommand{\N}         {\mathbb{N}}
  \newcommand{\Nplus}     {\mathbb{N}^{+}}
  \newcommand{\Nzero}     {\mathbb{N}_{0}}
  \newcommand{\Z}         {\mathbb{Z}}
}

% Sigma sum sign with indices to the right, not below and above
\newcommand{\sumnodisplay}{{\textstyle \sum}}

% Absolute value and norm
\providecommand{\abs}[1]{\lvert#1\rvert}
\providecommand{\Abs}[1]{\bigl\lvert#1\bigr\rvert}
\providecommand{\ABS}[1]{\left\lvert#1\right\rvert}
\providecommand{\norm}[1]{\lVert#1\rVert}
\providecommand{\Norm}[1]{\bigl\lVert#1\bigr\rVert}
\providecommand{\NORM}[1]{\left\lVert#1\right\rVert}

% Exists unique
\newcommand{\existsunique}{\exists!}

% Rounding
\newcommand{\ceiling}[1]{\lceil #1 \rceil}
\newcommand{\Ceiling}[1]{\bigl \lceil #1 \bigr \rceil}
\newcommand{\CEILING}[1]{\left \lceil #1 \right \rceil}

\newcommand{\floor}[1]{\lfloor #1 \rfloor}
\newcommand{\Floor}[1]{\bigl \lfloor #1 \bigr \rfloor}
\newcommand{\FLOOR}[1]{\left \lfloor #1 \right \rfloor}

\newcommand{\intpart}[1]{\lceil #1 \rfloor}
\newcommand{\Intpart}[1]{\bigl \lceil #1 \bigr \rfloor}
\newcommand{\INTPART}[1]{\left \lceil #1 \right \rfloor}

% Max and min
% Don't use \maxof and \minof to avoid conflict with calc package
\newcommand{\MAXOFEXPR}[2][]{\max_{#1} \left\{ #2 \right\}}
\newcommand{\MINOFEXPR}[2][]{\min_{#1} \left\{ #2 \right\}}
\newcommand{\Maxofexpr}[2][]{\max_{#1} \bigl\{ #2 \bigr\}}
\newcommand{\Minofexpr}[2][]{\min_{#1} \bigl\{ #2 \bigr\}}
\newcommand{\maxofexpr}[2][]{\max_{#1} \{ #2 \}}
\newcommand{\minofexpr}[2][]{\min_{#1} \{ #2 \}}

\newcommand{\maxofset}[3][:]{\max \{ #2 #1 #3 \}}
\newcommand{\minofset}[3][:]{\min \{ #2 #1 #3 \}}
 
\newcommand{\MAXOFSET}[3][:]%
     {\ifthenelse{\equal{#1}{;}}%
     {\MAXOFEXPR{ #2 \,;\, #3 }}
     {\ifthenelse{\equal{#1}{:}}%
     {\MAXOFEXPR{ #2 \,:\, #3 }}
     {\max \twincommandJN{\left\{}{#2}{\left#1}{\right}{\,#3}{\right\}}}}}
\newcommand{\MINOFSET}[3][:]%
     {\ifthenelse{\equal{#1}{;}}%
     {\MINOFEXPR{ #2 \,;\, #3 }}
     {\ifthenelse{\equal{#1}{:}}%
     {\MINOFEXPR{ #2 \,:\, #3 }}
     {\min \twincommandJN{\left\{}{#2}{\left#1}{\right}{\,#3}{\right\}}}}}

\newcommand{\Maxofset}[3][:]%
     {\ifthenelse{\equal{#1}{;}}%
     {\Maxofexpr{ #2 \,;\, #3 }}
     {\ifthenelse{\equal{#1}{:}}%
     {\Maxofexpr{ #2 \,:\, #3 }}
     {\max \twincommandJN{\bigl\{}{#2}{\bigl#1}{\bigr}{\,#3}{\bigr\}}}}}
\newcommand{\Minofset}[3][:]%
     {\ifthenelse{\equal{#1}{;}}%
     {\Minofexpr{ #2 \,;\, #3 }}
     {\ifthenelse{\equal{#1}{:}}%
     {\Minofexpr{ #2 \,:\, #3 }}
     {\min \twincommandJN{\bigl\{}{#2}{\bigl#1}{\bigr}{\,#3}{\bigr\}}}}}


%
% A L G E B R A
%---------------
%

% Some linear algebra
\newcommand{\transpose}[1]{\ensuremath{#1^{\top}}}
\newcommand{\innerproduct}[2]{\langle #1, #2 \rangle}
\newcommand{\Innerproduct}[2]{\bigl\langle #1, #2 \bigr\rangle}
\newcommand{\INNERPRODUCT}[2]{\left\langle #1, #2 \right\rangle}

% Generic field
\newcommand{\fieldstd}{\mathbb{F}}
\newcommand{\fieldf}{\mathbb{F}}
\newcommand{\F}{\mathbb{F}}

% Finite fields
\newcommand{\GF}[1]{\mathrm{GF} ( #1 )}
\newcommand{\gf}[1]{\mathrm{GF} ( #1 )}
\newcommand{\GFmul}[1]{\mathrm{GF} ( #1 )^{*}}
\newcommand{\gfmul}[1]{\mathrm{GF} ( #1 )^{*}}


%
% P R O B A B I L I T Y   T H E O R Y 
%-------------------------------------

% AMS-TeX defines an operator name \Pr
\DeclareMathOperator{\Expop}{E}
\DeclareMathOperator{\Varianceop}{Var}

% Probability
\newcommand{\PROB}[2][]{\Pr_{#1} \left[ #2 \right]}
\newcommand{\Prob}[2][]{\Pr_{#1} \bigl[ #2 \bigr]}
\ifthenelse{\boolean{detectedLMCS}}
{\renewcommand{\prob}[2][]{\Pr_{#1} [ #2 ]}}
{\newcommand{\prob}[2][]{\Pr_{#1} [ #2 ]}}

% Expectation
\newcommand{\EXPECTATION}[2][]{\Expop_{#1} \left[ #2 \right]}
\newcommand{\Expectation}[2][]{\Expop_{#1} \bigl[ #2 \bigr]}
\newcommand{\expectation}[2][]{\Expop_{#1} [ #2 ]}
\newcommand{\VARIANCE}[1]{\Varianceop \left( #1 \right)}
\newcommand{\Variance}[1]{\Varianceop \bigl( #1 \bigr)}
\newcommand{\variance}[1]{\Varianceop ( #1 )}

% 
% INTERLUDE: MATCHING MIDDLE SEPARATORS (FROM THE UK TeX FAQ)
% 
% 
% One of the few glaring omissions from TeX's mathematical typesetting
% capabilities is a means of setting separators in the middle of
% mathematical expressions. In all sorts of mathematical enterprises one
% may find oneself needing a \middle command, to be used in expressions
% like \left\{ x \in \mathbb{N} \middle| x \mbox{ even} \right\} to
% specify the set of even natural numbers. The e-TeX system defines just
% such a command, but users of Knuth's original need some support.
% Donald Arseneau's braket package provides commands for set
% specifications (as above) and for Dirac brackets (and bras and kets).
% The package uses the e-TeX built-in command if it finds itself running
% under e-TeX.
% 
% See ftp://cam.ctan.org/tex-archive/macros/latex/contrib/misc/braket.sty .
% 
% Or one can do as below.
%

\newcommand{\twincommandJN}[6]%
    {#1#2#3\vphantom{#2#5}\mspace{-2.05mu}#4.#5#6}

% Perhaps this is superfluous---in text mode there is no need for measuring
% with \vphantom, I think, since \bigl[ and \bigr] are what they are 
% independent of what is inside (are they not?).
%
% The length -2.25mu probably should be set instead by doing sth like
%    
%    \newlength{\lengthJN}
%    \settowidth{\lengthJN}{$\left.\right.$}
%    \setlength{\lengthJN}{0.5\lengthJN}
%
% and then using \mspace{-\lengthJN}, but the difference appears to be
% very small so I have not implemented this.


% CONDITIONAL EXPECTATION
\newcommand{\condexp}[2]{\Expop{#1  \mid  #2}}
\newcommand{\CondExp}[2]%
    {\Expop\twincommandJN{\bigl[}{#1}{\bigl|}{\bigr}{\,#2}{\bigr]}}
\newcommand{\CONDEXP}[2]%
     {\Expop\twincommandJN{\left[}{#1}{\left|}{\right}{\,#2}{\right]}}

% CONDITIONAL PROBABILITY
\newcommand{\condprob}[3][]{\prob[#1]{#2  \mid  #3}}
\newcommand{\Condprob}[3][]%
    {\Pr_{#1}\twincommandJN{\bigl[}{#2}{\bigl|}{\bigr}{\,#3}{\bigr]}}
\newcommand{\CONDPROB}[3][]%
    {\Pr_{#1}\twincommandJN{\left[}{#2}{\left|}{\right}{\,#3}{\right]}}

%
% Example code:
%    
%    \begin{displaymath}
%     \CONDEXP{\sum_{i=1}^kX_i}{Z}\quad\mbox{and}\quad%
%     \CONDPROB{B\land C}{\bigwedge_{i\in S}A_i}
%    \end{displaymath}
%    
%    $\CondExp{\sum_{i=1}^k X_i}{Z}$  
%    and
%    $\CondProb{B\land C}{\bigwedge_{i\in S}A_i}$
%    


%
% F U N C T I O N S
%-------------------
%

% DESCRIPTION OF FUNCTION
\newcommand{\funcdescr}[3]{\ensuremath{ #1 : #2 \to #3}}

% DOMAIN
\newcommand{\domainof}[1]{\ensuremath{\mathrm{dom} ( #1 )}}
\newcommand{\Domainof}[1]{\ensuremath{\mathrm{dom}\bigl( #1 \bigr)}}

% INVERSE IMAGE
\newcommand{\invimageof}[2]{{\ensuremath{{#1}^{-1} \left( #2 \right)}}}

%
% G R A P H S
%-------------
%

\newcommand{\edges}[1]{E( #1 )}
\newcommand{\Edges}[1]{E\bigl( #1 \bigr)}
\newcommand{\vertices}[1]{V( #1 )}
\newcommand{\Vertices}[1]{V\bigl( #1 \bigr)}

\newcommand{\vdegree}[2][]{\mathrm{deg}_{#1}(#2)}
\newcommand{\Vdegree}[2][]{\mathrm{deg}_{#1}\bigl(#2\bigr)}
\newcommand{\vneighbour}[2][]{N_{#1}({#2})}
\newcommand{\Vneighbour}[2][]{N_{#1}\bigl({#2}\bigr)}

% Boundary
\newcommand{\boundary}[1]{\ensuremath{\partial #1}}

\newcommand{\pathstd}{\ensuremath{P}}
\newcommand{\pathalt}{\ensuremath{Q}}
\newcommand{\pathfromto}[3]{#1 : #2 \rightsquigarrow #3}


%
% S E T S   A N D   T U P L E S
%-------------------------------
%

\newcommand{\set}[1]{\{ #1 \}}
\newcommand{\Set}[1]{\bigl\{ #1 \bigr\}}
\newcommand{\SET}[1]{\left\{ #1 \right\}}

\newcommand{\setdescr}[3][\mid]{\set{ #2 #1 #3 }}
\newcommand{\Setdescr}[3][|]%
     {\ifthenelse{\equal{#1}{;}}%
     {\Set{ #2 \,;\, #3 }}
     {\ifthenelse{\equal{#1}{:}}%
     {\Set{ #2 \,:\, #3 }}
     {\twincommandJN{\bigl\{}{#2\,}{\bigl#1}{\bigr}{\,#3}{\bigr\}}}}}
\newcommand{\SETDESCR}[3][|]%
     {\twincommandJN{\left\{}{#2\,}{\left#1}{\right}{\,#3}{\right\}}}

\newcommand{\setbrackets}[1]{[ #1 ]}
\newcommand{\Setbrackets}[1]{\bigl[ #1 \bigr]}
\newcommand{\SETBRACKETS}[1]{\left[ #1 \right]}

\newcommand{\setdescrbrackets}[3][\mid]{{\setbrackets{ #2 #1 #3 }}}
\newcommand{\Setdescrbrackets}[3][|]%
     {\twincommandJN{\bigl[}{#2}{\bigl#1}{\bigr}{\,#3}{\bigr]}}
\newcommand{\SETDESCRBRACKETS}[3][|]%
     {\twincommandJN{\left[}{#2}{\left#1}{\right}{\,#3}{\right]}}

\newcommand{\SETSIZE}[1]{\left\lvert#1\right\rvert}
\newcommand{\Setsize}[1]{\bigl\lvert#1\bigr\rvert}
\newcommand{\setsize}[1]{\lvert#1\rvert}

% Set complement
\newcommand{\setcompl}[1]{\overline{#1}}

% Intersection and union
\newcommand{\intersection}{\cap}
\newcommand{\Intersection}{\bigcap}
\newcommand{\Intersectionnodisplay}{\textstyle \bigcap}

\newcommand{\union}{\cup}
\newcommand{\Union}{\bigcup}
\newcommand{\Unionnodisplay}{\textstyle \bigcup}

% Intersection and union with some space
\newcommand{\unionSP}{\, \union \, }
\newcommand{\intersectionSP}{\, \intersection \, }

% Disjoint union 
%
% Can see no difference between below definition and \mathbin{\dot{\cup}}
\newcommand{\disjointunion}{\overset{.}{\cup}}
\newcommand{\disjointunionSP}{\disjointunion}
\newcommand{\Disjointunion}{\overset{.}{\bigcup}}
\newcommand{\disjunion}{\disjointunion}
\newcommand{\Disjunion}{\Disjointunion}

% First n positive integers
\newcommand{\nset}[1]{[{#1}]}
\newcommand{\Nset}[1]{\bigl[ {#1} \bigr]}


%
% L O G I C
%-----------
%

%
% Logic connectives
%
% Logic or is \lor. Logic and is \land. Logic not is \lnot.
% They can be used in math mode only.
\newcommand{\Lor}{\bigvee}
\newcommand{\Land}{\bigwedge}

% nodisplay = indices to the right, not below and above
\newcommand{\Lornodisplay}{{\textstyle \bigvee}}
\newcommand{\Landnodisplay}{{\textstyle \bigwedge}}

\newcommand{\limpl}{\rightarrow}
\newcommand{\lequiv}{\leftrightarrow}

% Prefixed NOT (pfnot):  \lnot x
% Overlined NOT (olnot): \overline{x}
% \stdnot{lit} is the standard NOT notation for variables and literals

\newcommand{\pfnot}[1]{\lnot #1}
\newcommand{\olnot}[1]{\overline{#1}}
\newcommand{\stdnot}[1]{\olnot{#1}}
    
% syntactic equivalence
\newcommand{\synteq}{\doteq}

% constants "true" and "false"
\newcommand{\FALSE}{\mathit{FALSE}}
\newcommand{\TRUE}{\mathit{TRUE}}

\newcommand{\false}{\bot}
\newcommand{\true}{\top}

\newcommand{\falsenum}{0}
\newcommand{\truenum}{1}

%
% Notation and terms for CNF/DNF formulas
%
% Standard notation for parameters in k-CNF formulas
\newcommand{\nvar}{n}
\newcommand{\nvars}{\nvar}
\newcommand{\nclause}{m}
\newcommand{\nclauses}{\nclause}
\newcommand{\clwidth}{k}
\newcommand{\density}{\Delta}

% Formatting of k-CNF/k-DNF in running text
\newcommand{\xdnf}[1]{\mbox{\ensuremath{#1}-DNF}\xspace}
\newcommand{\kdnf}{\xdnf{\clwidth}}
\newcommand{\xdnfform}[1]{\mbox{\ensuremath{#1}-DNF} formula\xspace}
\newcommand{\kdnfform}{\xdnfform{\clwidth}}

\newcommand{\xcnf}[1]{\mbox{\ensuremath{#1}-CNF}\xspace}
\newcommand{\kcnf}{\xcnf{\clwidth}}
\newcommand{\xcnfform}[1]{\mbox{\ensuremath{#1}-CNF} formula\xspace}
\newcommand{\kcnfform}{\xcnfform{\clwidth}}
\newcommand{\xclause}[1]{\mbox{\ensuremath{#1}-clause}\xspace}
\newcommand{\kclause}{\xclause{\clwidth}}

\newcommand{\Excnf}[1]{\mbox{\ensuremath{\mathrm{E}#1}-CNF}\xspace}
\newcommand{\Ekcnf}{\Excnf{\clwidth}}
\newcommand{\Excnfform}[1]{\mbox{\ensuremath{\mathrm{E}#1}-CNF} formula\xspace}
\newcommand{\Ekcnfform}{\Excnfform{\clwidth}}

\newcommand{\xterm}[1]{\mbox{\ensuremath{#1}-term}\xspace}
\newcommand{\kterm}{\xterm{\clwidth}}
\newcommand{\Exterm}[1]{\mbox{\ensuremath{\mathrm{E}#1}-term}\xspace}
\newcommand{\Ekterm}{\Exterm{\clwidth}}

%
% Distributions for random k-CNF formulas
%
% fixed # random clauses chosen with replacement
% \randkcnfnclwrepl{clause width}{#variables}{#clauses}
%
\newcommand{\randkcnfnclwrepl}[3][\clwidth]%
        {\ensuremath{\mathcal{F}^{#2, #3}_{#1}}}
\newcommand{\randkcnfnclwreplstd}% 
        {\randkcnfnclwrepl{\clwidth}{\nvar}{\nclause}}


%
% C O M P L E X I T Y   T H E O R Y 
%-----------------------------------
%

% NOTATION FOR LANGUAGES/PROBLEMS
\newcommand{\problemlanguageformat}[1]{\textsc{#1}\xspace}
\newcommand{\problemlanguageformatnospace}[1]{\textsc{#1}}
\newcommand{\langstd}{\ensuremath{L}}
\newcommand{\langcompl}[1]{\ensuremath{\overline{#1}}}

\newcommand{\PROP}{\problemlanguageformat{PROP}}
\newcommand{\TAUT}{\problemlanguageformat{TAUT}}
\newcommand{\TAUTOLOGY}{\problemlanguageformat{Tautology}}
\newcommand{\SAT}{\problemlanguageformat{Sat}}
\newcommand{\CNFSAT}{\problemlanguageformat{CnfSat}}
\newcommand{\SATISFIABILITY}{\problemlanguageformat{Satisfiability}}
\newcommand{\THREESAT}{\text{$3$-}\problemlanguageformat{Sat}}
\newcommand{\TWOSAT}{\text{$2$-}\problemlanguageformat{Sat}}
\newcommand{\ONEINTHREESAT}{\text{$1$-}\problemlanguageformatnospace{In}%
        \text{-$3$-}\problemlanguageformat{Sat}}
\newcommand{\MAXSAT}{\problemlanguageformat{MaxSat}}
\newcommand{\MAXTWOSAT}{\problemlanguageformatnospace{Max}%
        \text{-$2$-}\problemlanguageformat{Sat}}
\newcommand{\CIRCUITSAT}{\problemlanguageformat{CircuitSat}}
\newcommand{\NAESAT}{\problemlanguageformat{NotAllEqualSat}}
\newcommand{\DOMINATINGSET}{\problemlanguageformat{DominatingSet}}
\newcommand{\VERTEXCOVER}{\problemlanguageformat{VertexCover}}
\newcommand{\MAXCUT}{\problemlanguageformat{MaxCut}}
\newcommand{\DOMATICNUMBER}{\problemlanguageformat{DomaticNumber}}
\newcommand{\MONOCHROMTRI}{\problemlanguageformat{MonochromaticTriangle}}
\newcommand{\CLIQUECOVER}{\problemlanguageformat{CliqueCover}}
\newcommand{\INDSET}{\problemlanguageformat{IndependentSet}}
\newcommand{\GCLIQUE}{\problemlanguageformat{Clique}}
\newcommand{\GCOLOURING}{\problemlanguageformat{Colouring}}
\newcommand{\SUBGRAPHISO}{\problemlanguageformat{Subgraph\-Isomorphism}}
\newcommand{\GKERNEL}{\problemlanguageformat{Kernel}}
\newcommand{\MINMAXMATCHING}{\problemlanguageformat{MinMaxMatching}}
\newcommand{\CUBICSUBGRAPH}{\problemlanguageformat{CubicSubgraph}}

% NOTATION FOR COMPLEXITY CLASSES
\newcommand{\complclassformat}[1]%
        {\textrm{\upshape{\textsf{#1}}}\xspace}
\newcommand{\cocomplclass}[1]%
        {\textrm{\upshape{\textsf{co#1}}}\xspace}

\newcommand{\TIMEclass}[1]{\ensuremath{\complclassformat{TIME}\bigl(#1\bigr)}}
\newcommand{\SPACEclass}[1]{\ensuremath{\complclassformat{SPACE}\bigl(#1\bigr)}}
\newcommand{\DTIMEclass}[1]{\ensuremath{\complclassformat{DTIME}\bigl(#1\bigr)}}
\newcommand{\DTIMEadviceclass}[2]%
    {\ensuremath{\complclassformat{DTIME}\bigl(#1\bigr)/{#2}}}
\newcommand{\Pclass}{\complclassformat{P}}
\newcommand{\NP}{\complclassformat{NP}}
\newcommand{\NPclass}{\NP}
\newcommand{\coNP}{\cocomplclass{NP}}
\newcommand{\coNPclass}{\coNP}
\newcommand{\CoNP}{\coNP}
\newcommand{\Ppoly}{\complclassformat{P/poly}}
\newcommand{\Pslashpoly}{\Ppoly}
\newcommand{\Pslashpolyclass}{\Pslashpoly}
\newcommand{\PSPACE}{\complclassformat{PSPACE}}
\newcommand{\PSPACEclass}{\PSPACE}
\newcommand{\NPSPACE}{\complclassformat{NPSPACE}}
\newcommand{\NPSPACEclass}{\NPSPACE}
\newcommand{\pspace}{\PSPACE}
\newcommand{\EXPTIME}{\complclassformat{EXPTIME}}
\newcommand{\exptime}{\complclassformat{EXPTIME}}
\newcommand{\EXPSPACE}{\complclassformat{EXPSPACE}}
\newcommand{\expspace}{\complclassformat{EXPSPACE}}
\newcommand{\EXP}{\complclassformat{EXP}}
\newcommand{\NEXP}{\complclassformat{NEXP}}
\newcommand{\coNEXP}{\cocomplclass{NEXP}}
\newcommand{\Lclass}{\complclassformat{L}}
\newcommand{\LOGSPACE}{\complclassformat{L}}
\newcommand{\NL}{\complclassformat{NL}}
\newcommand{\NLclass}{\NL}
\newcommand{\coNL}{\cocomplclass{NL}}
\newcommand{\NCclass}{\complclassformat{NC}}
\newcommand{\DP}{\complclassformat{DP}}
\newcommand{\DPclass}{\DP}
\newcommand{\SIGMACLASS}[1]{\ensuremath{\Sigma^p_{#1}}}
\newcommand{\Sigmaclass}[1]{\SIGMACLASS{#1}}
\newcommand{\PICLASS}[1]{\ensuremath{\Pi^p_{#1}}}
\newcommand{\Piclass}[1]{\PICLASS{#1}}
\newcommand{\PH}{\complclassformat{PH}}
\newcommand{\EXPEXP}{\complclassformat{EXPEXP}}
\newcommand{\NEXPEXP}{\complclassformat{NEXPEXP}}
\newcommand{\coNEXPEXP}{\cocomplclass{NEXPEXP}}        
\newcommand{\BPP}{\complclassformat{BPP}}
\newcommand{\ZPP}{\complclassformat{ZPP}}
\newcommand{\RP}{\complclassformat{RP}}
\newcommand{\coRP}{\cocomplclass{RP}}
\newcommand{\MIParg}[1]{\ensuremath{\complclassformat{MIP}[#1]}}
\newcommand{\MIP}{\complclassformat{MIP}}
\newcommand{\IP}{\complclassformat{IP}}
\newcommand{\PCPnot}{\complclassformat{PCP}}
% PCP{completeness}{soundness}{randomness}{queries}{alphabet size}
\newcommand{\PCPalph}[5]%
    {\ensuremath{\complclassformat{PCP}_{{#1},{#2}}[{#3}, {#4}, {#5}]}}
\newcommand{\PCP}[4]%
    {\ensuremath{\complclassformat{PCP}_{{#1},{#2}}[{#3}, {#4}]}}

\newcommand{\SAC}{\complclassformat{SAC}}
\newcommand{\NC}{\complclassformat{NC}}
\newcommand{\ACzero}{\ensuremath{\complclassformat{AC}^{0}}}
\newcommand{\SC}{\complclassformat{SC}}
\newcommand{\TISP}{\complclassformat{TISP}}
\newcommand{\LOGCFL}{\complclassformat{LOGCFL}}


%
% T E X T   F O R M A T T I N G
%-------------------------------
%

% introducing a new term or mentioning a familiar one for the first time
%\newcommand{\introduceterm}[1]{{\textsl{#1}}}
\newcommand{\introduceterm}[1]{{\emph{#1}}}

% Spacing before punctuation in displayed equations according to LLNCS
% (But SIAM disagrees, so it is convenient to have a macro for this.)
\newcommand{\eqperiod}{\enspace .}
\newcommand{\eqcomma}{\enspace ,}

% For description lists with items in italics (to avoid confusion
% with theorem headers, for instance)
\newcommand{\italicitem}[1][]{\item[\textit{#1}]}


%
% A B B R E V I A T I O N S   O F   F R E Q U E N T     E X P R E S S I O N S
%-----------------------------------------------------------------------------
%

% GENERAL EXPRESSIONS
\newcommand{\wrt}{with respect to\xspace}
\newcommand{\wrtabbrev}{w.r.t.\ }
\ifthenelse{\boolean{detectedToC}}{}
  {\newcommand{\eg}{for instance\xspace} % should be surrounded by commas 
    \newcommand{\Eg}{For instance\xspace}}
\ifthenelse{\boolean{detectedToC}}{}{
\newcommand{\ie}{i.e.,\ }
\newcommand{\Ie}{I.e.,\ }
}
\newcommand{\ieComma}{i.e.,\ }  %%%% Special hack to use when adapting to ToC
\newcommand{\ieNoComma}{i.e.\ }
\ifthenelse{\boolean{detectedLIPIcs} \or \boolean{detectedIJCAI}}
{\renewcommand{\st}{\errmessage{Please do not use st}}}
{\ifthenelse{\isundefined{\st}}
  {\newcommand{\st}{such that\xspace}}
  {}}      
%   \newcommand{\st}{such that\xspace}}

\newcommand{\etal}{et al.\@\xspace}
\newcommand{\etalS}{et al\@. }

\newcommand{\vs}{vs.\ }

% TYPICAL MATHEMATICAL EXPRESSIONS
\newcommand{\ifaoif}{if and only if\xspace}
\newcommand{\wolog}{without loss of generality\xspace}
\newcommand{\Wolog}{Without loss of generality\xspace}
\newcommand{\iid}{independently and identically distributed\xspace}
\ifthenelse{\boolean{detectedIEEE}}{}{\newcommand{\QED}{Q.E.D.}}
\newcommand{\qedlong}{which was to be proved\xspace}

\newcommand{\whp}{with high probability\xspace}
\newcommand{\Whp}{With high probability\xspace}
\newcommand{\aas}{asymptotically almost surely\xspace}
\newcommand{\Aas}{Asymptotically almost surely\xspace}


%
% R E F E R E N C E S 
%---------------------
%
% Macros with capital initial letter intended for use at start of sentence.
%

%
% REFERENCES TO SECTIONAL UNITS AND PAGE INTERVALS
%
% The LLNCS document class and instructions says that in the running text, 
% but not at the beginning of a sentence, Sect., Chap. and Fig. should be 
% abbreviated.
%
% LLNCS wants, e.g., "Section 4.3" with capital S, SIAM doesn't.
%

% Sections
\newcommand{\refsec}[1]{Section~\ref{#1}}
\newcommand{\refsecP}[1]{Section~\vref{#1}}
\newcommand{\Refsec}[1]{Section~\ref{#1}}
\newcommand{\RefsecP}[1]{Section~\vref{#1}}
\newcommand{\reftwosecs}[2]{Sections~\ref{#1} and~\ref{#2}}
\newcommand{\refthreesecs}[3]{Sections~\ref{#1}, \ref{#2}, and~\ref{#3}}

% Chapters
\newcommand{\refch}[1]{Chapter~\ref{#1}}
\newcommand{\Refch}[1]{Chapter~\ref{#1}}
\newcommand{\reftwochs}[2]{Chapters~\ref{#1} and~\ref{#2}}
\newcommand{\refthreechs}[3]{Chapters~\ref{#1}, \ref{#2} and~\ref{#3}}

% Appendices
\newcommand{\refapp}[1]{Appendix~\ref{#1}}
\newcommand{\Refapp}[1]{Appendix~\ref{#1}}
\newcommand{\reftwoapps}[2]{Appendices~\ref{#1} and~\ref{#2}}

% Figures
\newcommand{\reffig}[1]{Figure~\ref{#1}}
\newcommand{\reffigP}[1]{Figure~\vref{#1}}
\newcommand{\Reffig}[1]{Figure~\ref{#1}}
\newcommand{\ReffigP}[1]{Figure~\vref{#1}}
\newcommand{\reftwofigs}[2]{Figures~\ref{#1} and~\ref{#2}}
\newcommand{\refthreefigs}[3]{Figures~\ref{#1}, \ref{#2}, and~\ref{#3}}


%
% REFERENCES TO THEOREM-LIKE ENVIRONMENTS
%
% Requires \usepackage{varioref}
%
% Adapted to LLNCS document class and instructions. Definition, Theorem etc
% should be capitalized when followed by a number.
%

% Theorems
\newcommand{\refth}[1]{Theorem~\ref{#1}}
\newcommand{\reftwoths}[2]{Theorems~\ref{#1} and~\ref{#2}}
\newcommand{\refthreeths}[4][and]{Theorems~\ref{#2}, \ref{#3}, {#1}~\ref{#4}}
\newcommand{\refthm}[1]{Theorem~\ref{#1}}
\newcommand{\reftwothms}[2]{Theorems~\ref{#1} and~\ref{#2}}
\newcommand{\refthreethms}[4][and]{Theorems~\ref{#2}, \ref{#3}, {#1}~\ref{#4}}

% Lemmas
\newcommand{\reflem}[1]{Lemma~\ref{#1}}
\newcommand{\reftwolems}[2]{Lemmas~\ref{#1} and~\ref{#2}}
\newcommand{\refthreelems}[4][and]{Lemmas~\ref{#2}, \ref{#3}, {#1}~\ref{#4}}

% Propositions
\newcommand{\refpr}[1]{Proposition~\ref{#1}}
\newcommand{\reftwoprs}[2]{Propositions~\ref{#1} and~\ref{#2}}

% Corollaries
\newcommand{\refcor}[1]{Corollary~\ref{#1}}
\newcommand{\reftwocors}[2]{Corollaries~\ref{#1} and~\ref{#2}}

% Definitions
\newcommand{\refdef}[1]{Definition~\ref{#1}}
\newcommand{\reftwodefs}[2]{Definitions~\ref{#1} and~\ref{#2}}
\newcommand{\refthreedefs}[3]{Definitions~\ref{#1}, \ref{#2}, and~\ref{#3}}

% Remarks
\newcommand{\refrem}[1]{Remark~\ref{#1}}
\newcommand{\reftworems}[2]{Remarks~\ref{#1} and~\ref{#2}}

% Observations
\newcommand{\refobs}[1]{Observation~\ref{#1}}
\newcommand{\reftwoobs}[2]{Observations~\ref{#1} and~\ref{#2}}

% Facts
\newcommand{\reffact}[1]{Fact~\ref{#1}}

% Conjectures
\newcommand{\refconj}[1]{Conjecture~\ref{#1}}
\newcommand{\reftwoconjs}[2]{Conjectures~\ref{#1} and~\ref{#2}}

% Examples
\newcommand{\refex}[1]{Example~\ref{#1}}
\newcommand{\reftwoexs}[2]{Examples~\ref{#1} and~\ref{#2}}

% Properties
\newcommand{\refproperty}[1]{Property~\ref{#1}}
\newcommand{\reftwoproperties}[2]{Properties~\ref{#1} and~\ref{#2}}
 
% Claims
\newcommand{\refclaim}[1]{Claim~\ref{#1}}
\newcommand{\reftwoclaims}[2]{Claims~\ref{#1} and~\ref{#2}}

% References at start of sentence
\newcommand{\Refth}[1]{Theorem~\ref{#1}}
\newcommand{\Reflem}[1]{Lemma~\ref{#1}}
\newcommand{\Refpr}[1]{Proposition~\ref{#1}}
\newcommand{\Refcor}[1]{Corollary~\ref{#1}}
\newcommand{\Refdef}[1]{Definition~\ref{#1}}
\newcommand{\Refrem}[1]{Remark~\ref{#1}}
\newcommand{\Refobs}[1]{Observation~\ref{#1}}
\newcommand{\Refconj}[1]{Conjecture~\ref{#1}}
\newcommand{\Refex}[1]{Example~\ref{#1}}
\newcommand{\Refclaim}[1]{Claim~\ref{#1}}

% References with page numbers
\newcommand{\refthP}[1]{Theorem~\vref{#1}}
\newcommand{\reflemP}[1]{Lemma~\vref{#1}}
\newcommand{\refprP}[1]{Proposition~\vref{#1}}
\newcommand{\refcorP}[1]{Corollary~\vref{#1}}
\newcommand{\refdefP}[1]{Definition~\vref{#1}}
\newcommand{\refremP}[1]{Remark~\vref{#1}}
\newcommand{\refobsP}[1]{Observation~\vref{#1}}
\newcommand{\refconjP}[1]{Conjecture~\vref{#1}}
\newcommand{\refexP}[1]{Example~\vref{#1}}
\newcommand{\refpropertyP}[1]{Property~\vref{#1}}

% Some more references
\newcommand{\refrule}[1]{rule~\ref{#1}}
\newcommand{\reftworules}[2]{rules~\ref{#1} and~\ref{#2}}

\newcommand{\refpart}[1]{part~\ref{#1}}
\newcommand{\Refpart}[1]{Part~\ref{#1}}
\newcommand{\reftwoparts}[2]{parts~\ref{#1} and~\ref{#2}}
\newcommand{\Reftwoparts}[2]{Parts~\ref{#1} and~\ref{#2}}

\newcommand{\refitem}[1]{item~\ref{#1}}
\newcommand{\Refitem}[1]{Item~\ref{#1}}
\newcommand{\reftwoitems}[2]{items~\ref{#1} and~\ref{#2}}
\newcommand{\Reftwoitems}[2]{Items~\ref{#1} and~\ref{#2}}

\newcommand{\refcase}[1]{case~\ref{#1}}
\newcommand{\Refcase}[1]{Case~\ref{#1}}
\newcommand{\reftwocases}[2]{cases~\ref{#1} and~\ref{#2}}
\newcommand{\Reftwocases}[2]{Cases~\ref{#1} and~\ref{#2}}

% References to equations (alias for \eqref just for simplicity)
%
% The definition of \refeq overwrites a command in the mathtools package
% if this package has been loaded

\ifthenelse
{\isundefined{\refeq}}
{\newcommand{\refeq}[1]{\eqref{#1}}}
{\renewcommand{\refeq}[1]{\eqref{#1}}}
\newcommand{\refeqP}[1]{\eqref{#1} on page~\pageref{#1}}




%%%
%%% SOME LOCAL MACROS FOR THE IDMA COURSE
%%%

% GCD
%   \DeclareMathOperator{\gcd}{gcd}

% Two-norm
\newcommand{\twonorm}[1]{\lVert#1\rVert_2}
\newcommand{\Twonorm}[1]{\bigl\lVert#1\bigr\rVert_2}
\newcommand{\TWONORM}[1]{\left\lVert#1\right\rVert_2}

% Formal language grammars
%   \newcommand{\produces}{\rightarrow}
%   \newcommand{\terminalf}[1]{\mathtt{#1}}
%   \newcommand{\tokenf}[1]{\text{\textbf{#1}}}
%   \newcommand{\emptystring}{\varepsilon}
%   \newcommand{\numtoken}{\tokenf{num}}

% For checkmark and xmark
\usepackage{pifont}
\newcommand{\xmark}{\ding{55}}

% For formatting induction proofs
\newcommand{\indproofstep}[1]%
{
  \smallskip
  \noindent
  \textbf{\textit{#1:}}
}

\newcommand{\indbase}[1]{\indproofstep{Base case (#1)}}
\newcommand{\indstep}{\indproofstep{Induction step}}
\newcommand{\indclaim}{\indproofstep{Claim}}





% For getting watermark "DRAFT" across all pages (for instance, 
% when posting preliminary version of problem set)
%    \usepackage{draftwatermark}
%    % \SetWatermarkFontSize{20 cm}
%    \SetWatermarkScale{5}

% For METAPOST logo as \hologo{METAPOST}
%   \usepackage{hologo}

% For TikZ
%   \input{Figures/tikz-packages.tex}

%%%
%%% TITLE
%%%

\author{\courseinstructor}
\course{\coursenamelong{}}
\semester{\courseperiod}
\title{\coursenameshort: Problem Set \psetno}

\begin{document}

\maketitle



\begin{abstract}
  \noindent
  \textbf{Due:} \duedate.

  \noindent
  \textbf{Submission:}
  Please submit your solutions
  via \emph{Absalon}
  as a PDF file.
  State your name and e-mail address 
  close to the top   of the first page.
  Solutions should be written in \LaTeX{} or some other math-aware
  typesetting system with reasonable margins on all sides (at least 2.5~cm).
  Please try to be precise and to the point in your solutions and
  refrain from vague statements.
  % Make sure to explain your reasoning.
  Never, ever just state the answer, but always
  make sure to explain your reasoning.
  \emph{Write so that a fellow student of yours can read, understand, and
    verify your solutions.}
  In addition to what is stated below, the general rules 
  for problem sets stated on \emph{Absalon} always apply.

%    
%      \noindent
%      \textbf{Hints:}
%      For most or all problems, ``hints'' can be purchased at a cost of 
%      \mbox{5--10~points}. In this way, you can configure yourself whether
%      you want the problems to be more creative and open-ended, where
%      sometimes a lot can depend on finding the right idea, or whether you
%      want them to be more of guided exercises providing a useful work-out
%      on the concepts of proof complexity. If you do not solve a problem,
%      there is no charge for the hint (i.e., it is not deducted from the
%      score on other problems).  
%    

  \noindent
  \textbf{Collaboration:}
  Discussions of ideas in groups of 
  two to three people 
  are allowed---and indeed, encouraged---but 
  you should always write up your solutions completely on your own,
  from start to finish, and you should understand all aspects of them
  fully. It is not allowed to compose draft solutions together and
  then continue editing individually, or to share any text, formulas,
  or pseudocode. Also, no such material may be downloaded from or
  generated via the internet to be used in draft or final solutions.
  Submitted solutions will be checked for plagiarism.

% 
%     You should also clearly acknowledge any collaboration.
%     State   close to the top 
%     of the first page of your problem set
%     solutions if you have been collaborating with someone and if so with
%     whom.  
%     \emph{Note that collaboration is on a per problem set basis,
%       so you should not discuss different problems on the same problem
%       set with different people.}
%    
%    
%   
%     \noindent
%     \textbf{Reference material:} 
%     Some of the problems are ``classic'' and hence it might be easy to
%     find solutions on the Internet, in textbooks or in research
%     papers. It is not allowed to use such material in any way unless
%     explicitly stated otherwise. Anything said during the lectures or in
%     the lecture notes 
%   %      , or which can be found in chapters of Arora-Barak covered in
%   %      the course,  
%     should be fair game, though, unless you are specifically asked to
%     show something that we claimed without proof in class. All
%     definitions 
%   %      used 
%     should be as given in class
%     or in Arora-Barak 
%     and cannot be substituted by 
%   %      definitions
%     versions
%     from other sources.  It is
%     hard to pin down 100\% watertight formal rules on what all of this
%     means---when in doubt, ask the main instructor.
%     

  \noindent
  \textbf{Grading:}
  A score of 
  \thresholdforpass
  is guaranteed to be enough to pass this problem set.


  \noindent
  \textbf{Questions:}
  Please do not hesitate to ask the instructor or TAs if any problem
  statement is unclear, but please make sure to send private
  messages---sometimes specific enough questions could give away the
  solution to your fellow students, and we want all of you to benefit
  from working on, and learning from, the problems.
%   
  Good luck!
\end{abstract}





\begin{problem}
  (60 p)
  The purpose of this problem is to gain 
  an understanding of 
  strongly connected components in directed graphs.


\begin{subproblem}%
  \label{problem:ssc}%
  (30 p)
  Compute the strongly connected components of the graph in
  \reffig{fig:ssc-graph}
  by making a dry-run of the algorithm in
  CLRS. 
%     CLRS (which is also discussed in the part of the lecture notes
%     that we did not have time to discuss in class).
  Make sure to explain carefully the different steps in the algorithm
  execution,
  including in which order vertices are dealt with during graph traversal
  and why,
  and also what the final output is.


\begin{figure}[t] %%  [tp]
  \centering
  \includegraphics{Figures/ssc-graph.1}
%
  \caption{Directed graph $G$ for which to compute
    strongly connected components in
    Problem~\ref{problem:ssc}.
  }
  \label{fig:ssc-graph}
\end{figure}



  We assume that 
  the graph is given to us
  in adjacency list representation,
  with the  out-neighbours in each adjacency list sorted in lexicographic
  order, so that this is the order
  in which vertices  are encountered when going through neighbour
  lists.
  (For instance, the out-neighbour list of $a$ is
  $(b, d, e)$
  sorted in this order.)

\end{subproblem}

\begin{solution}
  The algorithm for computing the strongly connected
  components of a graph~$G$   in CLRS and the lecture notes is as follows:
  \begin{enumerate}
  \item 
    Run a depth-first search for~$G$ 
    in \reffig{fig:ssc-graph}, and note the finishing times.

  \item
    Run a depth-first search for the inverted graph~$G^T$
    \reffig{fig:ssc-graph-inverted},
    but iterate over the vertices
    in decreasing order of finishing times
    in step~$1$.


\begin{figure}[t] %%  [tp]
  \centering
  \includegraphics{Figures/ssc-graph.2}
%
  \caption{Inverted graph~$G^T$
    for computation of
    strongly connected components in
    Problem~\ref{problem:ssc}.
  }
  \label{fig:ssc-graph-inverted}
\end{figure}




  \item
    Output the subset of vertices in each tree
    in the forest constructed in step~$2$ as the strongly connected
    components of~$G$.

  \end{enumerate}  
% 
  Note that we are not asked here to argue why this algorithm is
  correct, so it is sufficient to make an accurate dry-run including a
  clear but concise explanation of the different steps in the computations.  

  
\begin{figure}[t] %%  [tp]
  \centering
  \includegraphics{Figures/ssc-graph.3}
%
  \caption{Directed graph $G$     with 
    depth-first search 
%       DFS
    tree and start and finishing times.
  }
  \label{fig:ssc-graph-with-DFS-tree}
\end{figure}


  
  The result of the depth-first search for~$G$ is illustrated in
  \reffig{fig:ssc-graph-with-DFS-tree}, with discovery and finish times
  marked as
  $
  \langle
  \mathit{discovery\ time}
  \rangle
  /
  \langle
  \mathit{finish\ time}
  \rangle
  $
  above or below each vertex.
  Note that when we write ``neighbour'' below we mean
  ``out-neighbour'' unless specified otherwise.
  The chain of recursive calls are as follows (starting at time~$1$):
  \begin{itemize}
  \item 
    At time~$1$ we visit the vertex~$a$, the first neighbour of which
    is~$b$, which is undiscovered and so we recursively visit this vertex.

  \item
    At time~$2$ we visit~$b$. The first neighbour~$a$ of~$b$ has
    already been discovered, but the second neighbour~$c$ has not yet
    been visited and so we recursively visit it.

  \item
    At time~$3$ we visit~$c$. The first neighbour~$a$ of~$c$ has
    already been discovered, but the second neighbour~$e$
    is now visited.
    
  \item
    At time~$4$ we visit $e$, and since the first neighbour~$d$ of
    this vertex has not yet been discovered we visit it.
    
  \item
    At time~$5$ we visit~$d$. The first neighbour~$e$ is marked as
    discovered, but the second neighbour~$f$ gets a visit.
    
  \item
    At time~$6$ we visit~$f$, whose only neighbour~$g$ has not yet
    been discovered and therefore gets a visit.
    
  \item
    At time~$7$ we visit~$g$ and immediately continue to its only
    neighbour~$h$, since this vertex has not yet been discovered.
    
  \item
    At time~$8$, we visit~$h$.
    
  \item
    At this point, all vertices in~$G$ have been discovered, and
    so we will just finish the processing of all vertices in the
    reverse order of the calls above and stamp them with finishing
    times accordingly
    (so that $h$~gets finishing time~$9$,
    $g$~gets finishing time~$10$,
    $f$~gets finishing time~$11$,
    et cetera).
  \end{itemize}

  
\begin{figure}[t] %%  [tp]
  \centering
  \includegraphics{Figures/ssc-graph.4}
%
  \caption{Directed graph $G^T$
    with DFS forest identifying strongly connected components.
  }
  \label{fig:ssc-graph-with-DFS-tree-inverted}
\end{figure}



  We now invert all vertices in~$G$ to obtain~$G^T$
  as in \reffig{fig:ssc-graph-inverted}
  and run a DFS for~$G^T$ where we iterate over vertices in decreasing
  order with respect to their finishing times above.
  The result of this second DFS is shown in
  \reffig{fig:ssc-graph-with-DFS-tree-inverted}
  (where we include also the discovery and finishing times for
  completeness, although they are not needed for the discussion below).
  \begin{itemize}
  \item 
    We start with~$a$, since this vertex has finishing time~$16$.
    From~$a$ we visit~$b$ and immediately finish processing of this
    latter vertex, since the only out-neighbour of~$b$ in~$G^T$ is the
    already discovered vertex~$a$.
    Returning to~$a$, we visit the second neighbour~$c$, which is also
    immediately finished since its only out-neighbour~$b$ in~$G^T$ has
    already been processed.
    There are no more neighbours of~$a$, and so the first DFS-visit
    call at top level terminates, leading to the conclusion that
    $\set{a,b,c}$
    is a strongly connected component in~$G$.

  \item
    The currently non-discovered vertex with the highest finishing
    time from step~$1$ is~$e$, and so we make a DFS-visit call for
    this vertex. The neighbours $a$ and $c$ of~$e$ have already been
    processed, but $d$~has not yet been discovered and so gets a
    visit. When we visit~$d$,
    all out-neighbours $a$, $b$, and~$e$ in~$G^T$ have already been
    discovered, and so $d$~is immediately finished. When we return
    to~$e$ there are no more out-neighbours in~$G^T$ to process, so
    the second DFS-visit call at top level terminates, resulting in
    the strongly connected \mbox{component $\set{d,e}$.}

  \item
    Among the remaining  non-discovered vertices the one with the
    highest finishing time is~$f$. From~$f$ we visit the only
    out-neighbour~$h$ in~$G^T$, and from~$h$ we visit the only
    out-neighbour~$g$. Now all vertices in~$G^T$ have been discovered,
    and so we just return from all recursive visit calls.
    The final connected component reported by the algorithm is
    $\set{f,g,h}$.
  \end{itemize}
  % 
  Summing up, at the end of execution the algorithm has determined
  that the strongly connected components of the graph~$G$ in
  \reffig{fig:ssc-graph}
  are
  $\set{a,b,c}$,
  $\set{d,e}$,
  and
  $\set{f,g,h}$.
\end{solution}


\begin{subproblem}
  (30 p)
% 15 p for explanation, 15 p for complexity analysis
  In order to get a deeper understanding of operations on Boolean
  matrices, Jakob has performed some fairly extensive experiments on
  adjacency matrices~$A_G$ for small directed graphs~$G$.
  He now claims to have made the discovery that if he computes
  \begin{equation*}
    \Lor_{i=1}^{\infty}
%     \bigodot
    \bigl( A_G
    \bigr)_{\odot}^{i}
    =
    A_G \lor
    (A_G \odot A_G) \lor
    (A_G \odot A_G \odot A_G) \lor 
    (A_G \odot A_G \odot A_G \odot A_G) \lor 
    \ldots    
    \eqcomma
  \end{equation*}
  then it holds 
  (possibly after reordering the vertices in~$G$, corresponding to swapping
  rows and columns in~$A_G$)
  that this matrix can be written on the form
  \begin{equation*}
    \Lor_{i=1}^{\infty}
%     \bigodot
    \bigl( A_G
    \bigr)_{\odot}^{i}
    =
    \begin{pmatrix}
      M_{1, 1} & M_{1, 2} & \cdots & M_{1,s}
      \\ 
      M_{2, 1} & M_{2, 2} & \cdots & M_{2,s}
      \\
      \vdots & \vdots & \ddots & \vdots 
      \\
      M_{s, 1} & M_{s, 2} & \cdots & M_{s,s}
    \end{pmatrix}
  \end{equation*}
  where the submatrices
  $M_{i,j}$
  provide information about the strongly connected components of~$G$
  in the following sense.
  If
  $G$ has $s$~strongly connnected components
  $
  C_1, \, C_2, \, \ldots, \, C_s
  $
  of sizes
  $
  n_1, \, 
  n_2, \, 
  \ldots , 
  n_s 
  $, respectively, then:
  \begin{itemize}
  \item
    Each matrix $M_{i,j}$ has dimensions $n_i \times n_j$.

  \item
    If there is a path from some
    $u \in C_i$
    to some
    $v \in C_j$
    in~$G$,
    then $M_{i,j}$ contains $1$s everywhere.
    (In particular, all matrices
    $M_{i,i}$ on the diagonal contain only~$1$s.)

  \item
    If there is no path from any
    $u \in C_i$
    to any
    $v \in C_j$
    in~$G$,
    then $M_{i,j}$ contains $0$s everywhere.

  \end{itemize}
%
  Sadly, Jakob is completely unable to explain this amazing fact,
  and he also cannot determine any upper bound on the time complexity
  of computing
  $
  \Lor_{i=1}^{\infty}
  \bigl( A_G
  \bigr)_{\odot}^{i}
  $.
  
  
  Is Jakob right about his claim? If so, present a concise and clear
  explanation
  to help Jakob see why this is so. If he is wrong,
%     why this is correct. If not, 
  give a simple, concrete
  counter-example. 
%   
  Also, regardless of whether Jakob is correct or not, can you 
  provide an efficient algorithm for computing
  $
  \Lor_{i=1}^{\infty}
  \bigl( A_G
  \bigr)_{\odot}^{i}
  $
  (for the plain adjacency matrix~$A_G$ without any row or column swaps)
  together with as tight an upper bound as possible for the time complexity?
\end{subproblem}

\begin{solution}
  Yes, Jakob is actually right this time.
%%
  In what follows, let us identify the integer indices of the rows and
  columns in~$A_G$ with the vertices, so that we think of the vertices
  in~$G$ as numbered from~$1$ to~$n$.

  First we note that   for any positive integer~$t$ it holds that
  the $(i,j)$-entry of the Boolean matrix~%
  $
  \bigl( A_G  \bigr)_{\odot}^{t}
  $
  is~$1$ precisely when there is a path from vertex~$i$ to vertex~$j$ of length 
  exactly~$t$.
  This can be shown by induction. The base case is that
  $A_G$ contains all length-$1$ paths, also known as edges.
  Suppose the claim is true for~$t$.
  Consider any path from $i$ to~$j$ of length exactly~$t+1$ and let
  $\ell$ be the last vertex before~$j$ on this path.
  Then by the induction hypothesis it holds that entry
  $(i,\ell)$ in 
  $
  \bigl( A_G  \bigr)_{\odot}^{t}
  $  
  is~$1$, and since
  $(\ell, j)$ is an edge we have that the entry
  $(\ell,j)$ in~$A_G$ is also~$1$.
  By the definition of Boolean matrix multiplication, 
  it follows that entry $(i,j)$ in
  $
  \bigl( A_G  \bigr)_{\odot}^{t+1}
  =
  \bigl( A_G  \bigr)_{\odot}^{t}
  \odot A_G
  $
  is~$1$.
  Our claim now follows by the induction principle.
%%%
%%%
  Furthermore, we can conclude that the
  $(i,j)$-entry of
  $
  \Lor_{t=1}^{\infty}
  \bigl( A_G  \bigr)_{\odot}^{t}
  $
  is~$1$ precisely when there is some path from~$i$ to~$j$ in~$G$
  of \emph{some length}~$t$, regardless of what this exact length is.

  Rearrange the vertices of~$G$ so that
  the $n_1$~first vertices are in the strongly connected component~$C_1$,
  the $n_2$~next vertices are in the strongly connected
  component~$C_2$,
  et cetera. 
  This renumbering of vertices just corresponds to a sequence of
  pairwise swaps of rows and the same pair of columns of 
  the matrix~$A_G$.
  Let us next write
  \begin{equation}
    \label{eq:A_G-infty-partition}
    \Lor_{i=1}^{\infty}
%     \bigodot
    \bigl( A_G
    \bigr)_{\odot}^{i}
    =
    \begin{pmatrix}
      M_{1, 1} & M_{1, 2} & \cdots & M_{1,s}
      \\ 
      M_{2, 1} & M_{2, 2} & \cdots & M_{2,s}
      \\
      \vdots & \vdots & \ddots & \vdots 
      \\
      M_{s, 1} & M_{s, 2} & \cdots & M_{s,s}
    \end{pmatrix}
  \end{equation}
  where we simply define the submatrices
  $M_{i,j}$ to have dimensions $n_i \times n_j$.
  Let us analyse what we know about these submatrices.
%%
  \begin{itemize}
  \item
    Consider first any two vertices $k,\ell \in C_i$.
    Since these two vertices are in the same strongly connected
    component, there are paths from~$k$ to~$\ell$ and from~$\ell$
    to~$k$, and hence     the 
    $(k,\ell)$-
    and
    $(\ell,k)$-entries in
    \refeq{eq:A_G-infty-partition}
    are both~$1$ by the reasoning above.
    Hence, the submatrix $M_{i,i}$ consists of all~$1$s.

  \item
    Consider next any pair of vertices
    $k \in C_i$
    and
    $\ell \in C_j$
    for $i \neq j$
    and suppose that there is
    a path from~$k$ to~$\ell$. Then the
    $(k,\ell)$-entry in~\refeq{eq:A_G-infty-partition}
    is~$1$, again by the reasoning above. 
    Furthermore, since there is a path from any vertex
    in~$C_i$ to~$k$ and from~$\ell$  to any vertex in~$C_j$,
    we conclude that 
    the submatrix $M_{i,j}$ must consist of all~$1$s in this case.


  \item
    Consider finally any pair of vertices
    $k \in C_i$
    and
    $\ell \in C_j$
    for $i \neq j$
    such that there is no 
    path from~$k$ to~$\ell$. Then 
    it follows from the previous case in our case analysis
    that there cannot exist any path from any vertex in~$C_i$
    to any vertex in~$C_j$, and so 
    $M_{i,j}$ must contain $0$s everywhere.

  \end{itemize}
% 
  This establishes all the properties that Jakob claimed.
     
  Let us finally consider the time complexity of computing
  $
  \Lor_{i=1}^{\infty}
  \bigl( A_G
  \bigr)_{\odot}^{i}
  $.
  We make three key observations:
  \begin{enumerate}
  \item 
    If there is a path between any two vertices in an $n$-vertex
    graph, then a shortest such path cannot have length larger than~$n$.
    Hence, it holds that
    $
    \Lor_{i=1}^{\infty}
    \bigl( A_G
    \bigr)_{\odot}^{i}
    =
    \Lor_{i=1}^{n}
    \bigl( A_G  \bigr)_{\odot}^{i}
    $.

  \item 
    If we have already computed
    $
    \bigl( A_G  \bigr)_{\odot}^{t}
    $, 
    then we can compute
    $
    \bigl( A_G  \bigr)_{\odot}^{t+1}
    =
    \bigl( A_G  \bigr)_{\odot}^{t}
    \odot
    A_G
    $
    in time~$\Bigoh{n^3}$
    just by coding up the definition of Boolean matrix multiplication
    in the most straightforward way possible.

  \item 
    The Boolean disjunction $A \lor B$
    of two $n \times n$ matrices~$A$ and~$B$
    can be computed in time~$\Bigoh{n^2}$,
    again by just writing down the definition in code.
    Hence, if we have already computed 
    the two matrices
    $
    \Lor_{i=1}^{t}
    \bigl( A_G  \bigr)_{\odot}^{i}
    $
    and
    $
    \bigl( A_G  \bigr)_{\odot}^{t+1}
    $,
    then 
    we can compute
    $
    \Lor_{i=1}^{t+1}
    \bigl( A_G  \bigr)_{\odot}^{i}
    =
    \Lor_{i=1}^{t}
    \bigl( A_G  \bigr)_{\odot}^{i}
    \lor
    \bigl( A_G  \bigr)_{\odot}^{t+1}
    $
    in time~$\Bigoh{n^2}$.
  \end{enumerate}
  Putting all of this together, 
  we can compute
  $
  \Lor_{i=1}^{t+1}
  \bigl( A_G  \bigr)_{\odot}^{i}
  $
  from
  $
  \Lor_{i=1}^{t}
  \bigl( A_G  \bigr)_{\odot}^{i}
  $
  in time~$\Bigoh{n^3}$,
  and doing this for $t = 1, 2, \ldots, n$
  yields a total time complexity of~$\Bigoh{n^4}$.
  Note that we do not claim that this is a tight bound, 
%     but it is certainly as tight as possible 
  but this certainly enough for a full score 
  given what we have learned
  during this course.
\end{solution}
\end{problem}


\begin{problem}
    (80 p)
%%% a: 30 p, b: 10 p, c: 20 p, d: 20 p
    The purpose of this problem is
    to deepen our understanding of minimum spanning trees in
    undirected graphs.
  
  \begin{subproblem}%
    \label{problem:mst}%
    (30 p)
%       We assume again that this graph is represented in adjacency list
%       format, with the neighbours in each adjacency list sorted in
%       lexicographic order.
%       
    Generate a minimum spanning tree by running Kruskal's
    algorithm by hand on the 
    graph    in \reffig{fig:mst-graph}.
%
    Assume that edges of the same weight are sorted in lexicographic
    order 
    (so that for three hypothetical edges
    $(u,v)$, $(u,w)$, and
    $(v,w)$ of the same weight, 
    we would have~$(u,v)$ coming before $(u,w)$, which would in turn come before
    $(v,w)$).
    
    Describe how the forest
    changes at each step (but you do not need to describe in detail
    how the set operations are implemented).
    Also show the final tree produced by the algorithm.

    
\begin{figure}[t] %%  [tp]
  \centering
  \includegraphics{Figures/mst-graph.1}
%
  \caption{Undirected graph for
    which to compute
    minimum spanning tree in 
    Problem~\ref{problem:mst}.
  }
  \label{fig:mst-graph}
\end{figure}

  

  \end{subproblem}

\begin{solution}
  We illustrate the result of running Kruskal's algorithm in 
  \reffig{fig:mst-graph-with-spanning-tree}.
  A detailed description of the execution follows.
  After sorting the edges in increasing order of weight
  and splitting ties as specified in the problem statement, 
  we will have the edges listed in the following order, which is also
  the order in which they will be processed:
  \begin{description}
  \item[\emph{Weight 1:}]
    $(b,c)$,
    $(g,h)$.

    
\begin{figure}[t] %%  [tp]
  \centering
  \includegraphics{Figures/mst-graph.2}
%
  \caption{Graph in \reffig{fig:mst-graph}
    with minimum spanning tree as computed by Kruskal's algorithm.
  }
  \label{fig:mst-graph-with-spanning-tree}
\end{figure}

  

  \item[\emph{Weight 2:}]
    $(c,d)$.

  \item[\emph{Weight 3:}]
    $(b,d)$.

  \item[\emph{Weight 4:}]
    $(a,d)$,
    $(e,f)$.

  \item[\emph{Weight 5:}]
    $(a,b)$.    

  \item[\emph{Weight 6:}]
    $(a,c)$,
    $(d,e)$.


  \item[\emph{Weight 7:}]
    $(c,e)$,
    $(d,f)$.

  \item[\emph{Weight 8:}]
    $(c,f)$,
    $(e,g)$.

  \item[\emph{Weight 9:}]
    $(f,h)$.


  \end{description}
  The algorithm now does the following, starting with a set of
  singleton subtrees
  $\set{a}$,
  $\set{b}$,
  $\set{c}$,
  $\set{d}$,
  $\set{e}$,
  $\set{f}$,
  $\set{g}$,
  and
  $\set{h}$:    
  \begin{enumerate}
  \item
    Looking first at edges of weight~$1$, 
    according to the sorting order specified in the problem statement the edge
    $(b,c)$
    is considered first. 
    This edge merges
    $\set{b}$ and
    $\set{c}$
    into a subtree $\set{b,c}$ and is added.

  \item
    Next we consider the edge
    $(g,h)$,
    which merges 
    $\set{g}$ and
    $\set{h}$
    into  another subtree $\set{g,h}$ and is added.

  \item 
    Moving on to edges of weight~$2$, 
    the edge $(c,d)$ grows the subtree $\set{b,c}$ further to
    $\set{b,c,d}$ and is added.

  \item
    The edge
    $(b,d)$
    of weight $3$ creates a cycle and so is discarded.


  \item 
    For the edges of weight~$4$,
    according to our sorting order the edge
    $(a,d)$ is considered first.
    This edge
    grows the subtree $\set{b,c,d}$ to
    $\set{a,b,c,d}$ and is added.

  \item The edge $(e,f)$
    creates a subtree $\set{e,f}$
    and is added.

  \item
    The edge $(a,b)$ of weight~$5$ creates a cycle and is thrown away.

  \item
    Considering next edges of weight~$6$, 
    $(a,c)$ also creates a cycle and is discarded.

  \item
    The edge $(d,e)$ joins the two trees
    $\set{a,b,c,d}$
    and
    $\set{e,f}$
    into a larger tree
    $\set{a,b,c,d,e,f}$,
    and so is added.

  \item
    For edges of weight~$7$, 
    both edges
    $(c,e)$
    and
    $(d,f)$
    create cycles and are discarded.

  \item
    For weight~$8$,
    we first consider the edge
    $(c,f)$,
    which also creates a cycle and so is thrown away.

  \item
    The edge
    $(e,g)$
    creates a spanning tree for the full graph, and so is added.
    Since we know we now have the  the right number of
    edges, and adding any more edge must create a cycle somewhere, 
    we can terminate without considering the final remaining edge
    $(f,h)$.
  \end{enumerate}
\end{solution}


  \begin{subproblem}
    (10 p)
    Suppose that some vertex~$v$ 
    with several neighbours
    in a graph~$G$ has a unique
    neighbour~$u$
    such that the edge~$(u,v)$ has strictly smaller weight than any
    other edge incident to~$v$. Is it true that the edge~$(u,v)$ must
    be included in any minimum spanning tree? 
    Prove this or give a simple counter-example.
  \end{subproblem}

  
\begin{solution}
  Yes, any
%     MST 
  minimum spanning tree
  has to include the edge $(u,v)$,
  since this is the unique light edge in a cut with vertex~$v$ on one side and
  the rest of the graph on the other side.
  If there were an MST without the edge~$(u,v)$,
  then adding that edge to the MST would create a cycle, from which we
  could remove a heavier edge and get another spanning tree for the
  graph. 
  This contradicts that the tree we started with was an MST.
\end{solution}

  \begin{subproblem}
    (20 p)
    Suppose that some vertex~$v$ 
    with several neighbours
    in a graph~$G$ has a unique
    neighbour~$u$
    such that the edge~$(u,v)$ has strictly larger weight than any
    other edge incident to~$v$. Is it true that the edge~$(u,v)$ 
    can never 
    be included in any minimum spanning tree? 
    Prove this or give a simple counter-example.
  \end{subproblem}
  
  
\begin{solution}
  No, this is not true. Consider the case when $u$ has only one
  neighbour and  this neighbour is~$v$. Then the edge~$(u,v)$ has to
  be included in any MST.  
\end{solution}


  \begin{subproblem}
    (20 p)
    Suppose that $T$ is a minimum spanning tree for a weighted,
    undirected graph~$G$. Modify~$G$ by adding some 
    constant~$c \in \R^+$ to all edge weights. Is
    $T$ still a minimum spanning tree? 
    Prove this or give a simple counter-example.
  \end{subproblem}

\begin{solution}
  If the spanning tree has $M$~edges, then after the edge weight increase
  the new tree has a total weight that increased by an amount
  $M \cdot c$.
  But any spanning tree for a graph~$G$ will have the same number of
  edges, and hence the weight increase will be exactly the same for
  all spanning trees. This means, in particular, that all MSTs before
  the weight increase remain MSTs also after the weight increase.
\end{solution}

\end{problem}


\begin{problem}
  (100 p)
% a: 60 p; b, c: 10 p; d: 20 p
%     \label{problem:dijkstra}%
  Assume that we are given the directed graph
  in  \reffig{fig:directed-graph}.
  The graph is given to us
  in adjacency list representation,
  with the  out-neighbours in each adjacency list sorted in lexicographic
  order, so that this is the order
  in which vertices  are encountered when going through neighbour
  lists.
  (For instance, the out-neighbour list of $a$ is
  $(b, d, e, g)$
  sorted in this order.)


\begin{figure}[t] %%  [tp]
  \centering
  \includegraphics{Figures/directedgraph.1}
%
  \caption{Directed graph for 
    Dijkstra's algorithm in 
    Problem~\ref{problem:dijkstra}.
  }
  \label{fig:directed-graph}
\end{figure}


%%% Local Variables:
%%% mode: latex
%%% TeX-master: "DMFS_Exam_210414"
%%% End:


  \begin{subproblem}%
    \label{problem:dijkstra}%
    (60 p)
    Run Dijkstra's algorithm by hand on  this graph, starting in the
    vertex~$a$. Use a heap for the priority queue implementation.
    Assume that in the array representing the heap, the vertices
    are initially listed in lexicographic order.

    During the execution of the algorithm,
    describe for every vertex which neighbours are being
    considered and how they are dealt with.
    Show for the first two dequeued vertices how the heap changes
    after the dequeueing operations and all ensuing key value updates
    in the priority queue.
    For the rest of the vertices, it is sufficient to just describe
    how the key values are updated, without redrawing the heap after
    each operation, but you still have to describe how the algorithm
    considers all neighbours of the dequeued vertices.
    Finally, show the directed tree~$T$ produced at the end of the algorithm.
  \end{subproblem}

\begin{solution}
  We illustrate the heap used for the priority queue and how it changes in 
  \reffig{fig:heaps}. 
  At the outset, the vertex $a$ has key~$0$ and all other vertices
  have key~$\infty$.
  We will use the
  notation  $v:k$  in the heap when vertex $v$ has key value~$k$.
  
  
\begin{figure}[t] %%  [tp]
  \begin{subfigure}[b]{0.32\textwidth}
    \centering
    \includegraphics{Figures/heapsDijkstra-large.1}
    \caption{After dequeueing $a$.}
    \label{fig:heap-1}
  \end{subfigure}
  \hfill
  \begin{subfigure}[b]{0.32\textwidth}
    \centering
    \includegraphics{Figures/heapsDijkstra-large.2}
    \caption{After relaxing $(a,b)$.}
    \label{fig:heap-2}
  \end{subfigure}
  \hfill
  \begin{subfigure}[b]{0.32\textwidth}
    \centering
    \includegraphics{Figures/heapsDijkstra-large.3}
    \caption{After relaxing $(a,d)$.}
    \label{fig:heap-3}
  \end{subfigure}

  \begin{subfigure}[b]{0.32\textwidth}
    \centering
    \includegraphics{Figures/heapsDijkstra-large.4}
    \caption{After relaxing $(a,e)$.}
    \label{fig:heap-4}
  \end{subfigure}
  \hfill
  \begin{subfigure}[b]{0.32\textwidth}
    \centering
    \includegraphics{Figures/heapsDijkstra-large.5}
    \caption{After relaxing $(a,g)$.}
    \label{fig:heap-5}
  \end{subfigure}
  \hfill
  \begin{subfigure}[b]{0.32\textwidth}
    \centering
    \includegraphics{Figures/heapsDijkstra-large.6}
    \caption{After dequeueing $b$.}
    \label{fig:heap-6}
  \end{subfigure}

  \begin{subfigure}[b]{0.32\textwidth}
    \centering
    \includegraphics{Figures/heapsDijkstra-large.7}
    \caption{After relaxing $(b,c)$.}
    \label{fig:heap-7}
  \end{subfigure}
  \hfill
  \begin{subfigure}[b]{0.32\textwidth}
    \centering
    \includegraphics{Figures/heapsDijkstra-large.8}
    \caption{After relaxing $(b,e)$.}
    \label{fig:heap-8}
  \end{subfigure}
  \hfill
  \begin{subfigure}[b]{0.32\textwidth}
    \centering
    \includegraphics{Figures/heapsDijkstra-large.9}
    \caption{After relaxing $(b,f)$.}
    \label{fig:heap-9}
  \end{subfigure}

  \caption{Heap configurations for priority queue in
    Problem~\ref{problem:dijkstra}.
  }
  \label{fig:heaps}
\end{figure}


%%% Local Variables:
%%% mode: latex
%%% TeX-master: "DMFS_Exam_210414_WithSolutions"
%%% End:


  \begin{enumerate}
  \item 
    After $a$ has been dequeued, vertex~$h$ is moved to the top of the heap
    and we have the configuration in
    \reffig{fig:heap-1}.
    Relaxing the edge $(a,b)$ shifts $b$ to the top
    and pushes~$h$ down, yielding
    \reffig{fig:heap-2}.
    Relaxing~$(a,d)$ swaps $d$ and~$h$, yielding
    \reffig{fig:heap-3}.
    Relaxing~$(a,e)$ updates the key of~$e$, but since it is
    still larger than the key of the parent~$d$ nothing moves in the
    heap (see \reffig{fig:heap-4}).
    Finally, relaxing~$(a,g)$ makes~$g$ bubble up and $c$ bubble down,
    so that we have the configuration in 
    \reffig{fig:heap-5}
    when all outgoing edges from~$a$ have been relaxed.

  \item 
    Since $b$ is now at the top of the heap, it is the vertex dequeued
    next. This will add the edge~$(a,b)$ to the shortest path tree,
    which we indicate in
    \reffig{fig:directed-graph-with-tree}.
    When $b$ is removed, $c$~is moved to the top. This violates the
    min-heap property, since the key of~$c$ is larger than that of its
    children.  Since $d$~has     smaller key than~$g$, we swap $d$
    and~$c$. The min-heap property in the subtree rooted at~$c$, i.e., the
    left subtree of the heap, is now violated
    since the key of~$c$ is still not smaller than or equal to that of
    its children. Since $e$ has smaller key than~$h$, $c$~and $e$
    trade places. The subtree rooted at $c$ is now a single vertex,
    and so is a legal min-heap, and the full heap after removal of~$b$
    looks as in 
    \reffig{fig:heap-6}.    

  
\begin{figure}[t] %%  [tp]
  \centering
  \includegraphics{Figures/directedgraph.2}
%
  \caption{Shortest paths computed by Dijkstra's algorithm
    for graph in
    \reffig{fig:directed-graph}.
  }
  \label{fig:directed-graph-with-tree}
\end{figure}


%%% Local Variables:
%%% mode: latex
%%% TeX-master: "DMFS_Exam_210414_WithSolutions"
%%% End:


    Relaxing the edge~$(b,c)$ decreases the key value of~$c$ to
    $1 + 1 = 2$. Since the key value of~$c$ is now smaller than that
    of~$e$, $c$~bubbles up and $e$ bubbles down, and since $c$ also
    has a smaller key than~$d$ these two vertices also trade places,
    yielding the heap in
    \reffig{fig:heap-7}.
    Relaxing~$(b,e)$ swaps $d$ and~$e$, resulting in    \reffig{fig:heap-8}.    
    Finally, relaxing $(b,f)$ updates the key of~$f$ but does not
    change the structure of the heap, since the parent~$g$ of~$f$  has
    a smaller key (see \reffig{fig:heap-9}).

  \item
    Since $c$ is now at the top of the heap, it is dequeued next,
    and the edge~$(b,c)$ is added to the shortest paths tree.
    When we relax~$(c,e)$, we see that the distance~$2$ to~$c$ plus
    the edge weight~$4$ sum to~$6$, which is larger than the current
    key~$3$ of~$c$, so no update of~$c$ is made. 
    Relaxing the edge~$(c,f)$ decreases the key of~$f$ to $2+7 = 9$,
    however.

  \item
    Vertex~$e$ now has the smallest key~$3$ and is dequeued next.
    This adds the edge~$(b,e)$ to the shortest paths tree, since
    the key of~$e$ was last updated when $(b,e)$ was relaxed.
    Relaxing 
    $(e,d)$,
    $(e,f)$,
    and
    $(e,h)$
    yields updated key values $4$, $8$, and~$7$, respectively.

  \item
    Now vertex~$d$ has the smallest key~$4$ and so is dequeued,
    adding~$(e,d)$ to the shortest paths tree.
    When we relax~$(d,g)$, we see that the distance~$4$ to~$d$ plus
    the edge weight~$5$ sum to~$9$, which is not smaller than the current
    key~$9$ of~$g$, so no update is made. 
    Relaxing~$(d,h)$ decreases the key of~$h$ to~$6$, however.

  \item
    Vertex~$h$ is dequeued next, adding the just relaxed edge~$(d,h)$
    to the shortest paths tree. 
    Relaxing
    $(h,g)$
    and
    $(h,f)$
    leads to decreased keys $8$ and~$7$, respectively.

  \item
    Next, $f$ is dequeued, adding the just relaxed edge~$(h,f)$.

  \item
    Since $f$~has no out-neighbours, we proceed to dequeueing the
    final vertex~$g$ in the queue, adding the edge    $(h,g)$.

  \end{enumerate}
%   
  The computed shortest paths tree is indicated by the bold edges in
  \reffig{fig:directed-graph-with-tree}.
\end{solution}

  \begin{subproblem}
    (10 p)
    Consider the directed tree of shortest paths~$T$
    produced in 
    Problem~\ref{problem:dijkstra}.
    Suppose that all edge weights in~$G$ are changed by some additive
    constant
    $c \in \Rplus$.
    Is it true that $T$ is still a directed tree of shortest paths
    for the modified graph?
    Please make sure to motivate your answer clearly.
  \end{subproblem}

\begin{solution}
  No, this is not true.
  If we choose $c$ to be large enough, 
  then the shortest path between two vertices will always be the one with
  the fewest edges. This is not the case for the tree~$T$
  computed in   Problem~\ref{problem:dijkstra}.
  (Consider, e.g., the shortest path
  $a 
  \to b 
  \to e 
  \to d 
  \to h 
  \to f
  $
  from~$a$ to~$f$, which has many more edges than
  $
  a \to b \to f$.)
%     Incidentally, we discussed precisely this question in class during
%     one of the lectures,  so for that reason this should hopefully have
%     been a very easy subproblem.
\end{solution}


  \begin{subproblem}
    (10 p)
    Consider the directed tree of shortest paths~$T$
    produced in 
    Problem~\ref{problem:dijkstra}.
    Suppose that all edge weights in~$G$ are changed by some multiplicative
    constant
    $c \in \Rplus$.
    Is it true that $T$ is still a directed tree of shortest paths
    for the modified graph?
    Please make sure to motivate your answer clearly.
  \end{subproblem}

\begin{solution}
  Yes, this is true.
  Since all weights change by a factor~$c$,
  the cost of all paths also change by a factor~$c$.
  Hence, all shortest paths remain shortest paths, and the tree~$T$ is
  still good. 
%     (As far as the instructor can remember, this was not discussed in
%     class, but it was an exercise for one of the exercise sessions.)    
\end{solution}


  \begin{subproblem}
    (20 p)
    Suppose that we want to give an extra bonus to paths with few
    hops, so that the length of a path is calculated as
    the sum of the weight of all the edges in the path 
    \emph{plus} the number of edges in the path.
    Describe an algorithm that can solve this problem
    (for any directed graph~$G$ with non-negative edge weights)
    and analyze its time complexity.
  \end{subproblem}

\begin{solution}
  This problem can be solved in different ways.
  The easiest way is probably to read in the graph, 
  add~$1$ to all edge weights, and then run Dijkstra's algorithm as
  before.
  The length of any path after this change will clearly be 
  the sum of the weight of all the edges in the path 
  plus the number of edges in the path,
  and so Dijkstra's algorithm will return the answers we are looking for.
  Since the preprocessing step can be performed in time
  $\bigoh
  {
    \setsize{V} + \setsize{E}
  }$  
  for a graph with vertex set~$V$ and edge set~$E$, the running time of
  Dijkstra's algorithm dominates, and the asymptotic time complexity
  is the same as for the original version of Dijkstra's algorithm,
  i.e.,~$\bigoh{\setsize{E} \log \setsize{V}}$.
\end{solution}


\end{problem}




\end{document}


